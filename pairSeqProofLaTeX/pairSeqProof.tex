\documentclass[dvipdfmx,uplatex]{jsarticle}
\usepackage[utf8]{inputenc}
\usepackage[
	backend=bibtex,
	style=alphabetic,
	sorting=ynt
]{biblatex}
\addbibresource{references.bib}
\usepackage{comment}
\usepackage{amsfonts}
\usepackage{framed}
\usepackage[thref,amsthm,framed,thmmarks,hyperref]{ntheorem}
\usepackage{enumitem}
\usepackage[dvipdfmx]{hyperref}
\usepackage{pxjahyper}

\newif\iffull
\fulltrue
%\fullfalse

\makeatletter
\newtheoremstyle{customnonumberbreakfortheorem}%
	{\item[\rlap{\vbox{\hbox{\hskip\labelsep\theorem@headerfont##1\theorem@separator}\hbox{\strut}}}]}%
	{\item[\rlap{\vbox{\hbox{\hskip\labelsep\theorem@headerfont##1 (##3)\gdef\@currentlabelname{##3}\theorem@separator}\hbox{\strut}}}]}
\makeatother
\theoremstyle{customnonumberbreakfortheorem}
\theoremheaderfont{\normalfont\bfseries}
\theorembodyfont{\normalfont}
\theoremseparator{}
\theorempreskip{\topsep}
\theorempostskip{\topsep}

\newframedtheorem{theorem}{定理}
\newframedtheorem{proposition}{命題}
\newframedtheorem{lemma}{補題}
\newframedtheorem{corollary}{系}

\iffull
	\makeatletter
	\newtheoremstyle{customnonumberbreakforproof}%
		{\item[\rlap{\vbox{\hbox{\hskip\labelsep\theorem@headerfont##1\theorem@separator}\hbox{\strut}}}]}%
		{\item[\rlap{\vbox{\hbox{\hskip\labelsep\theorem@headerfont##3\theorem@separator}\hbox{\strut}}}]}
	\makeatother
	\theoremstyle{customnonumberbreakforproof}
	\theoremheaderfont{\normalfont\bfseries}
	\theoremseparator{:}
	\theorempreskip{\topsep}
	\theorempostskip{\topsep}
	\theorembodyfont{\normalfont}
	\theoremsymbol{\ensuremath{\Box}}
	\newtheorem{hideableproof}{証明}

	\newcommand{\qedhere}{\hfill\hideableproofSymbol}
\else
	\excludecomment{hideableproof}
\fi

\setlistdepth{10}
\def\indentwidth{17pt}

\newlist{nenumerate}{enumerate}{10}
\setlist[nenumerate]{
	label=\arabic*.,
	leftmargin=\indentwidth
}

\newlist{penumerate}{enumerate}{2}
\setlist[penumerate]{
	label=(\arabic*),
	leftmargin=\indentwidth
}

\newlist{indented}{itemize}{10}
\setlist[indented]{
	label=,
	leftmargin=\indentwidth
}

\newcommand{\documenttitle}{\iffull{ペア数列の停止性}\else{ペア数列の停止性 (要約版)}\fi}

\hypersetup{
	colorlinks=true,
	linkcolor=blue,
	urlcolor=cyan,
	citecolor=red,
	pdftitle={\documenttitle},
	pdfauthor={Naruyoko},
	pdfpagemode=FullScreen
}

\title{\documenttitle}
\author{
	p進大好きbot 著
	\and
	Naruyoko 編集 %これでいいのだろうか
}
\date{コンパイル日: \today}

\begin{document}

\maketitle

\renewcommand{\thefootnote}{$\diamondsuit$}

このドキュメントはp進大好きbot氏が公開したブログ記事\href{https://googology.fandom.com/ja/wiki/User\_blog:P\%E9\%80\%B2\%E5\%A4\%A7\%E5\%A5\%BD\%E3\%81\%8Dbot/\%E3\%83\%9A\%E3\%82\%A2\%E6\%95\%B0\%E5\%88\%97\%E3\%81\%AE\%E5\%81\%9C\%E6\%AD\%A2\%E6\%80\%A7}{ペア数列の停止性}を\LaTeX{}に変換したものです\footnote[0]{原文は\href{https://www.fandom.com/ja/licensing-ja}{CC-BY-SA 3.0 非移植ライセンスのもとで提供されています}。}。\iffull\else{このバージョンは要約版であり、序文や証明などが省かれています。}\fi 使用した版は\href{https://googology.fandom.com/ja/wiki/User\_blog:P\%E9\%80\%B2\%E5\%A4\%A7\%E5\%A5\%BD\%E3\%81\%8Dbot/\%E3\%83\%9A\%E3\%82\%A2\%E6\%95\%B0\%E5\%88\%97\%E3\%81\%AE\%E5\%81\%9C\%E6\%AD\%A2\%E6\%80\%A7?oldid=65412}{2024年4月28日の版}です。\href{https://creativecommons.org/licenses/by-sa/3.0/}{CC-BY-SA 3.0 非移植ライセンス}のもとで公開します。

\renewcommand{\thefootnote}{\arabic{footnote}}

\clearpage

\tableofcontents

\iffull

\section{概要}

ペア数列の停止性を示す。より正確には、数あるペア数列システムのうち1つの定式化を記述し、その中で標準形という概念を定義し、標準形のペア数列システムが停止することを証明する\footnote{当初の予定ではペア数列の解析、すなわちペア数列システムに伴う順序数表記系の限界を決定するつもりだったが、停止性の証明までを書き切るだけで大変疲れてしまった上に需要もさほどないと思うため、停止性までに留めた。}。

\fi

\section{序文}

\iffull

バシク行列システムは巨大数を研究する世界中の多くの人を魅了した巨大数生成法であり様々な亜種が作られ続けているが、オリジナルのバシク行列システム自体は固定した厳密な定義を持たず、従って停止性も一切証明されていない。バシク行列システムを取り巻く歴史的背景については\href{https://googology.wikia.com/wiki/User_blog:P\%E9\%80\%B2\%E5\%A4\%A7\%E5\%A5\%BD\%E3\%81\%8Dbot/Summary\_on\_historical\_background\_of\_BMS}{Summary on historical background of BMS}を参考にすると良い。

バシク行列システムを\(1\)行の行列(つまりベクトル)だけに制限した原始数列システムに関しては先人の度重なる議論によりほぼ万人に共通の定式化が築き上げられており、単純な木構造を用いた整列順序を用いて停止性と近似的な解析がなされている\footnote{その証明を厳密に記述した記事は知らないが、初等的であるためわざわざ書くほどでもないのだろう。}。一方でバシク行列システムを\(2\)行の行列までに制限したものがペア数列システムであり、その標準形と呼ばれる部分システムに関しては既に停止性の反例が挙がっているものを除いて停止性が強く信じられているだけで、その証明はもちろん知られておらず証明できたとしても\href{http://ja.googology.wikia.com/wiki/\%E3\%82\%AB\%E3\%83\%86\%E3\%82\%B4\%E3\%83\%AA:\%E9\%A0\%86\%E5\%BA\%8F\%E6\%95\%B0\%E5\%B4\%A9\%E5\%A3\%8A\%E9\%96\%A2\%E6\%95\%B0}{順序数崩壊関数}を用いた極めて非自明な議論を要することが想像に難くない。この記事の目標は、新たに標準形のペア数列システムの停止性を証明することである。

バシク行列システムに多くの亜種があることから、ペア数列システムという用語が指す具体的なアルゴリズムは一義的でない。厳密にペア数列システムの解析を行うためには、まずペア数列システムの定義を1つ固定しなければ始まらないため、\href{http://ja.googology.wikia.com/wiki/\%E3\%83\%A6\%E3\%83\%BC\%E3\%82\%B6\%E3\%83\%BC\%E3\%83\%96\%E3\%83\%AD\%E3\%82\%B0:Koteitan/\%E3\%83\%90\%E3\%82\%B7\%E3\%82\%AF\%E8\%A1\%8C\%E5\%88\%97\%E3\%81\%AE\%E4\%BA\%9C\%E7\%A8\%AE\%E3\%83\%AB\%E3\%83\%BC\%E3\%83\%AB\%E3\%81\%AE\%E5\%88\%86\%E9\%A1\%9E?oldid=15499}{2018/8/20時点でのkoteitan分類法}のBM1.1, 2, 2.3, 3.1, 3.2に合わせて定式化する。

Buchholzの\(\psi\)関数はBuchholzによって性質が十分に調べられていて使いやすいのでここではBuchholzの\(\psi\)関数を用いるが、ペア数列とBuchholzの\(\psi\)関数はそこまできれいに対応しないのでかなり記述が煩雑になる点が不満である。BMSの表記に特化しているだろうBashicu's OCFやUNOCFといった順序数崩壊関数を使えばきっともっと短く解析できるのかもしれないが、Bashicu's OCFとUNOCFもオリジナルのバシク行列システム同様明示的な定義を書かれたことがないため、正確な定義を必要とする厳密な証明には用いることが出来ない。もう1つの方向性としては、Bashicu's OCFの原型とされるMadoreの\(\psi\)関数を自己流に多変数化してそれを用いることが有力だろう。ただしその場合はその新たな順序数崩壊関数の性質(特に具体的な表記がどの順序数に対応するか)を自分で調べて証明しなければ解析に用いても表記限界を既知の順序数として表せないことに変わりはないので、それだったらやはり既に基本性質が証明されているBuchholzの\(\psi\)関数を用いる方が結局手早いかもしれない。

\fi

停止性の証明の方針は以下の通りである:
\begin{nenumerate}
	\item (通常の意味での標準形より広い)標準形という概念を導入し、更に広い簡約という概念を導入する。
	\item 簡約とは限らないペア数列に対して簡約化\(\textrm{Red}\)という操作を定義する。
	\item 簡約ペア数列からBuchholzの表記系への写像\(\textrm{Trans}\)を定義し、\(\textrm{Red}\)を用いて簡約とは限らないペア数列へ定義を拡張する。
	\item \(\textrm{Trans}\)の部分文字列の計算補助として\(\textrm{Mark}\)という写像を定義する。
	\item \(\textrm{Mark}\)がペア数列の部分列の\(\textrm{Trans}\)で計算できることを示す。
	\begin{nenumerate}
		\item これによりペア数列の\(\textrm{Trans}\)の部分文字列を元のペア数列の部分列の\(\textrm{Trans}\)を用いて計算できる。
		\item 簡約ペア数列の部分列は簡約とは限らないため、簡約ペア数列にしか興味がない場合でも簡約でないペア数列を用いる必要があった。
		\item 標準形とは限らないペア数列を標準形に置き換える操作は不明であるので、標準形にしか興味が無い場合も標準形でないペア数列を用いる必要があった。
	\end{nenumerate}
	\item ペア数列の展開規則と\(\textrm{Trans}\)による像の基本列を比較する。
	\begin{nenumerate}
		\item これにより、標準形ペア数列の\(\textrm{Trans}\)による像がBuchholzの順序数項を定めることを示す。
		\item 更にBuchholzの順序数項が標準的な全順序に関して整礎であることと基本列がその順序に関して降下することから、標準形ペア数列システムの停止性が従う。
	\end{nenumerate}
\end{nenumerate}
\href{https://googology.wikia.com/wiki/User_blog:P\%E9\%80\%B2\%E5\%A4\%A7\%E5\%A5\%BD\%E3\%81\%8Dbot/Introduction\_to\_the\_Termination\_of\_Pair\_Sequence\_System}{方針の概説}も参考にすると良い。

\iffull

なお\(n \in \mathbb{N}\)に対する命題\(P(n), Q(n)\)に対して「任意の\(n \in \mathbb{N}\)に対し\(P(n) \to Q(n)\)」を\(n\)に関する数学的帰納法で示す際に、\(P(0)\)が成り立たない場合は\(P(0) \to Q(0)\)が成り立つので「任意の\(n \in \mathbb{N}\)に対し\(P(n) \to Q(n)\)ならば\(P(n+1) \to Q(n+1)\)」を示すだけで良いのだが、読者の理解の助けになるようになるべく記事中では\(P(n_0)\)が成り立つ最小の\(n_0 \in \mathbb{N}\)を調べて\(Q(n_0)\)も示すようにする。

\fi

\printbibliography[heading=bibnumbered,title={参考文献}]

\section{記法}

\(\mathbb{N}\)で非負整数全体の集合を表し、\(\mathbb{N}_{+}\)で正整数全体の集合を表す。

クラス\(A\)に対し、集合\(a\)が\(A\)値配列であるとは、ある\(n \in \mathbb{N}\)が存在して\(a \in A^n\)ということである。このような\(n\)は\(a\)に対して一意であるので\(\textrm{Lng}(a)\)と表す。\(\textrm{Lng}(a) = 0\)の時、\(A\)に紛れのない限り\(a\)を\(()\)と表す。\(i < \textrm{Lng}(a)\)を満たす各\(i \in \mathbb{N}\)に対し、\(a\)の第\(i\)成分を\(a_i \in A\)と表す。

\(i_0 \leq i_1\)を満たす各\(i_0,i_1 \in \mathbb{N}\)と、\(\{i \in \mathbb{N} \mid i_0 \leq i \leq i_1\}\)を定義域に含む各\(A\)値関数\(f\)に対し、長さが\(i_1-i_0+1\)であって任意の\(i \in \mathbb{N}\)に対し\(i \leq i_1-i_0\)ならば第\(i\)成分が\(f(i_0+i)\)であるような\(A\)値配列を\((f(i))_{i=i_0}^{i_1}\)と表す。\(i_0 \leq i_1\)を満たさないかまたは\(\{i \in \mathbb{N} \mid i_0 \leq i \leq i_1\}\)を\(f\)が定義域に含まないような各\(i_0,i_1 \in \mathbb{N}\)と各\(A\)値関数\(f\)に対し、\((a_i)_{i=i_0}^{i_1} := ()\)と置く。

クラス\(A\)に対し、\(A\)値配列全体のクラスを\(A^{< \omega}\)と置き、\(A^{< \omega}\)上の二項演算
\begin{eqnarray*}
\oplus_A \colon (A^{< \omega})^2 & \to & A^{< \omega} \\
(a,b) & \mapsto & a \oplus_A b
\end{eqnarray*}
を以下のように定める:
\begin{nenumerate}
	\item \(j_0 := \textrm{Lng}(a) - 1\)と置く。
	\item \(j_1 := j_0 + \textrm{Lng}(b)\)と置く。
	\item \(j \leq j_0\)を満たす各\(j \in \mathbb{N}\)に対し、\(c_j := a_j\)と置く。
	\item \(j_0 < j \leq j_1\)を満たす各\(j \in \mathbb{N}\)に対し、\(c_j = b_{j - j_0 - 1}\)と置く。
	\item \(a \oplus_A b := (c_j)_{j=0}^{j_1}\)である。
\end{nenumerate}

\(\oplus_A\)は結合的であるため、\(\oplus_A\)の反復的適用において省略可能なカッコは省略する。

写像
\begin{eqnarray*}
\bigoplus_A \colon (A^{< \omega})^{< \omega} & \to & A^{< \omega} \\
a & \mapsto & \bigoplus_A a
\end{eqnarray*}
を以下のように再帰的に定める:
\begin{nenumerate}
	\item \(j_1 := \textrm{Lng}(a) - 1\)と置く。
	\item \(j_1 = -1\)ならば\(\bigoplus_A a := ()\)である。
	\item \(j_1 = 0\)ならば\(\bigoplus_A a := a_0\)である。
	\item \(j_1 > 0\)ならば\(\bigoplus_A a := \left( \bigoplus_A (a_i)_{i=0}^{j_1-1} \right) \oplus_A a_{j_1}\)である。
\end{nenumerate}

何らかの条件を満たす配列の一意存在性を示す際、一意性が文字列の末尾または先頭からの比較により即座に従う場合は一意性の存在を省略する。


\section{定式化}

集合\(M\)がペア数列であるとは、\(M\)が\(()\)でない\(\mathbb{N}^2\)値配列であるということである。\(i \in \{0,1\}\)と\(\textrm{Lng}(M)\)未満の\(j \in \mathbb{N}\)の組\((i,j)\)全体の集合を\(\textrm{Idx}(M)\)と置く。各\((i,j) \in \textrm{Idx}(M)\)に対し、\((M_j)_i\)を\(M_{i,j}\)と置く。

ペア数列全体の集合を\(T_{\textrm{PS}}\)と置く。


\subsection{親子関係}

\(M \in T_{\textrm{PS}}\)とする。\(\mathbb{Z}^2\)上の二項関係\(<_M^{\textrm{Next}}\)と\(\leq_M\)を以下のように同時に再帰的に定義する。
\begin{nenumerate}
	\item \((i_0,j_0), (i_1,j_1) \in \mathbb{Z}^2\)に対し、\((i_0,j_0)  <_M^{\textrm{Next}} (i_1,j_1)\)であるとは、以下を満たすということである:
	\begin{nenumerate}
		\item \((i_0,j_0),(i_1,j_1) \in \textrm{Idx}(M)\)である。
		\item \(i_0 = i_1\)である。
		\item \(j_0 < j_1\)である。
		\item \(M_{i_0,j_0} < M_{i_1,j_1}\)である。
		\item \(i_0 = 0\)ならば、任意の\(j \in \mathbb{N}\)に対し、\(j_0 < j < j_1\)ならば\(M_{0,j} \geq M_{0,j_1}\)である。
		\item \(i_0 = 1\)ならば、\((0,j_0) \leq_M (0,j_1)\)かつ任意の\(j \in \mathbb{N}\)に対し、\(j_0 < j\)かつ\((0,j) \leq_M (0,j_1)\)ならば\(M_{1,j} \geq M_{1,j_1}\)である。
	\end{nenumerate}
	\item \((i_0,j_0) \leq_M (i_1,j_1)\)であるとは、以下を満たすということである:
	\begin{nenumerate}
		\item \((i_0,j_0),(i_1,j_1) \in \textrm{Idx}(M)\)である。
		\item \(i_0 = i_1\)である。
		\item ある\(a \in \mathbb{N}^{< \omega}\)が存在し、\(J := \textrm{Lng}(a) - 1\)と置くと以下を満たす:
		\begin{nenumerate}
			\item \(a \neq ()\)である。
			\item \(a_0 = j_0\)である。
			\item 任意の\(k \in \mathbb{N}\)に対し、\(k < J\)ならば\((i_0,a_k) <_M^{\textrm{Next}}(i_1,a_{k+1})\)である。
			\item \(a_J = j_1\)である。
		\end{nenumerate}
	\end{nenumerate}
\end{nenumerate}

\begin{proposition}[親の存在の判定条件]\label{親の存在の判定条件}
	任意の\(M \in T_{\textrm{PS}}\)と\(j_0,j_1 \in \mathbb{N}\)に対し、\(j_0 < j_1 < \textrm{Lng}(M)\)ならば以下が成り立つ:
	\begin{penumerate}
		\item \(M_{0,j_0} < M_{0,j_1}\)ならば、ある\(j \in \mathbb{N}\)が存在して\(j_0 \leq j < j_1\)かつ\((0,j) <_M^{\textrm{Next}} (0,j_1)\)である。
		\item \(M_{1,j_0} < M_{1,j_1}\)かつ\((0,j_0) \leq_M (0,j_1)\)ならば、ある\(j \in \mathbb{N}\)が存在して\(j_0 \leq j < j_1\)かつ\((1,j) <_M^{\textrm{Next}} (1,j_1)\)である。
		\item 条件「任意の\(j \in \mathbb{N}\)に対し\(j_0 < j \leq j_1\)ならば\(M_{0,j_0} < M_{0,j}\)である」を満たすならば、\((0,j_0) \leq_M (0,j_1)\)である。
		\item 条件「任意の\(j \in \mathbb{N}\)に対し\(j_0 < j\)かつ\((0,j) \leq_M (0,j_1)\)ならば\(M_{1,j_0} < M_{1,j}\)である」と\((0,j_0) \leq_M (0,j_1)\)を満たすならば、\((1,j_0) \leq_M (1,j_1)\)である。
	\end{penumerate}
\end{proposition}

\begin{hideableproof}
	\begin{penumerate}
		\item 仮定より\(j < j_1\)かつ\(M_{0,j} < M_{0,j_1}\)を満たす最大の\(j \in \mathbb{N}\)が存在し、その最大性から\(j_0 \leq j < j_1\)かつ\((0,j) <_M^{\textrm{Next}} (0,j_1)\)である。
		\item 仮定より\(M_{1,j} < M_{1,j_1}\)かつ\((0,j) \leq_M (0,j_1)\)を満たす最大の\(j \in \mathbb{N}\)が存在し、その最大性から\(j_0 \leq j < j_1\)かつ\((1,j) <_M^{\textrm{Next}} (1,j_1)\)である。
		\item 仮定と(1)よりある\(j \in \mathbb{N}\)が存在して\(j_0 \leq j < j_1\)かつ\((0,j) <_M^{\textrm{Next}} (0,j_1)\)である。従って\(j_1\)に関する数学的帰納法より従う。
		\setcounter{penumeratei}{2}
		\item 仮定と(2)よりある\(j \in \mathbb{N}\)が存在して\(j_0 \leq j < j_1\)かつ\((1,j) <_M^{\textrm{Next}} (1,j_1)\)である。従って\(j_1\)に関する数学的帰納法より従う。
	\end{penumerate}
\end{hideableproof}

\begin{proposition}[親の基本性質]\label{親の基本性質}
	任意の\(M \in T_{\textrm{PS}}\)と\(j_0,j,j_1 \in \mathbb{N}\)に対し、\(j_0 < j \leq j_1\)ならば以下が成り立つ:
	\begin{penumerate}
		\item \((0,j_0) <_M^{\textrm{Next}} (0,j_1)\)ならば\(M_{0,j} \geq M_{0,j_1}\)である。
		\item \((1,j_0) <_M^{\textrm{Next}} (1,j_1)\)かつ\((0,j) \leq_M (0,j_1)\)ならば\(M_{1,j} \geq M_{1,j_1}\)である。
	\end{penumerate}
\end{proposition}

\begin{hideableproof}
	\begin{penumerate}
		\item 成り立たないと仮定して矛盾を導く。仮定より\(j_0 < j \leq j_1\)かつ\((0,j_0) <_M^{\textrm{Next}} (0,j_1)\)かつ\(M_{0,j} < M_{0,j_1}\)を満たす\(j_0,j,j_1 \in \mathbb{N}\)が存在する。そのような\((j_0,j,j_1)\)の組み合わせにおいて\(j < \textrm{Lng}(M)\)であるため、\(j\)が最大となる組み合わせ\((j_0,j,j_1)\)が存在し、それを\((j'_0,j',j'_1)\)と置く。
		\item[] \(j'\)の最大性から、任意の\(j \in \mathbb{N}\)に対し\(j' < j \leq j'_1\)ならば\(M_{0,j} \geq M_{0,j'_1}\)である。従って\((0,j') <_M^{\textrm{Next}} (0,j'_1)\)となるが、これは\(j'_0 < j'\)に反する。
		\item 成り立たないと仮定して矛盾を導く。仮定より\(j_0 < j \leq j_1\)かつ\((1,j_0) <_M^{\textrm{Next}} (1,j_1)\)かつ\((0,j) \leq_M (0,j_1)\)かつ\(M_{1,j} < M_{1,j_1}\)を満たす\(j_0,j,j_1 \in \mathbb{N}\)が存在する。そのような\((j_0,j,j_1)\)の組み合わせにおいて\(j < \textrm{Lng}(M)\)であるため、\(j\)が最大となる組み合わせ\((j_0,j,j_1)\)が存在し、それを\((j'_0,j',j'_1)\)と置く。
		\item[] \(j'\)の最大性から、任意の\(j \in \mathbb{N}\)に対し\(j' < j \leq j'_1\)かつ\((0,j) \leq_M (0,j'_1)\)ならば\(M_{1,j} \geq M_{1,j'_1}\)である。従って\((1,j') <_M^{\textrm{Next}} (1,j'_1)\)となるが、これは\(j'_0 < j'\)に反する。
	\end{penumerate}
\end{hideableproof}

\begin{corollary}[直系先祖の基本性質]\label{直系先祖の基本性質}
	任意の\(M \in T_{\textrm{PS}}\)と\(j_0,j,j_1 \in \mathbb{N}\)に対し、\(j_0 < j \leq j_1\)ならば以下が成り立つ:
	\begin{penumerate}
		\item \((0,j_0) \leq_M (0,j_1)\)ならば\(M_{0,j_0} < M_{0,j}\)である。
		\item \((1,j_0) \leq_M (1,j_1)\)かつ\((0,j) \leq_M (0,j_1)\)ならば\(M_{1,j_0} < M_{1,j}\)である。
	\end{penumerate}
\end{corollary}

\begin{hideableproof}
	\begin{indented}
		\item \nameref{親の基本性質}から、\(\leq_M\)の定義における\(J\)に関する数学的帰納法より即座に従う。
	\end{indented}
\end{hideableproof}

\begin{corollary}[直系先祖の木構造]\label{直系先祖の木構造}
	任意の\(M \in T_{\textrm{PS}}\)と\(j'_0, j, j'_1 \in \mathbb{N}\)に対し、以下が成り立つ:
	\begin{penumerate}
		\item \((0,j'_0) \leq_M (0,j'_1)\)かつ\(j'_0 \leq j \leq j'_1\)ならば、\((0,j'_0) \leq_M (0,j)\)である。
		\item \((1,j'_0) \leq_M (1,j'_1)\)かつ\(j'_0 \leq j\)かつ\((0,j) \leq_M (0,j'_1)\)ならば、\((1,j'_0) \leq_M (1,j)\)である。
	\end{penumerate}
\end{corollary}

\begin{hideableproof}
	\begin{penumerate}
		\item \nameref{親の存在の判定条件} (3)と\nameref{直系先祖の基本性質} (1)から即座に従う。
		\item \nameref{親の存在の判定条件} (4)と\nameref{直系先祖の基本性質} (2)から即座に従う。
	\end{penumerate}
\end{hideableproof}


\subsection{前者関数}

写像
\begin{eqnarray*}
\textrm{Pred} \colon T_{\textrm{PS}} & \to & T_{\textrm{PS}} \\
M & \mapsto & \textrm{Pred}(M)
\end{eqnarray*}
を以下のように定める:
\begin{nenumerate}
	\item \(j_1 := \textrm{Lng}(M)-1\)と置く。
	\item \(j_1 = 0\)ならば\(\textrm{Pred}(M) := M\)である。
	\item \(j_1 > 0\)ならば\(\textrm{Pred}(M) := (M_j)_{j=0}^{j_1-1}\)である。
\end{nenumerate}

まだ順序数表記系の構造を与えていないが、\(\textrm{Pred}\)は\(\emptyset\)に対しては\(\emptyset\)を、後続順序数に対してその最大元を取る操作に対応する。

写像
\begin{eqnarray*}
\textrm{Derp} \colon T_{\textrm{PS}} & \to & T_{\textrm{PS}} \cup \{()\} \\
M & \mapsto & \textrm{Derp}(M)
\end{eqnarray*}
を以下のように定める:
\begin{nenumerate}
	\item \(j_1 := \textrm{Lng}(M)-1\)と置く。
	\item \(\textrm{Derp}(M) := (M_j)_{j=1}^{j_1}\)である。
\end{nenumerate}

\(\textrm{Derp}\)には順序数の操作に関連する意味が特にないが、後に導入する\(\textrm{Red}\)という写像を定義するために便利なので一度だけ用いる。


\subsection{基本列}

写像
\begin{eqnarray*}
\textrm{operator}[] \colon T_{\textrm{PS}} \times \mathbb{N}_{+} & \to & T_{\textrm{PS}} \\
(M,n) & \mapsto & M[n]
\end{eqnarray*}
を以下のように再帰的に定める:
\begin{nenumerate}
	\item \(j_1 := \textrm{Lng}(M)-1\)と置く。
	\item \(j_1 = 0\)ならば\(M[n] := M\)である。
	\item \(j_1 > 0\)とする。
	\begin{nenumerate}
		\item \(M_{j_1} = (0,0)\)ならば\(M[n] := \textrm{Pred}(M)\)である。
		\item \(M_{j_1} \neq (0,0)\)とする。
		\begin{nenumerate}
			\item \(i_1 := \max \{i \in \{0,1\} \mid M_{i,j_1} > 0\}\)と置く\footnote{\(M_{j_1} \neq (0,0)\)より\(\max\)は存在する。}。
			\item \((i_1,j_0) <_M^{\textrm{Next}} (i_1,j_1)\)を満たす一意な\(j_0 \in \mathbb{N}\)が存在しないならば\(M[n] := \textrm{Pred}(M)\)である。
			\item \((i_1,j_0) <_M^{\textrm{Next}} (i_1,j_1)\)を満たす一意な\(j_0 \in \mathbb{N}\)が存在するとする。
			\begin{nenumerate}
				\item \(i < i_1\)を満たす各\(i \in \mathbb{N}\)に対し、\(\delta_i := M_{i,j_1} - M_{i,j_0}\)と置く。
				\item \(i \geq i_1\)を満たす各\(i \in \mathbb{N}\)に対し、\(\delta_i := 0\)と置く。
				\item \(G := (M_j)_{j=0}^{j_0-1} \in T_{\textrm{PS}}\)と置く。
				\item \(B := (((M_{0,j} + k \delta_0, M_{1,j} + k \delta_1))_{j=j_0}^{j_1-1})_{k=0}^{n-1} \in T_{\textrm{PS}}^n\)と置く。
				\item \(M[n] := \left(G \oplus_{\mathbb{N}^2} \left( \bigoplus_{\mathbb{N}^2} B \right) \right)\)である。
			\end{nenumerate}
		\end{nenumerate}
	\end{nenumerate}
\end{nenumerate}

まだ順序数表記系の構造を与えていないが、\(\textrm{operator}[]\)は\(\emptyset\)に対しては\(\emptyset\)を、後続順序数に関してはその最大元を、極限順序数に対してはその基本列を与える操作に対応する。

\begin{proposition}[\(\textrm{Pred}\)が\(\lbrack1\rbrack\)で表されること]\label{Predが1で表されること}
	任意の\(M \in T_{\textrm{PS}}\)に対し、\(\textrm{Lng}(M) > 1\)ならば\(\textrm{Pred}(M) = M[1]\)である。
\end{proposition}

\begin{hideableproof}
	\begin{indented}
		\item \(\textrm{operator}[]\)の定義から即座に従う。
	\end{indented}
\end{hideableproof}


\subsection{ペア数列システム}

写像\(f \colon \mathbb{N}_{+} \to \mathbb{N}_{+}\)を1つ固定する。主に\(f(n) = n+1\)か\(f(n) = n^2\)である。

\(T_{\textrm{PS}} \times \mathbb{N}_{+}\)上の部分写像
\begin{eqnarray*}
F \colon (M,n) \mapsto F_M(n)
\end{eqnarray*}
を以下ように再帰的に定める:
\begin{nenumerate}
	\item \(j_1 := \textrm{Lng}(M)-1\)と置く。
	\item \(j_1 = 0\)ならば\(F_M(n) := f(n)\)である。
	\item \(j_1 > 0\)とする。
	\begin{nenumerate}
		\item \(M_{j_1} = (0,0)\)ならば\(F_M(n) := F_{\textrm{Pred}(M)}(f(n))\)である。
		\item \(M_{j_1} \neq (0,0)\)とする。
		\begin{nenumerate}
			\item \(i_1 := \max \{i \in \{0,1\} \mid M_{i,j_1} > 0\}\)と置く\footnote{\(M_{j_1} \neq (0,0)\)より\(\max\)は存在する。}。
			\item \((i_1,j_0) <_M^{\textrm{Next}} (i_1,j_1)\)を満たす一意な\(j_0 \in \mathbb{N}\)が存在しないならば\(F_M(n) := F_{\textrm{Pred}(M)}(f(n))\)である。
			\item \((i_1,j_0) <_M^{\textrm{Next}} (i_1,j_1)\)を満たす一意な\(j_0 \in \mathbb{N}\)が存在するならば\(F_M(n) := F_{M[n]}(f(n))\)である。
		\end{nenumerate}
	\end{nenumerate}
\end{nenumerate}
ペア数列自体は元々ペア数列数という1つの数として定義されているだけなのでペア数列システムという概念はバシク行列のバージョンを固定しても人によって微妙にずれがある。上記もその1つの例に過ぎないが、この記事で示すことは基本的にどのペア数列システムの解釈でも通用するので、この定式化自体に深い意味はないことに注意する。

\(F\)の定義域を\(\textrm{Dom}(F) \subset T_{\textrm{PS}} \times \mathbb{N}_{+}\)と置く。

\begin{proposition}[\(F_M\)と基本列の関係]\label{F_Mと基本列の関係}
	任意の\((M,n) \in T_{\textrm{PS}} \times \mathbb{N}_{+}\)に対し、以下は同値である:
	\begin{penumerate}
		\item \((M,n) \in \textrm{Dom}(F)\)である。
		\item \((M[n],n) \in \textrm{Dom}(F)\)である。
		\item \((M,n) \in \textrm{Dom}(F)\)かつ\((M[n],n) \in \textrm{Dom}(F)\)かつ\(F_M(n) = F_{M[n]}(n)\)である。
	\end{penumerate}
\end{proposition}

\begin{hideableproof}
	\begin{indented}
		\item \(\textrm{Pred}\)を用いた\(\textrm{operator}[]\)の定義と\(F\)の再帰的定義より即座に従う。
	\end{indented}
\end{hideableproof}


\section{ペア数列の基本性質}

後に定義する標準形のペア数列システムの停止性を証明するための準備として、ペア数列からBuchholzの表記系への翻訳写像\(\textrm{Trans}\)を定めるための準備として、ペア数列の基本操作や基本性質を調べる。


\subsection{最上行のインクリメント}

写像
\begin{eqnarray*}
\textrm{IncrFirst} \colon T_{\textrm{PS}} & \to & T_{\textrm{PS}} \\
M & \mapsto & \textrm{IncrFirst}(M)
\end{eqnarray*}
を以下のように定める:
\begin{nenumerate}
	\item \(j_1 := \textrm{Lng}(M)-1\)と置く。
	\item \(\textrm{IncrFirst}(M) := ((M_{0,j}+1,M_{1,j}))_{j=0}^{j_1}\)である。
\end{nenumerate}

\begin{proposition}[\(\leq_M\)の\(\textrm{IncrFirst}\)不変性]\label{leq_MのIncrFirst不変性}
	任意の\(M \in T_{\textrm{PS}}\)に対し、\(\leq_M\)と\(\leq_{\textrm{IncrFirst}(M)}\)は一致する。
\end{proposition}

\begin{hideableproof}
	\begin{indented}
		\item \(\leq_M\)の定義より即座に従う。
	\end{indented}
\end{hideableproof}


\subsection{単項性}

\(M \in T_{\textrm{PS}}\)とする。
\begin{nenumerate}
	\item \(j_1 := \textrm{Lng}(M)-1\)と置く。
	\item \(M\)が零項であるとは、以下を満たすということである:
	\begin{nenumerate}
		\item \(j_1= 0\)である。
		\item \(M_{1,0} = 0\)である。
	\end{nenumerate}
	\item \(M\)が単項であるとは、以下を満たすということである:
	\begin{nenumerate}
		\item \(M\)は零項でない。
		\item \((0,0) \leq_M (0,j_1)\)である。
	\end{nenumerate}
	\item \(M\)が複項であるとは、以下を満たすということである:
	\begin{nenumerate}
		\item \(M\)は零項でない。
		\item \(M\)は単項でない
	\end{nenumerate}
	\item 零項ペア数列全体の部分集合を\(ZT_{\textrm{PS}} \subset T_{\textrm{PS}}\)と置く。
	\item 単項ペア数列全体の部分集合を\(PT_{\textrm{PS}} \subset T_{\textrm{PS}}\)と置く。
	\item 複項ペア数列全体の部分集合を\(MT_{\textrm{PS}} \subset T_{\textrm{PS}}\)と置く。
\end{nenumerate}

\begin{proposition}[複項性の判定条件]\label{複項性の判定条件}
	任意の\(M \in T_{\textrm{PS}}\)に対し、以下は同値である:
	\begin{penumerate}
		\item \(M\)は複項でない。
		\item 任意の\(j \in \mathbb{N}\)に対し、\(0 < j < \textrm{Lng}(M)\)ならば\(M_{0,0} < M_{0,j}\)である。
		\item \((0,0) \leq_M (0,j_1)\)である。
	\end{penumerate}
\end{proposition}

\begin{hideableproof}
	\begin{indented}
		\item (2)と(3)の同値性は\nameref{親の存在の判定条件} (3)と\nameref{直系先祖の基本性質} (1)より従う。
		\item (1)が成り立つならば、\(\textrm{Lng}(M) = 1\)であるかまたは\((0,0) \leq_M (0,j_0)\)であるので、\nameref{直系先祖の基本性質} (1)より(2)が成り立つ。
		\item (2)が成り立つならば、\(\textrm{Lng}(M) = 1\)ならば\(M\)は複項でなく、\(\textrm{Lng}(M) > 1\)ならば\nameref{親の存在の判定条件} (3)より\((0,0) \leq_M (0,j_1)\)であるため\(M\)は単項であるので、(1)が成り立つ。
	\end{indented}
\end{hideableproof}

\begin{corollary}[単項性の始切片への遺伝性]\label{単項性の始切片への遺伝性}
	任意の\(M \in PT_{\textrm{PS}}\)と\(j_0 \in \mathbb{N}\)に対し、\(0 < j_0 < \textrm{Lng}(M)\)ならば\((M_j)_{j=0}^{j_0}\)は単項である。
\end{corollary}

\begin{hideableproof}
	\begin{indented}
		\item 単項性の定義より、\nameref{複項性の判定条件}から即座に従う。
	\end{indented}
\end{hideableproof}

\begin{proposition}[単項性の直系先祖による切片への遺伝性]\label{単項性の直系先祖による切片への遺伝性}
	任意の\(M \in T_{\textrm{PS}}\)と\(j'_0,j'_1 \in \mathbb{N}\)に対し、\(j'_0 < j'_1\)かつ\((0,j'_0) \leq_M (0,j'_1)\)ならば\((M_j)_{j=j'_0}^{j'_1}\)は単項である。
\end{proposition}

\begin{hideableproof}
	\begin{indented}
		\item \(M' := (M_j)_{j=j'_0}^{j'_1}\)と置く。任意の\(j \in \mathbb{N}\)に対し\(0 < j \leq j'_1-j'_0\)を満たすならば、\nameref{直系先祖の基本性質} (1)と\((0,j'_0) \leq_M (0,j'_1)\)より\(M'_{0,0} = M_{0,j'_0} < M_{0,j'_0+j} = M'_{0,j}\)である。従って\nameref{親の存在の判定条件} (3)より\((0,0) \leq_{M'} (0,j'_1-j'_0)\)である。\(\textrm{Lng}(M') = j'_1 - j'_0 > 0\)であるので、\(M'\)は零項でない。以上より\(M'\)は単項である。
	\end{indented}
\end{hideableproof}

写像
\begin{eqnarray*}
P \colon T_{\textrm{PS}} & \to & T_{\textrm{PS}}^{< \omega} \\
M & \mapsto & P(M)
\end{eqnarray*}
を以下のように再帰的に定める:
\begin{nenumerate}
	\item \(M\)が零項または単項ならば\(P(M) := (M)\)である。
	\item \(M\)が複項とする。
	\begin{nenumerate}
		\item \(j_1 := \textrm{Lng}(M) - 1\)と置く。
		\item \(j_0 := \min \{j \in \mathbb{N} \mid 0 < j \leq j_1 \wedge (0,j) \leq_M (0,j_1)\}\)と置く\footnote{\(M\)が複項より\(0 < j_1\)であり、かつ\((0,j_1) \leq_M (0,j_1)\)であるので\(\min\)は存在する。}。
		\item \(P(M) := P((M_j)_{j=0}^{j_0-1}) \oplus_{T_{\textrm{PS}}} (M_j)_{j=j_0}^{j_1}\)である。
	\end{nenumerate}
\end{nenumerate}

\(P\)の再帰的定義から\(P(M) \neq ()\)である。

\begin{proposition}[\(P\)の\(\textrm{IncrFirst}\)同変性]\label{PのIncrFirst同変性}
	任意の\(M \in T_{\textrm{PS}}\)に対し、\(J_1 := \textrm{Lng}(P(M))-1\)と置くと、\(P(\textrm{IncrFirst}(M)) = (\textrm{IncrFirst}(P(M)_J))_{J=0}^{J_1}\)である。
\end{proposition}

\begin{hideableproof}
	\begin{indented}
		\item \nameref{leq_MのIncrFirst不変性}より即座に従う。
	\end{indented}
\end{hideableproof}

\begin{proposition}[\(P\)の各成分の非複項性]\label{Pの各成分の非複項性}
	任意の\(M \in T_{\textrm{PS}}\)に対し、以下が成り立つ:
	\begin{penumerate}
		\item \(P(M)\)の各成分は零項か単項である。
		\item \(M\)が複項である必要十分条件は\(\textrm{Lng}(P(M)) > 1\)である。
	\end{penumerate}
\end{proposition}

\begin{hideableproof}
	\begin{indented}
		\item \(P\)の再帰的定義から\(\textrm{Lng}(M)\)に関する数学的帰納法より即座に従う。
	\end{indented}
\end{hideableproof}

\begin{proposition}[\(P\)の加法性]\label{Pの加法性}
	任意の\(M \in T_{\textrm{PS}}\)と\(j_0 \in \mathbb{N}\)に対し、\(j_1 := \textrm{Lng}(M)-1\)と置き、\(0 < j_0 \leq j_1\)とし、任意の\(j \in \mathbb{N}\)に対し\(j < j_0\)ならば\(M_{0,j} \geq M_{0,j_0}\)であるとすると、\(P(M) = P((M_j)_{j=0}^{j_0-1}) \oplus_{T_{\textrm{PS}}} P((M_j)_{j=j_0}^{j_1})\)である。
\end{proposition}

\begin{hideableproof}
	\begin{indented}
		\item \(j'_0 := \min \{j \in \mathbb{N} \mid (0,j) \leq_M (0,j_1)\}\)と置く\footnote{\(j = j_1\)が条件を満たすため、\(\min\)は存在する。}。
		\item 任意の\(j \in \mathbb{N}\)に対し\(j < j_0\)ならば\(M_{0,j} \geq M_{0,j_0}\)であることから、\nameref{直系先祖の基本性質} (1)より\(0 < j_0 \leq j'_0\)である。従って\(P\)の定義より\(P(M) = P((M_j)_{j=0}^{j'_0-1}) \oplus_{\mathbb{N}^2} ((M_j)_{j=j'_0}^{j_1})\)かつ\(P((M_j)_{j=j'_0}^{j_1}) = ((M_j)_{j=j'_0}^{j_1})\)であるので、\(P(M) = P((M_j)_{j=0}^{j'_0-1}) \oplus_{\mathbb{N}^2} P((M_j)_{j=j'_0}^{j_1})\)となる。
		\item \(j'_0\)の定義と\nameref{親の存在の判定条件}より、任意の\(j \in \mathbb{N}\)に対し、\(j < j'_0\)ならば\(M_{0,j} \geq M_{0,j'_0}\)である。
		\item \(j'_0-j_0\)と\(j_0\)の辞書式順序に関する数学的帰納法で示す\footnote{\(j'_0-j_0\)と\(j_0\)の辞書式順序の順序型は\(\omega \times \omega\)であるため整礎であり、数学的帰納法が適用可能である。}。
		\item \(j'_0-j_0 = 0\)とする。
		\begin{eqnarray*}
		P(M) = P((M_j)_{j=0}^{j'_0-1}) \oplus_{\mathbb{N}^2} P((M_j)_{j=j'_0}^{j_1}) = P((M_j)_{j=0}^{j_0-1}) \oplus_{\mathbb{N}^2} P((M_j)_{j=j_0}^{j_1})
		\end{eqnarray*}
		\begin{indented}
			\item である。
		\end{indented}
		\item \(j'_0-j_0 > 0\)とする。
		\begin{indented}
			\item 帰納法の仮定より\(P((M_j)_{j=0}^{j'_0-1}) = P((M_j)_{j=0}^{j_0-1}) \oplus_{T_{\textrm{PS}}} P((M_j)_{j=j_0}^{j'_0-1})\)かつ\(P((M_j)_{j=j_0}^{j_1}) = P((M_j)_{j=j_0}^{j'_0-1}) \oplus_{T_{\textrm{PS}}} P((M_j)_{j=j'_0}^{j_1})\)であり、従って
			\begin{eqnarray*}
			& & P(M) = P((M_j)_{j=0}^{j'_0-1}) \oplus_{\mathbb{N}^2} P((M_j)_{j=j'_0}^{j_1}) = P((M_j)_{j=0}^{j_0-1}) \oplus_{T_{\textrm{PS}}} P((M_j)_{j=j_0}^{j'_0-1}) \oplus_{\mathbb{N}^2} P((M_j)_{j=j'_0}^{j_1}) \\
			& = & P((M_j)_{j=0}^{j_0-1}) \oplus_{\mathbb{N}^2} P((M_j)_{j=j_0}^{j_1})
			\end{eqnarray*}
			\item である。
		\end{indented}
	\end{indented}
\end{hideableproof}

\begin{proposition}[\(P\)と基本列の関係]\label{Pと基本列の関係}
	任意の\(M \in T_{\textrm{PS}}\)と\(n \in \mathbb{N}_{+}\)に対し、\(J_1 := \textrm{Lng}(P(M))-1\)と置くと以下が成り立つ:
	\begin{penumerate}
		\item \(\textrm{Lng}(P(M)_{J_1}) = 1\)ならば\(M[n] = \textrm{Pred}(M)\)であり、更に\(J_1 = 0\)か否かに従って\(P(M[n]) = (M[n])\)または\(P(M[n]) = (P(M)_J)_{J=0}^{J_1-1}\)である。
		\item \(\textrm{Lng}(P(M)_{J_1}) > 1\)ならば\(M[n] = (\bigoplus_{\mathbb{N}^2} (P(M)_J)_{J=0}^{J_1-1}) \oplus_{\mathbb{N}^2} P(M)_{J_1}[n]\)かつ\(P(M[n]) = (P(M)_J)_{J=0}^{J_1-1} \oplus_{T_{\textrm{PS}}} P(P(M)_{J_1}[n])\)である。
	\end{penumerate}
\end{proposition}

\begin{hideableproof}
	\begin{indented}
		\item (1)は\(\textrm{operator}[]\)の定義と\nameref{単項性の始切片への遺伝性}から即座に従う。
		\item (2)は\((P(M)_{J_1})_0 = P(M)_{J_1}[n]_0\)より\(P\)の再帰的定義から即座に従う。
	\end{indented}
\end{hideableproof}

\begin{proposition}[非複項性と基本列の関係]\label{非複項性と基本列の関係}
	任意の\(M \in T_{\textrm{PS}}\)と\(n \in \mathbb{N}_{+}\)に対し、\(j_1 := \textrm{Lng}(M)-1\)と置くと、\(M\)が複項でないならば以下が成り立つ:
	\begin{penumerate}
		\item \((0,0) <_M^{\textrm{Next}} (0,j_1)\)かつ\(M_{1,j_1} = 0\)ならば、\(P(M[n]) = (\textrm{Pred}(M))_{k=0}^{n-1}\)である。
		\item \((0,0) <_M^{\textrm{Next}} (0,j_1)\)でないまたは\(M_{1,j_1} > 0\)ならば、\(P(M[n]) = (M[n])\)である。
	\end{penumerate}
\end{proposition}

\begin{hideableproof}
	\begin{penumerate}
		\item \(M[n] = \bigoplus_{\mathbb{N}^2} (\textrm{Pred}(M))_{k=0}^{n-1}\)であり、\nameref{単項性の始切片への遺伝性}より\(\textrm{Pred}(M)\)は複項でない。従って\nameref{Pの加法性}から、\(n\)に関する数学的帰納法により従う。
		\item \(M\)は複項でないので、\nameref{複項性の判定条件}から任意の\(j \in \mathbb{N}\)に対し\(0 < j \leq j_1\)ならば\(M[n]_{0,0} = M_{0,0} < M_{0,j}\)である、
		\item[] \((0,0) <_M^{\textrm{Next}} (0,j_1)\)でなくかつ\(M_{1,j_1} = 0\)ならば、任意の\(j \in \mathbb{N}\)に対し\(0 < j < \textrm{Lng}(M[n])\)ならば\(M[n]_{0,j}\)は\((M_{0,j})_{j=1}^{j_1}\)のいずれかの成分であるので\(M[n]_{0,0} < M[n]_{0,j}\)となる。
		\item[] \(M_{1,j_1} > 0\)ならば、任意の\(j \in \mathbb{N}\)に対し\(0 < j < \textrm{Lng}(M[n])\)ならば\(M[n]_{0,j}\)は\((M_{0,j})_{j=1}^{j_1}\)のいずれかの成分に\(n\)未満の自然数を足したものなので\(M[n]_{0,0} < M[n]_{0,j}\)となる。
		\item[] 従っていずれの場合も\nameref{複項性の判定条件}より\(M[n]\)は複項でなく、\nameref{Pの各成分の非複項性} (2)より\(P(M[n]) = (M[n])\)である。
	\end{penumerate}
\end{hideableproof}


\subsection{許容性}

\(M \in T_{\textrm{PS}}\)とし、\(j \in \mathbb{N}\)とする。
\begin{nenumerate}
	\item \(j_1 := \textrm{Lng}(M)\)と置く。
	\item \(j\)が非\(M\)許容であるとは、以下のいずれかを満たすということである:
	\begin{nenumerate}
		\item \(j > j_1\)である。
		\item \((1,j-1) <_M^{\textrm{Next}} (1,j) <_M^{\textrm{Next}} (1,j+1)\)である。
	\end{nenumerate}
	\item \(j\)が\(M\)許容であるとは、\(j\)が非\(M\)許容でないということである。
	\item \(M\)許容である自然数全体のなす部分集合を\(\mathbb{N}_M \subset \mathbb{N}\)と置く。
\end{nenumerate}

\begin{proposition}[許容性の切片への遺伝性]\label{許容性の切片への遺伝性}
	任意の\(M \in T_{\textrm{PS}}\)と\(j'_0, j_0, j'_1 \in \mathbb{N}\)に対し、\(j_1 := \textrm{Lng}(M) - 1\)と置き、\(j'_0 \leq j_0 \leq j'_1 \leq j_1\)とし\(N := (M_j)_{j=j'_0}^{j'_1}\)と置くと、以下は同値である:
	\begin{penumerate}
		\item \(j_0\)が\(M\)許容であるまたは\(j'_0 = j_0\)または\(j_0 = j'_1\)である。
		\item \(j_0-j'_0\)が\(N\)許容である。
	\end{penumerate}
\end{proposition}

\begin{hideableproof}
	\begin{indented}
		\item \(j'_0 = j'_0\)または\(j_0 = j'_1\)または\(j'_1 = j_1\)ならば、許容性の定義より従う。以下\(j'_0 < j_0 < j'_1 < j_1\)とする。
		\item \(j'_0 < j_0 < j'_1\)より、\((1,j_0-1) <_M^{\textrm{Next}} (1,j_0) <_M^{\textrm{Next}} (1,j_0+1)\)でないことと\((1,j_0-j'_0-1) <_N^{\textrm{Next}} (1,j_0-j'_0) <_N^{\textrm{Next}} (1,j_0-j'_0+1)\)でないことは同値である。従って\(j_0\)の\(M\)許容性と\(j_0-j'_0\)の\(N\)許容性は同値である。
	\end{indented}
\end{hideableproof}

\begin{eqnarray*}
\textrm{Adm}_M \colon T_{\textrm{PS}} \times \mathbb{N} & \to & \mathbb{N}_M \\
(M,j) & \mapsto & \textrm{Adm}_M(j)
\end{eqnarray*}
を以下のように定める:
\begin{nenumerate}
	\item \(j\)が\(M\)許容ならば、\(\textrm{Adm}_M(j) := j\)である。
	\item \(j\)は非\(M\)許容ならば、\(\textrm{Adm}_M(j) := \max \{j' \in \mathbb{N}_M \mid j' < j\}\)と置く\footnote{\(0\)は\(M\)許容であり、かつ\(j\)は非\(M\)許容より\(j > 0\)であるので、\(\max\)は存在する。}。
\end{nenumerate}

\begin{proposition}[許容化の切片への遺伝性]\label{許容化の切片への遺伝性}
	任意の\(M \in T_{\textrm{PS}}\)と\(j'_0,j'_1 \in \mathbb{N}\)に対し、\(j_1 := \textrm{Lng}(M) - 1\)と置き、\(j'_0 \leq \textrm{Adm}_M(j_0)\)かつ\(j_0 < j'_1 \leq j_1\)とし\(N := (M_j)_{j=j'_0}^{j'_1}\)と置くと、\(\textrm{Adm}_N(j_0-j'_0) = \textrm{Adm}_M(j_0)-j'_0\)である。
\end{proposition}

\begin{hideableproof}
	\begin{indented}
		\item \(\textrm{Adm}_N(j_0-j'_0) \leq j_0-j'_0 < j_1-j'_0\)かつ\(j'_0 \leq \textrm{Adm}_M(j_0) \leq j_0 < j_1\)と\nameref{許容性の切片への遺伝性}より、\(\textrm{Adm}_N(j_0-j'_0)\)と\(\textrm{Adm}_M(j_0)-j'_0\)が\(N\)許容かつ\(\textrm{Adm}_M(j_0)\)が\(M\)許容である。従って\(\textrm{Adm}_N(j_0-j'_0)\)の最大性から\(\textrm{Adm}_N(j_0-j'_0) \geq \textrm{Adm}_M(j_0)-j'_0\)である。
		\item \(\textrm{Adm}_M(j_0) < \textrm{Adm}_N(j_0-j'_0)+j'_0\)と仮定し矛盾を導く。
		\begin{indented}
			\item \(\textrm{Adm}_M(j_0) < \textrm{Adm}_N(j_0-j'_0)+j'_0 \leq (j_0-j'_0)+j'_0 = j_0\)より\((1,\textrm{Adm}_N(j_0-j'_0)+j'_0-1) <_M^{\textrm{Next}} (1,\textrm{Adm}_N(j_0-j'_0)+j'_0)\)であり、\(j'_0 \leq \textrm{Adm}_M(j_0) < \textrm{Adm}_N(j_0-j'_0)+j'_0\)より\((1,\textrm{Adm}_N(j_0-j'_0)-1) <_N^{\textrm{Next}} (1,\textrm{Adm}_N(j_0-j'_0))\)となる。
			\item 従って\(\textrm{Adm}_N(j_0-j'_0)\)の\(N\)許容性から\((1,\textrm{Adm}_N(j_0-j'_0)) <_N^{\textrm{Next}} (1,\textrm{Adm}_N(j_0-j'_0)+1)\)でなく、\((1,\textrm{Adm}_N(j_0-j'_0)+j'_0) <_M^{\textrm{Next}} (1,\textrm{Adm}_N(j_0-j'_0)+j'_0+1)\)でない。再度\(\textrm{Adm}_M(j_0) < \textrm{Adm}_N(j_0-j'_0)+j'_0 \leq j_0\)より\(\textrm{Adm}_N(j_0-j'_0)+j'_0 = j_0\)となる。
			\item すると\nameref{許容性の切片への遺伝性}と\(j_0 < j_1\)から\(j_0\)が\(M\)許容となるので\(\textrm{Adm}_M(j_0) = j_0\)であり、これは\(\textrm{Adm}_M(j_0) < \textrm{Adm}_N(j_0-j'_0)+j'_0 \leq j_0\)に反する。
		\end{indented}
		\item 以上より\(\textrm{Adm}_M(j_0) = \textrm{Adm}_N(j_0-j'_0)+j'_0\)である。
	\end{indented}
\end{hideableproof}

\((M,m) \in T_{\textrm{PS}} \times \mathbb{N}\)とする。
\begin{nenumerate}
	\item \((M,m)\)が基点付きペア数列であるとは、\(j_1 := \textrm{Lng}(M)-1\)と置くと以下を満たすということである:
	\begin{nenumerate}
		\item \(m\)は\(M\)許容である。
		\item \((0,m) \leq_M (0,j_1)\)である。
	\end{nenumerate}
	\item 基点付きペア数列全体の部分集合を\(T_{\textrm{PS}}^{\textrm{Marked}} \subset T_{\textrm{PS}} \times \mathbb{N}\)と置く。
	\item \(RT_{\textrm{PS}}^{\textrm{Marked}} := \{(M,m) \in T_{\textrm{PS}}^{\textrm{Marked}} \mid M \in RT_{\textrm{PS}}\}\)である。
\end{nenumerate}

\begin{proposition}[基点の切片への遺伝性]\label{基点の切片への遺伝性}
	任意の\((M,m) \in T_{\textrm{PS}}^{\textrm{Marked}}\)と\(j'_0,j'_1 \in \mathbb{N}\)に対し、\(j_1 := \textrm{Lng}(M) - 1\)と置くと、\(j'_0 \leq m \leq j'_1 \leq j_1\)ならば\(((M_j)_{j=j'_0}^{j'_1},m-j'_0) \in T_{\textrm{PS}}^{\textrm{Marked}}\)である。
\end{proposition}

\begin{hideableproof}
	\begin{indented}
		\item \nameref{直系先祖の木構造} (1)と\nameref{許容性の切片への遺伝性}から即座に従う。
	\end{indented}
\end{hideableproof}


\subsection{幹と枝}

写像
\begin{eqnarray*}
\textrm{IdxSum} \colon T_{\textrm{PS}}^{< \omega} & \to & \mathbb{N}^{< \omega} \\
Q & \mapsto & \textrm{IdxSum}(Q)
\end{eqnarray*}
を以下のように定める:
\begin{nenumerate}
	\item \(J_1 := \textrm{Lng}(Q)-1\)と置く。
	\item \(J \leq J_1+1\)を満たす各\(J \in \mathbb{N}\)に対し、\(j_J \in \mathbb{N}\)を以下のように再帰的に定める:
	\begin{nenumerate}
		\item \(J = 0\)ならば\(j_J := 0\)である。
		\item \(J > 0\)ならば\(j_J := j_{J-1} + \textrm{Lng}(Q_{J-1})\)である。
	\end{nenumerate}
	\item \(\textrm{IdxSum}(Q) := (j_J)_{J=0}^{J_1+1}\)である。
\end{nenumerate}

\begin{proposition}[\(P\)と\(\textrm{IdxSum}\)の関係]\label{PとIdxSumの関係}
	任意の\(M \in T_{\textrm{PS}}\)と\(J \in \mathbb{N}\)に対し、\(J_1 := \textrm{Lng}(P(M))-1\)と置くと、\(J \leq J_1\)として\(j'_0 := \textrm{IdxSum}(P(M))_J\)と置き\(j'_1 := \textrm{IdxSum}(P(M))_{J+1}\)と置くと、\(P(M)_J = (M_j)_{j=j'_0}^{j'_1-1}\)である。
\end{proposition}

\begin{hideableproof}
	\begin{indented}
		\item \(\textrm{IdxSum}\)の定義から即座に従う。
	\end{indented}
\end{hideableproof}

\begin{corollary}[\(P\)と\(\textrm{IdxSum}\)の合成の特徴付け]\label{PとIdxSumの合成の特徴付け}
	任意の\(M \in T_{\textrm{PS}}\)に対し、\(j_1 := \textrm{Lng}(M)-1\)と置き、\(J_1 := \textrm{Lng}(P(M))-1\)と置くと、以下が成り立つ:
	\begin{penumerate}
		\item 任意の\(J \in \mathbb{N}\)に対し、\(J \leq J_1\)ならば、\((0,j_0) <_M^{\textrm{Next}} (0,\textrm{IdxSum}(P(M))_J)\)を満たす一意な\(j_0 \in \mathbb{N}\)が存在しない。
		\item 任意の\(j \in \mathbb{N}\)に対し、\(j \leq j_1\)かつ\((0,j_0) <_M^{\textrm{Next}} (0,j)\)を満たす一意な\(j \in \mathbb{N}\)が存在しないならば、ある\(J \in \mathbb{N}\)が存在して\(J \leq J_1\)かつ\(j = \textrm{IdxSum}(P(M))_J\)である。
	\end{penumerate}
\end{corollary}

\begin{hideableproof}
	\begin{indented}
		\item \(P\)の再帰的定義と\nameref{PとIdxSumの関係}から即座に従う。
	\end{indented}
\end{hideableproof}

\begin{proposition}[\(P\)の各成分の左端の単調性]\label{Pの各成分の左端の単調性}
	任意の\(M \in T_{\textrm{PS}}\)と\(J'_0,J'_1 \in \mathbb{N}\)に対し、\(J_1 := \textrm{Lng}(P(M))-1\)と置くと、\(J'_0 \leq J'_1 \leq J_1\)ならば\((P(M)_{J'_0})_{0,0} \geq (P(M)_{J'_1})_{0,0}\)である。
\end{proposition}

\begin{hideableproof}
	\begin{indented}
		\item \nameref{Pの各成分の非複項性}と\nameref{PとIdxSumの関係}から、任意の\(j \in \mathbb{N}\)に対し\(j < \textrm{IdxSum}(P(M))_{J'_1}\)ならば\((0,j) \leq_M (0,j_{J'_1})\)でない。
		\item 従って\nameref{親の存在の判定条件} (1)から、任意の\(j \in \mathbb{N}\)に対し\(j < j_{J'_1}\)ならば\(M_{0,j} \geq M_{0,j_{J'_1}}\)である。
		\item \(J'_0 \leq J'_1 \leq J_1\)より\(\textrm{IdxSum}(P(M))_{J'_0} < \textrm{IdxSum}(P(M))_{J'_1}\)であるので、\((P(M)_{J'_0})_{0,0} = M_{0,j_{J'_0}} \geq M_{0,j'_{J'_1}} = (P(M)_{J'_1})_{0,0}\)である。
	\end{indented}
\end{hideableproof}

\begin{proposition}[切片の単項成分と\(<_M^{\textrm{Next}}\)の関係]\label{切片の単項成分とNextの関係}
	任意の\(M \in PT_{\textrm{PS}}\)と\(j_0,J \in \mathbb{N}\)に対し、\(j_1 := \textrm{Lng}(M)-1\)と置き、\(0 < j_0 \leq j_1\)として\(M' := (M_j)_{j=j_0}^{j_1}\)と置き、\(J_1 := \textrm{Lng}(P(M'))-1\)と置くと、\(J \leq J_1\)ならば一意な\(j \in \mathbb{N}\)が存在して\(j < j_0\)かつ\((0,j) <_M^{\textrm{Next}} (0,j_0 + \textrm{IdxSum}(P(M'))_J)\)である。
\end{proposition}

\begin{hideableproof}
	\begin{indented}
		\item \(j'_0 := j_0 + \textrm{IdxSum}(P(M'))_J\)と置く。
		\item \(j'_0 \leq j_0 + (j_1-j_0) = j_1\)である。
		\item \(M\)が単項より\((0,0) \leq_M (0,j'_0)\)であり、かつ\(j'_0 \geq j_0 > 0\)より\(M_{0,0} < M_{0,j'_0}\)である。従って\nameref{親の存在の判定条件} (1)より一意な\(j \in \mathbb{N}\)が存在して\((0,j) <_M^{\textrm{Next}} (0,j'_0)\)である。
		\item 一方で\nameref{PとIdxSumの関係}と\(P\)の定義から任意の\(j \in \mathbb{N}\)に対し\(j_0 \leq j \leq j'_0\)ならば\((0,j-j_0) \leq_{M'} (0,j'_0-j_0)\)でなくすなわち\((0,j) \leq_M (0,j'_0)\)でない。以上より\(j < j_0\)である。
	\end{indented}
\end{hideableproof}

写像
\begin{eqnarray*}
\textrm{TrMax} \colon PT_{\textrm{PS}} & \to & \mathbb{N} \\
M & \mapsto & \textrm{TrMax}(M)
\end{eqnarray*}
と
\begin{eqnarray*}
\textrm{Br} \colon PT_{\textrm{PS}} & \to & T_{\textrm{PS}}^{< \omega} \\
M & \mapsto & \textrm{Br}(M)
\end{eqnarray*}
と
\begin{eqnarray*}
\textrm{FirstNodes} \colon PT_{\textrm{PS}} & \to & \mathbb{N}^{< \omega} \\
M & \mapsto & \textrm{FirstNodes}(M)
\end{eqnarray*}
と
\begin{eqnarray*}
\textrm{Joints} \colon PT_{\textrm{PS}} & \to & \mathbb{N}^{< \omega} \\
M & \mapsto & \textrm{Joints}(M)
\end{eqnarray*}
と
を以下のように定める:
\begin{nenumerate}
	\item \(\textrm{TrMax}(M) := \max \{j \in \mathbb{N} \mid \forall j' \in \mathbb{N}, (j' < j) \to ((1,j') <_M^{\textrm{Next}} (1,j'+1))\}\)である\footnote{\(j = 0\)が条件を満たし、かつ条件を満たす\(j\)は\(\textrm{Lng}(M)\)未満であるため、最大値が存在する。}。
	\item \(j_1 := \textrm{Lng}(M) - 1\)と置く。
	\item \(j'_1 := \textrm{TrMax}(M)\)と置く。
	\item \(j'_1 = j_1\)ならば\(\textrm{Br}(M) := ()\)である。
	\item \(j'_1 < j_1\)ならば\(\textrm{Br}(M) := P((M_j)_{j=j'_1+1}^{j_1})\)である。
	\item \(J_1 := \textrm{Lng}(\textrm{Br}(M))-1\)と置く。
	\item \(\textrm{FirstNodes}(M) := (\textrm{TrMax}(M) + 1 + \textrm{IdxSum}(\textrm{Br}(M))_J)_{J=0}^{J_1+1}\)である。
	\item \(J \leq J_1\)を満たす各\(J \in \mathbb{N}\)に対し、\((0,j) <_M^{\textrm{Next}} (0,\textrm{FirstNodes}(M)_J)\)を満たす一意な\(j \in \mathbb{N}\)を\(a_J\)と置く\footnote{\(M\)の単項性から\((0,0) \leq_M (0,\textrm{FirstNodes}(M)_J)\)であり、\(\textrm{FirstNodes}(M)_J > j'_1 \geq 0\)より\(j\)は存在する。}。
	\item \(\textrm{Joints}(M) := (a_J)_{J=0}^{J_1}\)である。
\end{nenumerate}

\(\textrm{TrMax}\)の定義から\((1,\textrm{TrMax}(M)) <_M^{\textrm{Next}} (1,\textrm{TrMax}(M)+1)\)でなくかつ\(\textrm{TrMax}(M) < \textrm{Lng}(M)\)であるので、\(\textrm{TrMax}(M)\)は\(M\)許容である。また\(\textrm{FirstNodes}(M)\)の定義から、\(\textrm{Br}(M) = ((M_j)_{j = \textrm{FirstNodes}(M)_J}^{\textrm{FirstNodes}(M)_{J+1}-1})_{J=0}^{J_1}\)となる。

\begin{proposition}[\(\textrm{FirstNodes}\)と\(\textrm{TrMax}\)と\(\textrm{Joints}\)の関係]\label{FirstNodesとTrMaxとJointsの関係}
	任意の\(M \in PT_{\textrm{PS}}\)と\(J \in \mathbb{N}\)に対し、\(J_1 := \textrm{Lng}(\textrm{Br}(M))-1\)と置くと、\(J \leq J_1\)ならば\(\textrm{Joints}(M)_J \leq \textrm{TrMax}(M) < \textrm{FirstNodes}(M)_J\)である。
\end{proposition}

\begin{hideableproof}
	\begin{indented}
		\item \(\textrm{TrMax}\)と\(\textrm{Br}\)と\(\textrm{FirstNodes}\)と\(\textrm{Joints}\)の定義と\nameref{切片の単項成分とNextの関係}から即座に従う。
	\end{indented}
\end{hideableproof}

\begin{corollary}[\(\textrm{FirstNodes}\)と\(\textrm{Joints}\)の単調性]\label{FirstNodesとJointsの単調性}
	任意の\(M \in PT_{\textrm{PS}}\)と\(J'_0,J'_1 \in \mathbb{N}\)に対し、\(J_1 := \textrm{Lng}(\textrm{Br}(M))-1\)と置くと、\(J'_0 < J'_1 \leq J_1\)ならば以下が成り立つ:
	\begin{penumerate}
		\item \(\textrm{FirstNodes}(M)_{J'_0} \leq \textrm{FirstNodes}(M)_{J'_1}\)である。
		\item \(\textrm{Joints}(M)_{J'_0} \geq \textrm{Joints}(M)_{J'_1}\)である。
		\item \(M_{0,\textrm{FirstNodes}(M)_{J'_0}} \geq M_{0,\textrm{FirstNodes}(M)_{J'_1}}\)である。
		\item 任意の\(i \in \{0,1\}\)に対し\(M_{i,\textrm{Joints}(M)_{J'_0}} > M_{i,\textrm{Joints}(M)_{J'_1}}\)である。
	\end{penumerate}
\end{corollary}

\begin{hideableproof}
	\begin{indented}
		\item (1)と(3)は\(P\)と\(\textrm{IdxSum}\)の定義から即座に従う。
		\item (2)と(4)は\nameref{FirstNodesとTrMaxとJointsの関係}と(3)から即座に従う。
	\end{indented}
\end{hideableproof}

\begin{corollary}[単項性の切片への遺伝性]\label{単項性の切片への遺伝性}
	任意の\(M \in PT_{\textrm{PS}}\)と\(j'_0,j'_1 \in \mathbb{N}\)に対し、\(j_1 := \textrm{Lng}(M)-1\)と置き、\(J_1 := \textrm{Lng}(\textrm{Br}(M))-1\)と置き、\(j'_0 < j'_1 \leq j_1\)として\(M' := (M_j)_{j=j'_0}^{j'_1}\)と置くと、\(j'_0 \leq \textrm{Joints}(M)_{J_1}\)ならば\(M'\)は単項である。
\end{corollary}

\begin{hideableproof}
	\begin{indented}
		\item \(j_0 := \textrm{TrMax}(M)\)と置く。
		\item \(\textrm{TrMax}(M') = j_0-j'_0\)かつ\(\textrm{Lng}(M')-1 = j'_1-j'_0\)である。
		\item \(\textrm{Lng}(M')-1 = j'_1-j'_0 > 0\)より、任意の\(j \in \mathbb{N}\)に対し、\(j \leq j'_1-j'_0\)ならば\((0,0) \leq_{M'} (0,j)\)となることを示せば良い。
		\item \(j \leq \textrm{TrMax}(M')\)ならば\((1,0) \leq_{M'} (1,j)\)なので\((0,0) \leq_{M'} (0,j)\)である。
		\item \(j > \textrm{TrMax}(M')\)とする。
		\begin{indented}
			\item \(j+j'_0 > \textrm{TrMax}(M')+j'_0 = j_0\)であるので、\nameref{FirstNodesとJointsの単調性}から一意な\(J \in \mathbb{N}\)が存在して\(J \leq J_1\)かつ\(\textrm{FirstNodes}(M)_J \leq j+j'_0 < \textrm{FirstNodes}(M)_{J+1}\)となる。\nameref{Pの各成分の非複項性}より\(\textrm{Br}(M)_J\)は複項でないので、\((0,0) \leq_{\textrm{Br}(M)_J} (0,j+j'_0 - \textrm{FistNodes}(M)_J)\)すなわち\((0,\textrm{FirstNodes}(M)_J) \leq_M (0,j+j'_0)\)である。
			\item \nameref{FirstNodesとJointsの単調性}と\nameref{FirstNodesとTrMaxとJointsの関係}から\(j'_0 \leq \textrm{Joints}(M)_{J_1} \leq \textrm{Joints}(M)_J \leq \textrm{TrMax}(M)\)であるので、\((1,j'_0) \leq_M (1,\textrm{Joints}(M)_J)\)である。
			\item \((1,j'_0) \leq_M (1,\textrm{Joints}(M)_J)\)かつ\((0,\textrm{Joints}(M)_J) <_M^{\textrm{Next}} (0,\textrm{FirstNodes}(M)_J)\)かつ\((0,\textrm{FirstNodes}(M)_J) \leq_M (0,j+j'_0)\)より\((0,j'_0) \leq_M (0,j+j'_0)\)すなわち\((0,0) \leq_{M'} (0,j)\)である。
		\end{indented}
		\item 以上より\(M'\)は単項である。
	\end{indented}
\end{hideableproof}


\subsection{簡約化}

写像
\begin{eqnarray*}
\textrm{Red} \colon T_{\textrm{PS}} & \to & T_{\textrm{PS}} \\
M & \mapsto & \textrm{Red}(M)
\end{eqnarray*}
を以下のように再帰的に定める:
\begin{nenumerate}
	\item \(M\)が零項ならば\(\textrm{Red}(M) := ((0,0))\)である。
	\item \(M\)が単項とする。
	\begin{nenumerate}
		\item \(j_1 := \textrm{Lng}(M) - 1\)と置く。
		\item \(M_0 = (0,0)\)とする\footnote{\(M\)は零項でないので、この時\(j_1 > 0\)である。}。
		\begin{nenumerate}
			\item \(j'_1 := \textrm{TrMax}(M)\)と置く。
			\item \(j'_1 = j_1\)ならば\(\textrm{Red}(M) := ((j,j))_{j=M_{1,0}}^{M_{1,0}+j_1}\)である。
			\item \(j'_1 < j_1\)とする。
			\item \(J_1 := \textrm{Lng}(\textrm{Br}(M)) - 1\)と置く。
			\item \(J \leq J_1\)を満たす各\(J \in \mathbb{N}\)に対し、\(n_J \in \mathbb{N} \cup \{-1\}\)と\(N_J \in PT_{\textrm{PS}}\)を以下のように定める:
			\begin{nenumerate}
				\item \((\textrm{Br}(M)_J)_{1,0} = 0\)ならば、\(n_J := -1\)である。
				\item \((\textrm{Br}(M)_J)_{1,0} > 0\)ならば\((1,j) <_M^{\textrm{Next}} (1,\textrm{FirstNodes}(M)_J)\)を満たす一意な\(j \in \mathbb{N}\)を\(n_J\)と置く\footnote{\(M\)が単項より\((0,0) \leq_M (0,\textrm{FirstNodes}(M)_J)\)であり、\(M_{0,0} = 0 < (\textrm{Br}(M)_J)_{1,0} = M_{1,\textrm{FirstNode}(M)_J}\)と\nameref{親の存在の判定条件} (2)から\(j\)は存在する。}。
				\item \(N_J := ((M_{0,0} + \textrm{Joints}(M)_J + 1,M_{1,0} + n_J + 1)) \oplus_{\mathbb{N}^2} \textrm{Derp}(\textrm{Br}(M)_J)\)である\footnote{\nameref{FirstNodesとTrMaxとJointsの関係}より\(\textrm{Joints}(M)_J \leq \textrm{TrMax}(M)\)なので\(M_{0,0} + \textrm{Joints}(M)_J \leq M_{0,\textrm{Joints}(M)}\)であり、\((0,\textrm{Joints}(M)_J) <_M^{\textrm{Next}} (0,\textrm{FirstNodes}(M)_J)\)より\(M_{0,\textrm{Joints}(M)_J} < M_{0,\textrm{FirstNodes}(M)_J}\)であるので\(M_{0,0} + \textrm{Joints}(M)_J + 1 \leq M_{0,\textrm{FirstNodes}(M)_J}\)となり、\(\textrm{Br}(M)_J\)が非複項であったので\(N_J\)も非複項となる。}。
			\end{nenumerate}
			\item \(\textrm{Red}(M) := ((j,j))_{j=0}^{j'_1} \oplus_{\textrm{N}^2} \bigoplus_{\mathbb{N}^2} (\textrm{IncrFirst}^{\textrm{Joints}(M)_J-n_J}(\textrm{Red}(N_J)))_{J=0}^{J_1}\)である\footnote{\(\textrm{Joints}(M)_J\)と\(n_J\)の定義より\(\textrm{Joints}(M)_J \geq n_J\)なので\(\textrm{IncrFirst}^{\textrm{Joints}(M)_J-n_J}\)は意味を持つ。}。
		\end{nenumerate}
		\item \(M_0 \neq (0,0)\)とする。
		\begin{nenumerate}
			\item \(M_{1,0} = 0\)とする\footnote{\(M\)は零項でないので、この時\(j_1 > 0\)である。}。
			\begin{nenumerate}
				\item \(N := ((M_{0,j} - M_{0,0},M_{1,j}))_{j=0}^{j_1}\)と置く\footnote{\(N_0 = (0,0)\)であり、\(M\)が単項より\nameref{直系先祖の基本性質} (1)から任意の\(j \in \mathbb{N}\)に対し\(0 < j \leq j_1\)ならば\(M_{0,0} < M_{0,j}\)であるので\(N_j \in \mathbb{N}^2\)となり、\(\leq_M\)と\(\leq_N\)は等しいので\(N\)は単項である。}。
				\item \(\textrm{Red}(M) := \textrm{Red}(N)\)である。
			\end{nenumerate}
			\item \(M_{1,0} > 0\)とする。
			\begin{nenumerate}
				\item \(N := \textrm{Red}(((j,j))_{j=0}^{M_{1,0}-1} \oplus_{\mathbb{N}^2} \textrm{IncrFirst}^{M_{1,0}}(M))\)と置く\footnote{\(N_0 = (0,0)\)であり、任意の\(j \in \mathbb{N}\)に対し\(j < M_{1,0} + \textrm{Lng}(M)\)ならば\(N_{1,j} > 0\)なので\(N\)は単項である。}。
				\item \(j_1 := \textrm{Lng}(N) - 1\)と置く。
				\begin{nenumerate}
					\item \(M_{1,0} \leq j_1\)かつ\((N_j)_{j=M_{1,0}}^{j_1} \in PT_{\textrm{PS}}\)ならば\(\textrm{Red}(M) := ((N_{0,j}-N_{0,M_{1,0}}+N_{1,M_{1,0}},N_{1,j}))_{j=M_{1,0}}^{j_1}\)である\footnote{\((N_j)_{j=M_{1,0}}^{j_1} \in PT_{\textrm{PS}}\)より任意の\(j \in \mathbb{N}\)に対し\(M_{1,0} < j \leq j_1\)ならば\(N_{0,M_{1,0}} < N_{0,j}\)であるので\((N_{0,j}-N_{0,M_{1,0}}+N_{1,M_{1,0}},N_{1,j}) \in \mathbb{N}^2\)である。}。
					\item \(M_{1,0} \leq j_1\)かつ\((N_j)_{j=M_{1,0}}^{j_1} \in T_{\textrm{PS}} \setminus PT_{\textrm{PS}}\)ならば\(\textrm{Red}(M) := M\)である\footnote{後で証明する\nameref{単項性とRedの関係}により、この分岐が生じないことが分かる。}。
					\item \(M_{1,0} > j_1\)ならば\(\textrm{Red}(M) := M\)である\footnote{後で証明する\nameref{LngのRed不変性}により、この分岐が生じないことが分かる。}。
				\end{nenumerate}
			\end{nenumerate}
		\end{nenumerate}
	\end{nenumerate}
	\item \(M\)が複項とする。
	\begin{nenumerate}
		\item \(J_1 := \textrm{Lng}(P(M)) - 1\)と置く。
		\item \(\textrm{Red}(M) := \bigoplus_{\mathbb{N}^2} (\textrm{Red}(P(M)_J))_{J=0}^{J_1}\)である。
	\end{nenumerate}
\end{nenumerate}

\begin{proposition}[\(\textrm{Red}\)のwell-defined性]\label{Redのwell-defined性}
	上の条件を全て満たす写像\(\textrm{Red}\)が一意に存在する。
\end{proposition}

\begin{hideableproof}
	\begin{indented}
		\item \(\{M \in PT_{\textrm{PS}} \mid M_0 = (0,0)\}\)への制限が一意に存在することは、\(\textrm{Lng}(M) - \textrm{TrMax}(M)\)に関する数学的帰納法より従う。
		\item \(T_{\textrm{PS}}\)への延長が一意に存在することは\(T_{\textrm{PS}} \setminus \{M \in PT_{\textrm{PS}} \mid M_0 = (0,0)\}\)における\(\textrm{Red}\)の定義より即座に従う。
	\end{indented}
\end{hideableproof}

\begin{proposition}[\(\textrm{Red}\)の\(\textrm{IncrFirst}\)不変性]\label{RedのIncrFirst不変性}
	任意の\(M \in T_{\textrm{PS}}\)に対し、\(\textrm{Red}(\textrm{IncrFirst}(M)) = \textrm{Red}(M)\)である。
\end{proposition}

\begin{hideableproof}
	\begin{indented}
		\item \(\textrm{Red}\)の再帰的定義より即座に従う。
	\end{indented}
\end{hideableproof}

\begin{proposition}[\(\textrm{Lng}\)の\(\textrm{Red}\)不変性]\label{LngのRed不変性}
	任意の\(M \in T_{\textrm{PS}}\)に対し、\(\textrm{Lng}(\textrm{Red}(M)) = \textrm{Lng}(M)\)である。
\end{proposition}

\begin{hideableproof}
	\begin{indented}
		\item \(\textrm{Red}\)の再帰的定義により、\(\textrm{Lng}(M)\)に関する数学帰納法から即座に従う。
	\end{indented}
\end{hideableproof}

\begin{corollary}[\(\textrm{Red}\)が零項性を保つこと]\label{Redが零項性を保つこと}
	任意の\(M \in T_{\textrm{PS}}\)に対し、以下は同値である:
	\begin{penumerate}
		\item \(M\)は零項である。
		\item \(\textrm{Red}(M)\)は零項である。
	\end{penumerate}
\end{corollary}

\begin{hideableproof}
	\begin{indented}
		\item \nameref{LngのRed不変性}から\(\textrm{Lng}(M) = 1\)の場合に帰着される。\(\textrm{Lng}(M) = 1\)ならば、\(\textrm{Red}\)の再帰的定義より即座に従う。
	\end{indented}
\end{hideableproof}

\begin{corollary}[直系先祖の\(\textrm{Red}\)不変性]\label{直系先祖のRed不変性}
	任意の\(M \in T_{\textrm{PS}}\)に対し、\(\leq_M\)と\(\leq_{\textrm{Red}(M)}\)は一致する。
\end{corollary}

\begin{hideableproof}
	\begin{indented}
		\item \nameref{LngのRed不変性}と\(\textrm{Red}\)の再帰的定義により、\(\textrm{Lng}(M)\)に関する数学帰納法から即座に従う。
	\end{indented}
\end{hideableproof}

\begin{corollary}[\(\textrm{Red}\)が単項性を保つこと]\label{Redが単項性を保つこと}
	任意の\(M \in T_{\textrm{PS}}\)に対し、以下は同値である:
	\begin{penumerate}
		\item \(M\)は単項である。
		\item \(\textrm{Red}(M)\)は単項である。
	\end{penumerate}
\end{corollary}

\begin{hideableproof}
	\begin{indented}
		\item 単項性の定義と\nameref{直系先祖のRed不変性}より即座に従う。
	\end{indented}
\end{hideableproof}

\begin{corollary}[\(P\)の\(\textrm{Red}\)同変性]\label{PのRed同変性}
	任意の\(M \in T_{\textrm{PS}}\)に対し、\(J_1 := \textrm{Lng}(P(M))\)と置くと\(P(\textrm{Red}(M)) = (\textrm{Red}(P(M)_J))_{J=0}^{J_1}\)である。
\end{corollary}

\begin{hideableproof}
	\begin{indented}
		\item \(\textrm{Red}\)の再帰的定義から、\nameref{直系先祖のRed不変性}より即座に従う。
	\end{indented}
\end{hideableproof}

\begin{proposition}[単項性と\(\textrm{Red}\)の関係]\label{単項性とRedの関係}
	任意の\(M \in PT_{\textrm{PS}}\)に対し、\(N := \textrm{Red}(((j,j))_{j=0}^{M_{1,0}-1} \oplus_{\mathbb{N}^2} \textrm{IncrFirst}^{M_{1,0}}(M))\)と置き\(j_1 := \textrm{Lng}(N)-1\)と置くと、\((N_j)_{j=M_{1,0}}^{j_1} \in PT_{\textrm{PS}}\)である。
\end{proposition}

\begin{hideableproof}
	\begin{indented}
		\item \(N\)の定義より、\(\textrm{Lng}(M) = j_1 - M_{1,0} + 1\)でありかつ任意の\((i,j), (i',j') \in \mathbb{N}^2\)に対し以下は同値である:
		\begin{penumerate}
			\item \((i,j) \leq_M (i',j')\)である。
			\item \((i,j+M_{1,0}) \leq_N (i',j'+M_{1,0})\)である。
		\end{penumerate}
		\item 従って\nameref{直系先祖のRed不変性}より従う。
	\end{indented}
\end{hideableproof}

\begin{proposition}[\(\textrm{Red}\)の冪等性]\label{Redの冪等性}
	任意の\(M \in T_{\textrm{PS}}\)に対し、\(\textrm{Red}^2(M) = \textrm{Red}(M)\)である。
\end{proposition}

\begin{hideableproof}
	\begin{indented}
		\item \(\textrm{Red}\)の再帰的定義により、\(\textrm{Lng}(M)\)に関する数学帰納法から即座に従う。
	\end{indented}
\end{hideableproof}

\begin{proposition}[\(\textrm{Red}\)と\(\textrm{Pred}\)の可換性]\label{RedとPredの可換性}
	任意の\(M \in T_{\textrm{PS}}\)に対し、\(\textrm{Red}(\textrm{Pred}(M)) = \textrm{Pred}(\textrm{Red}(M))\)である。
\end{proposition}

\begin{hideableproof}
	\begin{indented}
		\item \(\textrm{Red}\)の再帰的定義から\(\textrm{Lng}(M)\)に関する数学的帰納法により即座に従う。
	\end{indented}
\end{hideableproof}

\begin{proposition}[\(\textrm{Red}\)と基本列の可換性]\label{Redと基本列の可換性}
	任意の\(M \in T_{\textrm{PS}}\)と\(n \in \mathbb{N}_{+}\)に対し、\(\textrm{Red}(M)[n] = \textrm{Red}(M[n])\)である。
\end{proposition}

\begin{hideableproof}
	\begin{indented}
		\item \(j_1 := \textrm{Lng}(M_1)-1\)と置く。\(j_1\)に関する数学的帰納法で示す。
		\item \(j_1 = 0\)ならば\(\textrm{Red}\)と\(\textrm{operator}[]\)の定義から即座に従う。
		\item \(j_1 > 0\)とする。
		\begin{indented}
			\item \((1,0) <_M^{\textrm{Next}} (1,j_1)\)でないならば、\(\textrm{Red}\)と\(\textrm{operator}[]\)の再帰的定義から帰納法の仮定より従う。
			\item \((1,0) <_M^{\textrm{Next}} (1,j_1)\)ならば、\nameref{RedとPredの可換性}と\(\textrm{operator}[]\)の定義に用いた\(B\)の定義から、\(n\)に関する数学的帰納法により従う。
		\end{indented}
	\end{indented}
\end{hideableproof}

\begin{proposition}[\(\textrm{Red}\)が許容性を保つこと]\label{Redが許容性を保つこと}
	任意の\(M \in T_{\textrm{PS}}\)に対し、\(\mathbb{N}_M = \mathbb{N}_{\textrm{Red}(M)}\)である。
\end{proposition}

\begin{hideableproof}
	\begin{indented}
		\item \nameref{直系先祖のRed不変性}から即座に従う。
	\end{indented}
\end{hideableproof}

\begin{corollary}[許容化の\(\textrm{Red}\)不変性]\label{許容化のRed不変性}
	任意の\(M \in T_{\textrm{PS}}\)と\(j \in \mathbb{N}\)に対し、\(\textrm{Adm}_M(j) = \textrm{Adm}_{\textrm{Red}(M)}(j)\)である。
\end{corollary}

\begin{hideableproof}
	\begin{indented}
		\item \nameref{直系先祖のRed不変性}と\nameref{Redが許容性を保つこと}から即座に従う。
	\end{indented}
\end{hideableproof}

\begin{corollary}[\(\textrm{Red}\)が基点を保つこと]\label{Redが基点を保つこと}
	任意の\((M,m) \in T_{\textrm{PS}}^{\textrm{Marked}}\)に対し、\((\textrm{Red}(M),m) \in RT_{\textrm{PS}}^{\textrm{Marked}}\)である。
\end{corollary}

\begin{hideableproof}
	\begin{indented}
		\item \nameref{直系先祖のRed不変性}と\nameref{Redが許容性を保つこと}から即座に従う。
	\end{indented}
\end{hideableproof}


\subsection{簡約性}

\(M \in T_{\textrm{PS}}\)とする。
\begin{nenumerate}
	\item \(M\)が簡約であるとは、\(\textrm{Red}(M) = M\)を満たすということである。
	\item 簡約ペア数列全体のなす部分集合を\(RT_{\textrm{PS}} \subset T_{\textrm{PS}}\)と置く。
\end{nenumerate}

すなわち\(RT_{\textrm{PS}} = \textrm{Im}(\textrm{Red})\)である。\(RT_{\textrm{PS}} \subset \textrm{Im}(\textrm{Red})\)であることは簡約性の定義から従い、\(\textrm{Im}(\textrm{Red}) \subset RT_{\textrm{PS}}\)であることは\nameref{Redの冪等性}から従う。

\begin{proposition}[簡約性の切片への遺伝性]\label{簡約性の切片への遺伝性}
	任意の\(M \in RT_{\textrm{PS}}\)に対し、\(j_1 := \textrm{Lng}(M)-1\)と置くと、任意の\(j'_0,j'_1 \in \mathbb{N}\)に対し\(j'_0 \leq \textrm{TrMax}(M) \leq j'_1 \leq j_1\)ならば\((M_j)_{j=j'_0}^{j'_1}\)は簡約である。
\end{proposition}

\begin{hideableproof}
	\begin{indented}
		\item \(\textrm{Red}\)の再帰的定義と\nameref{RedとPredの可換性}より即座に従う。
	\end{indented}
\end{hideableproof}

\begin{proposition}[\(P\)が簡約性を保つこと]\label{Pが簡約性を保つこと}
	任意の\(M \in T_{\textrm{PS}}\)に対し、以下は同値である:
	\begin{penumerate}
		\item \(M\)は簡約である。
		\item \(P(M)\)の各成分は簡約である。
	\end{penumerate}
\end{proposition}

\begin{hideableproof}
	\begin{indented}
		\item \nameref{PのRed同変性}より即座に従う。
	\end{indented}
\end{hideableproof}

\begin{proposition}[簡約性が基本列で保たれること]\label{簡約性が基本列で保たれること}
	任意の\(M \in RT_{\textrm{PS}}\)と\(n \in \mathbb{N}_{+}\)に対し、\(M[n] \in RT_{\textrm{PS}}\)である。
\end{proposition}

\begin{hideableproof}
	\begin{indented}
		\item \nameref{Redの冪等性}と\nameref{Redと基本列の可換性}から即座に従う。
	\end{indented}
\end{hideableproof}

\begin{proposition}[簡約性と係数の関係]\label{簡約性と係数の関係}
	任意の\(M \in T_{\textrm{PS}}\)に対し、\(j_1 := \textrm{Lng}(M)-1\)と置くと\(M\)が簡約である必要十分条件は以下が成り立つことである:
	\begin{indented}
		\item[(A)] 任意の\(i \in \{0,1\}\)と\(j'_1 \in \mathbb{N}\)に対し、\((i,j'_0) <_M^{\textrm{Next}} (i,j'_1)\)を満たす一意な\(j'_0 \in \mathbb{N}\)が存在するならば\(M_{i,j'_0}+1 = M_{i,j'_1}\)である。
		\item[(B)] 任意の\(j'_1 \in \mathbb{N}\)に対し、\((0,j'_0) <_M^{\textrm{Next}} (0,j'_1)\)を満たす一意な\(j'_0 \in \mathbb{N}\)が存在せずかつ\(j'_1 \leq j_1\)ならば、\(M_{0,j'_1} = M_{1,j'_1}\)である。
	\end{indented}
\end{proposition}

\nameref{簡約性と係数の関係}の証明の準備としていくつかの補題を用意する。

\begin{lemma}[Redと左端の関係]\label{Redと左端の関係}
	任意の\(M \in T_{\textrm{PS}}\)に対し、以下が成り立つ:
	\begin{penumerate}
		\item \(\textrm{Red}(M)_{1,0} = M_{1,0}\)である。
		\item 任意の\(u,j_0 \in \mathbb{N}\)に対し、\(j_1 := \textrm{Lng}(M)-1\)と置くと、\(M\)が単項かつ\(j_0 \leq j_1\)かつ\((M_j)_{j=0}^{j_0} = ((j,j))_{j=u}^{j_0+u}\)ならば、\(\textrm{Red}(M)_{j_0} = (j_0+u,j_0+u)\)である。
	\end{penumerate}
\end{lemma}

\begin{hideableproof}
	\begin{indented}
		\item \(\textrm{Red}\)の再帰的定義より即座に従う。
	\end{indented}
\end{hideableproof}

\begin{lemma}[簡約性と係数の基本性質]\label{簡約性と係数の基本性質}
	任意の\(M \in RT_{\textrm{PS}}\)と\(j \in \mathbb{N}\)に対し、\(j < \textrm{Lng}(M)\)ならば\(M_{0,j} \geq M_{1,j}\)である。
\end{lemma}

\begin{hideableproof}
	\begin{indented}
		\item \(\textrm{Red}\)の再帰的定義により、\(\textrm{Lng}(M)\)に関する数学帰納法から即座に従う。
	\end{indented}
\end{hideableproof}

\begin{lemma}[簡約性と左端の関係]\label{簡約性と左端の関係}
	任意の\(M \in RT_{\textrm{PS}} \cap PT_{\textrm{PS}}\)と\(u \in \mathbb{N}\)に対し、\(u \leq M_{1,0}\)ならば\(N := ((j,j))_{j=u}^{M_{1,0}-1} \oplus_{\mathbb{N}^2} M\)と置くと\(N\)は簡約かつ単項である。
\end{lemma}

\begin{hideableproof}
	\begin{indented}
		\item \(M_{1,0} = u\)ならば\(N = M\)より従う。以下\(M_{1,0} > u\)とする。
		\item \nameref{Redと左端の関係} (2)より、\(\textrm{Red}\)の定義と\nameref{Redの冪等性}から\(((j,j))_{j=0}^{u-1} \oplus_{\mathbb{N}^2} N = ((j,j))_{j=0}^{M_{1,0}-1} \oplus_{\mathbb{N}^2} M = ((j,j))_{j=0}^{M_{1,0}-1} \oplus_{\mathbb{N}^2} \textrm{Red}(M) = \textrm{Red}(((j,j))_{j=0}^{M_{1,0}-1} \oplus_{\mathbb{N}^2} M) \in RT_{\textrm{PS}}\)であり、\(((j,j))_{j=0}^{u-1} \oplus_{\mathbb{N}^2} N = \textrm{Red}(((j,j))_{j=0}^{u-1} \oplus_{\mathbb{N}^2} N) = ((j,j))_{j=0}^{u-1} \oplus_{\mathbb{N}^2} \textrm{Red}(N)\)となるので\(N = \textrm{Red}(N)\)である。従って\(N\)は簡約である。
		\item \nameref{簡約性と係数の基本性質}から\(u < M_{1,0} \leq M_{0,0}\)であるので\((0,0) \leq_N (0,u+M_{1,0})\)となり、\(M\)の単項性から\(N\)は単項である。
	\end{indented}
\end{hideableproof}

\begin{lemma}[条件(A)と(B)と係数の基本性質]\label{条件(A)と(B)と係数の基本性質}
	任意の\(M \in T_{\textrm{PS}}\)に対し、\(j_1 := \textrm{Lng}(M)-1\)と置くと、\(M_0 = (0,0)\)かつ\(M\)が条件(A)を満たすならば、以下が成り立つ:
	\begin{penumerate}
		\item 任意の\(j \in \mathbb{N}\)に対し、\(j \leq j_1\)ならば\(M_{0,j} \leq j\)である。
		\item \(M\)が条件(B)を満たすならば、任意の\(j \in \mathbb{N}\)に対し、\(j \leq j_1\)ならば\(M_{0,j} \geq M_{1,j}\)である。
		\item \(i \in \{0,1\}\)とし、\(i = 0\)であるか、または\(i = 1\)かつ\(M\)が条件(B)を満たすとする。任意の\(j \in \mathbb{N}\)に対し、ある\(j'_0,j'_1 \in \mathbb{N}\)が存在して\((i,j'_0) \leq_M (i,j'_1)\)でなくかつ\(j'_0 < j'_1 \leq j \leq j_1\)ならば、\(M_{i,j} < j\)である。
	\end{penumerate}
\end{lemma}

\begin{hideableproof}
	\begin{penumerate}
		\item 成り立たないと仮定して矛盾を導く。仮定から\(j \leq j_1\)かつ\(M_{0,j} > j\)を満たす\(j \in \mathbb{N}\)が存在し、その最小値を\(j_0\)と置けば、\(M_{0,0} = 0 \leq j_0 < M_{0,j_0}\)と\nameref{親の存在の判定条件} (1)より\(j_0 > 0\)かつ\((0,j'_0) <_M^{\textrm{Next}} (0,j_0)\)を満たす一意な\(j'_0 \in \mathbb{N}\)が存在する。一方で\(j_0\)の最小性から\(M_{0,j'_0} \leq j'_0\)となり、\(M\)が条件(A)を満たすことから\(M_{0,j_0} = M_{0,j'_0}+1 = j'_0+1 \leq j_0\)となり矛盾する。
		\item 成り立たないと仮定して矛盾を導く。仮定から\(j \leq j_1\)かつ\(M_{0,j} < M_{1,j}\)を満たす\(j \in \mathbb{N}\)が存在する。その最小値を\(j_0\)と置けば、\(M\)が条件(B)を満たすことから\((0,j'_0) <_M^{\textrm{Next}} (0,j_0)\)を満たす一意な\(j'_0 \in \mathbb{N}\)が存在する。一方で\(j_0\)の最小性から\(M_{0,j'_0} \geq M_{1,j'_0}\)となり、\(M\)が条件(A)を満たすことから\(M_{0,j_0} = M_{0,j'_0}+1 \geq M_{1,j'_0}+1\)となる。
		\item[] \((1,j'_0) <_M^{\textrm{Next}} (1,j_0)\)ならば、\(M\)が条件(A)を満たすことから\(M_{0,j_0} = M_{1,j'_0}+1 = M_{1,j_0}\)となり矛盾する。
		\item[] \((1,j'_0) <_M^{\textrm{Next}} (1,j_0)\)でないならば、\(M_{1,j'_0} \geq M_{1,j_0}\)より\(M_{0,j_0} = M_{1,j'_0}+1 > M_{1,j_0}\)となり矛盾する。
		\item \(M_0 = (0,0)\)かつ\(j > 0\)より、一意な\(J_1 \in \mathbb{N}\)と\(a \in \mathbb{N}^{< \omega}\)が存在して以下を満たす:
		\begin{indented}
			\item \(\textrm{Lng}(a) = J_1+1\)である。
			\item \((i,j'-1) <_M^{\textrm{Next}} (i,a_0)\)を満たす一意な\(j' \in \mathbb{N}\)が存在しない。
			\item 任意の\(J \in \mathbb{N}\)に対し、\(J < J_1\)ならば\((i,a_J) <_M^{\textrm{Next}} (i,a_{J+1})\)である。
			\item \(a_{J_1} = j\)である。
		\end{indented}
		\item[] \(j\)の条件から、ある\(j' \in \mathbb{N}\)が存在して\((1,j'-1) <_M^{\textrm{Next}} (1,j')\)でなくかつ\(j'_0 < j' \leq j'_1\)である。従って\(a\)は\(j'\)を成分に持たない狭義単調増大列となるので\(J_1 < j\)である。\(M_{0,0} = 0\)であることから\(a_{0,0} = 0\)であるので、\(i = 0\)ならば\(a_{i,0} = 0\)であり、\(i = 1\)かつ\(M\)が条件(B)を満たすならば(2)より\(a_{i,0} = 0\)である。従っていずれの場合も\(a_{i,0} = 0\)である。更に\(M\)が条件(A)を満たすことから\(M_{i,j} \leq a_{i,0} + J_1 < j\)である。
	\end{penumerate}
\end{hideableproof}

\iffull{それでは本題に戻る。}\fi

\begin{hideableproof}[\nameref{簡約性と係数の関係}の証明]
	\begin{indented}
		\item \(j_1 = 0\)ならば条件(A)は常に成り立ちかつ条件(B)が成り立つ必要十分条件は\(M_{0,0} = M_{1,0}\)であるので、\(\textrm{Red}\)の定義より従う。
		\item \(M\)が単項かつ\(M_{1,0} = 0\)とする。
		\begin{indented}
			\item この条件下で\(M\)の簡約性と\(M\)が条件(A)と(B)を満たすことが同値であることを\(j_1\)に関する数学的帰納法で示す。
			\item \(j_1 = 1\)とする。
			\begin{indented}
				\item この時\((0,0) <_M^{\textrm{Next}} (0,1)\)である。
				\item \((1,0) <_M^{\textrm{Next}} (1,1)\)ならば\(M_{1,0} < M_{1,1}\)であり、条件(A)は\(M_1 = (M_{0,0}+1,M_{1,0}+1)\)と同値であり条件(B)は\(M_{0,0} = M_{1,0}\)と同値であるので、\(\textrm{Red}\)の定義より従う。
				\item \((1,0) <_M^{\textrm{Next}} (1,1)\)でないならば\(M_{1,0} \geq M_{1,1}\)であり、条件(A)は\(M_{0,1} = M_{0,0}+1\)と同値であり条件(B)は\(M_{0,0} = M_{1,0}\)と同値であるので、\(\textrm{Red}\)の定義より従う。
			\end{indented}
			\item \(j_1 > 1\)とする。
			\begin{indented}
				\item まず\(M\)が簡約とする。
				\begin{indented}
					\item \(j_0 := \max \{j \in \mathbb{N} \mid j \leq j_1 \wedge \forall j' \in \mathbb{N}, (j' < j) \to ((1,j') <_M^{\textrm{Next}} (1,j'+1))\}\)と置く\footnote{\(j = 0\)が条件を満たすため\(\max\)は存在する。}。
					\item \(J_1 := \textrm{Lng}(P((M_j)_{j=j_0}^{j_1}))-1\)と置く。
					\item \(\textrm{Red}\)の再帰的定義から\(((M_j)_{j=0}^{j_0})_0 = M_0 = (0,0)\)であり、任意の\(j \in \mathbb{N}\)に対し\(j \leq j_0\)ならば\((0,0) \leq_M (0,j)\)より\((0,0) \leq_{(M_j)_{j=0}^{j_0}} (0,j)\)であるので、\((M_j)_{j=0}^{j_0}\)は単項である。更に\nameref{簡約性の切片への遺伝性}より\((M_j)_{j=0}^{j_0}\)は簡約であり、\(\textrm{Red}\)の定義より\((M_j)_{j=0}^{j_0} = \textrm{Red}((M_j)_{j=0}^{j_0}) = ((j,j))_{j=0}^{j_0}\)である。
					\item \(i \in \{0,1\}\)と\(j'_1 \in \mathbb{N}\)とし、\((i,j'_0) <_M^{\textrm{Next}} (i,j'_1)\)を満たす一意な\(j'_0 \in \mathbb{N}\)が存在するとする。この時\(j'_1 > j'_0 \geq 0\)である。
					\begin{indented}
						\item \(j'_1 < j_1\)ならば、\nameref{簡約性の切片への遺伝性}から\(\textrm{Pred}(M)\)が簡約であるので、帰納法の仮定より\(M_{i,j'_0}+1 = \textrm{Pred}(M)_{i,j'_0}+1 = \textrm{Pred}(M)_{i,j'_1} = M_{i,j'_1}\)が成り立つ。
						\item \(j_0 = j'_1 = j_1\)かつ\(J_1 = 0\)ならば、\(M = ((j,j))_{j=0}^{j_1}\)より\(j'_0+1 = j'_1\)であり\(M_{i,j'_0}+1 = j'_0+1 = j'_1 = M_{i,j'_1}\)である。
						\item \(j_0 < j'_1 = j_1\)かつ\(J_1 = 0\)とする。
						\begin{indented}
							\item この時\((M_j)_{j=j_0}^{j_1}\)が単項である。
							\item \(J_0 := \textrm{Lng}(P((M_j)_{j=j_0+1}^{j_1}))-1\)と置く。
							\item \(m := j_1 - \textrm{Lng}(P((M_j)_{j=j_0+1}^{j_1})_{J_0}) + 1\)と置く。
							\item \(N := ((j,j))_{j=0}^{M_{1,m}-1} \oplus_{\mathbb{N}^2} \textrm{Red}(P((M_j)_{j=j_0+1}^{j_1})_{J_0})\)と置く。
							\item \nameref{簡約性と左端の関係}から\(N\)は簡約である。
							\item \(M_{1,m}=0\)ならば、\nameref{Redと左端の関係} (2)と\(\textrm{Red}\)の再帰的定義から\(N_0 = \textrm{Red}(P((M_j)_{j=j_0+1}^{j_1})_{J_0})_0 = (0,0)\)となり、\(M_{1,m} \neq 0\)ならば\(N\)の定義からやはり\(N_0 = (0,0)\)となる。従っていずれの場合も\(N_0 = (0,0)\)である。
							\item \(\textrm{Lng}(N) - 1 = j_1 - m + M_{1,m} < j_1\)であり、\((0,m) \leq_M (0,j_1)\)より\((0,0) \leq_{P(M)_{J_1}} (0,j_1-m)\)であるので\((0,M_{1,0}) \leq_N (0,j_1-m+M_{1,m})\)である。
							\item \(j'_0 \leq j_0 = j_1 - 1\)とする。
							\begin{indented}
								\item \((M_j)_{j=j_0}^{j_1}\)が単項であるため、\((0,j_0) <_M^{\textrm{Next}} (0,j_1)\)である。
								\item \(M = ((j,j))_{j=0}^{j_0} \oplus_{\mathbb{N}^2} (M_{j_1})\)であり、\((1,j_0) <_M^{\textrm{Next}} (1,j_1)\)でなくかつ\((0,j_0) <_M^{\textrm{Next}} (0,j_1)\)より\(j_0 = M_{0,j_0} < M_{0,j_1}\)かつ\(j_0 = M_{1,j_0} \geq M_{1,j_1}\)である。従って\(j'_0 = M_{i,j_1}-1\)であるので\(M_{i,j'_0}+1 = j'_0+1 = M_{i,j_1}\)である。
							\end{indented}
							\item \(j'_0 \leq j_0 < j_1 - 1\)とする。
							\begin{indented}
								\item \((M_j)_{j=j_0}^{j_1}\)が単項であるため、\((0,j_0) <_M^{\textrm{Next}} (0,j_0+1)\)である。
								\item \((1,j_0) <_M^{\textrm{Next}} (1,j_0+1)\)でなくかつ\((0,j_0) <_M^{\textrm{Next}} (0,j_0+1)\)より\(M_{1,j_0} \geq M_{1,j_0+1}\)であり、\((i,j'_0) <_M^{\textrm{Next}} (i,j_1)\)かつ\(j'_0 < j_0+1 < j_1\)より\((1,j_0+1) \leq_M (1,j_1)\)でない。従って\((0,j_0+1) \leq_M (0,j_1)\)でない。すなわち\(J_0 > 0\)であり、\(j_0 + 1 < m\)である。\(j'_0 < j_0 + 1 < m\)かつ\((0,m) \leq_M (0,j_1)\)かつ\((i,j'_0) <_M^{\textrm{Next}} (i,j_1)\)より\(i = 1\)かつ\(j'_0 = M_{1,j'_0} < M_{1,j_1} \leq M_{1,m}\)となるので、\((1,j'_0) <_N^{\textrm{Next}} (1,j_1 - m + M_{1,m})\)から帰納法の仮定より\(M_{i,j'_0}+1 = N_{1,j'_0} + 1 = N_{1,j_1 - m + M_{1,m}} = M_{i,j'_1}\)である。
							\end{indented}
							\item \(j_0 < j'_0\)ならば、\((i,j'_0) <_M^{\textrm{Next}} (i,j_1)\)から\(P\)の定義より\(m \leq j'_0\)となり、再び\((i,j'_0) <_M^{\textrm{Next}} (i,j_1)\)より\((i,j'_0-m+M_{1,m}) <_N^{\textrm{Next}} (i,j_1-m+M_{1,m})\)となるので、帰納法の仮定より\(M_{i,j'_0}+1 = N_{1,j'_0} + 1 = N_{1,j_1 - m + M_{1,m}} = M_{i,j'_1}\)である。
						\end{indented}
						\item \(j'_1 = j_1\)かつ\(J_1 > 0\)とする。
						\begin{indented}
							\item \(m := j_1 - \textrm{Lng}(P(M)_{J_1}) + 1\)と置く。
							\item \(J_1 > 0\)より\(j_0 < m\)であり、\(P\)の定義から\(M_{0,m} < j_0\)である。\nameref{簡約性と係数の基本性質}から\(M_{0,m}-M_{1,m} \geq 0\)であるので\(M_{1,m} \leq M_{0,m} < j_0 < m\)となる。\(P(M)_{J_1} \in PT_{\textrm{PS}}\)より\(((j,j))_{j=0}^{M_{1,m}} \oplus_{\mathbb{N}^2} (M_j)_{j=m+1}^{j_1} \in PT_{\textrm{PS}}\)となり、\(M\)の簡約性と\(\textrm{Red}\)の定義から
							\begin{eqnarray*}
							P(M)_{J_1} = \textrm{IncrFirst}^{M_{0,m}-M_{1,m}}(\textrm{Red}((M_{0,m},M_{1,m}) \oplus_{\mathbb{N}^2} (M_j)_{j=m+1}^{j_1})) =  \textrm{IncrFirst}^{M_{0,m}-M_{1,m}}(\textrm{Red}(P(M)_{J_1}))
							\end{eqnarray*}
							\item となる。
							\item \(N := ((j,j))_{j=0}^{M_{1,m}-1} \oplus_{\mathbb{N}^2} \textrm{Red}(P(M)_{J_1})\)と置く。
							\item \nameref{簡約性と左端の関係}から\(N\)は簡約である。
							\item \(M_{1,m}=0\)ならば、\nameref{Redと左端の関係} (2)と\(\textrm{Red}\)の再帰的定義から\(N_0 = \textrm{Red}(P(M)_{J_1})_0 = (0,0)\)となり、\(M_{1,m} \neq 0\)ならば\(N\)の定義から\(N_0 = (0,0)\)となる。従っていずれの場合も\(N_0 = (0,0)\)である。
							\item \(\textrm{Lng}(N) - 1 = j_1 - m + M_{1,m} < j_1\)であり、\((0,m) \leq_M (0,j_1)\)より\((0,0) \leq_{P(M)_{J_1}} (0,j_1-m)\)であるので\((0,M_{1,0}) \leq_N (0,j_1-m+M_{1,m})\)である。
							\item \(m = j_1\)ならば、\(P(M)_{J_1} = (M_m)\)であり\(P\)の定義から\(j'_0 \leq j_0\)であるので、\(M_{1,m} \leq M_{0,m} < j_0\)より\(j'_0 = M_{i,m}-1\)となるので、\(M_{i,j'_0}+1 = j'_0+1 = M_{i,m} = M_{i,j'_1}\)である。
							\item \(m < j_1\)かつ\(j'_0 \leq j_0\)ならば、\(j'_0 < m\)かつ\((0,m) \leq_M (0,j_1)\)かつ\((i,j'_0) <_M^{\textrm{Next}} (i,j_1)\)より\(i = 1\)かつ\(j'_0 = M_{1,j'_0} < M_{1,j_1} \leq M_{1,m}\)となるので、\((1,j'_0) <_N^{\textrm{Next}} (1,j_1 - m + M_{1,m})\)から帰納法の仮定より\(M_{i,j'_0}+1 = N_{1,j'_0} + 1 = N_{1,j_1 - m + M_{1,m}} = M_{i,j'_1}\)である。
							\item \(m < j_1\)かつ\(j_0 < j'_0\)ならば、\(P\)の定義より\(m \leq j'_0\)であり\((i,j'_0) <_M^{\textrm{Next}} (i,j_1)\)より\((i,j'_0-m+M_{1,m}) <_N^{\textrm{Next}} (i,j_1-m+M_{1,m})\)となるので、帰納法の仮定より\(M_{i,j'_0}+1 = N_{1,j'_0} + 1 = N_{1,j_1 - m + M_{1,m}} = M_{i,j'_1}\)である。
						\end{indented}
					\end{indented}
				\end{indented}
				\item 以上より、いずれの場合も条件(A)と(B)が従う。
			\end{indented}
			\item 次に\(M\)が条件(A)と(B)を満たすとする。
			\begin{indented}
				\item \(M_{1,0} = 0\)かつ\(M\)が条件(B)を満たすことから、\(M_{0,0} = 0\)である。
				\item ここで、(*) 任意の\(j'_0, j'_1 \in \mathbb{N}\)に対し、\((1,j'_0-1) <_M^{\textrm{Next}} (1,j'_0)\)でなくかつ\(j'_0 \leq j'_1 \leq j_1\)かつ\((M_j)_{j=j'_0}^{j'_1} \in PT_{\textrm{PS}}\)ならば、\(N' := ((M_{0,j}-M_{0,j'_0}+M_{1,j'_0},M_{1,j}))_{j=j'_0}^{j'_1}\)と置くと\(N'\)は簡約であることを示す。ただし\nameref{条件(A)と(B)と係数の基本性質} (2)より\(N'\)の各成分は\(\mathbb{N}^2\)に属する。
				\begin{indented}
					\item \(N := ((j,j))_{j=0}^{M_{1,j'_0}-1} \oplus_{\mathbb{N}^2} N'\)と置く。
					\item \(\textrm{Lng}(N)-1 = j'_1-j'_0+M_{1,j'_0}\)であり、\nameref{条件(A)と(B)と係数の基本性質} (3)から\(M_{1,j'_0} < j'_0\)であるので\(\textrm{Lng}(N)-1 < j'_1 \leq j_1\)である。
					\item \(M_{1,j'_1} = 0\)ならば\(N_0 = N'_0 = (0,0)\)であり、\(M_{1,j'_1} > 0\)ならば\(N\)の定義から\(N_0 = (0,0)\)である。
					\item \(\textrm{IncrFirst}^{M_{0,j'_0}-M_{1,j'_0}}(N') = (M_j)_{j=j'_0}^{j'_1} \in PT_{\textrm{PS}}\)かつ\nameref{PのIncrFirst同変性}より\(N' \in PT_{\textrm{PS}}\)であり、\((0,0) \leq_N (0,M_{1,0})\)より\(N\)は単項である。更に\(N_0 = (0,0)\)であるので、\(N\)は条件(B)を満たす。
					\item \(N\)が条件(A)を満たすことを示す。\(i \in \{0,1\}\)と\(k_0,k_1 \in \mathbb{N}^2\)とし、\((i,k_0) <_N^{\textrm{Next}} (i,k_1)\)とする。
					\begin{indented}
						\item \(M_{1,j'_0} \leq k_0\)ならば、\((i,k_0-M_{1,j'_0}) <_M^{\textrm{Next}} (i,k_1-M_{1,j'_0})\)であり\(M\)が条件(A)を満たすので、\(i=0\)ならば\(N_{i,k_0}+1 = M_{0,k_0-M_{1,j'_0}}-M_{0,j'_0}+M_{1,j'_0}+1 = M_{0,k_1-M_{i,j'_0}}-M_{0,j'_0}+M_{1,j'_0} = N_{i,k_1}\)であり、\(i=1\)ならば\(N_{i,k_0}+1 = M_{1,k_0-M_{1,j'_0}}+1 = M_{1,k_1-M_{i,j'_0}} = N_{i,k_1}\)である。
						\item \(k_0 < M_{1,j'_0} < k_1\)ならば、\((M_j)_{j=j'_0}^{j'_1} \in PT_{\textrm{PS}}\)より\((0,j'_0) \leq_M (0,j'_0+k_1-M_{1,j'_0})\)であるので\((0,M_{1,j'_0}) \leq_N (0,k_1)\)であり、従って\((1,M_{1,j'_0}) \leq_N (1,k_1)\)でなくかつ\(i=1\)となり、\nameref{条件(A)と(B)と係数の基本性質} (2)と(3)から\(N_{1,k_1} \leq N_{0,k_1}\)かつ\(N_{1,k_1} < k_1\)となる。従って\(k_0 = N_{1,k_1}-1\)となり、\(N_{i,k_0}+1 = k_0+1 = N_{i,k_1}\)である。
						\item \(k_1 \leq M_{1,j'_0}\)ならば、\(N_{k_1} = (k_1,k_1)\)となるので\(k_0 = k_1-1\)であり\(N_{i,k_0}+1 = k_0+1 = k_1 = N_{i,k_1}\)である。
					\end{indented}
					\item 従って\(N\)は条件(A)を満たす。\(\textrm{Lng}(N)-1 < j_1\)かつ\(N_0 = (0,0)\)かつ\(N\)が条件(A)と(B)を満たすことから、帰納法の仮定より\(N\)は簡約である。
					\item \((M_j)_{j=j'_0}^{j'_1} \in PT_{\textrm{PS}}\)より、任意の\(j \in \mathbb{N}\)に対し\(j'_0 \leq j \leq j'_1\)ならば\(M_{0,j'_0} \leq M_{0,j}\)すなわち\(M_{0,j}-M_{0,j'_0}+M_{1,j'_0} \leq M_{1,j'_0}\)となるので、一意な\(M' \in PT_{\textrm{N}}\)が存在して\(N' = \textrm{IncrFirst}^{M_{1,j'_0}}(M')\)となる。従って\nameref{RedのIncrFirst不変性}と\nameref{Redと左端の関係} (2)と\(\textrm{Red}\)の再帰的定義から
					\begin{eqnarray*}
					& & ((j,j))_{j=0}^{M_{1,j'_0}-1} \oplus_{\mathbb{N}^2} N' = N = \textrm{Red}(N) \\
					& = & \textrm{Red}(((j,j))_{j=0}^{M_{1,0}-1} \oplus_{\mathbb{N}^2} \textrm{IncrFirst}^{M_{1,j'_0}}(M')) \\
					& = & \textrm{Red}(((j,j))_{j=0}^{M_{1,0}-1} \oplus_{\mathbb{N}^2} \textrm{IncrFirst}^{M'_{1,0}}(M')) \\
					& = & ((j,j))_{j=0}^{M_{1,0}-1} \oplus_{\mathbb{N}^2} \textrm{Red}(M') \\
					& = & ((j,j))_{j=0}^{M_{1,0}-1} \oplus_{\mathbb{N}^2} \textrm{Red}(\textrm{IncrFirst}^{M_{1,j'_0}}(M')) \\
					& = & ((j,j))_{j=0}^{M_{1,0}-1} \oplus_{\mathbb{N}^2} \textrm{Red}(N')
					\end{eqnarray*}
					\item となるので\(N' = \textrm{Red}(N')\)であり、すなわち\(N'\)は簡約である。
				\end{indented}
				\item \(\textrm{Red}\)の再帰的定義中に導入した記号を用いると、\(P\)の定義から\(\textrm{Red}(M)\)の計算に現れる\(N_J\)は全て(*)の仮定を満たし、\(\textrm{IncrFirst}^{\textrm{Joints}(M)_J-n_J}(\textrm{Red}(N_J)) = N_J\)となる。従って\(\textrm{Red}\)の定義から\(\textrm{Red}(M) = M\)となり、すなわち\(M\)は簡約である。
			\end{indented}
		\end{indented}
		\item \(M\)が単項かつ\(M_{1,0} > 0\)とする。
		\begin{indented}
			\item \(N := ((j,j))_{j=0}^{M_{1,0}-1} \oplus_{\mathbb{N}^2} M\)と置く。
			\item \(M\)が単項かつ\(M_{1,0} > 0\)より\(M\)は\((0,0)\)を成分に持たず、従って\(N\)は単項である。
			\item まず\(M\)が簡約とする。
			\begin{indented}
				\item \nameref{簡約性と左端の関係}から\(N\)は簡約である。更に\(N\)が単項かつ\(N_0 = (0,0)\)より、\(N\)は条件(A)と(B)を満たす。特に\(M\)は条件(A)を満たす。\nameref{簡約性と係数の基本性質}から\(M_{0,0} = N_{0,M_{1,0}} \geq N_{1,M_{1,0}} = M_{1,0}\)であり、よって\((0,M_{1,0}-1) <_M^{\textrm{Next}} (0,M_{1,0})\)となり\(N\)が条件(A)を満たすことから\(M_{0,0} = M_{1,0}\)となる。更に\(M\)が単項より、\(M\)は条件(B)を満たす。
			\end{indented}
			\item 次に\(M\)が条件(A)と(B)を満たすとする。
			\begin{indented}
				\item \(N\)が単項かつ\(N_0 = (0,0)\)かつ\(M\)が条件(B)を満たすことから、\(N\)は条件(B)を満たす。
				\item \(i \in \{0,1\}\)かつ\(j'_0, j'_1 \in \mathbb{N}\)として、\((i,j'_0) <_N^{\textrm{Next}} (i,j'_1)\)とする。
				\begin{indented}
					\item \(M_{1,0} \leq j'_0\)ならば\((i,j'_0-M_{1,0}) <_M^{\textrm{Next}} (i,j'_1-M_{1,0})\)であり、\(M\)が条件(A)を満たすので\(N_{i,j'_0}+1 = M_{i,j'_0-M_{1,0}}+1 = M_{i,j'_1-M_{1,0}} = N_{i,j'_1}\)である。
					\item \(j'_0 < M_{1,0} < j'_1\)ならば、\((0,0) \leq_M (0,j'_1-M_{1,0})\)かつ\((0,m_{1,0}) \leq_N (0,j'_1)\)であるので、\((i,j'_0) <_N^{\textrm{Next}} (i,j'_1)\)より\((1,0) \leq_M (1,j'_1-M_{1,0})\)でなくかつ\(i=1\)である。\(M\)が条件(B)を満たすので\(M_{0,j'_1} > M_{0,0} = M_{1,0} \geq M_{1,j'_1}\)となり、従って\(j'_0 = M_{1,j'_1-M_{1,0}}-1\)であり\(N_{i,j'_0}+1 = j'_0+1 = M_{1,j'_1-M_{1,0}} = N_{i,j'_1}\)である。
					\item \(j'_1 \leq M_{1,0}\)ならば\(j'_0+1 = j'_1\)かつ\(N_{i,j'_0}+1 = j'_0+1 = j'_1 = N_{i,j'_1}\)である。
				\end{indented}
				\item 以上より\(N\)は条件(A)を満たす。従って\(N\)は簡約である。一方で\(M\)は条件(B)を満たすので\(M_{0,0} = M_{1,0}\)であり、\(M\)が単項であるので\(\textrm{IncrFirst}^{M_{1,0}}(M') = M\)を満たす一意な\(M' \in PT_{\textrm{PS}}\)が存在する。\nameref{RedのIncrFirst不変性}と\nameref{Redと左端の関係} (2)と\(\textrm{Red}\)の再帰的定義から
				\begin{eqnarray*}
				& & ((j,j))_{j=0}^{M_{1,0}-1} \oplus_{\mathbb{N}^2} M = N = \textrm{Red}(N) \\
				& = & \textrm{Red}(((j,j))_{j=0}^{M_{1,0}-1} \oplus_{\mathbb{N}^2} \textrm{IncrFirst}^{M_{1,0}}(M')) \\
				& = & \textrm{Red}(((j,j))_{j=0}^{M_{1,0}-1} \oplus_{\mathbb{N}^2} \textrm{IncrFirst}^{M'_{1,0}}(M')) \\
				& = & ((j,j))_{j=0}^{M_{1,0}-1} \oplus_{\mathbb{N}^2} \textrm{Red}(M') \\
				& = & ((j,j))_{j=0}^{M_{1,0}-1} \oplus_{\mathbb{N}^2} \textrm{Red}(\textrm{IncrFirst}^{M_{1,0}}(M')) \\
				& = & ((j,j))_{j=0}^{M_{1,0}-1} \oplus_{\mathbb{N}^2} \textrm{Red}(M)
				\end{eqnarray*}
				\item となるので\(M = \textrm{Red}(M)\)であり、すなわち\(M\)は簡約である。
			\end{indented}
		\end{indented}
		\item \(M\)が複項とする。
		\begin{indented}
			\item まず\(M\)が簡約ならば、\nameref{Pが簡約性を保つこと}から\(P(M)\)の各成分は簡約かつ単項であるので特に条件(A)と(B)を満たし、\(M = \bigoplus_{\mathbb{N}^2} P(M)\)より\(M\)も条件(A)と(B)を満たす。
			\begin{indented}
				\item 次に\(M\)が条件(A)と(B)を満たすならば、\(M = \bigoplus_{\mathbb{N}^2} P(M)\)より\(P(M)\)の各成分は条件(A)と(B)を満たしかつ単項であるので特に簡約であるので、\nameref{Pが簡約性を保つこと}から\(M\)も簡約である。
			\end{indented}
		\end{indented}
	\end{indented}
\end{hideableproof}

\begin{corollary}[直系先祖による切片と\(\textrm{Red}\)と\(\textrm{IncrFirst}\)の関係]\label{直系先祖による切片とRedとIncrFirstの関係}
	任意の\(M \in RT_{\textrm{PS}}\)と\(j'_0,j'_1 \in \mathbb{N}\)に対し、\(j_1 := \textrm{Lng}(M)-1\)と置き\(j'_0 < j'_1 \leq j_1\)として\(N := \textrm{Red}((M_j)_{j=j'_0}^{j'_1})\)と置くと、\((0,j'_0) \leq_M (0,j'_1)\)ならば\(N\)は簡約かつ単項かつ\((M_j)_{j=j'_0}^{j'_1} = \textrm{IncrFirst}^{M_{0,m} - M_{1,m}}(N)\)である\footnotemark{}。
\end{corollary}
\footnotetext{\nameref{簡約性と係数の関係}より\(M_{0,j'_0} - M_{1,j'_0} \geq 0\)であるので\(\textrm{IncrFirst}^{M_{0,j'_0} - M_{1,j'_0}}\)は意味を持つ。}

\begin{hideableproof}
	\begin{indented}
		\item \(\textrm{Red}\)を用いた\(N\)の定義から\(N\)は簡約である。
		\item \nameref{簡約性と係数の基本性質}より\(M_{0,j'_0}-M_{1,j'_0} \geq 0\)である。\nameref{単項性の直系先祖による切片への遺伝性}より\((M_j)_{j=j'_0}^{j'_1}\)は単項であるので、任意の\(j \in \mathbb{N}\)に対し\(j'_0 \leq j \leq j'_1\)ならば\((0,j'_0) \leq_M (0,j)\)となるので\(M_{0,j}-M_{0,j_0} \geq 0\)である。
		\item \(M' := ((M_{0,j}-M_{0,j'_0}+M_{1,j'_0},M_{1,j}))_{j=j'_0}^{j'_1}\)と置く。
		\item \(\textrm{IncrFirst}^{M_{0,j'_0} - M_{1,j'_0}}(M') = (M_j)_{j=j'_0}^{j'_1}\)であるので、\nameref{leq_MのIncrFirst不変性}より\(M'\)は単項である。
		\item \nameref{簡約性と係数の関係}から\(M\)は条件(A)と(B)を満たす。特に\((M_j)_{j=j'_0}^{j'_1}\)は条件(A)を満たし、従って\nameref{leq_MのIncrFirst不変性}より\(M'\)は条件(A)を満たす。また\(M'\)は単項でかつ\(M'_{0,0} = M_{1,j'_0} = M'_{1,0}\)であるので\(M'\)は条件(B)を満たす。従って\nameref{簡約性と係数の関係}より\(M'\)は簡約である。
		\item \nameref{RedのIncrFirst不変性}より\(N = \textrm{Red}((M_j)_{j=j'_0}^{j'_1}) = \textrm{Red}(\textrm{IncrFirst}^{M_{0,j'_0} - M_{1,j'_0}}(M')) = \textrm{Red}(M') = M'\)である。従って\(\textrm{IncrFirst}^{M_{0,j'_0} - M_{1,j'_0}}(N) = \textrm{IncrFirst}^{M_{0,j'_0} - M_{1,j'_0}}(M') = (M_j)_{j=j'_0}^{j'_1}\)である。
	\end{indented}
\end{hideableproof}

\begin{corollary}[\(1\)列ペア数列の基本性質]\label{1列ペア数列の基本性質}
	任意の\(M \in T_{\textrm{PS}}\)に対し、以下は同値である:
	\begin{penumerate}
		\item \(\textrm{Lng}(M) = 1\)かつ\(M\)は簡約である。
		\item 一意な\(v \in \mathbb{N}\)が存在して\(M = ((v,v))\)である。
	\end{penumerate}
\end{corollary}

\begin{hideableproof}
	\begin{indented}
		\item \(\textrm{Lng}(M) = 1\)かつ\(M\)は簡約であるとする。
		\begin{indented}
			\item \nameref{簡約性と係数の関係}から\(M\)は条件(A)と(B)を満たす。特に\(M\)が条件(B)を満たすので一意な\(v \in \mathbb{N}\)が存在して\(M_0 = (v,v)\)である。
			\item \(\textrm{Lng}(M) = 1\)より、\(M = (M_0) = ((v,v))\)である。
		\end{indented}
		\item 一意な\(v \in \mathbb{N}\)が存在して\(M = ((v,v))\)であるとする。
		\begin{indented}
			\item \(M\)は条件(A)と(B)を満たすので、\nameref{簡約性と係数の関係}からは簡約である。また\(\textrm{Lng}(M) = 1\)である。
		\end{indented}
	\end{indented}
\end{hideableproof}

\subsection{標準形}

部分集合\(S \subset T_{\textrm{PS}}\)であって以下を満たすもののうち最小のものを\(ST_{\textrm{PS}}\)と置く\footnote{\(S = T_{\textrm{PS}}\)が条件を満たし、条件を満たす\(S\)全体の共通部分もまた条件を満たすため、最小の\(S\)が存在する。}:
\begin{nenumerate}
	\item 任意の\(u,v \in \mathbb{N}\)に対し、\(u \leq v\)ならば\(((j,j))_{j=u}^{v} \in S\)である。
	\item 任意の\(M \in S\)と\(n \in \mathbb{N}_{+}\)に対し、\(M[n] \in S\)である。
\end{nenumerate}

ここでは\(ST_{\textrm{PS}}\)に属するペア数列のことを標準形と呼ぶ。通常は標準形ペア数列と言ったら\(3\)行バシク行列\(((0,0,0)(1,1,1))\)の展開で現れるものを指すが、それは上の条件において\(u = 0\)としたものに対応するため、ここでの流儀では標準形が通常より広い対象を指す用語となる。

\begin{proposition}[標準形の簡約性]\label{標準形の簡約性}
	\(ST_{\textrm{PS}} \subset RT_{\textrm{PS}}\)である。
\end{proposition}

\begin{hideableproof}
	\begin{indented}
		\item 任意の\(u,v \in \mathbb{N}\)に対し、\(u \leq v\)ならば\(((j,j))_{j=u}^{v}\)は簡約である。従って\nameref{簡約性が基本列で保たれること}と\(ST_{\textrm{PS}}\)の定義に基づく最小性から従う。
	\end{indented}
\end{hideableproof}

各\(k \in \mathbb{N}\)に対し、部分集合\(S_kT_{\textrm{PS}} \subset ST_{\textrm{PS}}\)を以下のように再帰的に定める:
\begin{nenumerate}
	\item \(k = 0\)ならば\(S_kT_{\textrm{PS}} := \{((j,j))_{j=u}^{v} \mid (u,v) \in \mathbb{N}^2 \land u \leq v\}\)である。
	\item \(k > 0\)ならば\(S_kT_{\textrm{PS}} := \{M[n] \mid M \in S_{k-1}T_{\textrm{PS}} \wedge n \in \mathbb{N}\}\)である。
\end{nenumerate}

\(ST_{\textrm{PS}}\)の定義に基づく最小性より、\(ST_{\textrm{PS}} = \bigcup_{k \in \mathbb{N}} S_kT_{\textrm{PS}}\)である。

\begin{proposition}[標準形の単項成分が標準形であること]\label{標準形の単項成分が標準形であること}
	任意の\(k \in \mathbb{N}\)と\(M \in S_kT_{\textrm{PS}}\)に対し、\(P(M) \in S_kT_{\textrm{PS}}^{< \omega}\)である。
\end{proposition}

\begin{hideableproof}
	\begin{indented}
		\item \(k\)に関する数学的帰納法で示す。
		\item \(k = 0\)ならば\(S_kT_{\textrm{PS}}\)の定義から\(M\)が零項または単項であるので従う。
		\item \(k > 0\)とする。
		\begin{indented}
			\item \(J_1 := \textrm{Lng}(P(M))-1\)と置く。
			\item \(S_kT_{\textrm{PS}}\)の定義から、ある\(M' \in S_{k-1}T_{\textrm{PS}}\)と\(n \in \mathbb{N}_{+}\)が存在して\(M = M'[n]\)である。帰納法の仮定より\(P(M') \in S_{k-1}T_{\textrm{PS}}^{< \omega}\)である。
			\item \(J_0 := \textrm{Lng}(P(M'))-1\)と置く。
			\item \(\textrm{Lng}(P(M')_{J_0}) = 1\)とする。
			\begin{indented}
				\item \(J_0 = 0\)ならば、\(M = M'[n] = P(M')_{J_0}[n] = P(M')_{J_0} \in S_{k-1}T_{\textrm{PS}}\)であるので帰納法の仮定より従う。
				\item \(J_0 > 0\)ならば、\nameref{Pと基本列の関係} (1)より\(P(M) = (P(M')_J)_{J=0}^{J_0-1} \in S_{k_0-1}T_{\textrm{PS}}^{< \omega}\)である。
			\end{indented}
			\item \(\textrm{Lng}(P(M')_{J_0}) > 1\)とする。
			\begin{indented}
				\item \nameref{Pと基本列の関係} (2)より\(P(M) = (P(M')_J)_{J=0}^{J_1} \oplus_{\mathbb{N}^2} P(P(M')_{J_0}[n])\)である。
				\item \((P(M')_J)_{J=0}^{J_0-1} \in S_{k_0-1}T_{\textrm{PS}}^{< \omega}\)であり、\(P(M')_{J_0} \in S_{k_0-1}T_{\textrm{PS}}\)より帰納法の仮定から\(P(P(M')_{J_0}[n]) \in S_{k_0-1}T_{\textrm{PS}}^{< \omega}\)であるので、\(P(M) \in S_{k_0-1}T_{\textrm{PS}}^{< \omega}\)である。
			\end{indented}
		\end{indented}
	\end{indented}
\end{hideableproof}

\begin{proposition}[標準形の始切片への遺伝性]\label{標準形の始切片への遺伝性}
	任意の\(M \in ST_{\textrm{PS}}\)と\(j'_1 \in \mathbb{N}\)に対し、\(j_1 := \textrm{Lng}(M)-1\)と置くと、\(j'_1 \leq j_1\)ならば\((M_j)_{j=0}^{j'_1}\)は標準形である。
\end{proposition}

\begin{hideableproof}
	\begin{indented}
		\item \(M[1] = \textrm{Pred}(M)\)であるため、\(j_1-m_1\)に関する数学的帰納法より即座に従う。
	\end{indented}
\end{hideableproof}


\subsection{降順性}

\(Q \in T_{\textrm{PS}}^{< \omega}\)とする。
\begin{nenumerate}
	\item \(J_1 := \textrm{Lng}(Q)-1\)と置く。
	\item \(Q\)が降順であるとは、任意の\(J'_0,J'_1 \in \mathbb{N}\)に対し、\(J'_0 \leq J'_1 \leq J_1\)ならば以下が成り立つということである:
	\begin{nenumerate}
		\item \((Q_{J'_0})_{0,0} \geq (Q_{J'_1})_{0,0}\)である。
		\item \((Q_{J'_0})_{0,0} = (Q_{J'_1})_{0,0}\)ならば、\((Q_{J'_0})_{1,0} \geq (Q_{J'_1})_{1,0}\)である。
	\end{nenumerate}
\end{nenumerate}

\nameref{Pの各成分の左端の単調性}より、任意の\(M \in T_{\textrm{PS}}\)に対し以下は同値である:
\begin{penumerate}
	\item \(P(M)\)は降順である。
	\item 任意の\(J'_0,J'_1 \in \mathbb{N}\)に対し、\(J'_0 \leq J'_1 \leq J_1\)かつ\((P(M)_{J'_0})_{0,0} = (P(M)_{J'_1})_{0,0}\)ならば、\((P(M)_{J'_0})_{1,0} \geq (P(M)_{J'_1})_{1,0}\)である。
\end{penumerate}
従って特に\(M\)が単項の時、\(j'_1 := \textrm{TrMax}(M)\)と置き、\(j_1 := \textrm{Lng}(M)-1\)と置き、\(j'_1 < j_1\)である場合は\(\textrm{Br}(M)\)の降順性の判定には(2)を\((M_j)_{j=j'_1}^{j_1}\)に対し確認すれば良い。

\begin{proposition}[標準形の切片と\(\textrm{Br}\)の降順性の関係]\label{標準形の切片とBrの降順性の関係}
	任意の\(M \in ST_{\textrm{PS}}\)と\(j'_0,j'_1 \in \mathbb{N}\)に対し、\(j_1 := \textrm{Lng}(M)-1\)と置き、\(j'_0 < j'_1 \leq j_1\)として\(M' := (M_j)_{j=j'_0}^{j'_1}\)と置くと、\((0,j'_0) \leq_M (0,j'_1)\)ならば\(M'\)は単項かつ\(\textrm{Br}(M')\)は降順である。
\end{proposition}

\begin{hideableproof}
	\begin{indented}
		\item \(M'\)の単項性は\nameref{単項性の直系先祖による切片への遺伝性}より従う。
		\item \(ST_{\textrm{PS}} = \bigcup_{k \in \mathbb{N}} S_kT_{\textrm{PS}}\)より、一意な\(k_0 \in \mathbb{N}\)が存在して\(M \in S_{k_0}T_{\textrm{PS}}\)かつ任意の\(k \in \mathbb{N}\)に対し\(k < k_0\)ならば\(M \notin S_kT_{\textrm{PS}}\)である。
		\item \((0,j'_0) \leq_M (0,j'_1)\)と\nameref{Pの各成分の非複項性}と\nameref{標準形の単項成分が標準形であること}から、\(M\)は非複項であるとしてよい。\(M\)が零項であるならば、\(j'_0 = j'_1 = 0\)となり\(M' = M \in ST_{\textrm{PS}}\)である。以下では\(M\)が単項であるとする。
		\item
		\item \(M\)が単項であるという条件下で\(M'\)が標準形となることを\(k_0\)に関する数学的帰納法で示す。
		\item \(k_0 = 0\)ならば、\(M = ((j,j))_{j=0}^{j_1}\)であるため\(M' = ((j,j))_{j=j'_0}^{j'_1}\)であり、\(\textrm{Br}(M') = ()\)となるので\(\textrm{Br}(M')\)は降順である。
		\item \(k_0 > 0\)とする。
		\begin{indented}
			\item \(M \in S_{k_0}T_{\textrm{PS}}\)より、ある\(N \in S_{k_0-1}T_{\textrm{PS}}\)と\(n \in \mathbb{N}_{+}\)が存在して\(N[n] = M\)となる。
			\item \nameref{標準形の単項成分が標準形であること}から\(P(N)_0 \in S_{k_0-1}T_{\textrm{PS}}\)であるので、\(M \neq P(N)_0\)である。従って\(M\)の単項性と\nameref{Pと基本列の関係}より、\(N\)は単項である。
			\item \(j_1^N := \textrm{Lng}(N)-1\)と置く。
			\item
			\item \(M' = (N_j)_{j=j'_0}^{j'_1}\)ならば、\((0,j'_0) \leq_M (0,j'_1)\)より\((0,j'_0) \leq_N (0,j'_1)\)であるので、帰納法の仮定から\(\textrm{Br}(M')\)は降順である。
			\item \(M = \textrm{Pred}(N)\)ならば、\(j'_1 \leq j_1 = j_1^N-1 < j_1^N\)より\(M' = (N_j)_{j=j'_0}^{j'_1}\)であるので、既に示したように\(\textrm{Br}(M')\)は降順である。
			\item \(n = 1\)ならば、\(M = N[1] = \textrm{Pred}(N)\)であるので、既に示したように\(\textrm{Br}(M')\)は降順である。
			\item
			\item 以下では\(n > 1\)とする。
			\item \(N\)の単項性から\((0,0) \leq_N (0,j_1^N)\)かつ\(j_1^N > 0\)であるので、一意な\(j_0^N \in \mathbb{N}\)が存在して\((0,j_0^N) <_N^{\textrm{Next}} (0,j_1^N)\)となる。
			\item \(N' := (N_j)_{j=j'_0}^{j_1^N}\)と置く。
			\item \(J_1 := \textrm{Lng}(\textrm{Br}(N'))-1\)と置く。
			\item
			\item \(N_{1,j_1^N} = 0\)とする。
			\begin{indented}
				\item \(M = (N_j)_{j=0}^{j_0^N-1} \bigoplus_{\mathbb{N}^2} ((N_j)_{j=j_0^N}^{j_1^N-1})_{k=0}^{n-1}\)であり、\(j_1 = j_0^N+(n+1)(j_1^N-j_0^N)-1\)である。
				\item \(j'_1 \leq j_0^N\)ならば\(M' = (N_j)_{j=j'_0}^{j'_1}\)かつ\((0,j'_0) \leq_N (0,j'_1)\)となるので、帰納法の仮定より\(\textrm{Br}(M') = \textrm{Br}((N_j)_{j=j'_0}^{j'_1})\)は降順である。
				\item \(j'_0 < j_0^N < j'_1\)とする。
				\begin{indented}
					\item \(j'_1-j_0^N\)を\(j_1^N-j_0^N\)で割った商と余りをそれぞれ\(q,r \in \mathbb{N}\)と置く。
					\item \(q < n\)かつ\(r < j_1^N-j_0^N\)であり、\(j'_1 = j_0^N+q(j_1^N-j_0^N)+r\)である。
					\item \((N[q+1]_j)_{j=j'_0}^{j'_1} = M'\)であるので、\(q = n-1\)として良い。
					\item \((0,j_0^N) <_N^{\textrm{Next}} (0,j_1^N)\)と\nameref{直系先祖の木構造} (1)より\((0,j_0^N) \leq_N (0,j_1^N-1)\)かつ\((0,j_0^N) \leq_N (0,j_0^N+r)\)すなわち\((0,j_0^N+(n-1)(j_1^N-j_0^N)) \leq_M (0,j'_1)\)である。更に\((0,j'_0) \leq_M (0,j'_1)\)かつ\(j'_0 < j_0^N \leq j_0^N+(n-1)(j_1^N-j_0^N) \leq j'_1\)であるので、\nameref{Pの各成分の非複項性}より\((0,j'_0) \leq_M (0,j_0^N+(n-1)(j_1^N-j_0^N))\)である。
					\item \(M_{j_0^N+(n-1)(j_1^N-j_0^N)} = M_{j_0^N} = N_{j_0^N}\)より\((M_j)_{j=0}^{j_0^N-1} \oplus_{\mathbb{N}^2} (M_{j_0^N+(n-1)(j_1^N-j_0^N)}) = (M_j)_{j=0}^{j_0^N} = (N_j)_{j=0}^{j_0^N}\)であるので、\((0,j'_0) \leq_M (0,j_0^N+(n-1)(j_1^N-j_0^N))\)より\((0,j'_0) \leq_N (0,j_0^N)\)である。以上より\((0,j'_0) \leq_N (0,j_0^N) \leq_N (0,j_1^N-1)\)かつ\((0,j'_0) \leq_N (0,j_0^N) \leq_N (0,j_1^N)\)である。
					\item 帰納法の仮定から、\(N'\)は単項かつ\(\textrm{Br}(N')\)は降順である。
					\item \(N_{0,j_1^N} = 0\)かつ\(j_1^N-j'_0 > j_0^N-j'_0 > 0\)より、\(\textrm{TrMax}(N') < j_1^N-j'_0\)となる。従って\(J_1 \geq 0\)である。
					\item \(j_0^N-j'_0 > 0\)であるので、\((0,j_0^N) <_N^{\textrm{Next}} (0,j_1^N)\)より\((0,j_0^N-j'_0) <_{N'}^{\textrm{Next}} (0,j_1^N-j'_0)\)である。また\(N'\)の単項性から\((0,0) \leq_{N'} (0,j_0^N-j'_0)\)であるので、一意な\(j_{-1} \in \mathbb{N}\)が存在して\((0,j_{-1}) <_{N'}^{\textrm{Next}} (0,j_0^N-j'_0)\)である。
					\item
					\item \(j_0^N-j'_0 \leq \textrm{TrMax}(N')\)とする。
					\begin{indented}
						\item \((0,j_0^N-j'_0) <_{N'}^{\textrm{Next}} (0,j_1^N-j'_0)\)と\nameref{Pの各成分の非複項性}より\(\textrm{FirstNodes}(N')_{J_1} = j_1^N-j'_0\)かつ\(\textrm{Br}(N')_{J_1} = (N_{j_1^N})\)であるので、\(\textrm{Br}(M') = (\textrm{Br}(N')_J)_{J=0}^{J_1-1} \oplus_{T_{\textrm{PS}}} ((N_j)_{j=j_0^N}^{j_1^N-1})_{k=1}^{n-2} \oplus_{T_{\textrm{PS}}} ((N_j)_{j=j_0^N}^{j_0^N+r})\)となる。
						\item 従って\(\textrm{Lng}(\textrm{Br}(M'))-1 = J_1+n-2\)かつ\(((\textrm{Br}(M')_J)_0)_{J=0}^{J_1+n-2} = ((\textrm{Br}(N')_J)_0)_{J=0}^{J_1-1} \oplus_{\mathbb{N}^2} (N_{j_0^N})_{k=0}^{n-1}\)となる。\(N_{0,j_0^N} < N_{0,j_1^N} = (\textrm{Br}(N')_{J_1})_{0,0}\)より、\(\textrm{Br}(M')\)は降順である。
					\end{indented}
					\item
					\item \(j_{-1} \leq \textrm{TrMax}(N') < j_0^N-j'_1\)とする。
					\begin{indented}
						\item \((0,j_{-1}) <_{N'}^{\textrm{Next}} (0,j_0^N-j'_0)\)と\nameref{Pの各成分の非複項性}より\(\textrm{FirstNodes}(N')_{J_1} = j_0^N-j'_0\)かつ\(\textrm{Br}(N')_{J_1} = ((N_j)_{j=j_0^N}^{j_1^N})\)であるので、\(\textrm{Br}(M') = (\textrm{Br}(N')_J)_{J=0}^{J_1-1} \oplus_{T_{\textrm{PS}}} ((N_j)_{j=j_0^N}^{j_1^N-1})_{k=0}^{n-2} \oplus_{T_{\textrm{PS}}} ((N_j)_{j=j_0^N}^{j_0^N+r})\)となる。
						\item 従って\(\textrm{Lng}(\textrm{Br}(M'))-1 = J_1+n-1\)かつ\(((\textrm{Br}(M')_J)_0)_{J=0}^{J_1+n-1} = ((\textrm{Br}(N')_J)_0)_{J=0}^{J_1-1} \oplus_{\mathbb{N}^2} (N_{j_0^N})_{k=0}^{n-1}\)となる。\(N_{j_0^N} = N_{\textrm{FirstNodes}(N')_{J_1}+j'_0} = (\textrm{Br}(N')_{J_1})_0\)より、\(\textrm{Br}(M')\)は降順である。
					\end{indented}
					\item
					\item \(\textrm{TrMax}(N') < j_{-1}\)とする。
					\begin{indented}
						\item \nameref{Pの各成分の非複項性}より\(\textrm{FirstNodes}(N')_{J_1} \leq j_{-1}\)であるので、\(\textrm{Br}(M') = (\textrm{Br}(N')_J)_{J=0}^{J_1-1} \oplus_{T_{\textrm{PS}}} ((M_j)_{j = \textrm{FirstNodes}(N')_{J_1}+j'_0}^{j'_1})\)となる。
						\item 従って\(\textrm{Lng}(\textrm{Br}(M'))-1 = J_1\)かつ\(((\textrm{Br}(M')_J)_0)_{J=0}^{J_1} = ((\textrm{Br}(N')_J)_0)_{J=0}^{J_1-1} \oplus_{\mathbb{N}^2} (M_{\textrm{FirstNodes}(N')_{J_1}+j'_0})\)となる。\(\textrm{FirstNodes}(N')_{J_1}+j'_0 < j_{-1}+j'_0 < j_0^N < j_1^N\)より\(M_{\textrm{FirstNodes}(N')_{J_1}+j'_0} = N_{\textrm{FirstNodes}(N')_{J_1}+j'_0} = (\textrm{Br}(N')_{J_1})_0\)より、\(\textrm{Br}(M')\)は降順である。
					\end{indented}
				\end{indented}
				\item
				\item \(j_0^N \leq j'_0\)とする。
				\begin{indented}
					\item \(j'_0-j_0^N\)を\(j_1^N-j_0^N\)で割った商と余りをそれぞれ\(q,r \in \mathbb{N}\)と置く。
					\item \(r' := j_1-(j_0^N+q(j_1^N-j_0^N))\)と置く。
					\item \(q < n\)かつ\(r < j_1^N-j_0^N\)であり、\(j'_0 = j_0^N+q(j_1^N-j_0^N)+r\)かつ\(j'_0 = j_0^N+q(j_1^N-j_0^N)+r'\)である。
					\item \((0,j'_0) \leq_M (0,j'_1)\)と\nameref{Pの各成分の非複項性}から\(j'_1 < j_0^N+(q+1)(j_1^N-j_0^N)\)であるので、\(r \leq r' < j_1^N-j_0^N\)である。
					\item \((0,j'_0) \leq_M (0,j'_1)\)かつ\((M_j)_{j=j_0^N+q(j_1^N-j_0^N)}^{j_0^N+(q+1)(j_1^N-j_0^N)-1} = (N_j)_{j=j_0^N}^{j_1^N-1}\)より\((0,j_0^N+r) \leq_N (0,j_0^N+r')\)である。
					\item \(M' = (N_j)_{j=j_0^N+r}^{j_0^N+r'}\)であるので、帰納法の仮定より\(\textrm{Br}(M')\)は降順である。
				\end{indented}
			\end{indented}
			\item
			\item \(N_{1,j_1^N} > 0\)とする。
			\begin{indented}
				\item \((1,j_{-2}^N) <_N^{\textrm{Next}} (1,j_1^N)\)を満たす一意な\(j_{-2}^N \in \mathbb{N}\)が存在しないとすると、\(M = N[n] = \textrm{Pred}(N)\)となるので、既に示したように\(\textrm{Br}(M')\)は降順である。
				\item \((1,j_{-2}^N) <_N^{\textrm{Next}} (1,j_1^N)\)を満たす一意な\(j_{-2}^N \in \mathbb{N}\)が存在するとする。
				\begin{indented}
					\item \(\delta := N_{0,j_1^N} - N_{0,j_{-2}^N}\)と置く。
					\item \((1,j_{-2}^N) <_N^{\textrm{Next}} (1,j_1^N)\)より\(\delta > 0\)であり、\(M = (N_j)_{j=0}^{j_{-2}^N-1} \oplus_{\mathbb{N}^2} \bigoplus_{\mathbb{N}^2} (\textrm{IncrFirst}^{k \delta}((N_j)_{j=j_{-2}^N}^{j_1^N}))_{k=0}^{n-1}\)である。
					\item
					\item 任意の\(k \in \mathbb{N}\)に対し、\(k < n-1\)ならば\((0,j_{-2}^N+k(j_1^N-j_{-2}^N)) \leq_M (0,j_{-2}^N+(k+1)(j_1^N-j_{-2}^N))\)であることを示す。
					\begin{indented}
						\item \((M_j)_{j =j_{-2}^N+k(j_1^N-j_{-2}^N)}^{j_{-2}^N+(k+1)(j_1^N-j_{-2}^N)} = \textrm{IncrFirst}^{k \delta}((N_j)_{j=j_{-2}^N}^{j_1^N-1}) \oplus_{\mathbb{N}^2} \textrm{IncrFirst}^{(k+1)\delta}((N_{j=j_{-2}^N}))\)である。更に\(M_{0,j_1^N} = M_{0,j_{-2}^N}+\delta\)より\((M_{0,j})_{j =j_{-2}^N+k(j_1^N-j_{-2}^N)}^{j_{-2}^N+(k+1)(j_1^N-j_{-2}^N)} = ((N_{0,j}+k \delta)_{j=j_{-2}^N}^{j_1^N})\)である。
						\item \((1,j_{-2}^N) <_N^{\textrm{Next}} (1,j_1^N)\)より\((0,j_{-2}^N) \leq_M (0,j_1^N)\)であるので、\((M_{0,j})_{j =j_{-2}^N+k(j_1^N-j_{-2}^N)}^{j_{-2}^N+(k+1)(j_1^N-j_{-2}^N)} = ((N_{0,j}+k \delta)_{j=j_{-2}^N}^{j_1^N})\)より\((0,j_{-2}^N+k(j_1^N-j_{-2}^N)) \leq_M (0,j_{-2}^N+(k+1)(j_1^N-j_{-2}^N))\)である。
					\end{indented}
					\item
					\item \((0,j_{-2}^N+(n-1)(j_1^N-j_{-2}^N)) \leq_M (0,j_1)\)であることを示す。
					\begin{indented}
						\item \((M_j)_{j =j_{-2}^N+(n-1)(j_1^N-j_{-2}^N)}^{j_1} = \textrm{IncrFirst}^{(n-1)\delta}((N_j)_{j=j_{-2}^N}^{j_1^N-1})\)であるので、\((M_{0,j})_{j =j_{-2}^N+(n-1)(j_1^N-j_{-2}^N)}^{j_1} = ((N_{0,j}+(n-1)\delta)_{j=j_{-2}^N}^{j_1^N-1})\)である。
						\item \((1,j_{-2}^N) <_N^{\textrm{Next}} (1,j_1^N)\)より\((0,j_{-2}^N) \leq_M (0,j_1^N)\)であり、\nameref{直系先祖の木構造} (1)より\((0,j_{-2}^N) \leq_M (0,j_1^N-1)\)であるので、\((M_{0,j})_{j =j_{-2}^N+(n-1)(j_1^N-j_{-2}^N)}^{j_1} = ((N_{0,j}+(n-1)\delta)_{j=j_{-2}^N}^{j_1^N-1})\)より\((0,j_{-2}^N+(n-1)(j_1^N-j_{-2}^N)) \leq_M (0,j_1)\)である。
					\end{indented}
					\item 以上より、\((0,j_{-2}^N) \leq_M (0,j_1)\)である。
					\item
					\item \(j'_1 \leq j_{-2}^N\)ならば、\((N_j)_{j=0}^{j_{-2}^N} = (M_j)_{j=0}^{j_{-2}^N}\)であるので\(M' = (N_j)_{j=j'_0}^{j'_1}\)となり、既に示したように\(\textrm{Br}(M')\)は降順である。
					\item \(j'_0 < j_{-2}^N < j'_1\)とする。
					\begin{indented}
						\item \(j_{-2}^N < j'_1 \leq j_1\)と\((0,j_{-2}^N) \leq_M (0,j_1)\)と\nameref{直系先祖の木構造} (1)から\((0,j_{-2}^N) \leq_M (0,j'_1)\)である。更に\(j'_0 < j_{-2}^N < j'_1\)と\((0,j'_0) \leq_M (0,j'_1)\)から、\((0,j'_0) \leq_M (0,j_{-2}^N)\)すなわち\((0,j'_0) \leq_N (0,j_{-2}^N)\)である。\((0,j'_0) \leq_N (0,j_{-2}^N)\)と\((1,j_{-2}^N) <_N^{\textrm{Next}} (1,j_1^N)\)より\((0,j'_0) \leq_N (0,j_1^N)\)であるので、帰納法の仮定から\(N'\)が単項かつ\(\textrm{Br}(N')\)は降順である。
						\item
						\item \(J_1 = -1\)とする。
						\begin{indented}
							\item \(j_{-2}^N = j_0^N = j_1^N-1\)かつ\(j'_1 = j_1\)となるので、\(M' = \textrm{Pred}(N) \oplus_{\mathbb{N}^2} ((N_{0,j_{-2}^N}+k \delta,N_{1,j_{-2}^N}))_{k=1}^{n-1}\)である。
							\item 従って\(\textrm{Lng}(\textrm{Br}(M')) = 1\)かつ\(\textrm{Br}(M') = (((N_{0,j_{-2}^N}+k \delta,N_{1,j_{-2}^N}))_{k=1}^{n-1})\)となり、\(\textrm{Br}(M')\)は降順である。
						\end{indented}
						\item
						\item \(J_1 \geq 0\)とする。
						\begin{indented}
							\item \(\textrm{TrMax}(N') < j_1^N-j'_0\)である。
							\item \(j_{-2}^N-j'_0 > 0\)であるので、\((1,j_{-2}^N) <_N^{\textrm{Next}} (1,j_1^N)\)より\((0,j_{-2}^N-j'_0) \leq_{N'} (0,j_1^N-j'_0)\)である。また\(N'\)の単項性より\((0,0) \leq_N (0,j_{-2}^N-j'_0)\)であり、\(j_{-2}^N-j'_0 > 0\)より\((0,j_{-3}) <_{N'}^{\textrm{Next}} (0,j_{-2}^N-j'_0)\)を満たす一意な\(j_{-3} \in \mathbb{N}\)が存在する。
							\item
							\item \(j_{-2}^N-j'_0 \leq \textrm{TrMax}(N')\)とする。
							\begin{indented}
								\item \(j_{-1} := \textrm{FirstNodes}(N')_{J_1}\)と置く。
								\item \((0,j_{-2}^N-j'_0) \leq_{N'} (0,j_1^N-j'_0)\)かつ\(j_{-2}^N-j'_0 \leq \textrm{TrMax}(N') < j_1^N-j'_0\)と\nameref{Pの各成分の非複項性}より\((0,j_{-2}^N-j'_0) <_{N'}^{\textrm{Next}} (0,j_{-1})\)であるので、\(\textrm{Br}(N')_{J_1} = (N_j)_{j=j_{-1}+j'_0}^{j_1^N}\)より\(\textrm{Br}(M') = (\textrm{Br}(N')_J)_{J=0}^{J_1-1} \oplus_{T_{\textrm{PS}}} ((M_j)_{j=j_{-1}+j'_0}^{j'_1})\)となる。
								\item 従って\(\textrm{Lng}(\textrm{Br}(M'))-1 = J_1\)かつ\(((\textrm{Br}(M')_J)_0)_{J=0}^{J_1} = ((\textrm{Br}(N')_J)_0)_{J=0}^{J_1-1} \oplus_{\mathbb{N}^2} (M_{j_{-1}})\)となる。\nameref{FirstNodesとTrMaxとJointsの関係}から\(j_{-1}+j'_0 = \textrm{FirstNodes}(N')_{J_1}+j'_0 \leq \textrm{TrMax}(N')+j'_0 < j_1^N\)となるので\(M_{j_{-1}} = N_{j_{-1}} = (\textrm{Br}(N')_{J_1})_0\)である。よって\(\textrm{Br}(M')\)は降順である。
							\end{indented}
							\item
							\item \(j_{-3} \leq \textrm{TrMax}(N') < j_{-2}^N-j'_0\)とする。
							\begin{indented}
								\item \((0,j_{-3}) <_{N'}^{\textrm{Next}} (0,j_{-2}^N-j'_0)\)と\nameref{Pの各成分の非複項性}より\(\textrm{FirstNodes}(N')_{J_1} = j_{-2}^N-j'_0\)かつ\(\textrm{Br}(N')_{J_1} = ((N_j)_{j=j_{-2}^N}^{j_1^N})\)であるので、\(\textrm{Br}(M') = (\textrm{Br}(N')_J)_{J=0}^{J_1-1} \oplus_{T_{\textrm{PS}}} ((N_j)_{j=j_0^N}^{j_1^N-1})_{k=1}^{n-2} \oplus_{T_{\textrm{PS}}} ((M_j)_{j=j_{-2}^N}^{j_1})\)となる。
								\item 従って\(\textrm{Lng}(\textrm{Br}(M'))-1 = J_1\)かつ\(((\textrm{Br}(M')_J)_0)_{J=0}^{J_1} = ((\textrm{Br}(N')_J)_0)_{J=0}^{J_1-1} \oplus_{\mathbb{N}^2} (N_{j_{-2}^N})\)となる。\(N_{j_{-2}^N} = (\textrm{Br}(N')_{J_1})_0\)より、\(\textrm{Br}(M')\)は降順である。
							\end{indented}
							\item
							\item \(\textrm{TrMax}(N') < j_{-3}\)とする。
							\begin{indented}
								\item \nameref{Pの各成分の非複項性}より\(\textrm{FirstNodes}(N')_{J_1} \leq j_{-3}\)であるので、\(\textrm{Br}(M') = (\textrm{Br}(N')_J)_{J=0}^{J_1-1} \oplus_{T_{\textrm{PS}}} ((M_j)_{j = \textrm{FirstNodes}(N')_{J_1}+j'_0}^{j'_1})\)となる。
								\item 従って\(\textrm{Lng}(\textrm{Br}(M'))-1 = J_1\)かつ\(((\textrm{Br}(M')_J)_0)_{J=0}^{J_1} = ((\textrm{Br}(N')_J)_0)_{J=0}^{J_1-1} \oplus_{\mathbb{N}^2} (M_{\textrm{FirstNodes}(N')_{J_1}})\)となる。\(\textrm{FirstNodes}(N')_{J_1}+j'_0 \leq j_{-3}+j'_0 \leq j_{-2}+j'_0 < j_1^N\)より\(M_{\textrm{FirstNodes}(N')_{J_1}+j'_0} = N_{\textrm{FirstNodes}(N')_{J_1}+j'_0} = (\textrm{Br}(N')_{J_1})_0\)より、\(\textrm{Br}(M')\)は降順である。
							\end{indented}
						\end{indented}
					\end{indented}
					\item
					\item \(j_{-2}^N \leq j'_0\)とする。
					\begin{indented}
						\item \(j'_0-j_{-2}^N\)を\(j_1^N-j_{-2}^N\)で割った商と余りをそれぞれ\(q,r \in \mathbb{N}\)と置く。
						\item \(q < n\)かつ\(r < j_1^N-j_0^N\)であり、\(j'_0 = j_0^N+q(j_1^N-j_0^N)+r\)である。
						\item \((N[n-q]_j)_{j=j'_{-2}+r}^{j'_{-2}+(n-q)(j_1^N-j_{-2}^N)-1} = (M_j)_{j=j'_0}^{j_1}\)と\((0,j'_0) \leq_M (0,j'_1)\)より、\((0,j'_{-2}+r) \leq_{N[n-q]} (0,j'_1-q(j_1^N-j_{-2}^N))\)である。\((N[n-q]_j)_{j=j'_{-2}+r}^{j'_1-q(j_1^N-j_{-2}^N)} = M'\)より、\(q = 0\)として良い。
						\item \(j'_1 < j_1^N\)ならば、\((M_j)_{j=0}^{j_1^N-1} = \textrm{Pred}(N)\)より\(M' = (N_j)_{j=j'_0}^{j'_1}\)であり、\((0,j'_0) \leq_M (0,j'_1)\)より\((0,j'_0) \leq_N (0,j'_1)\)となるので、帰納法の仮定より\(\textrm{Br}(M')\)は降順である。
						\item \(j'_1 \geq j_1^N\)とする。
						\begin{indented}
							\item 任意の\(k \in \mathbb{N}\)に対し\(k < n-1\)ならば\((0,j_0^N+k(j_1^N-j_0^N)) \leq_M (0,j_0^N+(k+1)(j_1^N-j_0^N))\)であり、かつ\((0,j_0^N+(n-1)(j_1^N-j_0^N)) \leq_M (0,j_1)\)であるので、\((0,j_1^N) = (0,j_0^N+(j_1^N-j_0^N)) \leq_M (0,j_1)\)である。従って\nameref{直系先祖の木構造} (1)から\((0,j_1^N) \leq_M (0,j'_1)\)となる。
							\item \((0,j'_0) \leq_M (0,j'_1)\)かつ\(j'_0 = j_0^N+r < j_1^N\)かつ\((0,j_1^N) \leq_M (0,j'_1)\)より\((0,j'_0) \leq_M (0,j_1^N)\)である。更に\((M_j)_{j=0}^{j_1^N} = (N_j)_{j=0}^{j_1^N-1} \oplus_{\mathbb{N}^2} ((N_{0,j_0^N}+\delta,N_{1,j_0^N})) = (N_j)_{j=0}^{j_1^N-1} \oplus_{\mathbb{N}^2} ((N_{0,j_1^N},N_{1,j_0^N}))\)であるので\((0,j'_0) \leq_N (0,j_1^N)\)となる。
							\item 従って帰納法の仮定から\(N'\)が単項かつ\(\textrm{Br}(N')\)は降順である。また\(j'_0 < j_1^N\)かつ\((0,j'_0) \leq_N (0,j_1^N)\)かつ\((0,j_0^N) <_N^{\textrm{Next}} (0,j_1^N)\)から\((0,j'_0) \leq_M (0,j_0^N)\)となる。
							\item
							\item \(j_1^N-j'_0 \leq \textrm{TrMax}(N')\)とする。
							\begin{indented}
								\item \(j'_0 \leq j_0^N < j_1^N \leq \textrm{TrMax}(N')+j'_0\)より\((1,j_1^N-1) <_N^{\textrm{Next}} (1,j_1^N)\)となるので、\(j_{-2}^N = j_0^N = j_1^N-1\)である。また\(j_{-2}^N \leq j'_0 \leq j_0^N\)より\(j'_0 = j_0^N\)となる。\(M = N[n] = \textrm{Pred}(N) \oplus_{\mathbb{N}^2} ((N_{0,j_{-2}^N}+k \delta,N_{1,j_{-2}^N}))_{k=1}^{n-1}\)であるので、\(M' = ((N_{0,j_{-2}^N}+k \delta,N_{1,j_{-2}^N}))_{k=0}^{n-1}\)である。
								\item 従って\(\textrm{Lng}(\textrm{Br}(M'))-1 = 0\)かつ\(\textrm{Br}(M') = (((N_{0,j_{-2}^N}+k \delta,N_{1,j_{-2}^N}))_{k=1}^{j'_1-j'_0})\)であり、\(\textrm{Br}(M')\)は降順である。
							\end{indented}
							\item
							\item \(j_0^N-j'_0 \leq \textrm{TrMax}(N') < j_1^N-j'_0\)とする。
							\begin{indented}
								\item \nameref{Pの各成分の非複項性}より\(\textrm{FirstNodes}(N')_{J_1} = j_1^N-j'_0 \leq j'_1-j'_0\)である。\((M_j)_{j=0}^{j_1^N} = \textrm{Pred}(N) \oplus_{\mathbb{N}^2} ((N_{0,j_0^N}+\delta,N_{1,j_0^N})) = \textrm{Pred}(N) \oplus_{\mathbb{N}^2} ((N_{0,j_1^N},N_{1,j_{-2}^N}))\)であるので、\(\textrm{Br}(M') = (\textrm{Br}(N')_J)_{J=0}^{J_1-1} \oplus_{T_{\textrm{PS}}} ((M_j)_{j=j_1^N}^{j_1})\)である。
								\item 従って\(\textrm{Lng}(\textrm{Br}(M'))-1 = J_1\)かつ\(((\textrm{Br}(M')_J)_0)_{J=0}^{J_1} = ((\textrm{Br}(N')_J)_0)_{J=0}^{J_1-1} \oplus_{\mathbb{N}^2} (M_{j_1^N})\)となる。\(M_{0,j_1^N} = N_{0,j_1^N} = N_{0,\textrm{FirstNodes}(N')_{J_1}+j'_0} = (\textrm{Br}(N')_{J_1})_{0,0}\)かつ\(M_{1,j_1^N} = N_{1,j_{-2}^N} < N_{1,j_1^N} = N_{1,\textrm{FirstNodes}(N')_{J_1}+j'_0} = (\textrm{Br}(N')_{J_1})_{1,0}\)となるので、\(\textrm{Br}(M')\)は降順である。
							\end{indented}
							\item
							\item \(\textrm{TrMax}(N') < j_0^N-j'_0\)とする。
							\begin{indented}
								\item \(j_{-1} := \textrm{FirstNodes}(N')_{J_1}\)と置く。
								\item \nameref{Pの各成分の非複項性}より\(j_{-1} \leq j_0^N-j'_0 < j_1^N-j'_0 \leq j'_1-j'_0\)である。\((M_j)_{j=0}^{j_{-1}+j'_0} = (N_j)_{j=0}^{j_{-1}+j'_0}\)であるので、\(\textrm{Br}(M') = (\textrm{Br}(N')_J)_{J=0}^{J_1-1} \oplus_{T_{\textrm{PS}}} ((M_j)_{j=j_{-1}+j'_0}^{j_1})\)である。
								\item 従って\(\textrm{Lng}(\textrm{Br}(M'))-1 = J_1\)かつ\(((\textrm{Br}(M')_J)_0)_{J=0}^{J_1} = ((\textrm{Br}(N')_J)_0)_{J=0}^{J_1-1} \oplus_{\mathbb{N}^2} (M_{j'_{-1}+j'_0})\)となる。\nameref{FirstNodesとTrMaxとJointsの関係}から\(j_{-1}+j'_0 = \textrm{FirstNodes}(N')_{J_1}+j'_0 \leq \textrm{TrMax}(N')+j'_0 < j_0^N < j_1^N\)であるので、\(M_{j_{-1}+j'_0} = N_{j_{-1}+j'_0} = N_{\textrm{FirstNodes}(N')_{J_1}+j'_0} = (\textrm{Br}(N')_{J_1})_0\)である。よって\(\textrm{Br}(M')\)は降順である。
							\end{indented}
						\end{indented}
					\end{indented}
				\end{indented}
			\end{indented}
		\end{indented}
	\end{indented}
\end{hideableproof}

\begin{proposition}[標準形の単項成分が降順であること]\label{標準形の単項成分が降順であること}
	任意の\(M \in ST_{\textrm{PS}}\)と\(J'_0,J'_1 \in \mathbb{N}\)に対し、\(J_1 := \textrm{Lng}(P(M))-1\)と置くと、\(J'_0 \leq J'_1 \leq J_1\)かつ\((P(M)_{J'_0})_{0,0} = (P(M)_{J'_1})_{0,0}\)ならば、\((P(M)_{J'_0})_{1,0} \geq (P(M)_{J'_1})_{1,0}\)である。
\end{proposition}

\begin{hideableproof}
	\begin{indented}
		\item \(k_0 := \min \{k \in \mathbb{N} \mid M \in S_kT_{\textrm{PS}}\}\)と置く\footnote{\(ST_{\textrm{PS}} = \bigcup_{k \in \mathbb{N}} S_kT_{\textrm{PS}}\)より\(\min\)は存在する。}。\(J_1 \geq J'_1 > J'_0 \geq 0\)より\nameref{Pの各成分の非複項性} (2)から\(M\)は複項であるので、\(k_0 > 0\)である。従ってある\(M' \in S_{k_0-1}T_{\textrm{PS}}\)と\(n \in \mathbb{N}_{+}\)が存在して\(M = M'[n]\)となる。
		\item \(M\)が複項であることと\nameref{非複項性と基本列の関係}から、\(k_0-1 > 0\)すなわち\(k_0 > 1\)かつ\(n > 1\)である。
		\item 従ってある\(N \in S_{k_0-2}T_{\textrm{PS}}\)と\(n' \in \mathbb{N}_{+}\)が存在して\(M' = N[n']\)となる。
		\item \(k_0\)に関する数学的帰納法で示す。
		\item \(k_0 = 2\)とする。
		\begin{indented}
			\item \(N \in S_{k_0-2}T_{\textrm{PS}} = S_0T_{\textrm{PS}}\)より、ある\(u,v \in \mathbb{N}\)が存在して\(u \leq v\)かつ\(N = ((j,j))_{j=u}^{v}\)である。
			\item \(u = v\)と仮定すると、\(M' = N[n'] = ((u,u))[n'] = ((u,u))\)かつ\(M = M'[n] = ((u,u))[n] = ((u,u))\)となり、\(M\)の複項性に反する。
			\item \(u+1 < v\)と仮定する。
			\begin{indented}
				\item \(M' = N[n'] = ((j,j))_{j=u}^{v-2} \oplus_{\mathbb{N}} ((v-1+k,v-1))_{k=0}^{n'-1}\)となる。
				\item \(n' = 1\)ならば\(M = M'[n] = (((j,j))_{j=u}^{v-2} \oplus_{\mathbb{N}} ((v-1,v-1)))[n] = ((j,j))_{j=u}^{v-3} \oplus_{\mathbb{N}} ((v-2+k,v-2))_{k=0}^{n-1}\)となるので\((M_{0,j})_{j=0}^{j_1} = (j)_{j=u}^{v+n-3}\)である。
				\item \(n' > 1\)ならば\(M = M'[n] = (((j,j))_{j=u}^{v-2} \oplus_{\mathbb{N}} ((v-1+k,v-1))_{k=0}^{n'-1})[n] = ((j,j))_{j=u}^{v-3} \oplus_{\mathbb{N}} \bigoplus_{\mathbb{N}^2} ((v-2+k'n',v-2) \oplus_{\mathbb{N}^2} ((v-1+k+k'n',v-1))_{k=0}^{n'-2})_{k'=0}^{n-1}\)となるので\((M_{0,j})_{j=0}^{j_1} = (j)_{j=u}^{v+nn'-3}\)である。
				\item いずれの場合も\nameref{複項性の判定条件}から\(M\)の複項性に反する。
			\end{indented}
			\item 以上より\(u+1 = v\)となり、\(M' = N[n'] = ((u+k,u))_{k=0}^{n'-1}\)となる。
			\item \(n' = 1\)と仮定すると、\(M = M'[n] = ((u,u))[n] = ((u,u))\)となり、\(M\)の複項性に反する。従って\(n' > 1\)である。
			\item \(u > 0\)ならば\(M = M'[n] = ((u+k,u))_{k=0}^{n'-1}[n] = ((u+k,u))_{k=0}^{n'-2}\)となり、\nameref{複項性の判定条件}から\(M\)の複項性に反する。従って\(u = 0\)であり、\(M = M'[n] = ((k,0))_{k=0}^{n'-1}[n] = ((k,0))_{k=0}^{n'-2} \oplus_{\mathbb{N}^2} ((n'-1,0))_{k'=0}^{n-1}\)である。
			\item \(n' > 2\)と仮定すると、\((0,0) \leq_M (0,j_1)\)より\nameref{複項性の判定条件}から\(M\)の複項性に反する。従って\(n' = 2\)であり、\(M = ((n'-1,0))_{k'=0}^{n-1}\)である。\(P(M) = (((n'-1,0)))_{k'=0}^{n-1}\)であるので、\((P(M)_{J'_0})_{1,0} = 0 = (P(M)_{J'_1})_{1,0}\)である。
		\end{indented}
		\item \(k_0 > 3\)とする。
		\begin{indented}
			\item \(J_0 := \textrm{Lng}(P(M'))-1\)と置く。
			\item \(\textrm{Lng}(P(M')_{J_0}) = 1\)ならば、\nameref{Pと基本列の関係}と\nameref{Pの各成分の非複項性} (2)から\(J_0 > 0\)かつ\(P(M) = (P(M')_J)_{J=0}^{J_0-1}\)であるので、帰納法の仮定から従う。
			\item \(\textrm{Lng}(P(M')_{J_0}) > 1\)とする。
			\begin{indented}
				\item \nameref{Pと基本列の関係}から\(P(M) = (P(M')_J)_{J=0}^{J_0-1} \oplus_{T_{\textrm{PS}}} P(P(M')_{J_0}[n])\)である。
				\item \(J_{-1} := \textrm{Lng}(P(P(M')_{J_0}))-1\)と置く。
				\item \nameref{Pの各成分の非複項性} (1)から\(P(M')_{J_0}\)は複項でなく、\nameref{非複項性と基本列の関係}から\(((P(P(M')_{J_0}[n])_J)_{0,0})_{J=0}^{J_{-1}}\)は\((P(M')_{J_0})_{0,0}\)のみを成分に持つ。
				\item \(J'_1 < J_0\)ならば、帰納法の仮定から\((P(M)_{J'_0})_{0,0} = (P(M')_{J'_0})_{0,0} \geq (P(M')_{J'_1})_{0,0} = (P(M)_{J'_1})_{0,0}\)である。
				\item \(J'_0 < J_0 \leq J'_1\)ならば、帰納法の仮定から\((P(M)_{J'_0})_{0,0} = (P(M')_{J'_0})_{0,0} \geq (P(M')_{J_0})_{0,0} = (P(P(M')_{J_0}[n])_{J'_1-J_0})_{0,0} = (P(M)_{J'_1})_{0,0}\)である。
				\item \(J_0 \leq J'_0\)ならば、\((P(M)_{J'_0})_{0,0} = (P(M')_{J_0})_{0,0} = (P(M)_{J'_1})_{0,0}\)である。
			\end{indented}
		\end{indented}
	\end{indented}
\end{hideableproof}


\section{Buchholzの表記系への翻訳}

後に定義する標準形ペア数列システムの停止性を証明するための準備として、ペア数列からBuchholzの表記系への翻訳写像\(\textrm{Trans}\)を定め、その性質を調べる。


\subsection{Buchholzの表記系}

以下では\(T_{\textrm{B}}\)は\(D_{\omega}\)を含まない項全体のなすBuchholozの表記系\(T\)\footnote{\cite{buc1} p. 200参照。}の部分集合を表す。\(T_{\textrm{B}}\)は順序数項\footnote{\cite{buc1} p. 201参照。}とは限らないことに注意する。最終的に用いるのは順序数項であるが、最初から順序数項に制限すると議論の過程で現れる各項が順序数項であるか否かを逐一判定する必要があるので非常に冗長となる。従って標準形ペアシステムの停止性を証明する直前の段階までは順序数項に制限せず議論する。

\(T_{\textrm{B}}\)の要素間の関係\(<\)は通常の強順序\footnote{\cite{buc1} p. 200参照。}を表し、\(T_{\textrm{B}}\)の要素間の関係\(\leq\)は「\(<\)または\(=\)」の略記を表し、\(T_{\textrm{B}}\)の要素間の演算\(+\)は加法\footnote{\cite{buc1} p. 203参照。}を表し、\(n \in \mathbb{N}\)に対し\(T_{\textrm{B}}\)の要素への\(\times n\)は\(n\)倍\footnote{\cite{buc1} p. 203参照。}を表し、\(T_{\textrm{B}}\)の要素に対する\([]\)演算子は\cite{buc1} pp. 203--204の\([]\)演算子の再帰的定義のうち([].4) (ii)のみ\cite{buc2} p. 6のDefinitionの6の規則に変えたもの\footnote{すなわち\cite{buc1} ([].4) (ii)の場合分けにおいて、各\(i \in \mathbb{N}\)に対し\(x_i\)を「\(i = 0\)ならば\(x_i = D_u 0\)、\(i > 0\)ならば\(x_i = b[D_u x_{i-1}]\)」と定め、\(a[n]\)の定義を\(D_v b[x_n]\)に変えるということである。}として得られる基本列\footnote{\cite{buc1}の\([]\)演算子の再帰的定義は基本的に基本列を再帰的に翻訳したものだが、([].4) (ii)の場合分けは基本列の一部だけを再帰的に翻訳したものであるため、そこだけ変更する必要がある。}を表し、\(\textrm{dom}\)は各項ごとに\([]\)演算子の定義域\footnote{\cite{buc1} pp. 203--204参照。}を表す。

\(0\)でも単項\footnote{\cite{buc1} p. 200 (T2)参照。}でもないBuchholzの表記系の項を複項と呼ぶ。\(PT_{\textrm{B}} \subset T_{\textrm{B}}\)で\(D_{\omega}\)を含まない単項全体のなす部分集合を表し、\(MT_{\textrm{B}} \subset T_{\textrm{B}}\)で\(D_{\omega}\)を含まない複項全体のなす部分集合を表す。

\(\underline{(}\)で\(T_{\textrm{B}}\)における字母\(\textrm{(}\)を表し、\(\underline{,}\)で\(T_{\textrm{B}}\)における字母\(\textrm{,}\)を表し、\(\underline{)}\)で\(T_{\textrm{B}}\)における字母\(\textrm{)}\)を表す。\(T_{\textrm{B}}\)における字母\(\underline{(}\)と\(\underline{,}\)と\(\underline{)}\)と\(0\)と各\(u \in \omega+1\)に対する\(D_u\)全体の集合を\(\Sigma\)と置く。

\(t \in PT_{\textrm{B}}^{< \omega}\)とする。
\begin{nenumerate}
	\item \(t = ()\)ならば\(t' := 0 \in \Sigma\)と置く。
	\item \(t \neq ()\)とする。
	\begin{nenumerate}
		\item \(j_1 := \textrm{Lng}(t)-1\)と置く。
		\item \(s_0 := \underline{(}\)と置く。
		\item \(0 < j < j_1\)を満たす各\(j \in \mathbb{N}\)に対して\(s_j := s_{j-1} t_{j-1} \underline{,}\)と置く。
		\item \(t' := s_{j_1-1} t_{j_1} \underline{)} \in \Sigma\)と置く。
	\end{nenumerate}
\end{nenumerate}
\(j_1 = -1\)ならば\(t' = 0 \in T_{\textrm{B}}\)である。\(j_1 = 0\)ならば\(t' = \underline{(} t_0 \underline{)}\)より縮約規則\(\{(p) = p \mid p \in PT_{\textrm{B}}\}\)の下で\(t'\)は\(t_0 \in PT_{\textrm{B}}\)と同一視される。\(j_1 > 0\)ならば\(t' \in T_{\textrm{B}} \setminus (\{0\} \cup PT_{\textrm{B}})\)である。従っていずれの場合も\(T_{\textrm{B}}\)の項を定め、それを\(\Sigma_{\textrm{B}} t\)と表記する。

写像
\begin{eqnarray*}
P \colon T_{\textrm{B}} & \to & PT_{\textrm{B}}^{< \omega} \\
t & \mapsto & P(t)
\end{eqnarray*}
を以下のように定める:
\begin{nenumerate}
	\item \(t = 0\)ならば\(P(t) := ()\)である。
	\item \(t \in PT_{\textrm{B}}\)ならば\(P(t) := (t)\)である。
	\item \(t \in MT_{\textrm{B}}\)とする。
	\begin{nenumerate}
		\item Buchholzの表記系の再帰的定義より、一意な\(s \in \Sigma^{<\omega}\)と\(t' \in PT_{\textrm{PS}}\)が存在して以下を満たす:
		\begin{nenumerate}
			\item \(t = \underline{(} s \underline{,} t' \underline{)}\)である。
			\item \(s \in PT_{\textrm{B}}\)または\(\underline{(} s \underline{)} \in MT_{\textrm{B}}\)である。
		\end{nenumerate}
		\item \(s \in PT_{\textrm{B}}\)ならば\(P(t) := P(s) \oplus_{PT_{\textrm{B}}} (t')\)である。
		\item \(\underline{(} s \underline{)} \in MT_{\textrm{B}}\)ならば\(P(t) := P(\underline{(} s \underline{)}) \oplus_{PT_{\textrm{B}}} (t')\)である。
	\end{nenumerate}
\end{nenumerate}

\(t \in T_{\textrm{B}}\)に対し、\(P(t)\)の各成分を\(t\)の単項成分と呼ぶ。

\begin{proposition}[順序数項のカッコの個数が左右で等しいこと]\label{順序数項のカッコの個数が左右で等しいこと}
	任意の\(t \in T_{\textrm{B}}\)に対し、\(t\)に出現する\(\underline{(}\)の個数と\(t\)に出現する\(\underline{)}\)の個数は等しい。
\end{proposition}

\begin{hideableproof}
	\begin{indented}
		\item Buchholzの表記系の再帰的定義より、\(\textrm{Lng}(t)\)に関する数学的帰納法から即座に従う。
	\end{indented}
\end{hideableproof}

\begin{proposition}[順序数項の単項成分の基本性質]\label{順序数項の単項成分の基本性質}
	任意の\(t \in T_{\textrm{B}}\)に対し、\(J_1 := \textrm{Lng}(P(t))-1\)と置くと以下が成り立つ:
	\begin{penumerate}
		\item \(J_1 = -1\)である必要十分条件は\(t = 0\)である。
		\item \(t = \Sigma_{\textrm{B}} (P(t)_J)_{J=0}^{J_1}\)である。
	\end{penumerate}
\end{proposition}

\begin{hideableproof}
	\begin{indented}
		\item \(P\)の再帰的定義より、\(\textrm{Lng}(t)\)に関する数学的帰納法から即座に従う。
	\end{indented}
\end{hideableproof}

\begin{proposition}[部分表現の不等式の延長性]\label{部分表現の不等式の延長性}
	任意の\(s,b \in \Sigma^{< \omega}\)と\(t_0, t_1 \in T_{\textrm{B}}\)に対し、\(s t_0 b \in T_{\textrm{B}}\)かつ\(s t_1 b \in T_{\textrm{B}}\)ならば、以下は同値である:
	\begin{penumerate}
		\item \(t_0 < t_1\)である。
		\item \(s t_0 b < s t_1 b\)である。
	\end{penumerate}
\end{proposition}

\begin{hideableproof}
	\begin{indented}
		\item \(<\)の再帰的定義より、\(\textrm{Lng}(s)\)に関する数学的帰納法から従う。
	\end{indented}
\end{hideableproof}


\subsection{scb分解}

写像
\begin{eqnarray*}
\textrm{RightNodes} \colon T_{\textrm{B}} & \to & \mathbb{N}^{< \omega} \\
t & \mapsto & \textrm{RightNodes}(t)
\end{eqnarray*}
を以下のように再帰的に定める:
\begin{nenumerate}
	\item \(t = 0\)ならば\(\textrm{RightNodes}(t) := ()\)である。
	\item \(t \in PT_{\textrm{B}}\)とする。
	\begin{nenumerate}
		\item \(u \in \mathbb{N}\)と\(t' \in T_{\textrm{B}}\)を用いて\(t = D_u t'\)と置く。
		\item \(\textrm{RightNodes}(t) := (u) \oplus_{\mathbb{N}} \textrm{RightNodes}(t')\)である。
	\end{nenumerate}
	\item \(t \in T_{\textrm{B}} \setminus (\{0\} \cup PT_{\textrm{B}})\)とする。
	\begin{nenumerate}
		\item \(J_1 := \textrm{Lng}(P(t)) - 1\)と置く。
		\item \(\textrm{RightNodes}(t) := \textrm{RightNodes}(P(t)_{J_1})\)である。
	\end{nenumerate}
\end{nenumerate}

順序数項の不等式や加法や基本列の計算補助をするために、文字列の構文情報を与えるscb分解という概念を導入する。

\(t \in T_{\textrm{B}}\)とし、\((s,c,b) \in (\Sigma^{< \omega})^3\)とする。
\begin{nenumerate}
	\item \((s,c,b)\)が\(t\)のscb分解であるとは、以下を満たすということである:
	\begin{nenumerate}
		\item \(t = scb\)である。
		\item \(t \neq 0\)ならば\(c \in PT_{\textrm{B}}\)である。
		\item \(b\)は\(\underline{)}\)のみからなる文字列である。
	\end{nenumerate}
	\item \((s,c,b)\)が\(t\)の第\(0\)種scb分解であるとは、以下を満たすということである:
	\begin{nenumerate}
		\item \((s,c,b)\)は\(t\)のscb分解である。
		\item \(\textrm{Lng}(\textrm{RightNodes}(c)) = 2\)である。
		\item \(\textrm{RightNodes}(c)_1 = 0\)である。
	\end{nenumerate}
	\item \((s,c,b)\)が\(t\)の第\(1\)種scb分解であるとは、以下を満たすということである:
	\begin{nenumerate}
		\item \((s,c,b)\)は\(t\)のscb分解である。
		\item \(j_1 := \textrm{Lng}(\textrm{RightNodes}(c))-1\)と置くと\(j_1 \geq 1\)である。
		\item \(\textrm{RightNodes}(c)_0 < \textrm{RightNodes}(c)_{j_1}\)である。
		\item 任意の\(j \in \mathbb{N}\)に対し、\(0 < j < j_1\)ならば\(\textrm{RightNodes}(c)_j \geq \textrm{RightNodes}(c)_{j_1}\)である。
	\end{nenumerate}
\end{nenumerate}

\begin{proposition}[scb分解の置換可能性]\label{scb分解の置換可能性}
	任意の\(s,b \in \Sigma^{< \omega}\)と\(c_0, c_1 \in T_{\textrm{B}}\)に対し、「\(c_0\)が単項でないまたは\(c_1\)が単項である」かつ\(s c_0 b \in T_{\textrm{B}}\)かつ\((s,c_0,b)\)が\(s c_0 b\)のscb分解であるならば、\(s c_1 b \in T_{\textrm{B}}\)かつ\((s,c_1,b)\)は\(s c_1 b\)のscb分解である。
\end{proposition}

\begin{hideableproof}
	\begin{indented}
		\item \(s c_1 b \in T_{\textrm{B}}\)であることはBuchholzの表記系の再帰的定義から\(\textrm{Lng}(s)\)に関する数学的帰納法より即座に従う。\((s,c_1,b)\)が\(s c_1 b\)のscb分解であることはscb分解の定義より即座に従う。
	\end{indented}
\end{hideableproof}

\begin{proposition}[scb分解の合成則]\label{scb分解の合成則}
	任意の\(t \in T_{\textrm{B}}\)に対し、以下が成り立つ:
	\begin{penumerate}
		\item 任意の\(c_0 \in PT_{\textrm{B}}\)と\(s_0,s_1,c_1,b_1,b_0 \in \Sigma^{< \omega}\)に対し、\((s_0,c_0,b_0)\)が\(t\)のscb分解でかつ\((s_1,c_1,b_1)\)が\(c_0\)のscb分解ならば、\((s_0 s_1,c_1,b_1 b_0)\)は\(t\)のscb分解である。
		\item 任意の\(v \in \mathbb{N}\)と\(s,c,b \in \Sigma^{< \omega}\)に対し、\((s,c,b)\)が\(t\)のscb分解であるならば\((D_v s,c,b)\)は\(D_v t\)のscb分解である。
	\end{penumerate}
\end{proposition}

\begin{hideableproof}
	\begin{indented}
		\item scb分解の定義より即座に従う。
	\end{indented}
\end{hideableproof}

\((s,c,b)\)が\(t\)のscb分解となるような\((s,b) \in (\Sigma^{< \omega})\)が存在する\((t,c) \in T_{\textrm{B}}^2\)全体のなす部分集合を\(T_{\textrm{B}}^{\textrm{Marked}} \subset T_{\textrm{B}}^2\)と置く。

\begin{proposition}[scb分解の自明性の判定条件]\label{scb分解の自明性の判定条件}
	任意の\((t,c) \in T_{\textrm{B}}^{\textrm{Marked}}\)に対し、以下が同値である:
	\begin{penumerate}
		\item \(t = c\)である。
		\item 任意の\((s,b) \in (\Sigma^{< \omega})^2\)に対し、\((s,c,b)\)が\(t\)のscb分解であるならば\(s = ()\)かつ\(b = ()\)である。
		\item ある\(b \in \Sigma^{< \omega}\)が存在し、\(((),c,b)\)が\(t\)のscb分解である。
	\end{penumerate}
\end{proposition}

\begin{hideableproof}
	\begin{indented}
		\item (1)が成り立つならば、\(\textrm{Lng}(s) + \textrm{Lng}(b) = (\textrm{Lng}(s) + \textrm{Lng}(c) + \textrm{Lng}(b)) - \textrm{Lng}(b) = \textrm{Lng}(t) - \textrm{Lng}(c) = 0\)より\(s = ()\)かつ\(b = ()\)となり、(2)が成り立つ。
		\item (2)が成り立つならば、\((t,c) \in T_{\textrm{B}}^{\textrm{Marked}}\)より\((s,c,b)\)が\(t\)のscb分解となる\((s,b) \in (\Sigma^{< \omega})^2\)が存在し、(2)より\(s = ()\)かつ\(b = ()\)となるので\(t = scb = c\)であり、(1)と(3)が成り立つ。
		\item (3)が成り立つならば、\nameref{順序数項のカッコの個数が左右で等しいこと}と\(t = cb\)から\(b = ()\)となるので\(t = c\)であり、(1)が成り立つ。
	\end{indented}
\end{hideableproof}

\(t \in T_{\textrm{B}}\)とする。
\begin{nenumerate}
	\item \(t\)が第\(0\)種scb分解可能であるとは、\(t\)の第\(0\)種scb分解が存在するということである。
	\item \(t\)が第\(1\)種scb分解可能であるとは、\(t\)の第\(1\)種scb分解が存在するということである。
\end{nenumerate}

\begin{proposition}[scb分解の一意性]\label{scb分解の一意性}
	任意の\(t \in T_{\textrm{B}}\)に対し、以下が成り立つ:
	\begin{penumerate}
		\item 任意の\((s_0,s_1,c,b_0,b_1) \in (\Sigma^{< \omega})^5\)に対し、\((s_0,c,b_0)\)と\((s_1,c,b_1)\)が\(t\)のscb分解であるならば、\(s_0 = s_1\)かつ\(b_0 = b_1\)である。
		\item \(\textrm{dom}(t) = \mathbb{N}\)である必要十分条件は、\(t\)が第\(0\)種scb分解可能または第\(1\)種scb分解可能であることである。
		\item \(t\)は第\(0\)種scb分解可能でないかまたは\(t\)は第\(1\)種scb分解可能でない。
		\item \(t\)の第\(0\)種scb分解は一意である。
		\item \(t\)の第\(1\)種scb分解は一意である。
	\end{penumerate}
\end{proposition}

\begin{hideableproof}
	\begin{indented}
		\item (1)は\(c\)が単項であり単項は\(\underline{)}\)以外の文字を含むことから即座に従う。
		\item (2)は\(t\)が\(D_{\omega}\)を含まないことから\(\textrm{dom}\)の再帰的定義より即座に従う。
		\item (4)を示す。
		\begin{indented}
			\item \((s_0,c_0,b_0)\)と\((s_1,c_1,b_1)\)が\(t\)の第\(0\)種scb分解であると仮定する。
			\item \(j_1 := \textrm{Lng}(\textrm{Rightnodes}(t))-1\)と置く。仮定から\(t\)はscb分解可能であるため、\(t \neq 0\)であり\(j_1 \geq 0\)である。
			\item \(u := \textrm{RightNodes}(t)_{j_1}\)と置く。\(\textrm{RightNodes}\)の再帰的定義から、\(t\)から\(\underline{)}\)を除いた文字列の末尾\(2\)文字は\(D_u 0\)である。
			\item \(i \in \{0,1\}\)とする。\(j_{1,i} := \textrm{Lng}(\textrm{RightNodes}(c_i))-1\)と置く。scb分解の定義から、\(j_{1,i} \geq 1\)である。
			\item \(u_i := \textrm{RightNodes}(c_i)_{j_{1,i}}\)と置く。\(\textrm{RightNodes}\)の再帰的定義から、\(c_i\)から\(\underline{)}\)を除いた文字列の末尾\(2\)文字は\(D_{u_i} 0\)である。
			\item \(b_i\)は\(\underline{)}\)のみからなる文字列であるため、\(t\)から\(\underline{)}\)を除いた文字列の末尾\(2\)文字と\(c_i\)から\(\underline{)}\)を除いた文字列の末尾\(2\)文字は等しく、\(u = u_i\)である。従って\(u_0 = u_1\)である。
			\item \(v_i := \textrm{RightNodes}(c_i)_0\)と置く。\(b_i\)は\(\underline{)}\)のみからなる文字列でありかつ\(c_i\)は\(t\)に含まれる項であるため、\(j_{1,i} \leq j_1\)かつ\(\textrm{RightNodes}(c_i) = (\textrm{RightNodes}(t)_j)_{j=j_1-j_{1,i}}^{j_1}\)である。従って\(v_i = \textrm{RightNodes}(t)_{j_1-j_{1,i}} < u\)かつ任意の\(j \in \mathbb{N}\)に対して\(j_1-j_{1,i} < j < j_1\)ならば\(\textrm{RightNodes}(t)_j \geq u\)である。以上より\(j_1-j_{1,0} = j_1-j_{1,1}\)かつ\(v_0 = v_1\)である。
			\item \(j_0 := j_1 - j_{1,0}\)と置く。\(\textrm{RightNodes}(t)_{j_0} = v_0\)より、\(t\)は\(D_{v_0}\)を含む。\(t\)に出現する\(D_{v_0}\)のうちもっとも末尾に近いものより左側の文字列を\(s\)と置くと、\(c_i\)の先頭が\(D_{v_0}\)でかつ\(\textrm{RightNodes}(c_i) = (\textrm{RightNodes}(t)_j)_{j=j_0}^{j_1}\)であることから\(s = s_i\)である。従って\(s_0 = s_1\)である。
			\item \(c_i\)が項であることから、\(c_i\)の末尾に\(1\)個以上の\(\underline{)}\)を結合した文字列は項でない。更に\(b_i\)が\(\underline{)}\)のみからなることから、\(t\)の先頭から\(s\)を除いた文字列の部分文字列であって項であるもののうち最大のものが\(c_i\)である。従って\(c_0 = c_1\)かつ\(b_0 = b_1\)である。
			\item 以上より\((s_0,c_0,b_0) = (s_1,c_1,b_1)\)である。
		\end{indented}
		\item (5)を示す。
		\begin{indented}
			\item \((s_0,c_0,b_0)\)と\((s_1,c_1,b_1)\)が\(t\)の第\(1\)種scb分解であると仮定する。
			\item \(j_1 := \textrm{Lng}(\textrm{Rightnodes}(t))-1\)と置く。仮定から\(t\)はscb分解可能であるため、\(t \neq 0\)であり\(j_1 \geq 0\)である。
			\item \(u := \textrm{RightNodes}(t)_{j_1}\)と置く。\(\textrm{RightNodes}\)の再帰的定義から、\(t\)から\(\underline{)}\)を除いた文字列の末尾\(2\)文字は\(D_u 0\)である。
			\item \(i \in \{0,1\}\)とする。\(j_{1,i} := \textrm{Lng}(\textrm{RightNodes}(c_i))-1\)と置く。scb分解の定義から、\(j_{1,i} \geq 1\)である。
			\item \(u_i := \textrm{RightNodes}(c_i)_{j_{1,i}}\)と置く。\(\textrm{RightNodes}\)の再帰的定義から、\(c_i\)から\(\underline{)}\)を除いた文字列の末尾\(2\)文字は\(D_{u_i} 0\)である。
			\item \(b_i\)は\(\underline{)}\)のみからなる文字列であるため、\(t\)から\(\underline{)}\)を除いた文字列の末尾\(2\)文字と\(c_i\)から\(\underline{)}\)を除いた文字列の末尾\(2\)文字は等しく、\(u = u_i\)である。従って\(u_0 = u_1\)である。
			\item \(v_i := \textrm{RightNodes}(c_i)_0\)と置く。\(b_i\)は\(\underline{)}\)のみからなる文字列でありかつ\(c_i\)は\(t\)に含まれる項であるため、\(j_{1,i} \leq j_1\)かつ\(\textrm{RightNodes}(c_i) = (\textrm{RightNodes}(t)_j)_{j=j_1-j_{1,i}}^{j_1}\)である。従って\(v_i = \textrm{RightNodes}(t)_{j_1-j_{1,i}} < u\)かつ任意の\(j \in \mathbb{N}\)に対して\(j_1-j_{1,i} < j < j_1\)ならば\(\textrm{RightNodes}(t)_j \geq u\)である。以上より\(j_1-j_{1,0} = j_1-j_{1,1}\)かつ\(v_0 = v_1\)である。
			\item \(j_0 := j_1 - j_{1,0}\)と置く。\(\textrm{RightNodes}(t)_{j_0} = v_0\)より、\(t\)は\(D_{v_0}\)を含む。\(t\)に出現する\(D_{v_0}\)のうちもっとも末尾に近いものより左側の文字列を\(s\)と置くと、\(c_i\)の先頭が\(D_{v_0}\)でかつ\(\textrm{RightNodes}(c_i) = (\textrm{RightNodes}(t)_j)_{j=j_0}^{j_1}\)であることから\(s = s_i\)である。従って\(s_0 = s_1\)である。
			\item \(c_i\)が項であることから、\(c_i\)の末尾に\(1\)個以上の\(\underline{)}\)を結合した文字列は項でない。更に\(b_i\)が\(\underline{)}\)のみからなることから、\(t\)の先頭から\(s\)を除いた文字列の部分文字列であって項であるもののうち最大のものが\(c_i\)である。従って\(c_0 = c_1\)かつ\(b_0 = b_1\)である。
			\item 以上より\((s_0,c_0,b_0) = (s_1,c_1,b_1)\)である。
		\end{indented}
		\item (3)を示す。
		\begin{indented}
			\item (4)の証明から、\(t\)の第\(0\)種scb\((s_0,c_0,b_0)\)が存在するならば\(t\)から\(\underline{)}\)を除いた文字列の末尾\(2\)文字は\(c_0\)から\(\underline{)}\)を除いた文字列の末尾\(2\)文字と一致し、\(D_0 0\)となる。
			\item (5)の証明から、\(t\)の第\(1\)種scb\((s_1,c_1,b_1)\)が存在するならば\(t\)から\(\underline{)}\)を除いた文字列の末尾\(2\)文字は\(c_1\)から\(\underline{)}\)を除いた文字列の末尾\(2\)文字と一致し、それはある\(u \in \mathbb{N}\)を用いて\(D_u 0\)と表せる。一方で\(c_1\)の先頭は\(v < u\)を満たすある\(v \in \mathbb{N}\)を用いて\(D_v\)と表せるので、特に\(u > 0\)である。
			\item 以上より、\(t\)が第\(0\)種scb分解可能ならば\(t\)は第\(1\)種scb分解可能でない。
		\end{indented}
	\end{indented}
\end{hideableproof}

\begin{corollary}[加法とscb分解の関係]\label{加法とscb分解の関係}
	任意の\(t \in T_{\textrm{B}}\)と\(c \in PT_{\textrm{B}}\)に対し、以下が成り立つ:
	\begin{penumerate}
		\item \((t+c,c) \in T_{\textrm{B}}^{\textrm{Marked}}\)である。
		\item 任意の\((s,b) \in (\Sigma^{< \omega})^2\)と\(c' \in PT_{\textrm{B}}\)に対し、\((s,c,b)\)が\(t+c\)のscb分解であるならば\((s,c',b)\)は\(t+c'\)のscb分解でである。
		\item 任意の\(v \in \mathbb{N}\)と\(s_0,s_1,b_0,b_1 \in \Sigma^{< \omega}\)と\(c' \in PT_{\textrm{B}}\)に対し、\(s_1 D_v(t+c) b_1 \in T_{\textrm{B}}\)かつ\((s_0,c,b_0)\)が\(s_1 D_v(t+c) b_1\)のscb分解であるならば、\(s_1 D_v(t+c') b_1 \in T_{\textrm{B}}\)かつ\((s_0,c',b_0)\)は\(s_1 D_v(t+c') b_1\)のscb分解である。
	\end{penumerate}
\end{corollary}

\begin{hideableproof}
	\begin{indented}
		\item (1),(2)を示す。
		\item \(t = 0\)とする。
		\begin{indented}
			\item \(t+c = c\)より\(((),c,())\)が\(t+c\)のscb分解をなすので\((t+c,c) \in T_{\textrm{B}}^{\textrm{Marked}}\)である。\nameref{scb分解の一意性}より\(s = ()\)かつ\(b = ()\)であり、\((s,c',b) = ((),c',())\)は\(t+c' = c'\)のscb分解である。
		\end{indented}
		\item \(t \in PT_{\textrm{B}}\)とする。
		\begin{indented}
			\item \(t+c = \underline{(} t \underline{,} c \underline{)}\)より\((\underline{(} t \underline{,},c,\underline{)})\)が\(t+c\)のscb分解をなすので\((t+c,c) \in T_{\textrm{B}}^{\textrm{Marked}}\)である。\nameref{scb分解の一意性}より\(s = \underline{(} t \underline{,}\)かつ\(b = \underline{)}\)であり、\((s,c',b) = (\underline{(} t \underline{,},c',\underline{)})\)は\(t+c' = \underline{(} t \underline{,} c' \underline{)}\)のscb分解である。
		\end{indented}
		\item \(t \in MT_{\textrm{B}}\)とする。
		\begin{indented}
			\item \(s' \in \Sigma^{< \omega}\)を用いて\(t = \underline{(} s \underline{)}\)と置くと\(t+c = \underline{(} s' \underline{,} c \underline{)}\)より\((\underline{(} s' \underline{,},c,\underline{)})\)が\(t+c\)のscb分解をなすので\((t+c,c) \in T_{\textrm{B}}^{\textrm{Marked}}\)である。\nameref{scb分解の一意性}より\(s = \underline{(} s' \underline{,}\)かつ\(b = \underline{)}\)であり、\((s,c',b) = (\underline{(} s' \underline{,},c',\underline{)})\)は\(t+c' = \underline{(} s' \underline{,} c' \underline{)}\)のscb分解である。
		\end{indented}
		\item (3)を示す。
		\item \(t = 0\)とする。
		\begin{indented}
			\item \(s_1 D_v c b_1 = s_1 D_v(t+c) b_1 \in T_{\textrm{B}}\)より\nameref{scb分解の置換可能性}から\(s_1 D_v(t+c') b_1 = s_1 D_v c' b_1 \in T_{\textrm{B}}\)である。
			\item \(s_0 c b_0 = s_1 D_v(t+c) b_1 = s_1 D_v c b_1\)より、\nameref{scb分解の一意性} (1)から\(s_0 = s_1 D_v\)かつ\(b_0 = b_1\)である。従って\(s_0 c' b_0 = s_1 D_v c' b_1 = s_1 D_v(t+c') b_1\)となるので\((s_0,c',b_0)\)は\(s_1 D_v(t+c') b_1\)のscb分解である。
		\end{indented}
		\item \(t \in PT_{\textrm{B}}\)とする。
		\begin{indented}
			\item \(s_1 D_v \underline{(} t \underline{,} c \underline{)} b_1 = s_1 D_v(t+c) b_1 \in T_{\textrm{B}}\)より\nameref{scb分解の置換可能性}から\(s_1 D_v(t+c') b_1 = s_1 D_v \underline{(} t \underline{,} c' \underline{)} b_1 \in T_{\textrm{B}}\)である。
			\item \(s_0 c b_0 = s_1 D_v(t+c) b_1 = s_1 D_v \underline{(} t \underline{,} c \underline{)} b_1\)より、\nameref{scb分解の一意性} (1)から\(s_0 = s_1 D_v \underline{(} t \underline{,}\)かつ\(b_0 = \underline{)} b_1\)である。従って\(s_0 c' b_0 = s_1 D_v \underline{(} t \underline{,} c' \underline{)} b_1 = s_1 D_v(t+c') b_1\)となるので\((s_0,c',b_0)\)は\(s_1 D_v(t+c') b_1\)のscb分解である。
		\end{indented}
		\item \(t \in MT_{\textrm{B}}\)とする。
		\begin{indented}
			\item \(s' \in \Sigma^{< \omega}\)を用いて\(t = \underline{(} s' \underline{)}\)と置くと\(s_1 D_v \underline{(} s' \underline{,} c \underline{)} b_1 = s_1 D_v(t+c) b_1 \in T_{\textrm{B}}\)より\nameref{scb分解の置換可能性}から\(s_1 D_v(t+c') b_1 = s_1 D_v \underline{(} s' \underline{,} c' \underline{)} b_1 \in T_{\textrm{B}}\)である。
			\item \(s_0 c b_0 = s_1 D_v(t+c) b_1 = s_1 D_v \underline{(} s' \underline{,} c \underline{)} b_1\)より、\nameref{scb分解の一意性} (1)から\(s_0 = s_1 D_v \underline{(} s' \underline{,}\)かつ\(b_0 = \underline{)} b_1\)である。従って\(s_0 c' b_0 = s_1 D_v \underline{(} s' \underline{,} c' \underline{)} b_1 = s_1 D_v(t+c') b_1\)となるので\((s_0,c',b_0)\)は\(s_1 D_v(t+c') b_1\)のscb分解である。
		\end{indented}
	\end{indented}
\end{hideableproof}

\begin{proposition}[scb分解と基本列の関係]\label{scb分解と基本列の関係}
	任意の\(v,n \in \mathbb{N}\)に対し、以下が成り立つ:
	\begin{penumerate}
		\item 任意の\(t'_0,t'_1 \in T_{\textrm{B}}\)に対し以下が成り立つ。
		\begin{indented}
			\item[(1-1)] \(t'_0 + D_v(t'_1 + D_0 0)[n] = t'_0 + (D_v t'_1) \times (n+1)\)である。
			\item[(1-2)] 任意の\(t \in T_{\textrm{B}}\)と\(u \in \mathbb{N}\)と\((s,b) \in (\Sigma^{< \omega})^2\)に対し、\((s,D_u(t'_0 + D_v(t'_1+D_0 0)),b)\)が\(t\)のscb分解ならば、\((s,D_u(t'_0 + (D_v t'_1) \times (n+1)),b)\)は\(t[n]\)のscb分解である。
		\end{indented}
		\item 任意の\(t \in T_{\textrm{B}}\)と\(u \in \mathbb{N}\)と\((s_0,s_1,c_2,b_0,b_1) \in (\Sigma^{< \omega})^5\)に対し、\((s_1,c_2,b_1)\)が\(t\)の第\(1\)種scb分解でありかつ\((D_u s_0,D_v 0,b_0)\)が\(c_2\)のscb分解であるならば、\(v > u\)かつ\(t[n] = s_1 D_u (s_0 D_{v-1})^{n+1} 0 b_0^{n+1}b_1\)である。
	\end{penumerate}
\end{proposition}

\begin{hideableproof}
	\begin{indented}
		\item Buchholzの表記系における基本列と共終数の再帰的定義から、\([n]\)を取る項の長さに関する数学的帰納法により即座に従う。
	\end{indented}
\end{hideableproof}
\begin{proposition}[\(\textrm{RightNodes}\)と部分表現の関係]\label{RightNodesと部分表現の関係}
	任意の\(s,b \in \Sigma^{< \omega}\)と\(v \in \mathbb{N}\)\(t \in PT_{\textrm{B}}\)に対し、\(b\)が\(\underline{)}\)のみからなりかつ\(s D_v 0 b \in T_{\textrm{B}}\)ならば、\(s D_v t b \in T_{\textrm{B}}\)かつ\(\textrm{Lng}(P(s D_v t b)) = \textrm{Lng}(P(s D_v 0 b))\)であり一意な\(a_0, a_1 \in \mathbb{N}^{< \omega}\)が存在して以下を満たす:
	\begin{penumerate}
		\item \(\textrm{RightNodes}(s D_v t b) = a_0 \oplus_{\mathbb{N}} (v) \oplus_{\mathbb{N}} a_1\)である。
		\item \(\textrm{RightNodes}(s D_v 0 b) = a_0 \oplus_{\mathbb{N}} (v)\)である。
		\item \(\textrm{RightNodes}(D_v t) = (v) \oplus_{\mathbb{N}} a_1\)である。
	\end{penumerate}
\end{proposition}

\begin{hideableproof}
	\begin{indented}
		\item \(s D_v t b \in T_{\textrm{B}}\)であることは\nameref{scb分解の置換可能性}から従い、\(\textrm{Lng}(P(s D_v t b)) = \textrm{Lng}(P(s D_v 0 b))\)であることは\(P\)の再帰的定義から\(\textrm{Lng}(s)\)に関する数学的帰納法より即座に従う。
		\item \(a_1 := \textrm{RightNodes}(t)\)と置く。
		\item \(\textrm{RightNodes}\)の定義より、\(\textrm{RightNodes}(D_v t) = (v) \oplus_{\mathbb{N}} a_1\)である。
		\item ある\(a_0 \in \mathbb{N}^{< \omega}\)が存在して\(\textrm{RightNodes}(s D_v t b) = a_0 \oplus_{\mathbb{N}} (v) \oplus_{\mathbb{N}} a_1\)かつ\(\textrm{RightNodes}(s D_v 0 b) = a_0 \oplus_{\mathbb{N}} (v)\)となることを\(\textrm{Lng}(s)\)に関する数学的帰納法で示す。
		\item \(\textrm{Lng}(s) = 0\)ならば、\(b = 0\)となるので\(a_0 = ()\)とすれば良い。
		\item \(\textrm{Lng}(s) > 0\)とする。
		\begin{indented}
			\item \(s D_v 0 b \in PT_{\textrm{B}}\)とする。
			\begin{indented}
				\item \(u \in \mathbb{N}\)と\(s' \in \Sigma^{< \omega}\)を用いて\(s = D_u s'\)と置く。
				\item \(T_{\textrm{B}}\)の再帰的定義から\(s' D_v 0 b \in T_{\textrm{B}}\)である。従って\(s' D_v t b \in T_{\textrm{B}}\)である。
				\item \(\textrm{Lng}(s') = \textrm{Lng}(s)-1 < \textrm{Lng}(s)\)であるので、帰納法の仮定からある\(a'_0 \in \mathbb{N}^{< \omega}\)が存在して\(\textrm{RightNodes}(s' D_v t b) = a'_0 \oplus_{\mathbb{N}} (v) \oplus_{\mathbb{N}} a_1\)かつ\(\textrm{RightNodes}(s' D_v 0 b) = a'_0 \oplus_{\mathbb{N}} (v)\)となる。
				\item \(a_0 := (u) \oplus_{\mathbb{N}} a'_0\)と置くと、\(T_{\textrm{B}}\)の再帰的定義から\(\textrm{RightNodes}(s D_v t b) = (u) \oplus_{\mathbb{N}} \textrm{RightNodes}(s' D_v t b) = (u) \oplus_{\mathbb{N}} a'_0 \oplus_{\mathbb{N}} (v) \oplus_{\mathbb{N}} a_1 = a_0 \oplus_{\mathbb{N}} (v) \oplus_{\mathbb{N}} a_1\)かつ\(\textrm{RightNodes}(s D_v 0 b) = (u) \oplus_{\mathbb{N}} \textrm{RightNodes}(s' D_v 0 b) = (u) \oplus_{\mathbb{N}} a'_0 \oplus_{\mathbb{N}} (v) = a_0 \oplus_{\mathbb{N}} (v)\)である。
			\end{indented}
			\item \(s D_v 0 b \in MT_{\textrm{B}}\)とする。
			\item \(J_1 := \textrm{Lng}(P(s D_v 0 b))\)と置く。
			\item \(\Sigma_{\textrm{B}}\)の定義から、\(\underline{)}\)のみからなるある\(b'_1 \in \Sigma^{< \omega}\)が存在して\(\underline{,} P(s D_v 0 b)_{J_1} b'_1\)が\(s D_v 0 b\)の末尾に現れる。\(\underline{)}\)のみからなる\(b'_0 \in \Sigma^{< \omega}\)を用いて\(b = b'_0 b'_1\)と置く。
			\item \(s D_v 0 b\)の末尾に\(D_v 0 b = D_v 0 b'_0 b'_1\)が現れ、\(P(s D_v 0 b)_{J_1}\)の末尾に\(b'_0\)が現れ、そして\(P(s D_v 0 b)_{J_1}\)は単項であるので\(0\)でも\(\underline{)}\)でもない文字を含むため、\(P(s D_v 0 b)_{J_1}\)の末尾に\(D_v 0 b'_0\)が現れる。\(s'_0 \in \Sigma^{< \omega}\)を用いて\(P(s D_v 0 b)_{J_1} = s'_0 D_v 0 b'_0\)と置く。
			\item \(s D_v 0 b\)の末尾に\(\underline{,} P(s D_v 0 b)_{J_1} b'_1 = \underline{,} s'_0 D_v 0 b'_0 b'_1 = \underline{,} s'_0 D_v 0 b\)が現れることから、\(s\)の末尾に\(\underline{,} s'_0\)が現れる。\(s'_1 \in \Sigma^{< \omega}\)を用いて\(s = s'_1 \underline{,} s'_0\)と置く。
			\item \(s'_0 D_v 0 b'_0 = P(s D_v b)_{J_1} \in T_{\textrm{B}}\)より\(s'_0 D_v t b'_0 \in T_{\textrm{B}}\)であり、\(s D_v 0 b = s'_1 \underline{,} s'_0 D_v 0 b'_0 b'_1 = s'_1 \underline{,} P(s D_v b)_{J_1} b'_1\)である。\(s D_v 0 b\)の末尾\(P(s D_v 0 b)_{J_1} b'_1\)を\(P(s D_v t b)_{J_1} b'_1\)に置き換えたものは\(s D_v t b\)であり、かつ\(s D_v 0 b\)の末尾\(P(s D_v 0 b)_{J_1} b'_1 = s'_0 D_v 0 b'_0 b'_1\)を\(s'_0 D_v t b'_1\)に置き換えたものも\(s D_v t b\)であるので、\(P(s D_v t b)_{J_1} = s'_0 D_v t b'_0\)である。
			\item \(\textrm{Lng}(s') = \textrm{s} - \textrm{Lng}(s'_1) + 1 < \textrm{Lng}(s)\)であるので、帰納法の仮定からある\(a_0 \in \mathbb{N}^{< \omega}\)が存在して\(\textrm{RightNodes}(s'_0 D_v t b'_0) = a_0 \oplus_{\mathbb{N}} (v) \oplus_{\mathbb{N}} a_1\)かつ\(\textrm{RightNodes}(s'_0 D_v 0 b'_0) = a_0 \oplus_{\mathbb{N}} (v)\)となる。
			\item \(\textrm{RightNodes}\)の再帰的定義から\(\textrm{RightNodes}(s D_v t b) = \textrm{RightNodes}(P(s D_v t b)_{J_1}) = \textrm{RightNodes}(s'_0 D_v t b'_0) = a_0 \oplus_{\mathbb{N}} (v) \oplus_{\mathbb{N}} a_1\)であり、\(\textrm{RightNodes}(s D_v 0 b) = \textrm{RightNodes}(P(s D_v 0 b)_{J_1}) = \textrm{RightNodes}(s'_0 D_v 0 b'_0) = a_0 \oplus_{\mathbb{N}} (v)\)である。
		\end{indented}
	\end{indented}
\end{hideableproof}


\subsection{翻訳写像}

\((\textrm{Trans}(M),\textrm{Mark}(M,m)) \in T_{\textrm{B}}^{\textrm{Marked}}\)を満たす写像
\begin{eqnarray*}
\textrm{Trans} \colon T_{\textrm{PS}} & \to & T_{\textrm{B}} \\
M & \mapsto & \textrm{Trans}(M)
\end{eqnarray*}
と
\begin{eqnarray*}
\textrm{Mark} \colon T_{\textrm{PS}}^{\textrm{Marked}} & \to & T_{\textrm{B}} \\
(M,m) & \mapsto & \textrm{Mark}(M,m)
\end{eqnarray*}
を以下のように再帰的に定める:
\begin{nenumerate}
	\item \(j_1 := \textrm{Lng}(M) - 1\)と置く。
	\item \(M \in RT_{\textrm{PS}}\)かつ\(j_1 = 0\)とする。
	\begin{nenumerate}
		\item \(M_0 = (0,0)\)とする。
		\begin{nenumerate}
			\item \(\textrm{Trans}(M) := 0\)である。
			\item \(\textrm{Mark}(M,m) := 0\)である。
		\end{nenumerate}
		\item \(M_0 \neq (0,0)\)とする。
		\begin{nenumerate}
			\item \(\textrm{Trans}(M) := D_{M_{1,0}} 0\)である。
			\item \(\textrm{Mark}(M,m) := D_{M_{1,0}} 0\)である。
		\end{nenumerate}
	\end{nenumerate}
	\item \(M\)が簡約かつ単項かつ\(j_1 > 0\)とする。
	\begin{nenumerate}
		\item \(t_1 := \textrm{Trans}(\textrm{Pred}(M))\)と置く。
		\item \(t_1 = 0\)とする\footnote{後で示す\nameref{Transが零項性を保つこと}から、この条件は\(j_1 = 1\)かつ\(M_0 = (0,0)\)と同値である。}。
		\begin{nenumerate}
			\item \(\textrm{Trans}(M) := D_0 D_{M_{1,j_1}} 0\)である。
			\item \(m = 0\)ならば\(\textrm{Mark}(M,m) := D_0 D_{M_{1,j_1}} 0\)である。
			\item \(m > 0\)ならば\(\textrm{Mark}(M,m) := D_{M_{1,j_1}} 0\)である。
		\end{nenumerate}
		\item \(t_1 \neq 0\)とする。
		\begin{nenumerate}
			\item \(i_1 := \max \{i \in \{0,1\} \mid M_{i,j_1} > 0\}\)と置く\footnote{\(M\)が単項かつ\(j_1 > 0\)より\(M_{j_1} \neq (0,0)\)であるので、\(\max\)は存在する。}。
			\item \(j_0 := \max \{j \in \mathbb{N} \mid j < j_1 \wedge (0,j) \leq_M (0,j_1)\}\)と置く\footnote{\(M\)が単項より\((0,0) \leq_M (0,j_1)\)であり、\(j_1 > 0\)より\(\max\)は存在する。また\(M\)が簡約より、\nameref{簡約性と係数の関係}から\(M_{0,j_0} = M_{0,j_1} - 1\)となる。}。
			\item \(j_{-1} := \textrm{Adm}_M(j_0)\)と置く。
			\item 互いに背反な条件(I)~(VI)を以下のように定める:
			\begin{nenumerate}
				\item 条件(I)は「\(M_{1,j_1} = 0\)かつ\(j_0\)は\(M\)許容」である。
				\item 条件(II)は「\(M_{1,j_1} = 0\)かつ\(j_0\)は非\(M\)許容」である。
				\item 条件(III)は「\(M_{1,j_1} > 0\)かつ\(M_{1,j_0} \geq M_{1,j_1}\)かつ\(j_0\)は\(M\)許容」である。
				\item 条件(IV)は「\(M_{1,j_1} > 0\)かつ\(M_{1,j_0} \geq M_{1,j_1}\)かつ\(j_0\)は非\(M\)許容」である。
				\item 条件(V)は「\(M_{1,j_1} > 0\)かつ\(M_{1,j_0}+1 = M_{1,j_1}\)かつ\(j_0+1 < j_1\)」である。
				\item 条件(VI)は「\(M_{1,j_1} > 0\)かつ\(M_{1,j_0}+1 = M_{1,j_1}\)かつ\(j_0+1 = j_1\)」である。
			\end{nenumerate}
			\item \(c_1 := \textrm{Mark}(\textrm{Pred}(M),j_{-1})\)と置く\footnote{\(j_{-1} \leq j_0 < j_1\)から\nameref{基点の切片への遺伝性}より\((\textrm{Pred}(M),j_{-1}) \in T_{\textrm{PS}}^{\textrm{Marked}}\)である。}。
			\item \((t_1,c_1) \in T_{\textrm{B}}^\textrm{Marked}\)かつ\(t_1 \neq 0\)より、\(c_1 \in PT_{\textrm{B}}\)であり\nameref{scb分解の一意性} (1)より一意な\((s_1,b_1) \in (\Sigma^{< \omega})^2\)が存在して\((s_1,c_1,b_1)\)は\(t_1\)のscb分解をなす。
			\item \(v \in \mathbb{N}\)と\(t_2 \in T_{\textrm{B}}\)を用いて\(c_1 = D_v t_2\)と置く。
			\item \(J_1 := \textrm{Lng}(P(t_2))-1\)と置く。
			\item 条件(I)か(III)か(V)を満たすならば\(c_2 := D_v(t_2 + D_{M_{1,j_1}} 0)\)と置く。
			\item 条件(II)か(IV)を満たすとする。
			\begin{nenumerate}
				\item \(t_2 = 0\)ならば\footnote{後で示す\nameref{条件(II)か(IV)の下でt_2が0でないこと}から、この分岐が結果的に生じないことが分かる。}、\(c_2 := D_v D_{M_{1,j_0}} D_{M_{1,j_1}} 0\)と置く。
				\item \(t_2 \neq 0\)とする。
				\begin{nenumerate}
					\item \(P(t_2)_{J_1}\)の左端が\(D_{M_{1,j_0}}\)であるとする。
					\begin{nenumerate}
						\item \(t_3 := \Sigma_{\textrm{B}} (P(t_2)_J)_{J=0}^{J_1-1}\)と置く。
						\item \(t_4 \in T_{\textrm{PS}}\)を用いて\(P(t_2)_{J_1} = D_{M_{1,j_0}} t_4\)と置く。
					\end{nenumerate}
					\item \(P(t_2)_{J_1}\)の左端が\(D_{M_{1,j_0}}\)でないとする。
					\begin{nenumerate}
						\item \(t_3 := t_2\)と置く。
						\item \(t_4 := t_2\)と置く。
					\end{nenumerate}
					\item \(c_2 := D_v (t_3 + D_{M_{1,j_0}}(t_4 + D_{M_{1,j_1}} 0))\)と置く。
				\end{nenumerate}
			\end{nenumerate}
			\item 条件(VI)を満たすならば\(c_2 := D_v D_{M_{1,j_1}} 0\)と置く。
			\item \(\textrm{Trans}(M) := s_1 c_2 b_1\)である\footnote{\nameref{scb分解の置換可能性}より\(s_1 c_2 b_1 \in T_{\textrm{B}}\)である。}。
			\item \(m < j_1\)とする。
			\begin{nenumerate}
				\item \(c_0 := \textrm{Mark}(\textrm{Pred}(M),m)\)と置く\footnote{\((0,m) \leq_M (0,j_1)\)かつ\(m < j_1\)より\((0,m) \leq_M (0,j_1-1)\)であるので\((0,m) \leq_{\textrm{Pred}(M)} (0,j_1-1)\)である。従って\((\textrm{Pred}(M),m) \in T_{\textrm{PS}}^{\textrm{Marked}}\)である。}。
				\item \((c_0,c_1) \in T_{\textrm{B}}^{\textrm{Marked}}\)とする。
				\begin{nenumerate}
					\item \((t_1,c_0) \in T_{\textrm{B}}^{\textrm{Marked}}\)より、\nameref{scb分解の一意性} (1)から一意な\((s_0,b_0) \in (\Sigma^{< \omega})^2\)が存在して\((s_0,c_0,b_0)\)は\(t_1\)のscb分解である。
					\item \((c_0,c_1) \in T_{\textrm{B}}^\textrm{Marked}\)より、\nameref{scb分解の一意性} (1)から一意な\((s_{-1},b_{-1}) \in (\Sigma^{< \omega})^2\)が存在して\((s_{-1},c_1,b_{-1})\)は\(c_0\)のscb分解である\footnote{この時\(t_1 = s_1 c_1 b_1\)かつ\(t = s_0 c_0 b_0\)より\(s_1 = s_0 s_{-1}\)かつ\(b_1 = b_{-1} b_0\)である。}。
					\item \(\textrm{Mark}(M,m) := s_{-1} c_2 b_{-1}\)である\footnote{\(t_1 = s_0 c_0 b_0\)かつ\(t_1 = s_1 c_1 b_1\)より\(s_1 = s_0 s_{-1}\)かつ\(b_1 = b_{-1} b_0\)であるので\(\textrm{Trans}(M) = s_0 \textrm{Mark}(M,m) b_0\)である。従って\((\textrm{Trans}(M),\textrm{Mark}(M,m)) \in T_{\textrm{B}}^{\textrm{Marked}}\)である。}。
				\end{nenumerate}
				\item \((c_0,c_1) \in T_{\textrm{B}}^{\textrm{Marked}}\)でないならば\footnote{後で示す\nameref{Markが順序関係を保つこと}から、この分岐は生じないことが分かる。}、\(\textrm{Mark}(M,m) := D_{M_{1,j_1}} 0\)である\footnote{\(\textrm{Trans}(M)\)の定義から\((\textrm{Trans}(M),\textrm{Mark}(M,m)) \in T_{\textrm{B}}^{\textrm{Marked}}\)である。}。
			\end{nenumerate}
			\item \(m = j_1\)ならば\(\textrm{Mark}(M,m) := D_{M_{1,j_1}} 0\)である\footnote{\(\textrm{Trans}(M)\)の定義から\((\textrm{Trans}(M),\textrm{Mark}(M,m)) \in T_{\textrm{B}}^{\textrm{Marked}}\)である。}。
		\end{nenumerate}
	\end{nenumerate}
	\item \(M\)が簡約かつ複項とする。
	\begin{nenumerate}
		\item \(J_1 := \textrm{Lng}(P(M))-1\)と置く。
		\item \(j_0 := j_1 - \textrm{Lng}(P(M)_{J_1}) + 1\)と置く。
		\item \(P(M)_{J_1} = ((0,0))\)とする。
		\begin{nenumerate}
			\item \(\textrm{Trans}(M) := \textrm{Trans}((M_j)_{j=0}^{j_0-1}) + D_0 0\)である。
			\item \(\textrm{Mark}(M,m) := D_0 0\)である。
		\end{nenumerate}
		\item \(P(M)_{J_1} \neq ((0,0))\)とする。
		\begin{nenumerate}
			\item \(\textrm{Trans}(M) := \textrm{Trans}((M_j)_{j=0}^{j_0-1}) + \textrm{Trans}(P(M)_{J_1})\)である。
			\item \(\textrm{Mark}(M,m) := \textrm{Mark}(P(M)_{J_1},m-j_0)\)である\footnote{\((0,m) \leq_M (0,j_1)\)より\(j_0 \leq m\)であり、\(P(M)_{J_1} = (M_j)_{j=j_0}^{j_1}\)より\((0,m-j_0) \leq_{P(M)_{J_1}} (0,j_1-j_0)\)である。従って\((P(M)_{J_1},m-j_0) \in T_{\textrm{PS}}^{\textrm{Marked}}\)である。}。
		\end{nenumerate}
	\end{nenumerate}
	\item \(M\)が簡約でないとする。
	\begin{nenumerate}
		\item \(\textrm{Trans}(M) := \textrm{Trans}(\textrm{Red}(M))\)である。
		\item \(\textrm{Mark}(M,m) := \textrm{Mark}(\textrm{Red}(M),m)\)である\footnote{\nameref{直系先祖のRed不変性}より\((\textrm{Red}(M),m) \in T_{\textrm{PS}}^{\textrm{Marked}}\)である。}。
	\end{nenumerate}
\end{nenumerate}

\begin{proposition}[\(\textrm{Trans}\)のwell-defined性]\label{Transのwell-defined性}
	上の条件を全て満たす写像\(\textrm{Trans}\)と\(\textrm{Mark}\)が一意に存在する。
\end{proposition}

\begin{hideableproof}
	\begin{indented}
		\item \(\textrm{Lng}(M)\)に関する数学的帰納法より即座に従う。
	\end{indented}
\end{hideableproof}

\begin{proposition}[\(2\)列ペア数列の基本性質]\label{2列ペア数列の基本性質}
	任意の\(M \in RT_{\textrm{PS}} \cap PT_{\textrm{PS}}\)に対し、\(\textrm{Lng}(M) = 2\)ならば以下が成り立つ:
	\begin{penumerate}
		\item \(\textrm{Trans}(M) = D_{M_{1,0}} D_{M_{1,1}} 0\)である。
		\item \((M,0),(M,1) \in T_{\textrm{PS}}^{\textrm{Marked}}\)である。
		\item \(\textrm{Mark}(M,0) = D_{M_{1,0}} D_{M_{1,1}} 0\)かつ\(\textrm{Mark}(M,1) = D_{M_{1,1}} 0\)である。
	\end{penumerate}
\end{proposition}

\begin{hideableproof}
	\begin{indented}
		\item \(m \in \{0,1\}\)とし、\(\textrm{Trans}\)と\(\textrm{Mark}\)の再帰的定義中に導入した記号を用いる。
		\item \(M\)の簡約性と\nameref{簡約性と係数の関係}から\(M\)は条件(A)と(B)を満たす。
		\item \(j_1 = 1\)であり、\(M\)は単項より\((0,0) <_M^{\textrm{Next}} (0,1) = (0,j_1)\)なので\(j_0 = 0\)である。特に\(j_0\)は\(M\)許容であり、\(j_{-1} = j_0 = 0\)である。
		\item \(m=0\)ならば、\(M\)は単項なので\((0,m) = (0,0) <_M^{\textrm{Next}} (0,1) = (0,j_1)\)となり、\(m = 0\)は\(M\)許容であるので\((M,m) \in T_{\textrm{PS}}^{\textrm{Marked}}\)である。
		\item \(m=1\)ならば、\((0,m) = (0,1) \leq_M^{\textrm{Next}} (0,1) = (0,j_1)\)となり、\(m = 1 = j_1\)は\(M\)許容であるので\((M,m) \in T_{\textrm{PS}}^{\textrm{Marked}}\)である。
		\item \(D_v t_2 = c_1 = \textrm{Mark}(\textrm{Pred}(M),j_{-1}) = \textrm{Mark}(((M_{1,0},M_{1,0})),0) = D_{M_{1,0}} 0\)より\(v = M_{1,0}\)かつ\(t_2 = 0\)である。
		\item \(j_0\)の\(M\)許容性と\(j_0+1 = 1 = j_1\)から\(M\)は条件(I)か(III)か(VI)を満たす。
		\item 条件(I)か(III)を満たすならば\(c_2 = D_v(t_2 + D_{M_{1,j_1}} 0) = D_{M_{1,0}}(0 + D_{M_{1,j_1}}) 0 = D_{M_{1,0}} D_{M_{1,1}} 0\)である。
		\item 条件(VI)を満たすならば\(c_2 = D_v D_{M_{1,j_1}} 0 = D_{M_{1,0}} D_{M_{1,j_1}} 0\)である。
		\item 従っていずれの場合も\(c_2 = D_{M_{1,m}} D_{M_{1,j_1}} 0\)である。
		\item \(m=0\)ならば、\(m = j_{-1}\)より\(s_{-1} c_1 b_{-1} = c_0 = \textrm{Mark}(\textrm{Pred}(M),m) = \textrm{Mark}(\textrm{Pred}(M),j_{-1}) = c_1\)となるので\(s_{-1} = ()\)かつ\(b_{-1} = ()\)であり、\(\textrm{Mark}(M,m) = s_{-1} c_2 b_{-1} = D_{M_{1,m}} D_{M_{1,j_1}} 0\)である。
		\item \(m=1\)ならば、\(m = j_1\)より\(\textrm{Mark}(M,m) = D_{M_{1,j_1}} 0\)である。
		\begin{indented}
			\item \(M_{1,0} = 0\)ならば、\(t_1 = \textrm{Trans}(\textrm{Pred}(M)) = \textrm{Trans}((M_{1,0},M_{1,0})) = 0\)であるので、\(\textrm{Trans}(M) = D_0 D_{M_{1,1}} 0 = D_{M_{1,0}} D_{M_{1,1}} 0\)である。
			\item \(M_{1,0} > 0\)ならば、\(t_1 = \textrm{Trans}(\textrm{Pred}(N)) = \textrm{Trans}((M_{1,0},M_{1,0})) = D_{M_{1,0}} 0 \neq 0\)であり、\(s_1 D_{M_{1,0}} 0 b_1 = s_1 c_1 b_1 = t_1 = D_{M_{1,m}} 0\)より\(s_1 = ()\)かつ\(b_1 = ()\)であるので、\(\textrm{Trans}(M) = s_1 c_2 b_1 = D_{M_{1,0}} D_{M_{1,j_1}} 0\)である。
		\end{indented}
	\end{indented}
\end{hideableproof}

\begin{proposition}[\(\textrm{Trans}\)の\((\textrm{IncrFirst},\textrm{Red})\)不変\(P\)同変性]\label{Transの(IncrFirst,Red)不変P同変性}
	任意の\(M \in T_{\textrm{PS}}\)に対し以下が成り立つ:
	\begin{penumerate}
		\item \(\textrm{Trans}(M) = \textrm{Trans}(\textrm{Red}(M)) = \textrm{Trans}(\textrm{IncrFirst}(M))\)である。
		\item \(M\)が複項とする。\(J_1 := \textrm{Lng}(P(M))\)と置く。\(J \leq J_1\)を満たす各\(J \in \mathbb{N}\)に対し、\(P(M)_J\)が零項ならば\(t_J : = D_0 0\)と置き\(P(M)_J\)が零項でないならば\(t_J := \textrm{Trans}(P(M)_J)\)と置く。この時\(\textrm{Trans}(M) = \Sigma_{\textrm{B}} (t_J)_{J=0}^{J_1}\)である。
	\end{penumerate}
\end{proposition}

\begin{hideableproof}
	\begin{indented}
		\item \nameref{Redの冪等性}と\(\textrm{Trans}\)と\(\textrm{Red}\)の再帰的定義より即座に従う。
	\end{indented}
\end{hideableproof}

\begin{proposition}[\(\textrm{Mark}\)の\((\textrm{IncrFirst},\textrm{Red},P)\)不変性]\label{Markの(IncrFirst,Red,P)不変性}
	任意の\((M,m) \in T_{\textrm{PS}}^{\textrm{Marked}}\)に対し以下が成り立つ:
	\begin{penumerate}
		\item \(\textrm{Mark}(M,m) = \textrm{Mark}(\textrm{Red}(M),m) = \textrm{Mark}(\textrm{IncrFirst}(M),m)\)である。
		\item \(M\)が複項とする。\(J_1 := \textrm{Lng}(P(M))\)と置き、\(j_0 := \textrm{Lng}(M) - \textrm{Lng}(P(M)_{J_1}) - 1\)と置く。\(P(M)_{J_1}\)が零項ならば\(\textrm{Mark}(M,m) = D_0 0\)であり、\(P(M)_{J_1}\)が零項でないならば\(\textrm{Mark}(M,m) = \textrm{Mark}(P(M)_{J_1},m-j_0)\)である。
	\end{penumerate}
\end{proposition}

\begin{hideableproof}
	\begin{indented}
		\item \nameref{Redの冪等性}と\(\textrm{Mark}\)と\(\textrm{Red}\)の再帰的定義より即座に従う。
	\end{indented}
\end{hideableproof}

\begin{proposition}[\(\textrm{Trans}\)が零項性を保つこと]\label{Transが零項性を保つこと}
	任意の\(M \in T_{\textrm{PS}}\)に対し、以下は同値である:
	\begin{penumerate}
		\item \(M\)は零項である。
		\item \(\textrm{Trans}(M) = 0\)である。
	\end{penumerate}
\end{proposition}

\begin{hideableproof}
	\begin{indented}
		\item \nameref{Redが零項性を保つこと}と\nameref{Transの(IncrFirst,Red)不変P同変性}より\(M\)が簡約の場合に帰着される。
		\item \(M\)が簡約ならば、\(\textrm{Trans}\)の再帰的定義から、\(\textrm{Lng}(M)\)に関する数学的帰納法より即座に従う。
	\end{indented}
\end{hideableproof}

\begin{proposition}[\(c_1\)と\(c_2\)の大小関係]\label{c_1とc_2の大小関係}
	任意の\(M \in RT_{\textrm{PS}} \cap PT_{\textrm{PS}}\)に対し、\(\textrm{Trans}\)の再帰的定義中に導入した記号を用いると、\(j_1 > 0\)かつ\(t_1 \neq 0\)ならば、\(c_1\)と\(c_2\)は単項でありかつ\(c_1 < c_2\)である。
\end{proposition}

\begin{hideableproof}
	\begin{indented}
		\item \(c_1\)と\(c_2\)の定義より即座に従う。
	\end{indented}
\end{hideableproof}

\begin{proposition}[\(\textrm{Pred}\)の\(\textrm{Trans}\)に関する降下性]\label{PredのTransに関する降下性}
	任意の\(M \in T_{\textrm{PS}}\)に対し、\(\textrm{Lng}(M) > 1\)ならば\(\textrm{Trans}(\textrm{Pred}(M)) < \textrm{Trans}(M)\)である。
\end{proposition}

\begin{hideableproof}
	\begin{indented}
		\item \nameref{Transの(IncrFirst,Red)不変P同変性}と\nameref{RedとPredの可換性}から、\(M \in RT_{\textrm{PS}} \setminus \textrm{ZT}_{\textrm{PS}}\)の場合に帰着される。
		\item \(J_1 := \textrm{Lng}(J_1)\)とする。
		\begin{indented}
			\item \(\textrm{Lng}(P(M)_{J_1}) = 1\)ならば、\(\textrm{Trans}\)の再帰的定義より\(0 < D_0 0\)と\nameref{部分表現の不等式の延長性}から従う。
			\item \(\textrm{Lng}(P(M)_{J_1}) > 1\)とする。
			\begin{indented}
				\item \nameref{Pが簡約性を保つこと}から\(P(M)_{J_1} \in RT_{\textrm{PS}} \cap PT_{\textrm{PS}}\)である。\(\textrm{Trans}\)の定義と\nameref{c_1とc_2の大小関係}より、\nameref{部分表現の不等式の延長性}から\(\textrm{Trans}(\textrm{Pred}(P(M)_{J_1})) < \textrm{Trans}(P(M)_{J_1})\)である。
				\item \(\textrm{Lng}(P(M)_{J_1}) > 1\)より\(P(\textrm{Pred}(M)) = (P(M)_J)_{J=0}^{J_1-1} \oplus_{T_{\textrm{PS}}} \textrm{Pred}(P(M)_{J_1})\)であるので、\(\textrm{Trans}\)の定義より\nameref{部分表現の不等式の延長性}から\(\textrm{Trans}(\textrm{Pred}(M)) < \textrm{Trans}(M)\)である。
			\end{indented}
		\end{indented}
	\end{indented}
\end{hideableproof}

\begin{proposition}[右端第\(1\)基点のMarkの基本性質]\label{右端第1基点のMarkの基本性質}
	任意の\((M,m) \in RT_{\textrm{PS}}^{\textrm{Marked}}\)に対し、\(j_1 := \textrm{Lng}(M) - 1\)と置くと以下は同値である:
	\begin{penumerate}
		\item \(m = j_1\)である。
		\item \(\textrm{Mark}(M,m) = D_{M_{1,m}} 0\)である。
	\end{penumerate}
\end{proposition}

\begin{hideableproof}
	\begin{indented}
		\item \(\textrm{Trans}\)の再帰的定義中に導入した記号を用いる。
		\item \(\textrm{Mark}\)の再帰的定義から、\(M \in RT_{\textrm{PS}} \cap PT_{\textrm{PS}}\)の場合に帰着される。
		\item \(M \in RT_{\textrm{PS}} \cap PT_{\textrm{PS}}\)ならば、\(\textrm{Lng}(c_2) > 2\)から\(\textrm{Mark}\)の定義より即座に従う。
	\end{indented}
\end{hideableproof}

\begin{corollary}[\(\textrm{Mark}\)の左端の基本性質]\label{Markの左端の基本性質}
	任意の\((M,m) \in RT_{\textrm{PS}}^{\textrm{Marked}}\)に対し、\(\textrm{Mark}(M,m)\)の左端は\(D_{M_{1,m}}\)である。
\end{corollary}

\begin{hideableproof}
	\begin{indented}
		\item \(\textrm{Mark}\)の再帰的定義から、\(m = \textrm{Lng}(M) - 1\)の場合に帰着される。
		\item \(m = \textrm{Lng}(M) - 1\)ならば、\nameref{右端第1基点のMarkの基本性質}から即座に従う。
	\end{indented}
\end{hideableproof}

\begin{corollary}[条件(II)か(IV)の下で\(t_2\)が\(0\)でないこと]\label{条件(II)か(IV)の下でt_2が0でないこと}
	任意の\(M \in RT_{\textrm{PS}} \cap PT_{\textrm{PS}}\)に対し、\(\textrm{Trans}\)の再帰的定義中に導入した記号を用いると、\(t_1 \neq 0\)かつ\(M\)が条件(II)か(IV)を満たすならば、\(t_2 \neq 0\)である。
\end{corollary}

\begin{hideableproof}
	\begin{indented}
		\item \(M\)が条件(II)を満たすならば、\((1,j_0) <_M^{\textrm{Next}} (1,j_0+1)\)より\(M_{1,j_0+1} > 0\)であるので、\(i_1 = 0\)より\(j_0+1 < j_1\)である。
		\item \(M\)が条件(IV)を満たすならば、\((1,j_0) \leq_M (1,j_1)\)でないので\(j_0\)の非\(M\)許容性から\(j_0+1 < j_1\)である。
		\item 従っていずれの場合も\(j_{-1} \leq j_0 < j_1-1\)であり、\nameref{右端第1基点のMarkの基本性質}より\(D_v t_2 = c_1 \neq D_{M_{1,j_{-1}}} 0\)である。また\nameref{Markの左端の基本性質}より\(v = M_{1,j_{-1}}\)であるので、\(t_2 \neq 0\)となる。
	\end{indented}
\end{hideableproof}

\begin{proposition}[右端第\(2\)基点のMarkの基本性質]\label{右端第2基点のMarkの基本性質}
	\(M \in RT_{\textrm{PS}} \cap PT_{\textrm{PS}}\)とし、\(\textrm{Trans}\)の再帰的定義中に導入した記号を用いると、\(j_1 > 0\)かつ\(t_1 \neq 0\)ならば、\(\textrm{Mark}(M,j_{-1}) = c_2\)である。
\end{proposition}

\begin{hideableproof}
	\begin{indented}
		\item \(m := j_{-1}\)と置き、\(\textrm{Mark}\)の再帰的定義中に導入した記号を用いる。
		\item \(s_{-1} c_0 b_{-1} = c_1 = \textrm{Mark}(\textrm{Pred}(M),j_{-1}) = \textrm{Mark}(\textrm{Pred}(M),m) = c_0\)より\(s_{-1} = ()\)かつ\(b_{-1} = ()\)である。従って\(\textrm{Mark}(M,j_{-1}) = \textrm{Mark}(M,m) = s_{-1} c_2 b_{-1} = c_2\)である。
	\end{indented}
\end{hideableproof}

\begin{proposition}[\(\textrm{Trans}\)の最左単項成分の左端の基本性質]\label{Transの最左単項成分の左端の基本性質}
	任意の\(M \in RT_{\textrm{PS}}\)に対し以下が成り立つ:
	\begin{penumerate}
		\item \(P(M)_0 = ((0,0))\)かつ\(\textrm{Lng}(P(M)) > 1\)ならば\(\textrm{Trans}(M)\)の最左単項成分の左端は\(D_{M_{1,1}}\)である。
		\item \(P(M)_0 \neq ((0,0))\)ならば\(\textrm{Trans}(M)\)の最左単項成分は\(\textrm{Trans}(P(M)_0)\)でありその左端は\(D_{M_{1,0}}\)である。
		\item \((1,0) <_M^{\textrm{Next}} (1,1)\)ならば\(\textrm{Trans}(M)\)の最左単項成分の左端\(2\)文字はある\(u \in \mathbb{N}\)を用いて\(D_{M_{1,0}} D_u\)と表せる。
	\end{penumerate}
\end{proposition}

\begin{hideableproof}
	\begin{indented}
		\item \(\textrm{Trans}\)の再帰的定義から、\(M \in RT_{\textrm{PS}} \cap PT_{\textrm{PS}}\)の場合に帰着される。
		\item \(M \in RT_{\textrm{PS}} \cap PT_{\textrm{PS}}\)ならば、\(\textrm{Trans}\)の再帰的定義から\(j_1\)に関する数学的帰納法より従う。
	\end{indented}
\end{hideableproof}

\begin{proposition}[\(\textrm{Trans}\)が単項性を保つこと]\label{Transが単項性を保つこと}
	任意の\(M \in T_{\textrm{PS}}\)に対し、以下は同値である:
	\begin{penumerate}
		\item \(M\)は単項である。
		\item \(\textrm{Trans}(M)\)は単項であるか、\(P(M)_0\)が零項でありかつ\(\textrm{Lng}(P(M)) = 2\)である。
	\end{penumerate}
\end{proposition}

\begin{hideableproof}
	\begin{indented}
		\item \nameref{Redが単項性を保つこと}と\nameref{Transの(IncrFirst,Red)不変P同変性}と\nameref{Transの最左単項成分の左端の基本性質}より\(M\)が簡約の場合に帰着される。
		\item \(M\)が簡約ならば、\(\textrm{Trans}\)の再帰的定義から、\(\textrm{Lng}(M)\)に関する数学的帰納法より即座に従う。
	\end{indented}
\end{hideableproof}

\begin{corollary}[\(\textrm{Trans}\)と非可算基数の関係]\label{Transと非可算基数の関係}
	任意の\(M \in RT_{\textrm{PS}}\)と\(v \in \mathbb{N}\)に対し、以下は同値である:
	\begin{penumerate}
		\item \(\textrm{Trans}(M) = D_v 0\)である。
		\item \(v = 0\)かつ\(M = ((0,0),(0,0))\)であるか、または\(v > 0\)かつ\(M = ((v,v))\)である。
	\end{penumerate}
\end{corollary}

\begin{hideableproof}
	\begin{indented}
		\item (2)ならば(1)であることは\(\textrm{Trans}\)の定義より従う。
		\item (1)とする。
		\begin{indented}
			\item \(M\)の簡約性と\nameref{簡約性と係数の関係}から\(M\)は条件(A)と(B)を満たす。特に\(M\)が条件(B)を満たすことから、\(M_{0,0} = M_{1,0}\)である。
			\item \(v = 0\)とする。
			\begin{indented}
				\item \(M_{1,0} > 0\)と仮定すると、\nameref{Transの最左単項成分の左端の基本性質} (2)から\(\textrm{Trans}(M)\)の最左単項成分の左端は\(D_{M_{1,0}} \neq D_0 = D_v\)となり矛盾する。従って\(M_{1,0} = 0\)である。
				\item \(\textrm{Lng}(M) = 1\)ならば\(M= ((0,0))\)すなわち\(\textrm{Trans}(M) = 0 \neq D_v 0\)となり矛盾する。従って\(\textrm{Lng}(M) > 1\)である。
				\item \(M' := (M_j)_{j=0}^{1}\)と置く。
				\item \(\textrm{Lng}(M') = 2\)より\(M'\)は零項でない。\(M'\)が単項ならば、\nameref{2列ペア数列の基本性質}と\nameref{PredのTransに関する降下性}から\(D_0 0 = D_v 0 = \textrm{Trans}(M) \geq \textrm{Trans}(M') = D_0 D_{M_{1,1}} 0 > D_0 0\)となり矛盾する。従って\(M'\)は複項であり、\(M_{0,1} \leq M_{0,0} = 0\)すなわち\(M_{0,1} = 0\)となる。
				\item \nameref{簡約性の切片への遺伝性}より\(M'\)は簡約であり、\nameref{簡約性と係数の関係}から\(M'\)は条件(A)と(B)を満たす。特に\(M'\)が複項でかつ条件(B)を満たすことから、\(M_{1,1} = M_{0,1} = 0\)となる。
				\item \(\textrm{Trans}(M') = \textrm{Trans}(M') = ((0,0),(0,0)) = D_0 0 = \textrm{Trans}(M)\)であるので、\nameref{PredのTransに関する降下性}より\(M = M' = ((0,0),(0,0))\)である。
			\end{indented}
			\item \(v > 0\)とする。
			\begin{indented}
				\item \(P(M)_0\)が零項であると仮定すると、\(M_0 = (0,0)\)であり\(P\)の定義より\(M_{0,1} = 0\)となり、\(M\)が条件(B)を満たすことから\(M_{1,1} = M_{0,1} = 0\)となるが、\nameref{Transの最左単項成分の左端の基本性質}より\(D_v 0 = \textrm{Trans}(M)\)の唯一の単項成分\(D_v 0\)の左端が\(D_{M_{1,1}} = D_0 \neq D_v\)となり矛盾する。従って\(P(M)_0\)は零項でない。
				\item \(P(M)_0\)が零項でないことと\nameref{Transが単項性を保つこと}から\(M\)は単項であり、\nameref{Transの最左単項成分の左端の基本性質}から\(\textrm{Trans}(M)\)の左端は\(D_{M_{1,0}}\)となるので、\(M_{1,0} = v > 0\)である。
				\item \(\textrm{Trans}((M_0)) = D_{M_{1,0}} 0 = D_v 0 = \textrm{Trans}(M)\)となるので、\nameref{PredのTransに関する降下性}より\(M = (M_0) = ((M_{1,0},M_{1,0})) = ((v,v))\)である。
			\end{indented}
		\end{indented}
	\end{indented}
\end{hideableproof}

\begin{corollary}[左端第\(1\)基点のMarkの基本性質]\label{左端第1基点のMarkの基本性質}
	任意の\(M \in RT_{\textrm{PS}} \cap PT_{\textrm{PS}}\)に対し以下が成り立つ:
	\begin{penumerate}
		\item \((M,0) \in T_{\textrm{PS}}^{\textrm{Marked}}\)である。
		\item 任意の\(m \in \mathbb{N}\)に対し、\((M,m) \in T_{\textrm{PS}}^{\textrm{Marked}}\)ならば、\(\textrm{Mark}(M,m) = \textrm{Trans}(M)\)である必要十分条件は\(m = 0\)である。
	\end{penumerate}
\end{corollary}

\begin{hideableproof}
	\begin{indented}
		\item \(\textrm{Mark}\)の再帰的定義中に導入した記号を\((M,m)\)に対して定める。
		\item \(j_1\)に関する数学的帰納法により示す。
		\item \(j_1 = 1\)とする。
		\begin{indented}
			\item \nameref{2列ペア数列の基本性質}より\((M,0) \in T_{\textrm{PS}}^{\textrm{Marked}}\)かつ\(\textrm{Mark}(M,0) = D_{M_{1,0}} D_{M_{1,1}} 0 = \textrm{Trans}(M)\)である。
			\item \(m \neq 0\)ならば\(m = 1\)であり、\nameref{2列ペア数列の基本性質}より\(\textrm{Mark}(M,m) = D_{M_{1,m}} 0 = D_{M_{1,1}} 0 \neq D_{M_{1,0}} D_{M_{1,1}} 0 = \textrm{Trans}(M)\)である。
		\end{indented}
		\item \(j_1 > 1\)とする。
		\begin{indented}
			\item \(M\)は単項であるので\(M \neq ((0,0),(0,0))\)である。
			\item \(m = j_1\)ならば\(m \neq 0\)であり、\nameref{右端第1基点のMarkの基本性質}より\(\textrm{Mark}(M,m) = D_{M_{1,j_1}} 0\)であり、\(j_1 > 1\)かつ\nameref{Transと非可算基数の関係}より\(\textrm{Trans}(M) \neq D_{M_{1,j_1}} 0\)である。
			\item \(m < j_1\)とする。
			\begin{indented}
				\item \(\textrm{Lng}(\textrm{Pred}(M)) = j_1 > 1\)より\(\textrm{Pred}(M)\)は零項でなく、\nameref{Transが零項性を保つこと}より\(t_1 \neq 0\)である。
				\item \nameref{簡約性の切片への遺伝性}と\nameref{単項性の始切片への遺伝性}から\(\textrm{Pred}(M)\)は簡約かつ単項であり、帰納法の仮定から\((\textrm{Pred}(M),0) \in T_{\textrm{PS}}^{\textrm{Marked}}\)かつ、\(c_0 = \textrm{Mark}(\textrm{Pred}(M),m)\)が\(t_1 = \textrm{Trans}(\textrm{Pred}(M))\)と一致する必要十分条件は\(m = 0\)である。
				\item \(m = 0\)ならば\(s_0 c_0 b_0 = t_1 = c_0\)より\(s_0 = ()\)かつ\(b_0 = ()\)であり、\(s_1 = s_0 s_{-1} = s_{-1}\)かつ\(b_1 = b_{-1} b_0 = b_{-1}\)となるので\(\textrm{Mark}(M,0) = \textrm{Mark}(M,m) = s_{-1} c_2 b_{-1} = s_1 c_2 b_1 = \textrm{Trans}(M)\)である。
				\item \(m \neq 0\)ならば\(s_0 c_0 b_0 = t_1 \neq c_0\)より\(s_0 \neq ()\)または\(b_0 \neq ()\)であり、\(\textrm{Mark}(M,m) = s_{-1} c_2 b_{-1} \neq s_0 s_{-1} c_2 b_{-1} b_0 = s_1 c_2 b_1 = \textrm{Trans}(M)\)である。
			\end{indented}
		\end{indented}
	\end{indented}
\end{hideableproof}

\begin{corollary}[\(s_1\)と\(b_1\)の空性と基点の関係]\label{s_1とb_1の空性と基点の関係}
	任意の\(M \in RT_{\textrm{PS}} \cap PT_{\textrm{PS}}\)に対し、\(\textrm{Trans}\)の再帰的定義中に導入した記号を用いると、以下は同値である:
	\begin{penumerate}
		\item \(j_{-1} = 0\)である。
		\item \(s_1 = ()\)かつ\(c_1 = t_1\)かつ\(b_1 = ()\)である。
		\item \(s_1 = ()\)である。
	\end{penumerate}
\end{corollary}

\begin{hideableproof}
	\begin{indented}
		\item (1)が成り立つならば、\nameref{左端第1基点のMarkの基本性質} (2)より\(c_1 = \textrm{Mark}(\textrm{Pred}(M),j_{-1}) = \textrm{Trans}(\textrm{Pred}(M)) = t_1\)となり、\nameref{scb分解の自明性の判定条件}より\(s_1 = ()\)かつ\(b_1 = ()\)となり、従って\(c_1 = s_1 c_1 b_1 = t_1\)である。
		\item (2)が成り立つならば、仮定から\(s_1 = ()\)である。
		\item (3)が成り立つならば、\nameref{scb分解の自明性の判定条件}より\(c_1 = t_1\)となるので、\nameref{左端第1基点のMarkの基本性質} (2)より\(j_{-1} = 0\)となる。
	\end{indented}
\end{hideableproof}

\begin{proposition}[\(\textrm{Mark}\)が順序関係を保つこと]\label{Markが順序関係を保つこと}
	任意の\(M \in T_{\textrm{PS}}\)と\(m_0,m_1 \in \mathbb{N}\)に対し、\((M,m_0), (M,m_1) \in T_{\textrm{PS}}^{\textrm{Marked}}\)ならば以下は同値である:
	\begin{penumerate}
		\item \(m_0 < m_1\)である。
		\item \(\textrm{Mark}(M,m_1) \neq \textrm{Mark}(M,m_0)\)かつ\((\textrm{Mark}(M,m_1),\textrm{Mark}(M,m_0)) \in T_{\textrm{B}}^{\textrm{Marked}}\)である。
	\end{penumerate}
\end{proposition}

\begin{hideableproof}
	\begin{indented}
		\item \(\textrm{Mark}\)の再帰的定義から、\(\textrm{Lng}(M)\)に関する数学的帰納法より即座に従う。
	\end{indented}
\end{hideableproof}

\begin{corollary}[\(s_{-1}\)と\(b_{-1}\)の空性と基点の関係]\label{s_-1とb_-1の空性と基点の関係}
	任意の\((M,m) \in T_{\textrm{PS}}^{\textrm{Marked}}\)に対し、\(M\)が簡約かつ単項ならば、\(\textrm{Trans}\)と\(\textrm{Mark}\)の再帰的定義中に導入した記号を用いると、以下が同値である:
	\begin{penumerate}
		\item \(m = j_{-1}\)である。
		\item \(s_{-1} = ()\)かつ\(b_{-1} = ()\)である。
	\end{penumerate}
\end{corollary}

\begin{hideableproof}
	\begin{indented}
		\item \nameref{scb分解の自明性の判定条件}と\nameref{Markが順序関係を保つこと}から即座に従う。
	\end{indented}
\end{hideableproof}

\begin{proposition}[\(\textrm{Mark}\)の\(\textrm{Trans}\)による表示]\label{MarkのTransによる表示}
	任意の\((M,m) \in T_{\textrm{PS}}^{\textrm{Marked}}\)に対し、\(j_1 := \textrm{Lng}(M)-1\)と置くと、\(j_1 - m > 0\)ならば\(\textrm{Mark}(M,m) = \textrm{Trans}((M_j)_{j=m}^{j_1})\)である。
\end{proposition}

\begin{hideableproof}
	\begin{indented}
		\item \nameref{Transの(IncrFirst,Red)不変P同変性}と\nameref{Markの(IncrFirst,Red,P)不変性}と\nameref{Redが許容性を保つこと}より\(M \in RT_{\textrm{PS}} \cap PT_{\textrm{PS}}\)の場合に帰着される。以下\(M \in RT_{\textrm{PS}} \cap PT_{\textrm{PS}}\)とし、\(\textrm{Trans}\)の再帰的定義中に導入した記号を\(M\)に対して定義し、\(M\)に対しての適用であることを明示するために右肩に\(M\)を乗せて表記する。
		\item \(m < j_1\)かつ\((0,m) \leq_M (0,j_1)\)かつ\((0,j_0^M) <_M^{\textrm{Next}} (0,j_1^M) = (0,j_1)\)より\(m \leq j_0^M\)である。
		\item \(m = 0\)ならば\nameref{左端第1基点のMarkの基本性質}より即座に従う。以下\(m > 0\)とする。この時\(j_1^M = j_1 > m > 0\)より\(j_1^M > 1\)であり、従って\(t_1^M \neq 0\)である。
		\item \(N := \textrm{Red}((M_j)_{j=m}^{j_1})\)と置く。
		\item \nameref{LngのRed不変性}より\(j_1^N = \textrm{Lng}(N)-1 = \textrm{Lng}((M_j)_{j=m}^{j_1})-1 = j_1-m\)である。\((0,j_0^M) <_M^{\textrm{Next}} (0,j_1)\)かつ\((0,m) \leq_M (0,j_1)\)より\(m \leq j_0^M\)かつ\((0,j_0^M-m) <_N^{\textrm{Next}} (0,j_1^M-m) = (0,j_1-m) = (0,j_1^N)\)であるので、\(j_0^N = j_0^M-m\)である。\(N\)は簡約であるので\nameref{簡約性と係数の関係}より\(M\)と\(N\)は条件(A)と(B)を満たす。\nameref{直系先祖による切片とRedとIncrFirstの関係}より\(N\)は単項かつ\(\textrm{IncrFirst}^{M_{0,m}-M_{1,m}}(N) = (M_j)_{j=m}^{j_1}\)であり、すなわち\(N = ((M_{0,j}-M_{0,0}+M_{1,0},M_{1,j}))_{j=m}^{j_1}\)である。
		\item \(\textrm{Trans}\)の再帰的定義中に導入した記号を\(N\)に対して定義し、\(N\)に対しての適用であることを明示するために右肩に\(N\)を乗せて表記する。
		\item
		\item \(j_1-m = 1\)とする。
		\begin{indented}
			\item \(N = ((M_{1,m},M_{1,m}),(M_{1,j_1}-M_{0,m}+M_{1,m},M_{1,j_1}))\)である。
			\item \(j_1^N = j_1-m = 1\)であり、\(N\)は単項より\((0,0) <_N^{\textrm{Next}} (0,1) = (0,j_1^N)\)なので\(j_0^N = 0\)である。特に\(j_0^N\)は\(N\)許容であり、\(j_{-1}^N = j_0^N = 0\)である。
			\item \(D_{v^N} t_2^N = c_1^N = \textrm{Mark}(\textrm{Pred}(N),j_{-1}^N) = \textrm{Mark}(((M_{1,m},M_{1,m})),0) = D_{M_{1,m}} 0\)より\(v^N = M_{1,m}\)かつ\(t_2^N = 0\)である。
			\item \((0,m) \leq_M (0,j_1)\)かつ\(m+1=j_1\)より\((0,m) <_M^{\textrm{Next}} (0,j_1)\)すなわち\(j_0^M = m\)であり、\(m\)の\(M\)許容性から\(m = j_{-1}\)である。従って\nameref{s_-1とb_-1の空性と基点の関係}から\(s_{-1}^M = ()\)かつ\(b_{-1}^M = ()\)である。
			\item \(m\)の\(M\)許容性と\nameref{右端第1基点のMarkの基本性質}から\(D_{v^M} t_2^M = c_1^M = \textrm{Mark}(\textrm{Pred}(M),j_{-1}^M) = \textrm{Mark}(\textrm{Pred}(M),m) = D_{M_{1,m}} 0\)である。従って\(v^M = M_{1,m}\)かつ\(t_2^M = 0\)である。
			\item \(m\)の\(M\)許容性と\(j_0^M+1 = m+1 = j_1 = j_1^M\)から\(M\)は条件(I)か(III)か(VI)を満たす。条件(I)か(III)を満たすならば\(c_2^M = D_{v^M}(t_2^M + D_{M_{1,j_1}} 0) = D_{M_{1,m}} D_{M_{1,j_1}} 0\)であり、条件(VI)を満たすならば\(c_2^M = D_{v^M} D_{M_{1,j_1}} 0 = D_{M_{1,m}} D_{M_{1,j_1}} 0\)であるので、いずれの場合も\(c_2^M = D_{M_{1,m}} D_{M_{1,j_1}} 0\)であり、\(\textrm{Mark}(M,m) = s_{-1}^M c_2^M b_{-1}^M = D_{M_{1,m}} D_{M_{1,j_1}} 0\)である。
			\item \(M_{1,m} = 0\)とする。
			\begin{indented}
				\item \(t_1^N = \textrm{Trans}(\textrm{Pred}(N)) = \textrm{Trans}((M_{1,m},M_{1,m})) = 0\)である。
				\item 従って\nameref{Transの(IncrFirst,Red)不変P同変性}と\(\textrm{Trans}\)の定義から\(\textrm{Trans}((M_j)_{j=m}^{j_1}) = \textrm{Trans}(N) = D_0 D_{N_{1,1}} 0 = D_{M_{1,m}} D_{M_{1,j_1}} 0\)である。
			\end{indented}
			\item \(M_{1,m} > 0\)とする。
			\begin{indented}
				\item \(t_1^N = \textrm{Trans}(\textrm{Pred}(N)) = \textrm{Trans}((M_{1,m},M_{1,m})) = D_{M_{1,m}} 0 \neq 0\)であるので、\(N\)に対し条件(I)~(VI)が意味を持つ。
				\item \(j_{-1}^N = 0\)と\nameref{s_1とb_1の空性と基点の関係}から\(s_1^N = ()\)かつ\(b_1^N = ()\)である。
				\item \(M_{1,m} < M_{1,j_1}\)とする。
				\begin{indented}
					\item \(N_{1,j_1^N} = N_{1,1} = M_{1,j_1} > M_{1,m} > 0\)である。\((0,0) <_N^{\textrm{Next}} (0,1)\)かつ\(N_{1,0} = M_{1,m} < M_{1,j_1} = N_{1,1}\)より\((1,0) <_N^{\textrm{Next}} (1,1)\)であり、\(j_0^N+1 = 1 = j_1^N\)と\(N\)が条件(B)を満たすことから\(N_{1,j_0^N}+1 = N_{1,j_1^N}\)である。従って\(N\)は条件(VI)を満たす。
					\item \(c_2^N = D_{v^N} D_{N_{1,j_1^N}} 0 = D_{M_{1,m}} D_{M_{1,j_1}} 0\)であるので、\(\textrm{Trans}(N) = s_1^N c_2^N b_1^N = D_{M_{1,m}} D_{M_{1,j_1}} 0\)である。
				\end{indented}
				\item \(M_{1,m} \geq M_{1,j_1}\)とする。
				\begin{indented}
					\item \(j_0^N\)が\(N\)許容かつ\(N_{1,j_0^N} = M_{1,m} \geq M_{1,j_1} = N_{1,j_1^N}\)より、\(N\)は条件(I)か(III)を満たす。
					\item \(c_2^N = D_{v^N}(t_2^N + D_{N_{1,j_1^N}} 0) = D_{M_{1,m}}(0+D_{M_{1,j_1}} 0) = D_{M_{1,m}} D_{M_{1,j_1}} 0\)であるので、\(\textrm{Trans}(N) = s_1^N c_2^N b_1^N = D_{M_{1,m}} D_{M_{1,j_1}} 0\)である。
				\end{indented}
			\end{indented}
			\item 従っていずれの場合も\(\textrm{Trans}(N) = s_1^N c_2^N b_1^N = D_{M_{1,m}} D_{M_{1,j_1}} 0\)であり、\nameref{Transの(IncrFirst,Red)不変P同変性}から\(\textrm{Trans}((M_j)_{j=m}^{j_1}) = \textrm{Trans}(N) = D_{M_{1,m}} D_{M_{1,j_1}} 0 = \textrm{Mark}(M,m)\)である。
		\end{indented}
		\item
		\item \(M \in RT_{\textrm{PS}} \cap PT_{\textrm{PS}}\)の条件下で\(\textrm{Mark}(M,m) = \textrm{Trans}((M_j)_{j=m}^{j_1})\)となることを、\(j_1\)に関する数学的帰納法で示す。
		\item \(j_1 = 2\)ならば\(j_1 > m > 0\)より\(m = 1\)であり\(j_1-m = 1\)となるので既に示した。
		\item \(j_1 > 2\)とする。\(j_1 - m = 1\)ならば従うことは既に示したので、以下\(j_1-m > 1\)とする。
		\item \(\textrm{Red}(R)\)が簡約かつ単項であり、 \(\textrm{Lng}(N)-1 = j_1-m > 1\)と\nameref{Transが零項性を保つこと}より\(t_1^N \neq 0\)であるので、\(\textrm{Red}(N)\)に対して条件(I)~(VI)が意味を持つ。
		\begin{indented}
			\item \nameref{基点の切片への遺伝性}より\((\textrm{Pred}(M),m) \in T_{\textrm{PS}}^{\textrm{Marked}}\)であり\(\textrm{Lng}(\textrm{Pred}(M))-1 = j_1-1 < j_1\)であるので、\nameref{RedとPredの可換性}と帰納法の仮定より
			\begin{eqnarray*}
			c_0^M & = & \textrm{Mark}(\textrm{Pred}(M),m) = \textrm{Trans}((\textrm{Pred}(M)_j)_{j=m}^{j_1-1}) = \textrm{Trans}((M_j)_{j=m}^{j_1-1}) = \textrm{Trans}((M_{0,j}-M_{0,m}+M_{1,m},M_{1,j})_{j=m}^{j_1-1}) = \textrm{Trans}(\textrm{Pred}(N)) = t_1^N \\
			c_1^M & = & \textrm{Mark}(\textrm{Pred}(M),j_{-1}^M) = \textrm{Trans}((M_j)_{j=j_{-1}^M}^{j_1-1})
			\end{eqnarray*}
			\item である。また\(N\)が単項より\((N,0) \in T_{\textrm{PS}}^{\textrm{Marked}}\)であり、\nameref{基点の切片への遺伝性}より\((\textrm{Pred}(N),0) \in T_{\textrm{PS}}^{\textrm{Marked}}\)となる。\(\textrm{Lng}(\textrm{Pred}(N))-1 = j_1^N-1 = j_1-m-1 < j_1\)であるので、\nameref{Transの(IncrFirst,Red)不変P同変性}と帰納法の仮定より
			\begin{eqnarray*}
			c_0^N & = & \textrm{Mark}(\textrm{Pred}(N),0) = \textrm{Trans}(\textrm{Pred}(N)) = \textrm{Trans}((M_{0,j}-M_{0,m}+M_{1,m},M_{1,j})_{j=m}^{j_1-1}) = \textrm{Trans}((M_j)_{j=m}^{j_1-1}) = c_0^M \\
			c_1^N & = & \textrm{Mark}(\textrm{Pred}(N),j_{-1}^N) = \textrm{Trans}((\textrm{Pred}(N)_j)_{j=j_{-1}^N}^{j_1^N-1}) = \textrm{Trans}((M_{0,j}-M_{0,m}+M_{1,m},M_{1,j})_{j=j_{-1}^M}^{j_1-1}) = \textrm{Trans}((M_j)_{j=j_{-1}^M}^{j_1-1}) = c_1^M
			\end{eqnarray*}
			\item である。\(D_{v^M} t_2^M = c_1^M = c_1^N = D_{v^N} t_2^N\)となるので\(v^M = v^N\)かつ\(t_2^M = t_2^N\)であり、\(s_{-1}^M c_1^M b_{-1}^M = c_0^M = t_1^N = s_1^N c_1^N b1^N = s_1^N c_1^M b_1^N\)より\(s_{-1}^M = s_1^N\)かつ\(b_{-1}^M = b_1^N\)である。以上より
			\begin{eqnarray*}
			\textrm{Trans}((M_j)_{j=m}^{j_1}) = \textrm{Trans}(N) = s_1^N c_2^N b_1^N = s_{-1}^M c_2^M b_{-1}^M = \textrm{Mark}(M,m)
			\end{eqnarray*}
			\item である。
		\end{indented}
	\end{indented}
\end{hideableproof}


\subsection{許容的親子関係}

\(M \in T_{\textrm{PS}}\)とする。\(\mathbb{Z}^2\)上の二項関係\(<_M^{\textrm{NextAdm}}\)を以下のように定める:
\begin{nenumerate}
	\item \((i_0,j_0), (i_1,j_1) \in \mathbb{Z}^2\)に対し、\((i_0,j_0)  <_M^{\textrm{NextAdm}} (i_1,j_1)\)であるとは、以下を満たすということである:
	\begin{nenumerate}
		\item \((i_0,j_0) \leq_M (i_1,j_1)\)である。
		\item \(j_0 < j_1\)である。
		\item \(j_0\)は\(M\)許容である。
		\item 任意の\(j \in \mathbb{N}\)に対し、\(j_0 < j < j_1\)ならば以下のいずれかを満たす:
		\begin{nenumerate}
			\item \((i_0,j) \leq_M (i_1,j_1)\)でない。
			\item \(j\)は非\(M\)許容である。
		\end{nenumerate}
	\end{nenumerate}
\end{nenumerate}

\begin{proposition}[\(\textrm{Adm}_M\)と\(<_M^{\textrm{NextAdm}}\)の関係]\label{AdmとNextAdmの関係}
	\(M \in T_{\textrm{PS}}\)とし、\(j_1 := \textrm{Lng}(M) - 1\)と置く。任意の\(i \in \{0,1\}\)に対し、\((i,j_0) <_M^{\textrm{Next}} (i,j_1)\)を満たす一意な\(j_0 \in \mathbb{N}\)が存在するならば。\(j_{-1} := \textrm{Adm}_M(j_0)\)と置くと\((i,j_{-1}) <_M^{\textrm{NextAdm}} (i,j_1)\)である。
\end{proposition}

\begin{hideableproof}
	\begin{indented}
		\item \((1,j_{-1}) \leq_M (1,j_0)\)かつ\((i,j_0) <_M^{\textrm{Next}} (i,j_1)\)より\((i,j_{-1}) \leq_M (i,j_1)\)である。更に\(j_{-1}\)は\(M\)許容である。
		\item \(j \in \mathbb{N}\)とし、\(j_{-1} < j < j_1\)とする。\((i,j) \leq_M (i,j_1)\)ならば、\((i,j_0) <_M^{\textrm{Next}} (i,j_1)\)より\(j \leq j_0\)であり、従って\(\textrm{Adm}_M(j) \leq j_{-1} < j\)となるので\(j\)は\(M\)許容でない。
		\item 以上より、\((i,j_{-1}) <_M^{\textrm{NextAdm}} (i,j_1)\)である。
	\end{indented}
\end{hideableproof}

\begin{proposition}[\(\textrm{Trans}\)と\(<_M^{\textrm{NextAdm}}\)の関係]\label{TransとNextAdmの関係}
	\(M \in T_{\textrm{PS}}\)とし、\(j_1 := \textrm{Lng}(M) - 1\)と置く。\((0,j_0) <_M^{\textrm{NextAdm}} (0,j_1)\)を満たす一意な\(j_0 \in \mathbb{N}\)が存在するならば、一意な\((s_0,b_0) \in (\Sigma^{< \omega})^2\)が存在し、以下を満たす:
	\begin{penumerate}
		\item \((s_0,\textrm{Mark}(\textrm{Pred}(M),j_0),b_0)\)は\(\textrm{Trans}(\textrm{Pred}(M))\)のscb分解である。
		\item \((s_0,\textrm{Mark}(M,j_0),b_0)\)は\(\textrm{Trans}(M)\)のscb分解である。
	\end{penumerate}
\end{proposition}

\begin{hideableproof}
	\begin{indented}
		\item \(\textrm{Trans}\)の再帰的定義と\nameref{直系先祖のRed不変性}から、\(M \in RT_{\textrm{PS}} \cap PT_{\textrm{PS}}\)の場合に帰着される。
		\item \(M \in RT_{\textrm{PS}} \cap PT_{\textrm{PS}}\)ならば\(\textrm{Trans}\)の定義と\nameref{直系先祖のRed不変性}から即座に従う。
	\end{indented}
\end{hideableproof}

\begin{corollary}[\(\textrm{Mark}\)と\(<_M^{\textrm{NextAdm}}\)の関係]\label{MarkとNextAdmの関係}
	\(M \in T_{\textrm{PS}}\)とし、\(j_1 := \textrm{Lng}(M) - 1\)と置く。\((0,j_0) <_M^{\textrm{NextAdm}} (0,j_1)\)を満たす一意な\(j_0 \in \mathbb{N}\)が存在するとする。任意の\(j \in \mathbb{N}\)に対し、\((0,j) \leq_M (0,j_0)\)ならば、一意な\((s_0,b_0) \in (\Sigma^{< \omega})^2\)が存在し、以下を満たす:
	\begin{penumerate}
		\item \((s_0,\textrm{Mark}(\textrm{Pred}(M),j_0),b_0)\)は\(\textrm{Mark}(\textrm{Pred}(M),j)\)のscb分解である。
		\item \((s_0,\textrm{Mark}(M,j_0),b_0)\)は\(\textrm{Mark}(M,j)\)のscb分解である。
	\end{penumerate}
\end{corollary}

\begin{hideableproof}
	\begin{indented}
		\item \nameref{TransとNextAdmの関係}と\nameref{Markが順序関係を保つこと}から即座に従う。
	\end{indented}
\end{hideableproof}

\begin{corollary}[\(\textrm{Trans}\)の\(\textrm{Mark}\)と\(\textrm{Pred}\)による表示]\label{TransのMarkとPredによる表示}
	任意の\((M,m) \in T_{\textrm{PS}}^{\textrm{Marked}}\)に対し、\(m < \textrm{Lng}(M) - 1\)ならば一意な\((s_0,b_0) \in (\Sigma^{< \omega})^2\)が存在し、以下を満たす:
	\begin{penumerate}
		\item \((s_0,\textrm{Mark}(\textrm{Pred}(M),m),b_0)\)は\(\textrm{Trans}(\textrm{Pred}(M))\)のscb分解である。
		\item \((s_0,\textrm{Mark}(M,m),b_0)\)は\(\textrm{Trans}(M)\)のscb分解である。
	\end{penumerate}
\end{corollary}

\begin{hideableproof}
	\begin{indented}
		\item \nameref{TransとNextAdmの関係}と\nameref{MarkとNextAdmの関係}から即座に従う。
	\end{indented}
\end{hideableproof}

\begin{corollary}[\(\textrm{Trans}\)の\(\textrm{Mark}\)と切片による表示]\label{TransのMarkと切片による表示}
	任意の\((M,m) \in RT_{\textrm{PS}}^{\textrm{Marked}}\)に対し、\(0 < m < \textrm{Lng}(M) - 1\)ならば一意な\((s_0,b_0) \in (\Sigma^{< \omega})^2\)が存在し、以下を満たす:
	\begin{penumerate}
		\item \((s_0,D_{M_{1,m}} 0,b_0)\)は\(\textrm{Trans}((M_j)_{j=0}^{m})\)のscb分解である。
		\item \((s_0,\textrm{Mark}(M,m),b_0)\)は\(\textrm{Trans}(M)\)のscb分解である。
	\end{penumerate}
\end{corollary}

\begin{hideableproof}
	\begin{indented}
		\item \(j_1 := \textrm{Lng}(M) - 1\)と置く。
		\item \(\textrm{Trans}\)の再帰的定義と\nameref{直系先祖のRed不変性}から、\(M \in RT_{\textrm{PS}} \cap PT_{\textrm{PS}}\)の場合に帰着される。以下\(M \in RT_{\textrm{PS}} \cap PT_{\textrm{PS}}\)として、\(\textrm{Trans}\)の再帰的定義中に導入した記号を\(M\)に対して定義し、\(M\)に対しての適用であることを明示するために右肩に\(M\)を乗せて表記する\footnote{\(M\)以外には適用しないが、命題の主張中の\(s_0, b_0\)と\(s_0^M, b_0^M\)を区別する必要がある。}。
		\item \(j_1-m\)に関する数学的帰納法で示す。
		\item \(j_1-m = 0\)ならば\nameref{右端第1基点のMarkの基本性質}より従う。
		\item \(j_1-m = 1\)とする。
		\begin{indented}
			\item \(s_0 := s_1^M\)と置く。
			\item \(b_0 := b_1^M\)と置く。
			\item \((0,m) \leq_M (0,j_1) = (0,j_1^M)\)かつ\(m+1 = j_1 = j_1^M\)より\(j_0^M = m\)であり、\(m\)が\(M\)許容であることから\(j_{-1}^M = j_0^M = m\)である。従って\nameref{s_-1とb_-1の空性と基点の関係}より\(s_{-1}^M = ()\)かつ\(b_{-1}^M = ()\)である。
			\item \nameref{右端第1基点のMarkの基本性質}より\(c_1^M = \textrm{Mark}(\textrm{Pred}(M),j_{-1}^M) = \textrm{Mark}(\textrm{Pred}(M),j_1-1) = D_{M_{1,m}} 0\)であるので、\(\textrm{Trans}((M_j)_{j=0}^{m}) = \textrm{Trans}(\textrm{Pred}(M)) = t_1^M = s_1^M c_1^M b_1^M = s_0 D_{M_{1,m}} b_0\)である。
			\item \(\textrm{Trans}(M) = s_1^M c_2^M b_1^M = s_1^M s_{-1}^M c_2^M b_{-1}^M b_1^M = s_0 \textrm{Mark}(M,m) b_0\)である。
		\end{indented}
		\item \(j_1-m > 1\)とする。
		\begin{indented}
			\item \nameref{簡約性の切片への遺伝性}より\(\textrm{Pred}(M)\)は簡約であり、\(0 < j_1-m-1 \leq j_1-1 < j_1\)かつ\nameref{単項性の始切片への遺伝性}より\(\textrm{Pred}(M)\)は単項であり、\(m < j_1 - 1\)かつ\nameref{基点の切片への遺伝性}より\((\textrm{Pred}(M),m) \in T_{\textrm{PS}}^{\textrm{Mark}}\)である。
			\item 従って帰納法の仮定から、一意な\((s_0,b_0) \in (\Sigma^{< \omega})^2\)が存在し、以下を満たす:
			\begin{penumerate}
				\item \(b_0\)は\(\underline{)}\)のみからなる。
				\item \(\textrm{Trans}((\textrm{Pred}(M)_j)_{j=0}^{m}) = s_0 D_{\textrm{Pred}(M)_{1,m}} 0 b_0 = s_0 D_{M_{1,m}} 0 b_0\)である。
				\item \(\textrm{Trans}(\textrm{Pred}(M)) = s_0 \textrm{Mark}(\textrm{Pred}(M),m) b_0 = s_0 c_0^M b_0\)である。
			\end{penumerate}
			\item \nameref{TransのMarkとPredによる表示}から\(\textrm{Trans}(M) = s_0 \textrm{Mark}(M,m) b_0\)である。更に\(m \leq j_1-1\)より\((M_j)_{j=0}^{m} = (\textrm{Pred}(M)_j)_{j=0}^{m}\)であるので、\(\textrm{Trans}((M_j)_{j=0}^{m}) = \textrm{Trans}((\textrm{Pred}(M)_j)_{j=0}^{m}) = s_0 D_{M_{1,m}} 0 b_0\)である。
		\end{indented}
	\end{indented}
\end{hideableproof}

\begin{corollary}[\(\textrm{RightNodes}\)と\(\textrm{Mark}\)の関係]\label{RightNodesとMarkの関係}
	任意の\((M,m) \in RT_{\textrm{PS}}^{\textrm{Marked}}\)に対し、\(0 < m < \textrm{Lng}(M) - 1\)ならば、ある\(a_0, a_1 \in \mathbb{N}^{< \omega}\)が存在して以下を満たす:
	\begin{penumerate}
		\item \(\textrm{RightNode}(\textrm{Trans}(M)) = a_0 \oplus_{\mathbb{N}} (M_{1,m}) \oplus_{\mathbb{N}} a_1\)である。
		\item \(\textrm{RightNode}(\textrm{Trans}((M_j)_{j=0}^{m})) = a_0 \oplus_{\mathbb{N}} (M_{1,m})\)である。
		\item \(\textrm{RightNode}(\textrm{Mark}(M,m)) = (M_{1,m}) \oplus_{\mathbb{N}} a_1\)である。
	\end{penumerate}
\end{corollary}

\begin{hideableproof}
	\begin{indented}
		\item \nameref{単項性の直系先祖による切片への遺伝性}と\nameref{Markの左端の基本性質}と\nameref{RightNodesと部分表現の関係}から即座に従う。
	\end{indented}
\end{hideableproof}

写像
\begin{eqnarray*}
\textrm{RightAnces} \colon T_{\textrm{PS}} & \to & \mathbb{N}^{< \omega} \\
M & \mapsto & \textrm{RightAnces}(M)
\end{eqnarray*}
を以下のように定義する:
\begin{nenumerate}
	\item \(\textrm{Trans}\)の再帰的定義中に導入した記号を用いる。
	\item \(M\)が簡約かつ\(j_1 = 0\)とする。
	\begin{nenumerate}
		\item \(M_0 = (0,0)\)ならば\(\textrm{RightAnces}(M) := ()\)である。
		\item \(M_0 \neq (0,0)\)ならば\(\textrm{RightAnces}(M) := (M_{1,0})\)である。
	\end{nenumerate}
	\item \(M\)が簡約かつ\(j_1 > 0\)かつ単項とする。
	\begin{nenumerate}
		\item \(\textrm{Pred}(M)\)が零項ならば\(\textrm{RightAnces}(M) := (0,M_{1,j_1})\)である。
		\item \(\textrm{Pred}(M)\)が零項でないとする\footnote{この時\nameref{Transが零項性を保つこと}から\(\textrm{Trans}(\textrm{Pred}(M)) \neq 0\)であるので、\(M\)に対して条件(I)~(VI)が意味を持つ。}。
		\begin{nenumerate}
			\item \((M_j)_{j=0}^{j_{-1}}\)が零項ならば\(a := (0)\)と置く。
			\item \((M_j)_{j=0}^{j_{-1}}\)が零項でないならば\(a := \textrm{RightAnces}((M_j)_{j=0}^{j_{-1}})\)と置く。
			\item \(M\)が条件(I)か(III)か(V)か(VI)を満たすならば\(\textrm{RightAnces}(M) := a \oplus_{\mathbb{N}} (M_{1,j_1})\)である。
			\item \(M\)が条件(II)か(IV)を満たすならば\(\textrm{RightAnces}(M) := a \oplus_{\mathbb{N}} (M_{1,j_0},M_{1,j_1})\)である。
		\end{nenumerate}
	\end{nenumerate}
	\item \(M\)が簡約かつ複項とする。
	\begin{nenumerate}
		\item \(J_1 := \textrm{Lng}(P(M))-1\)と置く。
		\item \(P(M)_{J_1} = ((0,0))\)ならば\(\textrm{RightAnces}(M) := (0)\)である。
		\item \(P(M)_{J_1} \neq ((0,0))\)ならば\(\textrm{RightAnces}(M) := \textrm{RightAnces}(P(M)_{J_1})\)である。
	\end{nenumerate}
	\item \(M\)が簡約でないならば\(\textrm{RightAnces}(M) := \textrm{RightAnces}(\textrm{Red}(M))\)である。
\end{nenumerate}

\begin{proposition}[\(\textrm{RightNodes}\)と\(\textrm{RightAnces}\)の関係]\label{RightNodesとRightAncesの関係}
	任意の\(M \in T_{\textrm{PS}}\)に対し、\(\textrm{RightAnces}(M) = \textrm{RightNodes}(\textrm{Trans}(M))\)である。
\end{proposition}

\begin{hideableproof}
	\begin{indented}
		\item \(\textrm{RightNodes}\)と\(\textrm{Trans}\)と\(\textrm{RightAnces}\)の再帰的定義から、\(M \in RT_{\textrm{PS}} \cap PT_{\textrm{PS}}\)である場合に帰着される。以下\(M \in RT_{\textrm{PS}} \cap PT_{\textrm{PS}}\)とし、\(\textrm{RightNodes}\)の再帰的定義中に導入した記号と\(\textrm{Trans}\)の再帰的定義中に導入した記号を用いる。
		\item \(M\)が単項より\(j_1 > 0\)である。また\(M\)が簡約より\nameref{簡約性と係数の関係}から\(M\)は条件(A)と(B)を満たす。
		\item \(\textrm{RightAnces}(M) = \textrm{RightNodes}(\textrm{Trans}(M))\)であることを\(j_1\)に関する数学的帰納法で示す。
		\item \(j_1 = 1\)とする。
		\begin{indented}
			\item \nameref{2列ペア数列の基本性質} (1)より\(\textrm{Trans}(M) = D_{M_{1,0}} D_{M_{1,1}} 0\)であり、\(\textrm{RightNodes}(\textrm{Trans}(M)) = \textrm{RightNodes}(D_{M_{1,0}} D_{M_{1,1}} 0) = (M_{1,0},M_{1,1})\)である。
			\item \(M\)が単項より\((0,0) <_M^{\textrm{Next}} (0,1) = (0,j_1)\)であるので\(j_0 = 0\)である。従って\(j_0\)は\(M\)許容であり\(j_{-1} = j_0 = 0\)である。
			\item \(M_{1,0} = 0\)ならば、\(\textrm{Pred}(M) = ((M_{0,0},0))\)であるので\(\textrm{Pred}(M)\)は零項であり\(\textrm{RightAnces}(M) = (0,M_{1,j_1}) = (M_{1,0},M_{1,1})\)である。
			\item \(M_{1,0} > 0\)とする。
			\begin{indented}
				\item \(\textrm{Pred}(M) = ((M_{0,0},M_{1,0}))\)より\(\textrm{Pred}(M)\)は零項でなく、\(a = \textrm{RightAnces}((M_j)_{j=0}^{j_{-1}}) = \textrm{RightAnces}((M_{0,0},M_{1,0})) = (M_{1,0})\)である。
				\item \(j_0\)が\(M\)許容であり\(j_0+1 = 1 = j_1\)であるので、\(M\)は条件(I)か(III)か(VI)を満たす。従って\(\textrm{RightAnces}(M) = a \oplus_{\mathbb{N}} (M_{1,j_1}) = (M_{0,1},M_{1,1})\)である。
			\end{indented}
			\item 以上よりいずれの場合も\(\textrm{RightAnces}(M) = \textrm{RightNodes}(\textrm{Trans}(M))\)である。
		\end{indented}
		\item \(j_1 > 1\)とする\footnote{この時\(\textrm{Pred}(M)\)は零項でないので、\nameref{Transが零項性を保つこと}から\(M\)に対して条件(I)~(VI)が意味を持つ。}。
		\begin{indented}
			\item \nameref{Markの左端の基本性質}より\(v = M_{1,0}\)である。
			\item \((M_j)_{j=0}^{j_{-1}}\)が零項であるとする。
			\begin{indented}
				\item この時\(a = (0)\)かつ\(j_{-1} = 0\)かつ\(v = M_{1,0} = 0\)であり、\nameref{s_1とb_1の空性と基点の関係}より\(s_1 = ()\)かつ\(b_1 = ()\)である。従って\(\textrm{Trans}(M) = s_1 c_2 b_1 = c_2\)である。
				\item \(M\)が条件(I)か(III)か(V)か(VI)を満たすとする。
				\begin{indented}
					\item \(\textrm{RightAnces}(M) = a \oplus_{\mathbb{N}} (M_{1,j_1}) = (0,M_{1,j_1})\)である。
					\item \(M\)が条件(VI)を満たすか否かに従って\(c_2 = D_v D_{M_{1,j_1}} 0 = D_0 D_{M_{1,j_1}} 0\)または\(c_2 = D_v(t_2 + D_{M_{1,j_1}} 0) = D_0(t_2 + D_{M_{1,j_1}} 0)\)となるので、いずれの場合も\(\textrm{RightNodes}(\textrm{Trans}(M)) = (0,M_{1,j_1}) = \textrm{RightAnces}(M)\)である。
				\end{indented}
				\item \(M\)が条件(II)か(IV)を満たすならば、\(\textrm{RightAnces}(M) = a \oplus_{\mathbb{N}} (M_{1,j_0},M_{1,j_1}) = (0,M_{1,j_0},M_{1,j_1})\)であり、\(c_2 = D_v(t_3 + D_{M_{1,j_0}}(t_4 + D_{M_{1,j_1}} 0)) = D_0(t_3 + D_{M_{1,j_0}}(t_4 + D_{M_{1,j_1}} 0))\)より\(\textrm{RightNodes}(M) = (0,M_{1,j_0},M_{1,j_1}) = \textrm{RightAnces}(M)\)である。
			\end{indented}
			\item \((M_j)_{j=0}^{j_{-1}}\)が零項であるとする。
			\begin{indented}
				\item \(N := (M_j)_{j=0}^{j_{-1}}\)と置く。
				\item \(\textrm{Lng}(N)-1 = j_{-1} \leq j_0 < j_1\)であるので、帰納法の仮定より\(\textrm{RightAnces}(N) = \textrm{RightNodes}(N)\)である。
				\item \nameref{右端第2基点のMarkの基本性質}より\(\textrm{Mark}(M,j_{-1}) = c_2\)であるので\(\textrm{Trans}(M) = s_1 c_2 b_1 = s_1 \textrm{Mark}(M,j_{-1}) b_1\)であり、\nameref{TransのMarkと切片による表示}より\(\textrm{Trans}(N) = s_1 D_{M_{1,j_{-1}}} b_1\)である。
				\item \(M\)が条件(I)か(III)か(V)か(VI)を満たすとする。
				\begin{indented}
					\item \(\textrm{RightAnces}(M) = a \oplus_{\mathbb{N}} (M_{1,j_1}) = \textrm{RightAnces}(N) \oplus_{\mathbb{N}} (M_{1,j_1})\)である。
					\item \(M\)が条件(VI)を満たすか否かに従って\(c_2 = D_v D_{M_{1,j_1}} 0 \)または\(c_2 = D_v(t_2 + D_{M_{1,j_1}} 0)\)であるので、いずれの場合も\(\textrm{RightNodes}(\textrm{Mark}(M,j_1)) = \textrm{RightNodes}(c_2) = (v,M_{1,j_1})\)である。従って\nameref{RightNodesとMarkの関係}から\(\textrm{RightNodes}(\textrm{Trans}(M)) = \textrm{RightNodes}(\textrm{Trans}(N)) \oplus_{\mathbb{N}} (M_{1,j_1}) = \textrm{RightAnsces}(N) \oplus_{\mathbb{N}} (M_{1,j_1})\)となる。
				\end{indented}
				\item \(M\)が条件(II)か(IV)を満たすとする。
				\begin{indented}
					\item \(\textrm{RightAnces}(M) = a \oplus_{\mathbb{N}} (M_{1,j_0},M_{1,j_1}) = \textrm{RightAnces}(N) \oplus_{\mathbb{N}} (M_{1,j_0},M_{1,j_1})\)である。
					\item \(c_2 = D_v(t_3 + D_{M_{1,j_0}}(t_4 + D_{M_{1,j_1}} 0))\)であるので、\(\textrm{RightNodes}(\textrm{Mark}(M,j_1)) = \textrm{RightNodes}(c_2) = (v,M_{1,j_0},M_{1,j_1})\)である。従って\nameref{RightNodesとMarkの関係}から\(\textrm{RightNodes}(\textrm{Trans}(M)) = \textrm{RightNodes}(\textrm{Trans}(N)) \oplus_{\mathbb{N}} (M_{1,j_0}M_{1,j_1}) = \textrm{RightAnsces}(N) \oplus_{\mathbb{N}} (M_{1,j_0},M_{1,j_1})\)となる。
				\end{indented}
				\item 以上より、いずれの場合も\(\textrm{RightAnces}(M) = \textrm{RightNodes}(\textrm{Trans}(M))\)である。
			\end{indented}
		\end{indented}
	\end{indented}
\end{hideableproof}

\begin{corollary}[非零項の\(\textrm{RightAnces}\)が非空であること]\label{非零項のRightAncesが非空であること}
	任意の\(M \in T_{\textrm{PS}}\)に対し、以下は同値である:
	\begin{penumerate}
		\item \(M\)は零項である。
		\item \(\textrm{RightAnces}(M) = ()\)である。
	\end{penumerate}
\end{corollary}

\begin{hideableproof}
	\begin{indented}
		\item \nameref{Transが零項性を保つこと}と\nameref{RightNodesとRightAncesの関係}から即座に従う。
	\end{indented}
\end{hideableproof}


\section{停止性}

まずは単項な標準形ペア数列に対し条件(I)~(VI)のそれぞれの下での展開規則を調べ、それによりBuchholzの表記系における展開規則との比較を行い、標準形ペア数列に伴う計算可能関数の全域性(すなわち計算規則の停止性)を証明する。


\subsection{条件(I)の下での展開規則}

\begin{proposition}[条件(I)の下での\(\textrm{Trans}\)と基本列の交換関係]\label{条件(I)の下でのTransと基本列の交換関係}
	任意の\(M \in RT_{\textrm{PS}} \cap PT_{\textrm{PS}}\)と\(n \in \mathbb{N}_{+}\)に対し、\(\textrm{Trans}\)の再帰的定義中に導入した記号を用いると、\(j_1 > 1\)かつ\(M\)が条件(I)を満たすならば\footnotemark{}、以下が成り立つ:
	\begin{penumerate}
		\item \(\textrm{Trans}(M[n]) = \textrm{Trans}(M)[n-1]\)である。
		\item \(\textrm{Trans}(M[n]) < \textrm{Trans}(M)\)である。
	\end{penumerate}
\end{proposition}
\footnotetext{\(j_1 > 1\)より\(t_1 \neq 0\)であるので\(M\)に対し条件(I)が意味を持つ。}

\nameref{条件(I)の下でのTransと基本列の交換関係}を証明するための準備としていくつかの補題を示す。

\begin{lemma}[公差\((1,1)\)のペア数列の\(\textrm{Trans}\)の基本性質]\label{公差(1,1)のペア数列のTransの基本性質}
	任意の\(u,v \in \mathbb{N}\)に対し、\(u < v\)ならば\(M = ((j,j))_{j=u}^{v}\)と置くと\(\textrm{Trans}(M) = D_u D_v 0\)である。
\end{lemma}

\begin{hideableproof}
	\begin{indented}
		\item \(M\)は単項であり、また条件(A)と(B)を満たすので\nameref{簡約性と係数の関係}から\(M\)は簡約である。\(\textrm{Trans}\)の再帰的定義中に導入した記号を\(M\)に対して定義し、\(M\)に対しての適用であることを明示するために右肩に\(M\)を乗せて表記する\footnote{\(M\)にしか適用しないが、主張の中の\(v\)と\(v^M\)を区別する必要がある。}。
		\item \(j_1^M = v-u\)かつ\(j_0^M = j_1^M-1 = v-u-1\)かつ\(j_{-1}^M = 0\)である。\(M_{v-u-1} = (v-1,v-1)\)かつ\(M_{v-u} = (v,v)\)より\((1,j_0^M) = (1,v-u-1) <_M^{\textrm{Next}} (1,v-u) = (1,j_1^M)\)である。
		\item \(v-u\)に関する数学的帰納法で示す。
		\item \(v-u = 1\)とする。
		\begin{indented}
			\item \(j_1^M = v-u = 1\)かつ\(j_0^M = v-u-1 = 0\)より\(j_{-1}^M = 0 = j_0\)となるので、\(j_0^M\)は\(M\)許容である。
			\item \(u = 0\)ならば、\(t_1^M = \textrm{Trans}(\textrm{Pred}(M)) = \textrm{Trans}((0,0)) = 0\)であるので\(\textrm{Trans}(M) = D_0 D_v 0 = D_u D_v 0\)である。
			\item \(u > 0\)とする。
			\begin{indented}
				\item \(t_1^M = \textrm{Trans}(\textrm{Pred}(M)) = \textrm{Trans}((u,u)) = D_u 0 \neq 0\)であるので\(M\)に対し条件(I)~(VI)が意味を持つ。
				\item \(j_{-1}^M = 0\)と\nameref{s_1とb_1の空性と基点の関係}から\(s_1^M = ()\)かつ\(b_1^M = ()\)である。従って\(D_{v^M} t_2^M = c_1^M = s_1^M c_1^M b_1^M = t_1^M = D_u 0\)より\(v^M = u\)かつ\(t_2^M = 0\)である。
				\item \(M_{1,j_1} = v = u+1 = M_{1,j_0}+1 > 0\)かつ\(j_0+1 = 1 = j_1\)であるので\(M\)は条件(VI)を満たす。従って\(c_2^M = D_{v^M} D_{M_{1,j_1^M}} 0 = D_u D_v 0\)であり、\(\textrm{Trans}(M) = s_1^M c_2^M b_1^M = D_u D_v 0\)である。
			\end{indented}
		\end{indented}
		\item \(v-u > 1\)とする。
		\begin{indented}
			\item \(\textrm{Pred}(M) = ((j,j))_{j=u}^{v-1}\)であるので、帰納法の仮定から\(t_1^M = \textrm{Trans}(\textrm{Pred}(M)) = D_u D_{v-1} 0 \neq 0\)であるので\(M\)に対し条件(I)~(VI)が意味を持つ。
			\item \(j_{-1}^M = 0\)と\nameref{s_1とb_1の空性と基点の関係}から\(s_1^M = ()\)かつ\(b_1^M = ()\)であり、\(D_{v^M} t_2^M = c_1^M = D_u D_{v-1} 0\)より\(v^M = u\)かつ\(t_2^M = D_{v-1} 0\)である。
			\begin{indented}
				\item \(M_{1,j_1} = v = u+1 = M_{1,j_0}+1 > 0\)かつ\(j_0+1 = 1 = j_1\)であるので\(M\)は条件(VI)を満たす。従って\(c_2^M = D_{v^M} D_{M_{1,j_1^M}} 0 = D_u D_v 0\)であり、\(\textrm{Trans}(M) = s_1^M c_2^M b_1^M = D_u D_v 0\)である。
			\end{indented}
		\end{indented}
	\end{indented}
\end{hideableproof}

\begin{corollary}[\(\textrm{Pred}\)が公差\((1,1)\)のペア数列の\(\textrm{Trans}\)の基本性質]\label{Predが公差(1,1)のペア数列のTransの基本性質}
	任意の\(u,v,w,w' \in \mathbb{N}\)に対し、\(u < v\)ならば\(M := ((j,j))_{j=u}^{v} \oplus_{\mathbb{N}^2} (w',w)\)と置くと以下が成り立つ:
	\begin{penumerate}
		\item \(w' = v+1\)かつ\(u < w \leq v\)ならば\(\textrm{Trans}(M) = D_u D_v D_w 0\)である。
		\item \(u < w' \leq v\)かつ\(w = w'\)ならば\(\textrm{Trans}(M) = D_u \underline{(} D_v 0 \underline{,} D_w 0 \underline{)}\)である。
		\item \(u+1 < w' \leq v\)かつ\(w < w'\)ならば\(\textrm{Trans}(M) = D_u \underline{(} D_v 0 \underline{,} D_{w'-1} \underline{(} D_v 0 \underline{,} D_w 0 \underline{)} \underline{)}\)である。
		\item \(u+1 = w'\)かつ\(w < w'\)ならば\(\textrm{Trans}(M) = D_u \underline{(} D_v 0 \underline{,} D_w 0 \underline{)}\)である。
	\end{penumerate}
\end{corollary}

\begin{hideableproof}
	\begin{indented}
		\item いずれの場合も\(M\)は単項であり、また条件(A)と(B)を満たすので\nameref{簡約性と係数の関係}から\(M\)は簡約である。\(\textrm{Trans}\)の再帰的定義中に導入した記号を\(M\)に対して定義し、\(M\)に対しての適用であることを明示するために右肩に\(M\)を乗せて表記する\footnote{\(M\)にしか適用しないが、主張の中の\(v\)と\(v^M\)を区別する必要がある。}。
		\item \(\textrm{Pred}(M) = ((j,j))_{j=u}^{v}\)であり、\nameref{公差(1,1)のペア数列のTransの基本性質}より\(t_1^M = \textrm{Trans}(\textrm{Pred}(M)) = D_u D_v 0 \neq 0\)であるので、\(M\)に対して条件(I)~(VI)が意味を持つ。
		\item \(j_1^M = v-u+1\)である。\nameref{左端第1基点のMarkの基本性質}より\(\textrm{Mark}(\textrm{Pred}(M),0) = t_1^M = D_u D_v 0\)であり、\nameref{右端第1基点のMarkの基本性質}より\(\textrm{Mark}(\textrm{Pred}(M),j_1^M-1) = D_{\textrm{Pred}(M)_{1,j_1^M-1}} 0 = D_v 0\)である。
		\item \(w' = v+1\)かつ\(u < w \leq v\)とする。
		\begin{indented}
			\item \(j_0^M = j_1^M-1 = v-u\)であり、\(M_{1,j_0^M} = M_{1,v-u} = v\)かつ\(M_{1,j_1^M} = M_{1,v-u+1} = w \leq v = M_{1,j_0^M}\)より\((1,j_0^M) <_M^{\textrm{Next}} (1,j_1^M)\)でなく、\(j_0^M\)は\(M\)許容であるので\(j_{-1}^M = j_0^M = j_1^M-1\)である。従って\(M\)は条件(I)か(III)を満たす。
			\item \(c_1^M = \textrm{Mark}(\textrm{Pred}(M),j_{-1}^M) = \textrm{Mark}(\textrm{Pred}(M),j_1^M-1) = D_v 0\)であり、\(s_1^M D_v 0 b_1^M = s_1^M c_1^M b_1^M = t_1^M = D_u D_v 0\)より\(s_1^M = D_u\)かつ\(b_1^M = ()\)である。また\(D_{v^M} t_2^M = c_1^M = D_v 0\)より\(v^M = v\)かつ\(t_2^M = 0\)であるので\(c_2^M = D_{v^M}(t_2^M + D_{M_{1,j_1^M}} 0) = D_v(0 + D_w 0) = D_v D_w 0\)である。
			\item 以上より\(\textrm{Trans}(M) = s_1^M c_2^M b_1^M = D_u D_v D_w 0\)である。
		\end{indented}
		\item \(u < w' \leq v\)かつ\(w = w'\)とする。
		\begin{indented}
			\item \(j_0^M = w'-u-1\)であり、\(M_{1,j_0^M} = M_{1,w'-u-1} = w'-1\)かつ\(M_{1,j_1^M} = M_{1,v-u+1} = w = M_{1,j_0^M}+1\)より\((1,j_0^M) <_M^{\textrm{Next}} (1,j_1^M)\)であり、\(j_0^M\)は非\(M\)許容であるので\(j_{-1}^M = 0\)である。従って\(M\)は条件(V)を満たす。
			\item \(c_1^M = \textrm{Mark}(\textrm{Pred}(M),j_{-1}^M) = \textrm{Mark}(\textrm{Pred}(M),0) = D_u D_v 0\)であり、\(j_{-1}^M = 0\)と\nameref{s_1とb_1の空性と基点の関係}から\(s_1^M = ()\)かつ\(b_1^M = ()\)である。また\(D_{v^M} t_2^M = c_1^M = D_u D_v 0\)より\(v^M = u\)かつ\(t_2^M = D_v 0\)であるので\(c_2^M = D_{v^M}(t_2^M + D_{M_{1,j_1^M}} 0) = D_u(D_v 0 + D_w 0) = D_u \underline{(} D_v 0 \underline{,} D_w 0 \underline{)}\)である。
			\item 以上より\(\textrm{Trans}(M) = s_1^M c_2^M b_1^M = D_u \underline{(} D_v 0 \underline{,} D_w 0 \underline{)}\)である。
		\end{indented}
		\item \(u+1 < w' \leq v\)かつ\(w < w'\)とする。
		\begin{indented}
			\item \(j_0^M = w'-u-1 > 0\)であり、\(M_{1,j_0^M} = M_{1,w'-u-1} = w'-1\)かつ\(M_{1,j_1^M} = M_{1,v-u+1} = w \leq w'-1 = M_{1,j_0^M}\)より\((1,j_0^M) <_M^{\textrm{Next}} (1,j_1^M)\)でなく、\(j_0^M\)は非\(M\)許容であるので\(j_{-1}^M = 0\)である。従って\(M\)は条件(IV)を満たす。
			\item \(c_1^M = \textrm{Mark}(\textrm{Pred}(M),j_{-1}^M) = \textrm{Mark}(\textrm{Pred}(M),0) = D_u D_v 0\)であり、\(j_{-1}^M = 0\)と\nameref{s_1とb_1の空性と基点の関係}から\(s_1^M = ()\)かつ\(b_1^M = ()\)である。また\(D_{v^M} t_2^M = c_1^M = D_u D_v 0\)より\(v^M = u\)かつ\(t_2^M = D_v 0\)であり、\(v \geq w' > w'-1\)より\(t_2^M\)の唯一の単項成分\(D_v 0\)の左端は\(D_v \neq D_{w'-1} = D_{M_{1,j_0^M}}\)であるので\(t_3^M = t_2^M = D_v  0\)かつ\(t_4^M = t_2^M = D_v 0\)となる。従って\(c_2^M = D_{v^M}(t_3^M + D_{M_{1,j_0^M}}(t_4^M + D_{M_{1,j_1^M}} 0)) = D_u(D_v 0 + D_{w'-1}(D_v 0 + D_w 0)) = D_u \underline{(} D_v 0 \underline{,} D_{w'-1} \underline{(} D_v 0 \underline{,} D_w 0 \underline{)} \underline{)}\)である。
			\item 以上より\(\textrm{Trans}(M) = s_1^M c_2^M b_1^M = D_u \underline{(} D_v 0 \underline{,} D_{w'-1} \underline{(} D_v 0 \underline{,} D_w 0 \underline{)} \underline{)}\)である。
		\end{indented}
		\item \(u+1 = w'\)かつ\(w < w'\)とする。
		\begin{indented}
			\item \(j_0^M = 0\)であり、\(M_{1,j_0^M} = M_{1,0} = u\)かつ\(M_{1,j_1^M} = M_{1,v-u+1} = w \leq w'-1 = u = M_{1,j_0^M}\)より\((1,j_0^M) <_M^{\textrm{Next}} (1,j_1^M)\)でなく、\(j_0^M\)は\(M\)許容であるので\(j_{-1}^M = j_0^M = 0\)である。従って\(M\)は条件(I)か(III)を満たす。
			\item \(c_1^M = \textrm{Mark}(\textrm{Pred}(M),j_{-1}^M) = \textrm{Mark}(\textrm{Pred}(M),0) = D_u D_v 0\)であり、\(j_{-1}^M = 0\)と\nameref{s_1とb_1の空性と基点の関係}から\(s_1^M = ()\)かつ\(b_1^M = ()\)である。また\(D_{v^M} t_2^M = c_1^M = D_u D_v 0\)より\(v^M = u\)かつ\(t_2^M = D_v 0\)であるので\(c_2^M = D_{v^M}(t_2^M + D_{M_{1,j_1^M}} 0) = D_u(D_v 0 + D_w 0) = D_u \underline{(} D_v 0 \underline{,} D_w 0 \underline{)}\)である。
			\item 以上より\(\textrm{Trans}(M) = s_1^M c_2^M b_1^M = D_u \underline{(} D_v 0 \underline{,} D_w 0 \underline{)}\)である。
		\end{indented}
	\end{indented}
\end{hideableproof}

\begin{lemma}[条件(I)か(III)の下での\(c_1\)前後の具体表示]\label{条件(I)か(III)の下でのc_1前後の具体表示}
	任意の\(M \in RT_{\textrm{PS}} \cap PT_{\textrm{PS}}\)に対し、\(\textrm{Trans}\)の再帰的定義中に導入した記号を用いると、\(j_0\)が\(M\)許容かつ\(j_1 > 1\)かつ\(M_{1,j_0} \geq M_{1,j_1}\)ならば、以下が成り立つ:
	\begin{penumerate}
		\item \(t_1 \neq 0\)であり\(M\)は条件(I)か(III)を満たし\(\textrm{Trans}((M_j)_{j=j_0}^{j_1-1}) = c_1 \in PT_{\textrm{B}}\)である。
	\end{penumerate}
	更に\((0,j'_0) <_M^{\textrm{Next}} (0,j_0)\)を満たす一意な\(j'_0 \in \mathbb{N}\)が存在するとし、\(j'_{-1} := \textrm{Adm}_M(j'_0)\)と置く。
	\begin{penumerate}
		\setcounter{penumeratei}{1}
		\item \(j'_0 \leq j_1-2\)かつ\((\textrm{Pred}(M),j'_{-1}) \in T_{\textrm{PS}}^{\textrm{Marked}}\)かつ\(((M_j)_{j=j'_{-1}}^{j_1-1},j_0-j'_{-1}) \in T_{\textrm{PS}}^{\textrm{Marked}}\)である。
		\item \(j'_0+1 = j_0\)ならば以下が成り立つ:
		\begin{indented}
			\item[(3-1)] \(j'_{-1} = j'_0\)または\(M_{1,j'_0}+1 = M_{1,j_0}\)ならば、\(\textrm{Mark}(\textrm{Pred}(M),j'_{-1}) = D_{M_{1,j'_{-1}}} c_1\)である。
			\item[(3-2)] \(j'_{-1} < j'_0\)かつ\(M_{1,j'_0} \geq M_{1,j_0}\)ならば、\(\textrm{Mark}(\textrm{Pred}(M),j'_{-1}) = D_{M_{1,j'_{-1}}} D_{M_{1,j'_0}} c_1\)である。
		\end{indented}
		\item \(j'_0+1 < j_0\)ならば以下が成り立つ:
		\begin{indented}
			\item[(4-1)] \(j'_{-1} = j'_0\)または\(M_{1,j'_0}+1 = M_{1,j_0}\)ならば、一意な\(t'_2 \in T_{\textrm{B}}^2\)が存在して\(\textrm{Mark}(\textrm{Pred}(M),j'_{-1}) = D_{M_{1,j'_{-1}}}(t'_2+c_1)\)である。
			\item[(4-2)] \(j'_{-1} < j'_0\)かつ\(M_{1,j'_0} \geq M_{1,j_0}\)ならば、一意な\((t'_3,t'_4) \in T_{\textrm{B}}^2\)が存在して\(\textrm{Mark}(\textrm{Pred}(M),j'_{-1}) = D_{M_{1,j'_{-1}}}(t'_3 + D_{M_{1,j'_0}}(t'_4+c_1))\)である。
		\end{indented}
	\end{penumerate}
	更に任意の\(n \in \mathbb{N}_+\)に対し、\(n > 1\)とし、\(N := (M[n]_j)_{j=0}^{j_0+(n-1)(j_1-j_0)}\)と置き\(\textrm{Trans}\)の再帰的定義中に導入した記号を\(N\)に対して定め\(N\)に対する適用であることを明示するために右肩に\(N\)を乗せて表記する。
	\begin{penumerate}
		\setcounter{penumeratei}{4}
		\item \((M[n],j_0+(n-1)(j_1-j_0)) \in T_{\textrm{B}}^{\textrm{Marked}}\)かつ\((0,j'_0) <_{M[n]}^{\textrm{Next}} (0,j_0+(n-1)(j_1-j_0))\)であり、\(j_1^N = j_0+(n-1)(j_1-j_0)\)かつ\(j_0^N = j'_0\)かつ\(j_{-1}^N = j'_{-1}\)かつ\(t_1^N \neq 0\)であり、\(N\)は条件(VI)を満たさない。
	\end{penumerate}
\end{lemma}

\begin{hideableproof}
	\begin{indented}
		\item (1)が成り立つことを示す。
		\item \(\textrm{Lng}(\textrm{Pred}(M))-1 = j_1 > 1\)より\(\textrm{Pred}(M)\)は零項でなく、従って\nameref{Transが零項性を保つこと}から\(t_1 \neq 0\)である。\(M_{1,j_0} \geq M_{1,j_1}\)かつ\(j_0\)が\(M\)許容より\(j_0 = j_{-1}\)であり\(M\)は条件(I)か(III)を満たす。
		\item \(j_0\)が\(M\)許容より\(j_{-1} = j_0\)である。従って\(c_1 = \textrm{Mark}(\textrm{Pred}(M),j_{-1}) = \textrm{Trans}((\textrm{Pred}(M)_j)_{j=j_{-1}}^{j_1-1}) = \textrm{Trans}((M_j)_{j=j_0}^{j_1-1})\)である。また\(t_1 \neq 0\)かつ\((t_1,c_1) \in T_{\textrm{B}}^{\textrm{Marked}}\)から\(c_1 \in PT_{\textrm{B}}\)である。
		\item
		\item (2)が成り立つことを示す。
		\item \(j'_{-1} \leq j'_0 < j_0 < j_1\)より\(j'_{-1} \leq j_1-2\)である。\((0,j'_{-1}) \leq_M (0,j'_0) <_M^{\textrm{Next}} (0,j_0) <_M^{\textrm{Next}} (0,j_1)\)より\((M,j'_{-1}) \in T_{\textrm{PS}}^{\textrm{Marked}}\)である。従って\nameref{基点の切片への遺伝性}より\((\textrm{Pred}(M),j'_{-1}) \in T_{\textrm{PS}}^{\textrm{Marked}}\)である。\((M,j_0) \in T_{\textrm{PS}}^{\textrm{Marked}}\)と\nameref{基点の切片への遺伝性}より\(((M_j)_{j=j'_{-1}}^{j_1-1},j_0-j'_{-1}) \in T_{\textrm{PS}}^{\textrm{Marked}}\)である。
		\item
		\item (3), (4)が成り立つことを示す。
		\item \(N := (M_j)_{j=j'_{-1}}^{j_0}\)と置く。\(\textrm{Red}(N)\)は簡約であるので\nameref{簡約性と係数の関係}より\(M\)と\(\textrm{Red}(N)\)は条件(A)と(B)を満たす。\(\textrm{Red}(N)\)が条件(A)と(B)を満たすことと\nameref{直系先祖による切片とRedとIncrFirstの関係}より
		\begin{eqnarray*}
		N & = & \textrm{IncrFirst}^{M_{0,j'_{-1}}-M_{1,j'_{-1}}}(\textrm{Red}(N)) \\
		\textrm{Red}(N) & = & ((j,j))_{j=M_{1,j'_{-1}}}^{M_{1,j'_0}} \oplus_{\mathbb{N}^2} ((M_{0,j}-M_{0,j'_{-1}}+M_{1,j'_{-1}},M_{1,j}))_{j=j'_0+1}^{j_0-1} \oplus_{\mathbb{N}^2} ((M_{1,j'_0}+1,M_{1,j_0}))
		\end{eqnarray*}
		\item となる。従って\nameref{Transの(IncrFirst,Red)不変P同変性}より
		\begin{eqnarray*}
		\textrm{Trans}(N) & = & \textrm{Trans}(\textrm{Red}(N)) \\
		& = & \textrm{Trans}(((j,j))_{j=M_{1,j'_{-1}}}^{M_{1,j'_0}} \oplus_{\mathbb{N}^2} ((M_{0,j}-M_{0,j'_{-1}}+M_{1,j'_{-1}},M_{1,j}))_{j=j'_0+1}^{j_0-1} \oplus_{\mathbb{N}^2} ((M_{1,j'_0}+1,M_{1,j_0})))
		\end{eqnarray*}
		\item である。
		\item
		\item \(j'_0+1 = j_0\)とする。
		\begin{eqnarray*}
		\textrm{Trans}(N) = \textrm{Trans}(\textrm{Red}(N)) = \textrm{Trans}(((j,j))_{j=M_{1,j'_{-1}}}^{M_{1,j'_0}} \oplus_{\mathbb{N}^2} ((M_{0,j}-M_{0,j'_{-1}}+M_{1,j'_{-1}},M_{1,j}))_{j=j'_0+1}^{j_0-1} \oplus_{\mathbb{N}^2} ((M_{1,j'_0}+1,M_{1,j_0}))) = \textrm{Trans}(((j,j))_{j=M_{1,j'_{-1}}}^{M_{1,j'_0}} \oplus_{\mathbb{N}^2} (M_{1,j'_0}+1,M_{1,j_0}))
		\end{eqnarray*}
		\begin{indented}
			\item である。\(M_{1,j'_0}+1 = M_{1,j_0}\)ならば\nameref{公差(1,1)のペア数列のTransの基本性質}より
			\begin{eqnarray*}
			\textrm{Trans}(N) = \textrm{Trans}(((j,j))_{j=M_{1,j'_{-1}}}^{M_{1,j'_0}} \oplus_{\mathbb{N}^2} (M_{1,j'_0}+1,M_{1,j_0})) = D_{M_{1,j'_{-1}}} D_{M_{1,j_0}} 0
			\end{eqnarray*}
			\item であり、\(j'_{-1} = j'_0\)かつ\(M_{1,j'_0} \geq M_{1,j_0}\)ならば\nameref{2列ペア数列の基本性質}より
			\begin{eqnarray*}
			\textrm{Trans}(N) = \textrm{Trans}(((j,j))_{j=M_{1,j'_{-1}}}^{M_{1,j'_0}} \oplus_{\mathbb{N}^2} (M_{1,j'_0}+1,M_{1,j_0})) = D_{M_{1,j'_{-1}}} D_{M_{1,j_0}} 0
			\end{eqnarray*}
			\item であり、\(j'_{-1} < j'_0\)かつ\(M_{1,j'_0} \geq M_{1,j_0}\)ならば\nameref{Predが公差(1,1)のペア数列のTransの基本性質} (1)より
			\begin{eqnarray*}
			\textrm{Trans}(N) = \textrm{Trans}(((j,j))_{j=M_{1,j'_{-1}}}^{M_{1,j'_0}} \oplus_{\mathbb{N}^2} (M_{1,j'_0}+1,M_{1,j_0})) = D_{M_{1,j'_{-1}}} D_{M_{1,j'_0}} D_{M_{1,j_0}} 0
			\end{eqnarray*}
			\item である。従って\nameref{MarkのTransによる表示}と\nameref{TransのMarkと切片による表示}より、\(j'_{-1} = j'_0\)または\(M_{1,j'_0}+1 = M_{1,j_0}\)ならば、
			\begin{eqnarray*}
			\textrm{Mark}(\textrm{Pred}(M),j'_{-1}) & = & \textrm{Trans}((M_j)_{j=j'_{-1}}^{j_1-1}) = D_{M_{1,j'_{-1}}} \textrm{Trans}((M_j)_{j=j_0}^{j_1-1}) \\
			& = & D_{M_{1,j'_{-1}}} \textrm{Mark}(\textrm{Pred}(M),j_{-1}) = D_{M_{1,j'_{-1}}} c_1
			\end{eqnarray*}
			\item であり、\(j'_{-1} < j'_0\)かつ\(M_{1,j'_0}+1 = M_{1,j_0}\)ならば
			\begin{eqnarray*}
			\textrm{Mark}(\textrm{Pred}(M),j'_{-1}) & = & \textrm{Trans}((M_j)_{j=j'_{-1}}^{j_1-1}) = D_{M_{1,j'_{-1}}} D_{M_{1,j'_0}} \textrm{Trans}((M_j)_{j=j_0}^{j_1-1}) \\
			& = & D_{M_{1,j'_{-1}}} \textrm{Mark}(\textrm{Pred}(M),j_{-1}) = D_{M_{1,j'_{-1}}} D_{M_{1,j'_0}} c_1
			\end{eqnarray*}
			\item である。
		\end{indented}
		\item
		\item \(j'_0+1 < j_0\)とする。
		\begin{indented}
			\item \(\textrm{Trans}\)の再帰的定義中に導入した記号を\(N\)に対して定め、\(N\)に対する適用であることを明示するために右肩に\(N\)を乗せて表記する。
			\item \(j_1^N = j_0-j'_{-1}\)かつ\(j_0^N = j'_0-j'_{-1}\)であり、\nameref{許容化の切片への遺伝性}より\(j_{-1}^N = \textrm{Adm}_N(j_0^N) = \textrm{Adm}_M(j'_0)-j'_{-1} = 0\)である。従って\nameref{s_1とb_1の空性と基点の関係}から\(s_1^N = ()\)かつ\(b_1^N = ()\)であり、
			\item \(j'_{-1} \leq j'_0 < j'_0+1 < j_0\)より\(j_1^N = j_0-j'_{-1} > 2\)であるので\(\textrm{Pred}(N)\)は零項でなく、\nameref{Transが零項性を保つこと}から\(t_1^N \neq 0\)である。従って\(N\)に対し条件(I)~(VI)が意味を持つ。
			\item \(j_0^N+1 = j'_0+1-j'_{-1} < j_0-j'_{-1} = j_1^N\)より\(N\)は条件(VI)を満たさない。\nameref{Transの最左単項成分の左端の基本性質}より\(v^N = N_{1,0} = M_{1,j'_{-1}}\)である。
			\item \(j'_{-1} = j'_0\)または\(M_{1,j'_0}+1 = M_{1,j_0}\)とする。
			\begin{indented}
				\item \(t'_2 := t_2^N\)と置く。
				\item \(j'_{-1} = j'_0\)ならば、\(j_{-1}^N = 0 = j'_0-j'_{-1} = j_0^N\)より\(j_0^N\)は\(N\)許容となり条件(I)か(III)を満たす。
				\item \(M_{1,j'_0}+1 = M_{1,j_0}\)ならば、\(N\)が条件(VI)を満たさないことから\(N\)は条件(V)を満たす。
				\item いずれの場合も
				\begin{eqnarray*}
				c_2^N & = & D_{v^N}(t_2^N + D_{N_{1,j_1^N}} 0) = D_{M_{1,j'_{-1}}}(t'_2+D_{M_{1,j_0}} 0) \\
				\textrm{Trans}(N) & = & s_1^N c_2^N b_1^N = D_{M_{1,j'_{-1}}}(t'_2+D_{M_{1,j_0}} 0)
				\end{eqnarray*}
				\item である。従って\nameref{加法とscb分解の関係}と\nameref{MarkのTransによる表示}と\nameref{TransのMarkと切片による表示}より
				\begin{eqnarray*}
				\textrm{Mark}(\textrm{Pred}(M),j'_{-1}) & = & \textrm{Trans}((M_j)_{j=j'_{-1}}^{j_1-1}) = D_{M_{1,j'_{-1}}}(t'_2 + \textrm{Trans}((M_j)_{j=j_0}^{j_1-1})) \\
				& = & D_{M_{1,j'_{-1}}}(t'_2 + \textrm{Mark}(\textrm{Pred}(M),j_{-1})) = D_{M_{1,j'_{-1}}}(t'_2+c_1)
				\end{eqnarray*}
				\item である。
			\end{indented}
			\item \(j'_{-1} < j'_0\)かつ\(M_{1,j'_0} \geq M_{1,j_0}\)とする。
			\begin{indented}
				\item \(j_{-1}^N = 0 < j'_0-j'_{-1} = j_0^N\)より\(j_0^N\)は非\(N\)許容となり、\(N_{1,j_0^N} = M_{1,j'_0} \geq M_{1,j_0} = N_{1,j_1^N}\)より\(N\)は条件(II)か(IV)を満たす。
				\item \(t'_3 := t_3^N\)と置く。
				\item \(t'_4 := t_4^N \)と置く。
				\begin{eqnarray*}
				c_2^N & = & D_{v^N}(t_3^N + D_{N_{1,j_0^N}}(t_4^N + D_{N_{1,j_1^N}} 0)) = D_{M_{1,j'_{-1}}}(t'_3 + D_{M_{1,j'_0}}(t'_4 + D_{M_{1,j_0}} 0)) \\
				\textrm{Trans}(N) & = & s_1^N c_2^N b_1^N = D_{M_{1,j'_{-1}}}(t'_3 + D_{M_{1,j'_0}}(t'_4 + D_{M_{1,j_0}} 0))
				\end{eqnarray*}
				\item である。従って\nameref{加法とscb分解の関係}と\nameref{MarkのTransによる表示}と\nameref{TransのMarkと切片による表示}より
				\begin{eqnarray*}
				\textrm{Mark}(\textrm{Pred}(M),j'_{-1}) & = & \textrm{Trans}((M_j)_{j=j'_{-1}}^{j_1-1}) = D_{M_{1,j'_{-1}}}(t'_3 + D_{M_{1,j'_0}}(t'_4 + \textrm{Trans}((M_j)_{j=j_0}^{j_1-1}))) \\
				& = & D_{M_{1,j'_{-1}}}(t'_3 + D_{M_{1,j'_0}}(t'_4 + \textrm{Mark}(\textrm{Pred}(M),j_{-1}))) = D_{M_{1,j'_{-1}}}(t'_3 + D_{M_{1,j'_0}}(t'_4+c_1))
				\end{eqnarray*}
				\item である。
			\end{indented}
		\end{indented}
		\item
		\item (5)が成り立つことを示す。
		\item \((0,j_0) <_M^{\textrm{Next}} (0,j_1)\)と\nameref{親の基本性質} (1)より、任意の\(j \in \mathbb{N}\)に対し\(j_0 < j \leq j_1\)ならば\(M_{0,j} \geq M_{0,j_1} > M_{0,j_0}\)である。従って\nameref{親の存在の判定条件} (3)より\((0,j_0) \leq_M (0,j_1-1)\)である。更に\((M_j)_{j=j_0}^{j_1-1} = (M[n]_j)_{j=j_0+(n-1)(j_1-j_0)}^{j_0+n(j_1-j_0)-1}\)であるので、\((0,j_0+(n-1)(j_1-j_0)) \leq_{M[n]} (0,j_0+n(j_1-j_0)-1)\)である。
		\item \(n > 1\)より\(M[n]_{0,j_0+(n-1)(j_1-j_0)-1} = M_{0,j_1-1} > M_{0,j_0} = M[n]_{0,j_0+(n-1)(j_1-j_0)}\)となるので、\((0,j_0+(n-1)(j_1-j_0)-1) <_{M[n]}^{\textrm{Next}} (0,j_0+(n-1)(j_1-j_0))\)でない。従って\(j_0+(n-1)(j_1-j_0)\)は\(M[n]\)許容である。以上より\((M[n],j_0+(n-1)(j_1-j_0)) \in T_{\textrm{PS}}^{\textrm{Marked}}\)である。
		\item 任意の\(j \in \mathbb{N}\)に対し、\(j_0 \leq j < j_0+(n-1)(j_1-j_0)\)ならば、\(j-j_0\)を\(j_1-j_0 > 0\)で割った余りを\(r \in \mathbb{N}\)と置くと\(r < j_1-j_0\)であり、\(r = 0\)ならば\(M[n]_{0,j} = M_{0,j_0+r} = M_{0,j_0} = M[n]_{0,j_0+(n-1)(j_1-j_0)}\)となり\(r > 0\)ならば\(M[n]_{0,j} = M_{0,j_0+r} \geq M_{0,j_1} > M_{0,j_0} = M[n]_{0,j_0+(n-1)(j_1-j_0)}\)であるのでいずれの場合も\((0,j) \leq_{M[n]} (0,j_0+(n-1)(j_1-j_0))\)でない。
		\item また任意の\(j \in \mathbb{N}\)に対し、\(j'_0 < j < j_0\)ならば、\((0,j'_0) <_M^{\textrm{Next}} (0,j_0)\)と\nameref{親の基本性質} (1)より\(M[n]_{0,j} = M_{0,j} \geq M_{0,j_0} = M[n]_{0,j_0+(n-1)(j_1-j_0)}\)であるので\((0,j) \leq_{M[n]} (0,j_0+(n-1)(j_1-j_0))\)でない。
		\item 更に\((0,j'_0) <_M^{\textrm{Next}} (0,j_0)\)より\(M[n]_{0,j'_0} = M_{0,j'_0} < M_{0,j_0} = M[n]_{0,j_0+(n-1)(j_1-j_0)}\)であるので、以上より\((0,j'_0) <_{M[n]}^{\textrm{Next}} (0,j_0+(n-1)(j_1-j_0))\)である。
		\item \(\textrm{Lng}(N) = j_0+(n-1)(j_1-j_0)+1\)より\(j_1^N = j_0+(n-1)(j_1-j_0) \geq j_0 + (j_1-j_0) \geq j_0+1 > 1\)であるので\(\textrm{Pred}(N)\)は零項でなく、\nameref{Transが零項性を保つこと}より\(t_1^N \neq 0\)である。従って\(N\)に対して条件(I)~(VI)が意味を持つ。
		\item \((0,j'_0) <_{M[n]}^{\textrm{Next}} (0,j_0+(n-1)(j_1-j_0))\)より\((0,j'_0) <_N^{\textrm{Next}} (0,j_0+(n-1)(j_1-j_0)) = (0,j_1^N)\)であるので\(j_0^N = j'_0\)であり、\(\textrm{Pred}(N) = (M[n]_j)_{j=0}^{j_0+(n-1)(j_1-j_0)-1} = M[n-1]\)である。従って\(t_1^N = \textrm{Trans}(\textrm{Pred}(N)) = \textrm{Trans}(M[n-1]) = \textrm{Trans}(M)[n-2]\)である。
		\item \(j'_0 < j_0 \leq j_1-1\)と\nameref{許容化の切片への遺伝性}より\(\textrm{Adm}_N(j'_0) = \textrm{Adm}_{\textrm{Pred}(M)}(j'_0) = j'_{-1}\)である。従って\(j_{-1}^N = \textrm{Adm}_N(j_0^N) = \textrm{Adm}_N(j'_0) = j'_{-1}\)である。
		\item \(j_0^N+1 = j'_0+1 \leq j_0 < j_1 \leq j_0+(n-1)(j_1-j_0) = j_1^N\)より、\(N\)は条件(VI)を満たさない。
	\end{indented}
\end{hideableproof}

\iffull{それでは本題に戻る。}\fi

\begin{hideableproof}[\nameref{条件(I)の下でのTransと基本列の交換関係}の証明]
	\begin{indented}
		\item \(M\)は単項であるので\nameref{Transが零項性を保つこと}から\(\textrm{Trans}(M) \neq 0\)である。
		\item (2)は(1)と\(\textrm{Trans}(M) \neq 0\)と\cite{buc1} Lemma 3.2 (a)より即座に従う。以下では(1)を示す。
		\item \(j_1 > 0\)より\(\textrm{Pred}(M)\)は零項でないので\nameref{Transが零項性を保つこと}から\(t_1 \neq 0\)である。また\((t_1,c_1) \in T_{\textrm{B}}^{\textrm{Marked}}\)であるので\(c_1 \in PT_{\textrm{B}}\)である。
		\item \(M\)は簡約であるので、\nameref{簡約性と係数の関係}から条件(A)と(B)を満たす。\(M\)の単項性と条件(B)から\(M_{0,0} = M_{1,0}\)である。\nameref{簡約性が基本列で保たれること}から\(M[n]\)も簡約である。
		\item \(j_0\)が\(M\)許容より\(j_{-1} = j_0\)である。\nameref{Markの左端の基本性質}より\(v = M_{1,j_{-1}} = M_{1,j_0}\)である。
		\item \(\textrm{Trans}\)の定義より\((s_1,c_2,b_1) = (s_1,D_{M_{1,j_0}}(t_2 + D_0 0), b_1)\)は\(\textrm{Trans}(M)\)の第\(0\)種scb分解である。
		\item \(n = 1\)ならば\(M[n] = \textrm{Pred}(M)\)より\(\textrm{Trans}(M[n]) = t_1 = s_1 c_1 b_1\)である。
		\item
		\item \(j_0 = 0\)とする。
		\begin{indented}
			\item \nameref{非複項性と基本列の関係} (1)より\(P(M[n]) = (\textrm{Pred}(M))_{k=0}^{n-1}\)である。
			\item \(j_0\)は\(M\)許容であるので\(j_{-1} = j_0 = 0\)である。従って\nameref{s_1とb_1の空性と基点の関係}より\(s_1 = ()\)かつ\(b_1 = ()\)であり、\nameref{scb分解と基本列の関係} (1-1)より\(\textrm{Trans}(M)[n-1] = D_{M_{1,j_0}}(t_2 + D_0 0)[n-1] = (D_{M_{1,j_0}} t_2) \times n  = c_1 \times n\)である。
			\item \(\textrm{Trans}(M[n]) = \textrm{Trans}(M)[n-1]\)となることを\(n\)に関する数学的帰納法で示す。
			\begin{indented}
				\item \(n = 1\)ならば\(\textrm{Trans}(M[n]) = s_1 c_1 b_1 =  c_1 = \textrm{Trans}(M)[n-1]\)である。
				\item \(n > 1\)ならば、帰納法の仮定より\(\textrm{Trans}(M[n-1]) = \textrm{Trans}(M)[n-2]\)であり、\(\textrm{Lng}(\textrm{Pred}(M)) = j_1 > 1\)より\(\textrm{Pred}(M) \neq (0,0)\)であるので\(\textrm{Trans}\)の再帰的定義より
				\begin{eqnarray*}
				\textrm{Trans}(M[n]) & = & \textrm{Trans}(\bigoplus_{\mathbb{N}^2} (\textrm{Pred}(M))_{k=0}^{n-2}) + \textrm{Trans}(\textrm{Pred}(M)) = \textrm{Trans}(M[n-1]) + t_1 = \textrm{Trans}(M)[n-2] + s_1 c_1 b_1 = c_1 \times (n-1) + c_1 \\
				& = & c_1 \times n = \textrm{Trans}(M)[n-1]
				\end{eqnarray*}
				\item である。
			\end{indented}
		\end{indented}
		\item
		\item \(j_0 > 0\)とする。
		\begin{indented}
			\item \(M\)が条件(I)を満たすことから、\(j_0\)が\(M\)許容かつ\(M_{1,j_0} \geq 0 = M_{1,j_1}\)である。従って\(M\)は\nameref{条件(I)か(III)の下でのc_1前後の具体表示}の仮定を満たす。
			\item \(M\)が単項であることから\((0,0) \leq_M (0,j_1)\)であるので\(M_{0,0} < M_{0,j_1}\)となり、\(0 < j_0\)であるので\nameref{親の存在の判定条件} (1)から\((0,j'_0) <_M^{\textrm{Next}} (0,j_0)\)を満たす一意な\(j'_0 \in \mathbb{N}\)が存在する。
			\item \nameref{条件(I)か(III)の下でのc_1前後の具体表示}で導入した記号を用いる。
			\item \((t_1,\textrm{Mark}(\textrm{Pred}(M),j'_{-1})) \in T_{\textrm{B}}^{\textrm{Marked}}\)より、一意な\((s'_{-1},b'_{-1}) \in (\Sigma^{< \omega})^2\)が存在して\((s'_{-1},\textrm{Mark}(\textrm{Pred}(M),j'_{-1}),b'_{-1})\)は\(t_1\)のscb分解をなす。特に\(\textrm{Trans}(\textrm{Pred}(M)) = t_1 = s'_{-1} \textrm{Mark}(\textrm{Pred}(M),j'_{-1}) b'_{-1}\)である。
			\item \(j_{-1} = j_0 > 0\)と\nameref{s_1とb_1の空性と基点の関係}と\nameref{scb分解の自明性の判定条件}より、\(s_1 \neq ()\)である。
			\item \(j'_{-1} = j'_0\)または\(M_{1,j'_0}+1 = M_{1,j_0}\)とする。
			\begin{indented}
				\item \(j'_0+1 = j_0\)とする。
				\begin{indented}
					\item \(t'_2 := 0\)と置く。
					\item \nameref{条件(I)か(III)の下でのc_1前後の具体表示} (3-1)より\(s_1 c_1 b_1 = \textrm{Trans}(\textrm{Pred}(M)) = s'_{-1} \textrm{Mark}(\textrm{Pred}(M),j'_{-1}) b'_{-1} = s'_{-1} D_{M_{1,j'_{-1}}} c_1 b'_{-1}\)である。従って\nameref{scb分解の一意性} (1)より\(s_1 = s'_{-1} D_{M_{1,j'_{-1}}}\)かつ\(b_1 = b'_{-1}\)である。
					\item \nameref{scb分解と基本列の関係} (1-2)より\(\textrm{Trans}(M)[n-1] = s'_{-1} D_{M_{1,j'_{-1}}}((D_{M_{1,j_0}} t_2) \times n) b_1 = s'_{-1} D_{M_{1,j'_{-1}}}(c_1 \times n) b'_{-1} = s'_{-1} D_{M_{1,j'_{-1}}}(t'_2 + c_1 \times n) b'_{-1}\)である。
				\end{indented}
				\item \(j'_0+1 < j_0\)とする。
				\begin{indented}
					\item \nameref{条件(I)か(III)の下でのc_1前後の具体表示} (4-1)より一意な\(t'_2 \in T_{\textrm{B}}\)が存在して\(\textrm{Mark}(\textrm{Pred}(M),j'_{-1}) = D_{M_{1,j'_{-1}}}(t'_2+c_1)\)である。
					\begin{eqnarray*}
					s_1 c_1 b_1 = t_1 = \textrm{Trans}(\textrm{Pred}(M)) = s'_{-1} \textrm{Mark}(\textrm{Pred}(M),j'_{-1}) b'_{-1} = s'_{-1} D_{M_{1,j'_{-1}}}(t'_2+c_1) b'_{-1}
					\end{eqnarray*}
					\item であるので、\nameref{加法とscb分解の関係}と\nameref{scb分解の合成則} (2)を反復して適用することで\footnote{単に「\(c_1\)を\(c_2\)に置き換える」と述べたいところだが、\(t'_2\)を用いて明示的に与えられている\(\textrm{Trans}(\textrm{Pred}(M))\)の表示はscb分解ではなく加法を含む表記であるため、\nameref{scb分解の置換可能性}を繰り返し適用してscb分解を得る必要がある。そのために直前の段落で\nameref{加法とscb分解の関係}と\nameref{scb分解の合成則} (2)を適用した。}
					\begin{eqnarray*}
					\textrm{Trans}(M) = s_1 c_2 b_1 = s'_{-1} D_{M_{1,j'_{-1}}}(t'_2+c_2) b'_{-1} = s'_{-1} D_{M_{1,j'_{-1}}}(t'_2 + D_{M_{1,j_0}}(t_2 + D_0 0)) b'_{-1}
					\end{eqnarray*}
					\item となる。従って\nameref{scb分解と基本列の関係} (1-2)より\(\textrm{Trans}(M)[n-1] = s'_{-1} D_{M_{1,j'_{-1}}}(t'_2 + (D_{M_{1,j_0}} t_2) \times n) b'_1 = s'_{-1} D_{M_{1,j'_{-1}}}(t'_2 + c_1 \times n) b'_{-1}\)である。
				\end{indented}
				\item 以上より、いずれの場合も\(\textrm{Trans}(M)[n-1] = s'_{-1} D_{M_{1,j'_{-1}}}(t'_2 + c_1 \times n) b'_{-1}\)である。
				\item \(\textrm{Mark}(M[n],j'_{-1}) = D_{M_{1,j'_{-1}}}(t'_2 + c_1 \times n)\)かつ\(\textrm{Trans}(M[n]) = s'_{-1} \textrm{Mark}(M[n],j'_{-1}) b'_{-1}\)となることを\(n\)に関する数学的帰納法で示す。
				\item \(n = 1\)ならば\(\textrm{Mark}(M[n],j'_{-1}) = \textrm{Mark}(\textrm{Pred}(M),j'_{-1}) = D_{M_{1,j'_{-1}}}(t'_2+c_1) = D_{M_{1,j'_{-1}}}(t'_2 + c_1 \times n)\)かつ\(\textrm{Trans}(M[n]) = \textrm{Trans}(\textrm{Pred}(M)) = s'_{-1} \textrm{Mark}(\textrm{Pred}(M),j'_{-1}) b'_{-1} = s'_{-1} \textrm{Mark}(M[n],j'_{-1}) b'_{-1}\)である。
				\item \(n > 1\)とする。
				\begin{indented}
					\item \nameref{条件(I)か(III)の下でのc_1前後の具体表示} (5)より\(j_1^N = j_0+(n-1)(j_1-j_0)\)かつ\(j_0^N = j'_0\)かつ\(j_{-1}^N = j'_{-1}\)かつ\(t_1^N \neq 0\)であり、\(N\)は条件(VI)を満たさない。
					\item 帰納法の仮定より\(c_1^N = \textrm{Mark}(M[n-1],j'_{-1}) = D_{M_{1,j'_{-1}}}(t'_2 + c_1 \times (n-1))\)かつ\(s_1^N c_1^N b_1^N = t_1^N = \textrm{Trans}(M[n-1]) = s'_{-1} \textrm{Mark}(M[n-1],j'_{-1}) b'_{-1} = s'_{-1} c_1^N b'_{-1}\)である。従って\nameref{scb分解の一意性} (1)より\(s_1^N = s'_{-1}\)かつ\(b_1^N = b'_{-1}\)である。また\(D_{v^N} t_2^N = c_1^N = D_{M_{1,j'_{-1}}}(t'_2 + c_1 \times (n-1))\)より\(v^N = M_{1,j'_{-1}}\)かつ\(t_2^N = t'_2 + c_1 \times (n-1)\)である。
					\item \(j'_{-1} = j'_0\)ならば\(j_{-1}^N = j'_{-1} = j'_0 = j_0^N\)より\(N\)は条件(I)か(III)を満たす。
					\item \(M_{1,j'_0}+1 = M_{1,j_0}\)ならば、\(N\)が条件(VI)を満たないことと\(N_{1,j_0^N}+1 = M_{1,j'_0}+1 = M_{1,j_0} = N_{1,j_1^N}\)より、\(N\)は条件(V)を満たす。
					\item 従っていずれの場合\(N\)は条件(I)か(III)か(V)を満たし、\(c_2^N = D_{v^N}(t_2^N + D_{N_{1,j_1^N}} 0) = D_{M_{1,j'_{-1}}}((t'_2 + c_1 \times (n-1)) + D_{M_{1,j_0}} 0)\)となるので\(\textrm{Trans}(N) = s_1^N c_2^N b_1^N = s'_{-1} D_{M_{1,j'_{-1}}}(t'_2 + c_1 \times (n-1) + D_{M_{1,j_0}} 0) b'_{-1}\)である。
					\item \nameref{加法とscb分解の関係}と\nameref{scb分解の合成則} (2) より\((D_{M_{1,j'_{-1}}}(t'_2 + c_1 \times (n-1) + D_{M_{1,j_0}} 0),D_{M_{1,j_0}} 0) \in T_{\textrm{B}}^{\textrm{Marked}}\)である。また\(\textrm{Mark}\)の定義より\((\textrm{Trans}(N),D_{M_{1,j'_{-1}}}(t'_2 + c_1 \times (n-1) + D_{M_{1,j_0}} 0)) \in T_{\textrm{B}}^{\textrm{Marked}}\)である。
					\item 更に\nameref{scb分解の合成則} (1)と\nameref{scb分解の置換可能性}を反復して適用することで\footnote{単に「\(D_0 0\)を\(\textrm{Mark}(M[n],j_0+(n-1)(j_1-j_0))\)に置き換える」と述べたいところだが、\(t'_2\)を用いて明示的に与えられている\(\textrm{Trans}(N)\)の表示はscb分解ではなく加法を含む表記であるため、\nameref{scb分解の置換可能性}を繰り返し適用してscb分解を得る必要がある。そのために直前の段落で\nameref{加法とscb分解の関係}と\nameref{scb分解の合成則} (2)を適用した。}\nameref{TransのMarkと切片による表示}より
					\begin{eqnarray*}
					\textrm{Trans}(M[n]) & = & s'_{-1} D_{M_{1,j'_{-1}}}((t'_2 + c_1 \times (n-1)) + \textrm{Mark}(M[n],j_0+(n-1)(j_1-j_0))) b'_{-1} = s'_{-1} D_{M_{1,j'_{-1}}}((t'_2 + c_1 \times (n-1)) + \textrm{Mark}(\textrm{Pred}(M),j_{-1})) b'_{-1} \\
					& = & s'_{-1} D_{M_{1,j'_{-1}}}((c_1 \times (n-1)) + c_1) b'_{-1} = s'_{-1} D_{M_{1,j'_{-1}}}(t'_2 + c_1 \times n) b'_{-1}
					\end{eqnarray*}
					\item となり、更に\(\textrm{Trans}(N) = s'_{-1} \textrm{Mark}(N,j'_{-1}) b'_{-1}\)より
					\begin{eqnarray*}
					s'_{-1} \textrm{Mark}(M[n],j'_{-1}) b'_{-1} & = & \textrm{Trans}(M[n]) = s'_{-1} D_{M_{1,j'_{-1}}}(t'_2 + c_1 \times n) b'_{-1}
					\end{eqnarray*}
					\item となるので\(\textrm{Mark}(M[n],j'_{-1}) = D_{M_{1,j'_{-1}}}(t'_2 + c_1 \times n)\)かつ\(\textrm{Trans}(M[n]) = s'_{-1} \textrm{Mark}(M[n],j'_{-1}) b'_{-1}\)である。
				\end{indented}
				\item 以上より
				\begin{eqnarray*}
				\textrm{Trans}(M[n]) = s'_{-1} \textrm{Mark}(M[n],j'_{-1}) b'_{-1} = s'_{-1} D_{M_{1,j'_{-1}}}(t'_2 + c_1 \times n) b'_{-1} = \textrm{Trans}(M)[n-1]
				\end{eqnarray*}
				\item である。
			\end{indented}
		\end{indented}
		\item
		\begin{indented}
			\item \(j'_{-1} < j'_0\)かつ\(M_{1,j'_0} \geq M_{1,j_0}\)とする。
			\begin{indented}
				\item \(j'_0+1 = j_0\)とする。
				\begin{indented}
					\item \(t'_3 := 0\)と置く。
					\item \(t'_4 := 0\)と置く。
					\item \nameref{条件(I)か(III)の下でのc_1前後の具体表示} (3-2)より\(s_1 c_1 b_1 = \textrm{Trans}(\textrm{Pred}(M)) = s'_{-1} \textrm{Mark}(\textrm{Pred}(M),j'_{-1}) b'_{-1} = s'_{-1} D_{M_{1,j'_{-1}}} D_{M_{1,j'_0}} c_1 b'_{-1} = s'_{-1} D_{M_{1,j'_{-1}}}(t'_3 + D_{M_{1,j'_0}}(t'_4+c_1)) b'_{-1}\)である。従って\nameref{scb分解の一意性} (1)より\(s_1 = s'_{-1} D_{M_{1,j'_{-1}}} D_{M_{1,j'_0}}\)かつ\(b_1 = b'_{-1}\)である。
					\item \nameref{scb分解と基本列の関係} (1-2)より\(\textrm{Trans}(M)[n-1] = s'_{-1} D_{M_{1,j'_{-1}}} D_{M_{1,j'_0}}((D_{M_{1,j_0}} t_2) \times n) b_1 = s'_{-1} D_{M_{1,j'_{-1}}} D_{M_{1,j'_0}}(c_1 \times n) b'_{-1} = s'_{-1} D_{M_{1,j'_{-1}}}(t'_3 + D_{M_{1,j'_0}}(t'_4 + c_1 \times n)) b'_{-1}\)である。
				\end{indented}
				\item \(j'_0+1 < j_0\)とする。
				\begin{indented}
					\item \nameref{条件(I)か(III)の下でのc_1前後の具体表示} (4-2)より一意な\((t'_3,t'_4) \in T_{\textrm{B}}^2\)が存在して\(\textrm{Mark}(\textrm{Pred}(M),j'_{-1}) = D_{M_{1,j'_{-1}}}(t'_3 + D_{M_{1,j'_0}}(t'_4+c_1))\)であり、\(s_1 c_1 b_1 = t_1 = \textrm{Trans}(\textrm{Pred}(M)) = s'_{-1} \textrm{Mark}(\textrm{Pred}(M),j'_{-1}) b'_{-1} = s'_{-1} D_{M_{1,j'_{-1}}}(t'_3 + D_{M_{1,j'_0}}(t'_4+c_1)) b'_{-1}\)である。
					\item よって\nameref{加法とscb分解の関係}と\nameref{scb分解の合成則} (2)を反復して適用することで\((D_{M_{1,j'_0}}(t'_4+c_1),c_1), (D_{M_{1,j'_{-1}}}(t'_3 + D_{M_{1,j'_0}}(t'_4+c_1)),D_{M_{1,j'_0}}(t'_4+c_1)) \in T_{\textrm{B}}^{\textrm{Marked}}\)が従う。また\(\textrm{Mark}\)の定義より\((s_1 c_1 b_1,D_{M_{1,j'_{-1}}}(t'_3 + D_{M_{1,j'_0}}(t'_4+c_1))), (s_1 c_1 b_1,c_1) \in T_{\textrm{B}}^{\textrm{Marked}}\)である。
					\item 更に\nameref{scb分解の合成則} (1)と\nameref{scb分解の置換可能性}を反復して適用することで\footnote{単に「\(c_1\)を\(c_2\)に置き換える」と述べたいところだが、\(t'_3\)と\(t'_4\)を用いて明示的に与えられている\(s_1 c_1 b_1\)の表示はscb分解ではなく加法を含む表記であるため、\nameref{scb分解の置換可能性}を繰り返し適用してscb分解を得る必要がある。そのために直前の段落で\nameref{加法とscb分解の関係}と\nameref{scb分解の合成則} (2)を反復して適用した。}\(\textrm{Trans}(M) = s_1 c_2 b_1 = s'_{-1} D_{M_{1,j'_{-1}}}(t'_3 + D_{M_{1,j'_0}}(t'_4 + D_{M_{1,j_0}}(t_2 + D_0 0))) b'_{-1}\)が従う。\nameref{scb分解と基本列の関係} (1-2)より\(\textrm{Trans}(M)[n-1] = s'_{-1} D_{M_{1,j'_{-1}}}(t'_3 + D_{M_{1,j'_0}}(t'_4 + (D_{M_{1,j_0}} t_2) \times n)) b'_1 = s'_{-1} D_{M_{1,j'_{-1}}}(t'_3 + D_{M_{1,j'_0}}(t'_4 + c_1 \times n)) b'_{-1}\)である。
				\end{indented}
				\item 従っていずれの場合も\(\textrm{Mark}(\textrm{Pred}(M),j'_{-1}) = D_{M_{1,j'_{-1}}}(t'_3 + D_{M_{1,j'_0}}(t'_4+c_1))\)かつ\(\textrm{Trans}(M)[n-1] = s'_{-1} D_{M_{1,j'_{-1}}}(t'_3 + D_{M_{1,j'_0}}(t'_4 + c_1 \times n)) b'_{-1}\)である。
				\item \(\textrm{Mark}(M[n],j'_{-1}) = D_{M_{1,j'_{-1}}}(t'_3 + D_{M_{1,j'_0}}(t'_4 + c_1 \times n))\)かつ\(\textrm{Trans}(M[n]) = s'_{-1} \textrm{Mark}(M[n],j'_{-1}) b'_{-1}\)となることを\(n\)に関する数学的帰納法で示す。
				\item \(n = 1\)ならば\(\textrm{Mark}(M[n],j'_{-1}) = \textrm{Mark}(\textrm{Pred}(M),j'_{-1}) = D_{M_{1,j'_{-1}}}(t'_3 + D_{M_{1,j'_0}}(t'_4 + c_1)) = D_{M_{1,j'_{-1}}}(t'_3 + D_{M_{1,j'_0}}(t'_4 + c_1 \times n))\)かつ\(\textrm{Trans}(M[n]) = \textrm{Trans}(\textrm{Pred}(M)) = s'_{-1} \textrm{Mark}(\textrm{Pred}(M),j'_{-1}) c_1 b'_{-1} = s'_{-1} \textrm{Mark}(M[n],j'_{-1}) b'_{-1}\)である。
				\item \(n > 1\)とする。
				\begin{indented}
					\item \nameref{条件(I)か(III)の下でのc_1前後の具体表示} (5)より\(j_1^N = j_0+(n-1)(j_1-j_0)\)かつ\(j_0^N = j'_0\)かつ\(j_{-1}^N = j'_{-1}\)かつ\(t_1^N \neq 0\)であり、\(N\)は条件(VI)を満たさない。
					\item 帰納法の仮定より\(c_1^N = \textrm{Mark}(M[n-1],j'_{-1}) = D_{M_{1,j'_{-1}}}(t'_3 + D_{M_{1,j'_0}}(t'_4 + c_1 \times (n-1)))\)かつ\(s_1^N c_1^N b_1^N = t_1^N = \textrm{Trans}(M[n-1]) = s'_{-1} \textrm{Mark}(M[n-1],j'_{-1}) b'_{-1} = s'_{-1} c_1^N b'_{-1}\)である。従って\nameref{scb分解の一意性} (1)より\(s_1^N = s'_{-1}\)かつ\(b_1^N = b'_{-1}\)である。また\(D_{v^N} t_2^N = c_1^N = D_{M_{1,j'_{-1}}}(t'_3 + D_{M_{1,j'_0}}(t'_4 + c_1 \times (n-1)))\)より\(v^N = M_{1,j'_{-1}}\)かつ\(t_2^N = t'_3 + D_{M_{1,j'_0}}(t'_4 + c_1 \times (n-1))\)である。
					\item \(j_{-1}^N = j'_{-1} < j'_0 = j_0^N\)より\(j_0^N\)は非\(N\)許容であるので\(N\)は条件(I)や(III)は満たさず、\(N_{1,j_0^N}+1 = M_{1,j'_0}+1 \geq M_{1,j_0} = N_{1,j_1^N}\)より\(N\)は条件(II)か(IV)を満たす。\(t_2^N\)の最右単項成分\(D_{M_{1,j'_0}}(t'_4 + c_1 \times (n-1))\)の左端は\(D_{M_{1,j'_0}} = D_{N_{1,j_0^N}}\)であるので、\(t_3^N = t'_3\)かつ\(D_{N_{1,j_0^N}} t_4^N = t_2^N = D_{M_{1,j'_0}}(t'_4 + c_1 \times (n-1))\)すなわち\(t_4^N = t'_4 + c_1 \times (n-1)\)である。
					\item 従って\(c_2^N = D_{v^N}(t_3^N + D_{N_{1,j_0^N}}(t_4^N + D_{N_{1,j_1^N}} 0)) = D_{M_{1,j'_{-1}}}(t'_3 + D_{M_{1,j'_0}}(t'_4 + c_1 \times (n-1) + D_{M_{1,j_0}} 0)) = D_{M_{1,j'_{-1}}}(t'_3 + D_{M_{1,j'_0}}(t'_4 + c_1 \times (n-1) + D_{M_{1,j_0}} 0))\)となるので\(\textrm{Trans}(N) = s_1^N c_2^N b_1^N = s'_{-1} D_{M_{1,j'_{-1}}}(t'_3 +  D_{M_{1,j'_0}}(t'_4 + c_1 \times (n-1) + D_{M_{1,j_0}} 0)) b'_{-1}\)である。
					\item \nameref{加法とscb分解の関係}と\nameref{scb分解の合成則} (2)を反復して適用することで\((D_{M_{1,j'_0}}(t'_4 + c_1 \times (n-1) + D_{M_{1,j_0}} 0),D_{M_{1,j_0}} 0), (D_{M_{1,j'_{-1}}}(t'_3 +  D_{M_{1,j'_0}}(t'_4 + c_1 \times (n-1) + D_{M_{1,j_0}} 0)),D_{M_{1,j'_0}}(t'_4 + c_1 \times (n-1) + D_{M_{1,j_0}} 0)) \in T_{\textrm{B}}^{\textrm{Marked}}\)が従う。また\(\textrm{Mark}\)の定義より\((\textrm{Trans}(N),D_{M_{1,j'_{-1}}}(t'_3 +  D_{M_{1,j'_0}}(t'_4 + c_1 \times (n-1) + D_{M_{1,j_0}} 0))) \in T_{\textrm{B}}^{\textrm{Marked}}\)である。
					\item 更に\nameref{scb分解の合成則} (1)と\nameref{scb分解の置換可能性}を反復して適用することで\footnote{単に「\(D_0 0\)を\(\textrm{Mark}(M[n],j_0+(n-1)(j_1-j_0))\)に置き換える」と述べたいところだが、\(t'_3\)と\(t'_4\)を用いて明示的に与えられている\(\textrm{Trans}(N)\)の表示はscb分解ではなく加法を含む表記であるため、\nameref{scb分解の置換可能性}を繰り返し適用してscb分解を得る必要がある。そのために直前の段落で\nameref{加法とscb分解の関係}と\nameref{scb分解の合成則} (2)を反復して適用した。}\nameref{TransのMarkと切片による表示}より
					\begin{eqnarray*}
					\textrm{Trans}(M[n]) & = & s'_{-1} D_{M_{1,j'_{-1}}}(t'_3 + D_{M_{1,j'_0}}(t'_4 + c_1 \times (n-1) + \textrm{Mark}(M[n],j_0+(n-1)(j_1-j_0)))) b'_{-1} = s'_{-1} D_{M_{1,j'_{-1}}}(t'_3 + D_{M_{1,j'_0}}(t'_4 + c_1 \times (n-1) + \textrm{Mark}(\textrm{Pred}(M),j_{-1}))) b'_{-1} = s'_{-1} D_{M_{1,j'_{-1}}}(t'_3 + D_{M_{1,j'_0}}(t'_4 + c_1 \times (n-1) + c_1)) b'_{-1} \\
					& = & s'_{-1} D_{M_{1,j'_{-1}}}(t'_3 + D_{M_{1,j'_0}}(t'_4 + c_1 \times n)) b'_{-1}
					\end{eqnarray*}
					\item となり、更に\(\textrm{Trans}(N) = s'_{-1} \textrm{Mark}(N,j'_{-1}) b'_{-1}\)より
					\begin{eqnarray*}
					s'_{-1} \textrm{Mark}(M[n],j'_{-1}) b'_{-1} & = & \textrm{Trans}(M[n]) = s'_{-1} D_{M_{1,j'_{-1}}}(t'_3 + D_{M_{1,j'_0}}(t'_4 + c_1 \times n)) b'_{-1}
					\end{eqnarray*}
					\item となるので\(\textrm{Mark}(M[n],j'_{-1}) = D_{M_{1,j'_{-1}}}(t'_3 + D_{M_{1,j'_0}}(t'_4 + c_1 \times n))\)かつ\(\textrm{Trans}(M[n]) = s'_{-1} \textrm{Mark}(M[n],j'_{-1}) b'_{-1}\)である。
				\end{indented}
				\item 以上より
				\begin{eqnarray*}
				\textrm{Trans}(M[n]) = s'_{-1} \textrm{Mark}(M[n],j'_{-1}) b'_{-1} = s'_{-1} D_{M_{1,j'_{-1}}}(t'_3 + D_{M_{1,j'_0}}(t'_4 + c_1 \times n)) b'_{-1} = \textrm{Trans}(M)[n-1]\end{eqnarray*}
				\item である。
			\end{indented}
		\end{indented}
	\end{indented}
\end{hideableproof}


\subsection{強単項性}

条件(II)の下での展開規則を調べるために、強単項性という概念を導入する。

\(M \in \textrm{T}_{\textrm{PS}}\)とする。
\begin{nenumerate}
	\item \(M\)が強単項であるとは、\(M\)が簡約かつ単項かつ\(\textrm{Br}(M)\)が降順であるということである。
	\item 強単項ペア数列全体のなす部分集合を\(DT_{\textrm{PS}} \subset T_{\textrm{PS}}\)と置く。
\end{nenumerate}

\begin{proposition}[標準形の直系先祖による切片の簡約化の強単項性]\label{標準形の直系先祖による切片の簡約化の強単項性}
	任意の\(M \in ST_{\textrm{PS}}\)と\(j'_0,j'_1 \in \mathbb{N}\)に対し、\(j_1 := \textrm{Lng}(M)-1\)と置き、\(j'_0 < j'_1 \leq j_1\)とし\(M' := (M_j)_{j=j'_0}^{j'_1} \)と置くと、\((0,j'_0) \leq_M (0,j'_1)\)ならば\(\textrm{Red}(M')\)は強単項である。
\end{proposition}

\begin{hideableproof}
	\begin{indented}
		\item \nameref{標準形の切片とBrの降順性の関係}より、\(M'\)は単項かつ\(\textrm{Br}(M')\)は降順である。従って\nameref{Redが単項性を保つこと}より\(\textrm{Red}(M')\)は単項である。
		\item \(J_1 := \textrm{Lng}(\textrm{Br}(\textrm{Red}(M')))-1\)と置く。
		\item \nameref{直系先祖による切片とRedとIncrFirstの関係}より\(M' = \textrm{IncrFirst}^{M_{0,j'_0}-M_{1,j'_0}}(\textrm{Red}(M'))\)であるので、\nameref{leq_MのIncrFirst不変性}から\(\textrm{Br}(M') = (\textrm{IncrFirst}^{M_{0,j'_0}-M_{1,j'_0}}(\textrm{Br}(\textrm{Red}(M'))_J))_{J=0}^{J_1}\)である。
		\item 従って\(\textrm{Br}(M')\)の降順性から\(\textrm{Br}(\textrm{Red}(M'))\)は降順である。以上より\(\textrm{Red}(M')\)は強単項である。
	\end{indented}
\end{hideableproof}

写像
\begin{eqnarray*}
\textrm{LastStep} \colon DT_{\textrm{PS}} & \to & \mathbb{N} \\
M & \mapsto & \textrm{LastStep}(M)
\end{eqnarray*}
を以下のように定める:
\begin{nenumerate}
	\item \(J_1 := \textrm{Lng}(\textrm{Br}(M))\)と置く。
	\item \(J_1 = 0\)ならば\(\textrm{LastStep}(M) = 0\)である\footnote{この分岐は使わない。}。
	\item \(J_1 > 0\)とする\footnote{この時\nameref{簡約性と係数の基本性質}より\((\textrm{Br}(M)_{J_1})_{0,0} \geq (\textrm{Br}(M)_{J_1})_{1,0}\)である。}。
	\begin{nenumerate}
		\item \((\textrm{Br}(M)_{J_1})_{0,0} = (\textrm{Br}(M)_{J_1})_{1,0}\)ならば\(\textrm{LastStep}(M) = J_1\)である\footnote{この分岐は使わない}。
		\item \((\textrm{Br}(M)_{J_1})_{0,0} > (\textrm{Br}(M)_{J_1})_{1,0}\)ならば\(\textrm{LastStep}(M) := \min \{J \in \mathbb{N} \mid (\textrm{Br}(M)_{J_1})_{0,0} = (\textrm{Br}(M)_J)_{0,0} > (\textrm{Br}(M)_J)_{1,0}\}\)である\footnote{\(J = J_1\)が条件を満たすため\(\min\)が存在する。}。
	\end{nenumerate}
\end{nenumerate}

\begin{proposition}[条件(II)か(IV)の下での終切片と\(\textrm{Trans}\)の関係]\label{条件(II)か(IV)の下での終切片とTransの関係}
	任意の\(M \in DT_{\textrm{PS}}\)に対し、\(j_1 := \textrm{Lng}(M)-1\)と置き、\(J_1 := \textrm{Lng}(\textrm{Br}(M))-1\)と置き、\(J_1 \geq 0\)として\(j'_0 := \textrm{Joints}(M)_{J_1}\)と置き、 \(j'_1 := \textrm{FirstNodes}(M)_{J_1}\)と置き、\(J_0 := \textrm{LastStep}(M)\)と置き、\(m_1 := \textrm{FirstNodes}(M)_{J_0}-1\)と置き、\(N := (M_j)_{j=0}^{m_1}\)と置き、\(N' := (M_j)_{j=j'_0}^{m_1}\)と置き\footnotemark{}、\(M' := (M_j)_{j=j'_0}^{j_1}\)と置くと、\(0 < j'_0 < \textrm{TrMax}(M)\)かつ\(M_{0,j'_1} > M_{1,j'_1}\)ならば一意な\(t_1,t_2 \in T_{\textrm{B}}\)が存在して以下を満たす:
	\begin{penumerate}
		\item \(\textrm{Trans}(N) = D_{M_{1,0}} t_1\)である。
		\item \(\textrm{Trans}(N') = D_{M_{1,j'_0}} t_1\)である。
		\item \(\textrm{Trans}(M') = D_{M_{1,j'_0}}(t_1 + t_2)\)かつ\(t_2 \neq 0\)である。
		\item \(\textrm{Trans}(M) = D_{M_{1,0}}(t_1 + D_{M_{1,j'_0}}(t_1 + t_2))\)である。
	\end{penumerate}
\end{proposition}
\footnotetext{\(m_1 \geq \textrm{TrMax}(M) \geq j'_0\)より\(N' \in T_{\textrm{PS}}\)である。}

\nameref{条件(II)か(IV)の下での終切片とTransの関係}を証明するための準備としていくつかの補題を示す。

\begin{lemma}[強単項性の切片への遺伝性]\label{強単項性の切片への遺伝性}
	任意の\(M \in DT_{\textrm{PS}}\)と\(j'_0,j'_1 \in \mathbb{N}\)に対し、\(j_1 := \textrm{Lng}(M)-1\)と置き、\(J_1 := \textrm{Lng}(\textrm{Br}(M))-1\)と置き、\(j'_0 < j'_1 \leq j_1\)とし\(M' := (M_j)_{j=j'_0}^{j'_1} \)と置くと、\(j'_0 \leq \textrm{Joints}(M)_{J_1}\)ならば\(M'\)は強単項である。
\end{lemma}

\begin{hideableproof}
	\begin{indented}
		\item \nameref{簡約性と係数の関係}より\(M\)は条件(A)と(B)を満たす。従って任意の\(j \in \mathbb{N}\)に対し\(j \leq \textrm{TrMax}(M)\)ならば\(M_j = (M_{1,0}+j,M_{1,0}+j)\)である。
		\item \nameref{FirstNodesとTrMaxとJointsの関係}より\(j'_0 \leq \textrm{Joints}(M)_{J_1} \leq \textrm{TrMax}(M)\)である。従って\nameref{簡約性の切片への遺伝性}から\(M'\)は簡約である。
		\item \(j_0 := \textrm{TrMax}(M)\)と置く。
		\item \(\textrm{TrMax}(M') = j_0-j'_0\)かつ\(\textrm{Lng}(M')-1 = j'_1-j'_0\)である。
		\item 任意の\(j \in \mathbb{N}\)に対し、\(j \leq j'_1-j'_0\)ならば\((0,0) \leq_M (0,j)\)となることを示す。
		\begin{indented}
			\item \(j \leq \textrm{TrMax}(M')\)ならば\((1,0) \leq_{M'} (1,j)\)なので\((0,0) \leq_{M'} (0,j)\)である。
			\item \(j > \textrm{TrMax}(M')\)とする。
			\begin{indented}
				\item \(j+j'_0 > \textrm{TrMax}(M')+j'_0 = j_0\)であるので、一意な\(J \in \mathbb{N}\)が存在して\(J \leq J_1\)かつ\(\textrm{FirstNodes}(M)_J \leq j+j'_0 < \textrm{FirstNodes}(M)_{J+1}\)となる。\nameref{Pの各成分の非複項性}より\(\textrm{Br}(M)_J\)は複項でないので、\((0,0) \leq_{\textrm{Br}(M)_J} (0,j+j'_0 - \textrm{FistNodes}(M)_J)\)すなわち\((0,\textrm{FirstNodes}(M)_J) \leq_M (0,j+j'_0)\)である。
				\item \(\textrm{Br}(M)\)の降順性から\(M_{0,\textrm{FirstNodes}(M)_J} \geq M_{0,\textrm{FirstNodes}(M)_{J_1}}\)であり、\(M\)が条件(A)を満たすことから\(M_{0,\textrm{Joints}(M)_J}+1 = M_{0,\textrm{FirstNodes}(M)_J}\)かつ\(M_{0,\textrm{Joints}(M)_{J_1}}+1 = M_{0,\textrm{FirstNodes}(M)_{J_1}}\)である。従って\(j'_0 \leq \textrm{Joints}(M)_{J_1} = M_{0,\textrm{Joints}(M)_{J_1}}-M_{1,0} = M_{0,\textrm{FirstNodes}(M)_{J_1}}-1-M_{1,0} \leq M_{0,\textrm{FirstNodes}(M)_J}-1-M_{1,0} = M_{0,\textrm{Joints}(M)_J}-M_{1,0} = \textrm{Joints}(M)_J\)である。
				\item \nameref{FirstNodesとTrMaxとJointsの関係}より\(j'_0 \leq \textrm{Joint}(M)_J \leq \textrm{TrMax}(M)\)であるので\((1,j'_0) \leq_M (1,\textrm{Joints}(M)_J)\)である。更に\((0,\textrm{Joints}(M)_J) <_M^{\textrm{Next}} (0,\textrm{FirstNodes}(M))\)かつ\((0,\textrm{FirstNodes}(M)_J) \leq_M (0,j+j'_0)\)より、\((0,j'_0) \leq_M (0,j+j'_0)\)すなわち\((0,0) \leq_{M'} (0,j)\)である。
			\end{indented}
		\end{indented}
		\item 以上より\(M'\)は単項である。
		\item \(J_0 := \textrm{Lng}(\textrm{Br}(M'))-1\)と置く。
		\item \nameref{PとIdxSumの合成の特徴付け}より、\(J_0 \leq J_1\)かつ\(\textrm{FirstNodes}(M') = (\textrm{FirstNodes}(M)_J-j'_0)_{J=0}^{J_0}\)である。従って\(\textrm{Br}(M')\)は降順である。
	\end{indented}
\end{hideableproof}

\begin{lemma}[部分表現の単項成分と\(\textrm{Pred}\)の関係]\label{部分表現の単項成分とPredの関係}
	任意の\(M \in RT_{\textrm{PS}} \cap PT_{\textrm{PS}}\)に対し、\(j_1 := \textrm{Lng}(M)-1\)と置き、\(J_1 := \textrm{Lng}(\textrm{Br}(M))-1\)と置き、\(J_1 \geq 0\)として\(j'_0 := \textrm{Joints}(M)_{J_1}\)と置き、\(j'_1 := \textrm{FirstNodes}(M)_{J_1}\)と置くと、\(j_1 > 1\)ならば以下のいずれかが成り立つ:
	\begin{penumerate}
		\item \(j'_1 = j_1\)であり、\(\textrm{TrMax}(M) = 0\)または\(j'_0 < \textrm{TrMax}(M)\)であり、\(M_{0,j'_1} = M_{1,j'_1}\)または\(j'_0\)が\(M\)許容であり、一意な\(t_1 \in T_{\textrm{B}}\)が存在して\(\textrm{Trans}(\textrm{Pred}(M)) = D_{M_{1,0}} t_1\)かつ\(\textrm{Trans}(M) = D_{M_{1,0}}(t_1 + D_{M_{1,j'_1}} 0)\)である。
		\item \(j'_1 = j_1\)であり、\(M_{0,j'_1} > M_{1,j'_1}\)かつ\(j'_0\)が非\(M\)許容であり、一意な\((t_1,t_2) \in T_{\textrm{B}}^2\)が存在して\(\textrm{Trans}(\textrm{Pred}(M)) = D_{M_{1,0}} t_1\)かつ\(\textrm{Trans}(M) = D_{M_{1,0}}(t_1 + D_{M_{1,j'_0}} t_2)\)である。
		\item 一意な\((t_1,t_2,t_3) \in T_{\textrm{B}}^2\)が存在して\(\textrm{Trans}(\textrm{Pred}(M)) = D_{M_{1,0}}(t_1 + D_{M_{1,j'_1}} t_2)\)かつ\(\textrm{Trans}(M) = D_{M_{1,0}}(t_1 + D_{M_{1,j'_1}} t_3)\)である。
		\item 一意な\((t_1,t_2,t_3) \in T_{\textrm{B}}^2\)が存在して\(\textrm{Trans}(\textrm{Pred}(M)) = D_{M_{1,0}}(t_1 + D_{M_{1,j'_0}} t_2)\)かつ\(\textrm{Trans}(M) = D_{M_{1,0}}(t_1 + D_{M_{1,j'_0}} t_3)\)である。
	\end{penumerate}
\end{lemma}

\begin{hideableproof}
	\begin{indented}
		\item \nameref{簡約性と係数の関係}から\(M\)は条件(A)と(B)を満たす。\(M\)が条件(A)と(B)を満たすことから任意の\(j \in \mathbb{N}\)に対し\(j \leq \textrm{TrMax}(M)\)ならば\(M_j = (M_{1,0} + j, M_{1,0} + j)\)である。また\(M\)が条件(A)を満たすことから\(M_{0,j'_0}+1 = M_{0,j'_1}\)である。
		\item \(j_1 > 1\)より\(\textrm{Pred}(M)\)は零項でなく、\nameref{Transが零項性を保つこと}から\(\textrm{Trans}(\textrm{Pred}(M)) \neq 0\)である。従って\(M\)に対し条件(I)~(VI)が意味を持つ。
		\item \(J_1 \geq 0\)より\(\textrm{TrMax}(M) < j_1\)である。また\(M\)の単項性から\((0,0) \leq_M (0,j_1)\)であり、\(j_1 > 1 > 0\)より\((0,j_0) <_M^{\textrm{Next}} (0,j_1)\)を満たす一意な\(j_0 \in \mathbb{N}\)が存在する。
		\item まず\(j_1-j'_1 = 0\)ならば\(M\)は条件(VI)を満たさないことを示す。
		\begin{indented}
			\item \((0,j'_0) <_M^{\textrm{Next}} (0,j'_1) = (0,j_1)\)より\(j_0 = j'_0\)である。また\nameref{FirstNodesとTrMaxとJointsの関係}より\(j'_0 \leq \textrm{TrMax}(M)\)である。
			\item \(j_0+1 = j_1\)かつ\((1,j_0) <_M^{\textrm{Next}} (1,j_1)\)と仮定すると\(\textrm{TrMax}(M) = j_1\)となり\(\textrm{TrMax}(M) < j_1\)と矛盾する。従って\(j_0+1 = j_1\)ならば\((1,j_0) <_M^{\textrm{Next}} (1,j_1)\)でない。特に\(M\)は条件(VI)を満たさない。
		\end{indented}
		\item \(j_{-1} := \textrm{Adm}_M(j_0)\)と置く。
		\item (1)~(4)のいずれかが成り立つことを\(j_1 - \textrm{TrMax}(M)\)に関する数学的帰納法で示す。
		\item \(j_1 - \textrm{TrMax}(M) = 1\)とする。
		\begin{indented}
			\item \(\textrm{Lng}((M_j)_{j=\textrm{TrMax}(M)+1}^{j_1}) = 1\)より\(J_1 = 0\)かつ\(j'_1 = j_1\)かつ\(\textrm{TrMax}(M) = j'_1-1\)かつ\(j'_0 = M_{0,j'_0}-M_{1,0} = M_{0,j'_1}-1-M_{1,0}\)となる。特に\(j_1-j'_1 = 0\)であるので、\(j_0 = j'_0\)かつ\(M\)は条件(VI)を満たさない。
			\item \(\textrm{Pred}(M) = (M_{1,0}+j,M_{1,0}+j)_{j=0}^{j'_1-1}\)であり\(j'_1-1 = j_1-1 > 0\)であるので、\nameref{公差(1,1)のペア数列のTransの基本性質}より\(\textrm{Trans}(\textrm{Pred}(M)) = D_{M_{1,0}} D_{M_{1,j'_1-1}} 0\)である。
			\item \(M\)が条件(VI)を満たさないことから、\(M\)は\nameref{Predが公差(1,1)のペア数列のTransの基本性質} (1)~(4)のいずれかの条件を満たす。
			\item \(M\)が\nameref{Predが公差(1,1)のペア数列のTransの基本性質} (1)の条件を満たすならば、\(j_0+1 = j_1\)となるので\(j'_0 = j_0 = j_1-1 = j'_1-1\)であり、従って\((t_1,t_2,t_3) := (0,0,D_{M_{1,j'_1}} 0)\)と置くと\(\textrm{Trans}(\textrm{Pred}(M)) = D_{M_{1,0}}(t_1 + D_{M_{1,j'_1-1}} t_2) = D_{M_{1,0}}(t_1 + D_{M_{1,j'_0}} t_2)\)かつ\(\textrm{Trans}(M) = D_{M_{1,0}}(t_1 + D_{M_{1,j'_1-1}} t_3) = D_{M_{1,0}}(t_1 + D_{M_{1,j'_0}} t_3)\)となるので、(4)が成り立つ。
			\item \(M\)が\nameref{Predが公差(1,1)のペア数列のTransの基本性質} (2)の条件を満たすならば、\(M_{0,j'_1} = M_{1,j'_1}\)であり、\((t_1,t_2) := (D_{M_{1,j'_1-1}} 0, 0)\)と置くと\(\textrm{Trans}(\textrm{Pred}(M)) = D_{M_{1,0}} t_1\)かつ\(\textrm{Trans}(M) = D_{M_{1,0}}(t_1 + D_{M_{1,j'_1}} t_2)\)となるので、(1)が成り立つ。
			\item \(M\)が\nameref{Predが公差(1,1)のペア数列のTransの基本性質} (3)の条件を満たすならば、\(M_{0,j'_1} > M_{1,j'_1}\)であり、\(j'_0 = M_{0,j'_1}-1-M_{1,0}\)かつ\(M_{1,0}+1 < M_{0,j'_1} \leq M_{0,j'_1-1}\)より\(0 < j'_0 \leq \textrm{TrMax}(M)-1\)となるので\(j'_0\)は非\(M\)許容であり、\((t_1,t_2) := (D_{M_{1,j'_1-1}} 0, \underline{(} D_{M_{1,j'_1-1}} 0 \underline{,} D_{M_{1,j'_0}} 0 \underline{)})\)と置くと\(\textrm{Trans}(\textrm{Pred}(M)) = D_{M_{1,0}} t_1\)かつ\(\textrm{Trans}(M) = D_{M_{1,0}}(t_1 + D_{M_{1,j'_0}} t_2)\)となるので、(2)が成り立つ。
			\item \(M\)が\nameref{Predが公差(1,1)のペア数列のTransの基本性質} (4)の条件を満たすならば、\(j'_0 = M_{0,j'_1}-1-M_{1,0} = 0\)となるので\(j'_0 = 0\)となり\(j'_0\)は\(M\)許容であり、\((t_1,t_2) := (D_{M_{1,j'_1-1}} 0, 0)\)と置くと\(\textrm{Trans}(\textrm{Pred}(M)) = D_{M_{1,0}} t_1\)かつ\(\textrm{Trans}(M) = D_{M_{1,0}}(t_1 + D_{M_{1,j'_1}} t_2)\)となるので、(1)が成り立つ。
		\end{indented}
		\item \(j_1 - \textrm{TrMax}(M) > 1\)とする。
		\begin{indented}
			\item \(\textrm{Adm}_M(j_0) > 0\)とする。
			\begin{indented}
				\item \(j_0 = j'_0 \leq \textrm{TrMax}(M)\)であるので\(\textrm{TrMax}\)の定義から\(\textrm{Adm}_M(j_0) = \textrm{TrMax}(M)\)となり、再び\(\textrm{Adm}_M(j_0) \leq j_0 \leq \textrm{TrMax}(M) = \textrm{Adm}_M(j_0)\)より\(j_0 = \textrm{Adm}_M(j_0) = \textrm{TrMax}(M)\)である。
				\item \nameref{簡約性の切片への遺伝性}と\nameref{単項性の始切片への遺伝性}から\(\textrm{Pred}(M)\)は簡約かつ単項である。\(j_1 \geq j_1 - \textrm{TrMax}(M) > 1\)より\(\textrm{Lng}(\textrm{Pred}(M))-1 = j_1 > 1\)である。
				\item \(J'_1 := \textrm{Lng}(\textrm{Br}(\textrm{Pred}(M)))-1\)と置く。
				\item \(j'_1 = j_1\)ならば、\(j_1 - \textrm{TrMax}(M) > 1\)より\(J_1 > 0\)であり、\nameref{PとIdxSumの合成の特徴付け}から、\(J'_1 = J_1-1 \geq 0\)である。
				\item \(j'_1 < j_1\)ならば、\nameref{PとIdxSumの合成の特徴付け}から、\(J'_1 = J_1 \geq 0\)である。
				\item 以上よりいずれの場合も\(J'_1 \geq 0\)である。
				\item \(\textrm{Lng}(\textrm{Pred}(M))-1 = j_1-1\)かつ\(\textrm{TrMax}(\textrm{Pred}(M')) = \textrm{TrMax}(M)\)より\((\textrm{Lng}(\textrm{Pred}(M))-1) - \textrm{TrMax}(\textrm{Pred}(M')) = j_1-1 - \textrm{TrMax}(M) < j_1 - \textrm{TrMax}(M)\)であるので、帰納法の仮定から\footnote{上までの議論により、\(\textrm{Pred}(M)\)も\(M\)と同じ条件を満たすことが確認されたので\(\textrm{Pred}(M)\)に対し帰納法の仮定が適用可能である。}、一意な\(i \in \{0,1\}\)と\((t_1,t_2) \in T_{\textrm{B}}^2\)が存在して\(\textrm{Trans}(\textrm{Pred}(M)) = D_{M_{1,0}}(t_1 + D_{M_{1,j'_i}} t_2)\)となる。
				\item \(\textrm{Adm}_M(j_0) > 0\)と\nameref{s_1とb_1の空性と基点の関係}から\(\textrm{Mark}(\textrm{Pred}(M),\textrm{Adm}_M(j_0)) \neq \textrm{Trans}(\textrm{Pred}(M)) = D_{M_{1,0}}(t_1 + D_{M_{1,j'_i}} t_2)\)となるので、\((\textrm{Trans}(\textrm{Pred}(M)),\textrm{Mark}(\textrm{Pred}(M),\textrm{Adm}_M(j_0))) \in T_{\textrm{B}}^{\textrm{Marked}}\)から\((D_{M_{1,j'_i}} t_2,\textrm{Mark}(\textrm{Pred}(M),\textrm{Adm}_M(j_0))) \in T_{\textrm{B}}^{\textrm{Marked}}\)となる。
				\item 従って\nameref{加法とscb分解の関係}と\(\textrm{Trans}\)の定義から、一意な\(t_3 \in T_{\textrm{B}}\)が存在して\(\textrm{Trans}(M) = D_{M_{1,0}}(t_1 + D_{M_{1,j'_i}} t_3)\)であり、(3)か(4)が成り立つ。
			\end{indented}
			\item \(\textrm{Adm}_M(j_0) = 0\)とする。
			\begin{indented}
				\item \(\textrm{TrMax}\)の定義から\(\textrm{TrMax}(M)\)は\(M\)許容である。\(\textrm{Adm}_M(j_0) = 0 \leq \textrm{TrMax}(M)\)より、\(j_0 \leq \textrm{TrMax}(M)\)となる。
				\item \(j'_1 < j_1\)と仮定すると、\nameref{Pの各成分の非複項性}から\((0,0) \leq_{\textrm{Br}_{J_1}} (0,j_1-j'_1)\)すなわち\((0,j'_1) \leq_M (0,j_1)\)であり、\(j'_1 < j_1\)と\nameref{親の存在の判定条件} (1)から\(\textrm{TrMax}(M) < j'_1 \leq j_0\)となり\(j_0 \leq \textrm{TrMax}(M)\)と矛盾する。従って\(j'_1 = j_1\)であり、\(j_0 = j'_0\)でありかつ\(M\)は条件(VI)を満たさない。
				\item \(\textrm{TrMax}(M) > 0\)ならば\(\textrm{Adm}_M(j'_0) = \textrm{Adm}_M(j_0) = 0 < \textrm{TrMax}(M) = \textrm{Adm}_M(\textrm{TrMax}(M))\)より\(j'_0 < \textrm{TrMax}(M)\)である。
				\item 従って、\(\textrm{TrMax}(M) = 0\)または\(j'_0 < \textrm{TrMax}(M)\)である。
				\item \(M\)が条件(I)か(III)か(V)を満たすとする。
				\begin{indented}
					\item \(M\)が条件(I)か(III)を満たすならば\(j'_0 = j_0\)は\(M\)許容である。
					\item \(M\)が条件(V)を満たすならば\(M_{0,j'_1} = M_{0,j'_0}+1 = M_{1,j'_0}+1 = M_{1,j_0}+1 = M_{1,j_1} = M_{1,j'_1}\)である。
					\item \nameref{Markの左端の基本性質}と\nameref{s_1とb_1の空性と基点の関係}と\(\textrm{Trans}\)の定義から、一意な\(t_1 \in T_{\textrm{B}}\)が存在して\(\textrm{Trans}(\textrm{Pred}(M)) = D_{M_{1,0}} t_1\)かつ\(\textrm{Trans}(M) = D_{M_{1,0}}(t_1 + D_{M_{1,j'_1}} 0)\)であり、(1)が成り立つ。
				\end{indented}
				\item \(M\)が条件(II)か(IV)を満たすとする。
				\begin{indented}
					\item \(M_{0,j'_1} \geq M_{0,j'_0} = M_{1,j'_0} = M_{1,j_0} = M_{1,j_1}-1 = M_{1,j'_1}-1\)すなわち\(M_{0,j'_1} > M_{1,j'_1}\)である。また\(j'_0 = j_0\)は\(M\)許容でない。
					\item \nameref{Markの左端の基本性質}と\nameref{s_1とb_1の空性と基点の関係}と\(\textrm{Trans}\)の定義から、一意な\((t_1,t_2) \in T_{\textrm{B}}^2\)が存在して\(\textrm{Trans}(\textrm{Pred}(M)) = D_{M_{1,0}} t_1\)かつ\(\textrm{Trans}(M) = D_{M_{1,0}}(t_1 + D_{M_{1,j'_0}} t_2)\)である。
					\item 以上より(2)が成り立つ。
				\end{indented}
			\end{indented}
		\end{indented}
	\end{indented}
\end{hideableproof}

\begin{lemma}[強単項性の下での部分表現の単項成分の基本性質]\label{強単項性の下での部分表現の単項成分の基本性質}
	任意の\(M \in DT_{\textrm{PS}}\)に対し、\(j_1 := \textrm{Lng}(M)-1\)と置き、\(J_1 := \textrm{Lng}(\textrm{Br}(M))-1\)と置き、\(J_1 \geq 0\)として\(j'_0 := \textrm{Joints}(M)_{J_1}\)と置き、\(j'_1 := \textrm{FirstNodes}(M)_{J_1}\)と置くと、一意な\(t' \in T_{\textrm{B}}\)が存在して以下を満たす:
	\begin{penumerate}
		\item \(\textrm{Trans}(M) = D_{M_{1,0}} t'\)である。
		\item \(j'_0 = 0\)または\(M_{0,j'_1} = M_{1,j'_1}\)ならば、\(t'\)の各単項成分は\(D_{M_{1,j'_1}} 0\)以上である。
		\item \(0 < j'_0 < \textrm{TrMax}(M)\)かつ\(M_{0,j'_1} > M_{1,j'_1}\)ならば、\(t'\)の各単項成分は\(D_{M_{1,j'_0}} 0\)以上である。
		\item \(0 < j'_0 = \textrm{TrMax}(M)\)ならば、\(t'\)の各単項成分は\(D_{M_{1,\textrm{TrMax}(M)}} 0\)以上である。
	\end{penumerate}
\end{lemma}

\begin{hideableproof}
	\begin{indented}
		\item 以下\nameref{FirstNodesとJointsの単調性}は断りなく用いる。
		\item \nameref{Transが単項性を保つこと}から\(\textrm{Trans}(M)\)は単項であり、\nameref{Pの各成分の非複項性}と\nameref{Transの最左単項成分の左端の基本性質}より一意な\(t' \in T_{\textrm{B}}\)が存在して\(\textrm{Trans}(M) = D_{M_{1,0}} t'\)である。
		\item \nameref{FirstNodesとTrMaxとJointsの関係}より、任意の\(J \in \mathbb{N}\)に対し\(J \leq J_1\)ならば\(\textrm{Joints}(M)_J \leq \textrm{TrMax}(M) < \textrm{FirstNodes}(M)_J\)である。特に\(j'_0 \leq\textrm{TrMax}(M) < j'_1\)である。また\(j_1 \geq j'_1 > j'_0 \geq 0\)より\(j_1 > 0\)である。
		\item \(\textrm{TrMax}\)の定義から\(\textrm{TrMax}(M)\)は\(M\)許容である。従って\((1,\textrm{TrMax}(M)) <_M^{\textrm{Next}} (1,\textrm{TrMax}(M)+1) = (1,\textrm{FirstNodes}(M)_0)\)でなく、すなわち\(M_{1,\textrm{TrMax}(M)} \geq M_{1,\textrm{FirstNodes}(M)_0}\)である。
		\item \nameref{簡約性と係数の関係}から\(M\)は条件(A)と(B)を満たす。\(M\)が条件(A)と(B)を満たすことから任意の\(j \in \mathbb{N}\)に対し\(j \leq \textrm{TrMax}(M)\)ならば\(M_j = (M_{1,0} + j, M_{1,0} + j)\)である。特に\(M_{1,\textrm{TrMax}(M)} = M_{1,0} + \textrm{TrMax}(M) \geq M_{1,0} + j'_0 = M_{1,j'_0}\)である。また\(M\)が条件(A)を満たすことから\(M_{1,j'_0} = M_{0,j'_0} = M_{0,j'_1}-1\)である。
		\item (2)と(3)と(4)が成り立つことを\(j_1 - \textrm{TrMax}(M)\)に関する数学的帰納法で示す。
		\item \(j_1 - \textrm{TrMax}(M) = 1\)とする。
		\begin{indented}
			\item \(\textrm{TrMax}(M) < j'_1 \leq j_1\)より\(j'_1 = j_1\)である。\(\textrm{TrMax}(M)\)の\(M\)許容性から\((1,j'_0) = (1,\textrm{TrMax}(M)) <_M^{\textrm{Next}} (1,\textrm{TrMax}(M)+1) = (1,j_1)\)でないので、\(M_{1,j'_0} \geq M_{1,j'_1}\)である。従って\(M_{0,j'_1} = M_{0,j'_0}+1 = M_{1,j'_0}+1 \geq M_{1,j'_1}+1 > M_{1,j'_1}\)である。
			\item \(j_1 = 1\)とする。
			\begin{indented}
				\item \(\textrm{TrMax}(M) = j_1-1 = 0\)である。\(j'_0 \leq \textrm{TrMax}(M) = 0\)より\(j'_0 = 0 = \textrm{TrMax}(M)\)である。
				\item \nameref{2列ペア数列の基本性質}より\(t' = D_{M_{1,j_1}} 0\)であり、その唯一の単項成分は\(D_{M_{1,j_1}} 0 = D_{M_{1,j'_1}} 0\)である。
			\end{indented}
			\item \(j_1 > 1\)とする。
			\begin{indented}
				\item \(j'_0 = 0\)ならば、\nameref{Predが公差(1,1)のペア数列のTransの基本性質} (4)から\(t' = \underline{(} D_{M_{1,\textrm{TrMax}(M)}} 0 \underline{,} D_{M_{1,j_1}} 0 \underline{)}\)であり、\(D_{M_{1,\textrm{TrMax}(M)}} 0 \geq D_{M_{1,j'_0}} 0 \geq D_{M_{1,j'_1}} 0\)かつ\(D_{M_{1,j_1}} 0 = D_{M_{1,j'_1}} 0\)である。
				\item \(0 < j'_0 < \textrm{TrMax}(M)\)ならば、\(\textrm{TrMax}\)の定義から\(j'_0\)は\(M\)許容でなく、\nameref{Predが公差(1,1)のペア数列のTransの基本性質} (3)から\(t' = \underline{(} D_{M_{1,\textrm{TrMax}(M)}} 0 \underline{,} D_{M_{0,j'_1}-1} \underline{(} D_{M_{1,\textrm{TrMax}(M)}} 0 \underline{,} D_{M_{1,j_1}} 0 \underline{)} \underline{)}\)であり、\(D_{M_{1,\textrm{TrMax}(M)}} 0 \geq D_{M_{1,j'_0}} 0\)かつ\(D_{M_{0,j'_1}-1} \underline{(} D_{M_{1,\textrm{TrMax}(M)}} 0 \underline{,} D_{M_{1,j_1}} 0 \underline{)} = D_{M_{1,j'_0}} \underline{(} D_{M_{1,\textrm{TrMax}(M)}} 0 \underline{,} D_{M_{1,j_1}} 0 \underline{)} \geq D_{M_{1,j'_0}} 0\)である。
				\item \(0 < j'_0 = \textrm{TrMax}(M)\)ならば、\nameref{Predが公差(1,1)のペア数列のTransの基本性質} (1)から\(t' = D_{M_{1,\textrm{TrMax}(M)}} D_{M_{1,j_1}} 0\)である。
			\end{indented}
		\end{indented}
		\item \(j_1 - \textrm{TrMax}(M) > 1\)とする。
		\begin{indented}
			\item \nameref{簡約性の切片への遺伝性}と\nameref{単項性の始切片への遺伝性}から\(\textrm{Pred}(M)\)は簡約かつ単項である。\(j_1 \geq j_1 - \textrm{TrMax}(M) > 1\)より\(\textrm{Lng}(\textrm{Pred}(M))-1 = j_1 > 1\)である。
			\item \(J'_1 := \textrm{Lng}(\textrm{Br}(\textrm{Pred}(M)))-1\)と置く。
			\item \(j'_1 = j_1\)ならば、\(j_1 - \textrm{TrMax}(M) > 1\)より\(J_1 > 0\)であり、\nameref{PとIdxSumの合成の特徴付け}から、\(J'_1 = J_1-1 \geq 0\)である。
			\item \(j'_1 < j_1\)ならば、\nameref{PとIdxSumの合成の特徴付け}から、\(J'_1 = J_1 \geq 0\)である。
			\item 以上よりいずれの場合も\(\textrm{Lng}(J'_1) \geq 0\)である。
			\item \(\textrm{Lng}(\textrm{Pred}(M))-1 = j_1-1\)かつ\(\textrm{TrMax}(\textrm{Pred}(M')) = \textrm{TrMax}(M)\)より\((\textrm{Lng}(\textrm{Pred}(M))-1) - \textrm{TrMax}(\textrm{Pred}(M')) = j_1-1 - \textrm{TrMax}(M) < j_1 - \textrm{TrMax}(M)\)であるので、帰納法の仮定から\footnote{上までの議論により、\(\textrm{Pred}(M)\)も\(M\)と同じ条件を満たすことが確認されたので\(\textrm{Pred}(M)\)に対し帰納法の仮定が適用可能である。}、一意な\(t \in T_{\textrm{B}}\)が存在して以下を満たす\footnote{\(0, \textrm{Joints}(\textrm{Pred}(M))_{J'_1}, \textrm{FirstNodes}(\textrm{Pred}(M))_{J'_1}\)はいずれも\(\textrm{Lng}(\textrm{Pred}(M))-1 = j_1-1\)以下であるので、\nameref{PとIdxSumの合成の特徴付け}から\(\textrm{Joints}(\textrm{Pred}(M))_{J'_1} = \textrm{Joints}(M)_{J'_1}\)かつ\(\textrm{FirstNodes}(\textrm{Pred}(M))_{J'_1} = \textrm{FirstNodes}(M)_{J'_1}\)であり、\(\textrm{Pred}(M)_0 = M_0\)かつ\(\textrm{Pred}(M)_{\textrm{Joints}(\textrm{Pred}(M))_{J'_1}} = M_{\textrm{Joints}(M)_{J'_1}}\)かつ\(\textrm{Pred}(M)_{\textrm{FirstNodes}(\textrm{Pred}(M))_{J'_1}} = M_{\textrm{FirstNodes}(M)_{J'_1}}\)であり、また\(\textrm{TrMax}(\textrm{Pred}(M)) = \textrm{TrMax}(M)\)であることに注意する。}:
			\begin{penumerate}
				\item \(\textrm{Trans}(\textrm{Pred}(M)) = D_{M_{1,0}} t\)である。
				\item \(\textrm{Joints}(M)_{J'_1} = 0\)または\(M_{0,\textrm{FirstNodes}(M)_{J'_1}} = M_{1,\textrm{FirstNodes}(M)_{J'_1}}\)ならば、\(t\)の各単項成分は\(D_{M_{1,\textrm{FirstNodes}(M)_{J'_1}}} 0\)以上である。
				\item \(0 < \textrm{Joints}(M)_{J'_1} < \textrm{TrMax}(M)\)かつ\(M_{0,\textrm{FirstNodes}(M)_{J'_1}} > M_{1,\textrm{FirstNodes}(M)_{J'_1}}\)ならば、\(t\)の各単項成分は\(D_{M_{1,\textrm{Joints}(M)_{J'_1}}} 0\)以上である。
				\item \(0 < \textrm{Joints}(M)_{J'_1} = \textrm{TrMax}(M)\)ならば、\(t\)の各単項成分は\(D_{M_{1,\textrm{TrMax}(M)_{J'_1}}} 0\)以上である。
			\end{penumerate}
			\item \(\textrm{Joints}(M)_{J'_1} = M_{0,\textrm{Joints}(M)_{J'_1}}-M_{1,0} = M_{0,\textrm{FirstNodes}(M)_{J'_1}}-1-M_{1,0} \geq M_{0,j'_1}-1-M_{1,0} = M_{0,j'_0}-M_{1,0} = j'_0\)である。また\nameref{簡約性と係数の基本性質}から\(M_{0,j'_1} \geq M_{1,j'_1}\)である。
			\item \(j'_0 = 0\)または\(M_{0,j'_1} = M_{1,j'_1}\)とする。
			\begin{penumerate}
				\setcounter{penumeratei}{1}
				\item \(\textrm{Joints}(M)_{J'_1} = 0\)または\(M_{0,\textrm{FirstNodes}(M)_{J'_1}} = M_{1,\textrm{FirstNodes}(M)_{J'_1}}\)とする。
				\begin{indented}
					\item \(t\)の各単項成分は\(D_{M_{1,\textrm{FirstNodes}(M)_{J'_1}}} 0\)以上である。
					\item \(\textrm{Joints}(M)_{J'_1} = 0\)とする。
					\begin{indented}
						\item \(j'_1 \leq \textrm{Joints}(M)_{J'_1} = 0\)より\(j'_0 = 0 = \textrm{Joints}(M)_{J'_1}\)である。
						\item \(M_{0,\textrm{FirstNodes}(M)_{J'_1}} = M_{0,\textrm{Joints}(M)_{J'_1}}+1 = M_{0,j'_0}+1 = M_{0,j'_1}\)であるので、\(M\)の強許容性から\(M_{1,\textrm{FirstNodes}(M)_{J'_1}} \geq M_{1,j'_1}\)である。
						\item \(M_{0,\textrm{FirstNodes}(M)_{J'_1}} = M_{1,\textrm{FirstNodes}(M)_{J'_1}}\)ならば、\(M_{1,\textrm{FirstNodes}(M)_{J'_1}} \geq M_{0,\textrm{FirstNodes}(M)_{J'_1}} \geq M_{0,j'_1} \geq M_{1,j'_1}\)である。
						\item 以上より、いずれの場合も\(M_{1,\textrm{FirstNodes}(M)_{J'_1}} \geq M_{1,j'_1}\)である。
					\end{indented}
					\item \(M_{0,\textrm{FirstNodes}(M)_{J'_1}} = M_{1,\textrm{FirstNodes}(M)_{J'_1}}\)ならば、\(P\)の定義から\(M_{1,\textrm{FirstNodes}(M)_{J'_1}} = M_{0,\textrm{FirstNodes}(M)_{J'_1}} \geq M_{0,j'_1} \geq M_{1,J'_1}\)である。
					\item 以上より、いずれの場合も\(t\)の各単項成分は\(D_{M_{1,j'_1}} 0\)以上である。
				\end{indented}
				\item \(0 < \textrm{Joints}(M)_{J'_1} < \textrm{TrMax}(M)\)かつ\(M_{0,\textrm{FirstNodes}(M)_{J'_1}} > M_{1,\textrm{FirstNodes}(M)_{J'_1}}\)ならば、\(t\)の各単項成分は\(D_{M_{0,\textrm{Joints}(M)_{J'_1}}} 0\)以上であり、\(M_{0,\textrm{FirstNodes}(M)_{J'_1}} \geq M_{0,j'_1} \geq M_{1,j'_1}\)であるので、\(t\)の各単項成分は\(D_{M_{1,j'_1}} 0\)以上である。
				\item \(0 < \textrm{Joints}(M)_{J'_1} = \textrm{TrMax}(M)\)とする。
				\begin{indented}
					\item \(t\)の各単項成分は\(D_{M_{1,\textrm{TrMax}(M)}} 0\)以上である。
					\item \(j'_0 = \textrm{TrMax}(M)\)と仮定して矛盾を導く。
					\begin{indented}
						\item \(j'_0 \neq 0\)より\(M_{1,j'_1} = M_{0,j'_1} = M_{0,j'_0}+1 = M_{0,\textrm{TrMax}(M)}+1 = M_{1,\textrm{TrMax}(M)}+1\)となる。
						\item \(M_{0,j'_1} \leq M_{0,\textrm{FirstNodes}(M)_0} = M_{0,\textrm{Joints}(M)_0}+1 \leq M_{0,\textrm{TrMax}(M)}+1 = M_{0,j'_1}\)より\(M_{0,\textrm{FirstNodes}(M)_0} = M_{0,\textrm{TrMax}(M)}+1 = M_{0,j'_1}\)であり、\(M\)の強許容性から\(M_{1,\textrm{FirstNodes}(M)_0} \geq M_{1,j'_1} = M_{1,\textrm{TrMax}(M)}+1\)である。
						\item 更に\(\textrm{FirstNodes}(M)_0 = \textrm{TrMax}(M)+1\)より\((0,\textrm{TrMax}(M)) <_M^{\textrm{Next}} (0,\textrm{TrMax}(M)+1)\)となり、これは\(\textrm{TrMax}\)の定義に矛盾する。
					\end{indented}
					\item 以上より\(j'_0 < \textrm{TrMax}(M)\)である。
					\item \(M_{1,\textrm{TrMax}(M)} = M_{0,\textrm{TrMax}(M)} \geq M_{0,j'_0}+1 = M_{0,j'_1} \geq M_{1,j'_1}\)となるので、\(t\)の各単項成分は\(D_{M_{1,j'_1}} 0\)以上である。
				\end{indented}
				\item[] 以上よりいずれの場合も\(t\)の各単項成分は\(D_{M_{1,j'_1}} 0\)以上である。従って\nameref{部分表現の単項成分とPredの関係} (1)と(3)と(4)から、一意な\(t' \in T_{\textrm{B}}\)が存在して以下を満たす:
				\begin{penumerate}
					\item \(\textrm{Trans}(M) = D_{M_{1,0}} t'\)である。
					\item \(t'\)の各単項成分は\(D_{M_{1,j'_1}} 0\)以上である。
				\end{penumerate}
			\end{penumerate}
			\item \(0 < j'_0 < \textrm{TrMax}(M)\)かつ\(M_{0,j'_1} > M_{1,j'_1}\)とする。
			\begin{penumerate}
				\item[] \(\textrm{Joints}(M)_{J'_1} \geq j'_0 > 0\)である。
				\setcounter{penumeratei}{1}
				\item \(M_{0,\textrm{FirstNodes}(\textrm{Pred}(M))_{J'_1}} = M_{1,\textrm{FirstNodes}(M)_{J'_1}}\)ならば、\(t\)の各単項成分は\(D_{M_{1,\textrm{FirstNodes}(M)_{J'_1}}} 0\)以上であり、\(P\)の定義から\(M_{1,\textrm{FirstNodes}(M)_{J'_1}} = M_{0,\textrm{FirstNodes}(M)_{J'_1}} \geq M_{0,j'_0} = M_{1,j'_0}\)であるので、\(t\)の各単項成分は\(D_{M_{1,j'_0}} 0\)以上である。
				\item \(0 < \textrm{Joints}(M)_{J'_1} < \textrm{TrMax}(M)\)かつ\(M_{0,\textrm{FirstNodes}(M)_{J'_1}} > M_{1,\textrm{FirstNodes}(M)_{J'_1}}\)とする。
				\begin{indented}
					\item \(t\)の各単項成分は\(D_{M_{1,\textrm{Joints}(M)_{J'_1}}} 0\)以上である。
					\item \(j'_0 = \textrm{Joints}(M)_{J'_1}\)ならば\(M\)の強許容性から\(M_{1,\textrm{Joints}(M)_{J'_1}} \geq M_{1,j'_0}\)である。
					\item \(j'_0 < \textrm{Joints}(M)_{J'_1}\)ならば\nameref{簡約性と係数の基本性質}から\(M_{1,\textrm{Joints}(M)_{J'_1}} = M_{0,\textrm{Joints}(M)_{J'_1}} > M_{0,j'_0} = M_{1,j'_0}\)である。
					\item 以上より、いずれの場合も\(t\)の各単項成分は\(D_{M_{1,j'_0}} 0\)以上である。
				\end{indented}
				\item \(0 < \textrm{Joints}(M)_{J'_1} = \textrm{TrMax}(M)\)ならば、\(t\)の各単項成分は\(D_{M_{1,\textrm{TrMax}(M)}} 0\)以上であり、\(M_{1,\textrm{TrMax}(M)} = M_{0,\textrm{TrMax}(M)} > M_{0,j'_0} = M_{1,j'_0}\)であるので、\(t\)の各単項成分は\(D_{M_{1,j'_0}} 0\)以上である。
				\item[] 以上より、いずれの場合も\(t\)の各単項成分は\(D_{M_{1,j'_0}} 0\)以上である。従って\nameref{部分表現の単項成分とPredの関係} (2)と(3)と(4)から、一意な\(t' \in T_{\textrm{B}}\)が存在して以下を満たす:
				\begin{penumerate}
					\item \(\textrm{Trans}(M) = D_{M_{1,0}} t'\)である。
					\setcounter{penumerateii}{2}
					\item \(0 < j'_0 < \textrm{TrMax}(M)\)かつ\(M_{0,j'_1} > M_{1,j'_1}\)ならば、\(t'\)の各単項成分は\(D_{M_{1,j'_0}} 0\)以上である。
				\end{penumerate}
			\end{penumerate}
			\item \(0 < j'_0 = \textrm{TrMax}(M)\)とする。
			\begin{penumerate}
				\item[] \(\textrm{TrMax}(M) \geq \textrm{Joints}(M)_{J'_1} \geq j'_0 = \textrm{TrMax}(M)\)より\(\textrm{Joints}(M)_{J'_1} = \textrm{TrMax}(M)\)である。
				\setcounter{penumeratei}{3}
				\item \(0 < \textrm{Joints}(M)_{J'_1} = \textrm{TrMax}(M)\)であるので、\(t\)の各単項成分は\(D_{M_{1,\textrm{TrMax}(M)}} 0\)以上であり、\(M_{1,\textrm{TrMax}(M)} = M_{1,j'_0}\)である。
				\item[] 以上より、\(t\)の各単項成分は\(D_{M_{1,j'_0}} 0\)以上である。従って\nameref{部分表現の単項成分とPredの関係} (3)と(4)から、一意な\(t' \in T_{\textrm{B}}\)が存在して以下を満たす:
				\begin{penumerate}
					\item \(\textrm{Trans}(M) = D_{M_{1,0}} t'\)である。
					\setcounter{penumerateii}{3}
					\item \(0 < j'_0 = \textrm{TrMax}(M)\)ならば、\(t'\)の各単項成分は\(D_{M_{1,\textrm{TrMax}(M)}} 0\)以上である。
				\end{penumerate}
			\end{penumerate}
		\end{indented}
	\end{indented}
\end{hideableproof}

\begin{lemma}[条件(V)の下での右端の親の基本性質]\label{条件(V)の下での右端の親の基本性質}
	任意の\(M \in RT_{\textrm{PS}} \cap PT_{\textrm{PS}}\)と\(m \in \mathbb{N}\)に対し、\(j_1 := \textrm{Lng}(M)-1\)と置き、\(J_1 := \textrm{Lng}(\textrm{Br}(M))-1\)と置き、\(J_1 \geq 0\)として\(j'_0 := \textrm{Joints}(M)_{J_1}\)と置き、\(j'_1 := \textrm{FirstNodes}(M)_{J_1}\)と置き、\(M' := (M_j)_{j=m}^{j_1}\)と置くと、「\(m < j'_0\)」または「\(m = j'_0\)かつ\(M_{0,j'_1} = M_{1,j'_1}\)かつ\(\textrm{Br}(M)\)が降順」ならば、一意な\(j_0 \in \mathbb{N}\)が存在して以下を満たす:
	\begin{penumerate}
		\item \((0,j_0) <_M^{\textrm{Next}} (0,j_1)\)である。
		\item \(j'_0 \leq j_0\)である。
		\item \(m < j_0\)または\(M_{0,j_1} = M_{1,j_1}\)である。
		\item \(m = j_0\)ならば\(j_0 < \textrm{TrMax}(M)\)である。
	\end{penumerate}
\end{lemma}

\begin{hideableproof}
	\begin{indented}
		\item \(j'_1 = j_1\)とする。
		\begin{indented}
			\item \(j_0 := j'_0\)と置く。
			\item \(m < j'_0\)ならば、\(m < j'_0 = j_0\)である。
			\item \(m = j'_0\)ならば、仮定より\(M_{0,j_1} = M_{0,j'_1} = M_{0,j'_0}+1 = M_{1,j'_0}+1 = M_{1,j'_1} = M_{1,j_1}\)である。
			\item 以上より、いずれの場合も\(m < j_0\)または\(M_{0,j_1} = M_{1,j_1}\)である。
		\end{indented}
		\item \(j'_1 < j_1\)とする。
		\begin{indented}
			\item \nameref{Pの各成分の非複項性}より\(\textrm{Br}(M)_{J_1}\)は複項でないので\((0,0) \leq_{\textrm{Br}(M)_{J_1}} (0,j_1-j'_1)\)すなわち\((0,j'_1) \leq_M (0,j_1)\)である。\(j'_1 < j_1\)より一意な\(j_0 \in \mathbb{N}\)が存在して\(j'_1 \leq j_0\)かつ\((0,j_0) <_M^{\textrm{Next}} (0,j_1)\)すなわち\((0,j_0) <_M^{\textrm{Next}} (0,j_1)\)である。
			\item \(m \leq j'_0 < j'_1 \leq j_0\)であるので、\(j'_0 < j_0\)かつ\(m < j_0\)である。
		\end{indented}
		\item \(m \leq j_0\)かつ\((0,j_0) <_M^{\textrm{Next}} (0,j_1)\)より、\((0,j_0-m) <_{M'}^{\textrm{Next}} (0,j_1-m)\)である。
		\item \(\textrm{TrMax}(M)+1 = \textrm{FirstNodes}(M)_0\)であり、\(\textrm{TrMax}\)の定義より\((1,\textrm{TrMax}(M)) <_M^{\textrm{Next}} (1,\textrm{FirstNodes}(M)_0)\)でない。
		\item
		\item \(m = j_0\)ならば\(j_0 < \textrm{TrMax}(M)\)であることを示す。
		\begin{indented}
			\item \(m < j_0\)または\(M_{0,j_1} = M_{1,j_1}\)であるので、\(m = j_0\)より\(M_{0,j_1} = M_{1,j_1}\)である。
			\item \nameref{簡約性と係数の基本性質}より\(M_{1,j_0} \leq M_{0,j_0} = M_{0,j_1}-1 = M_{1,j_1}-1 < M_{1,j_1}\)である。
			\item \((0,j_0) <_M^{\textrm{Next}} (0,j_1)\)かつ\(M_{1,j_0} < M_{1,j_1}\)より\((1,j_0) <_M^{\textrm{Next}} (1,j_1)\)となる。
			\item \(m \leq j'_0 \leq j_0\)より\(m = j'_0\)である。従って\(M_{0,j'_1} = M_{1,j'_1}\)かつ\(\textrm{Br}(M)\)が降順である。
			\item \((1,\textrm{TrMax}(M)) <_M^{\textrm{Next}} (1,\textrm{FirstNodes}(M)_0)\)でないので、\(M_{0,\textrm{TrMax}(M)} \geq M_{0,\textrm{FirstNodes}(M)_0}\)または\(M_{1,\textrm{TrMax}(M)} \geq M_{1,\textrm{FirstNodes}(M)_0}\)である。
			\item \(M_{0,\textrm{TrMax}(M)} \geq M_{0,\textrm{FirstNodes}(M)_0}\)とする。
			\begin{indented}
				\item \nameref{Pの各成分の左端の単調性}より\(M_{0,\textrm{TrMax}(M)} \geq M_{0,\textrm{FirstNodes}(M)_0} \geq M_{0,j'_1}\)であるので、\((0,\textrm{TrMax}(M)) \leq_M (0,j'_1)\)でなく、従って\(j'_0 < \textrm{TrMax}(M)\)である。
			\end{indented}
			\item \(M_{0,\textrm{TrMax}(M)} = M_{0,\textrm{FirstNodes}(M)_0}\)かつ\(M_{1,\textrm{TrMax}(M)} \geq M_{1,\textrm{FirstNodes}(M)_0}\)とする。
			\begin{indented}
				\item \(\textrm{Br}(M)\)の降順性から、\(M_{0,\textrm{FirstNodes}(M)_0} > M_{0,j'_1}\)または\(M_{1,\textrm{FirstNodes}(M)_0} \geq M_{1,j'_1}\)である。
				\item \(M_{0,\textrm{FirstNodes}(M)_0} > M_{0,j'_1}\)ならば\(M_{0,\textrm{TrMax}(M)} = M_{0,\textrm{FirstNodes}(M)_0} > M_{0,j'_1}\)であるので\((0,\textrm{TrMax}(M)) \leq_M (0,j'_1)\)でなく、従って\(j'_0 < \textrm{TrMax}(M)\)である。
				\item \(M_{1,\textrm{FirstNodes}(M)_0} \geq M_{1,j'_1}\)ならば\(M_{1,\textrm{TrMax}(M)} \geq M_{1,\textrm{FirstNodes}(M)_0} \geq M_{1,j'_1}\)であるので\((1,\textrm{TrMax}(M)) \leq_M (1,j'_1)\)でなく、従って\(j'_0 < \textrm{TrMax}(M)\)である。
				\item 以上より、いずれの場合も\(j'_0 < \textrm{TrMax}(M)\)である。
			\end{indented}
			\item 以上より、いずれの場合も\(j_0 = j'_0 < \textrm{TrMax}(M)\)である。
		\end{indented}
	\end{indented}
\end{hideableproof}

\begin{lemma}[条件(V)の下での終切片と\(\textrm{Trans}\)の関係]\label{条件(V)の下での終切片とTransの関係}
	任意の\(M \in RT_{\textrm{PS}} \cap PT_{\textrm{PS}}\)と\(m \in \mathbb{N}\)に対し、\(j_1 := \textrm{Lng}(M)-1\)と置き、\(J_1 := \textrm{Lng}(\textrm{Br}(M))-1\)と置き、\(J_1 \geq 0\)として\(j'_0 := \textrm{Joints}(M)_{J_1}\)と置き、\(j'_1 := \textrm{FirstNodes}(M)_{J_1}\)と置き、\(M' := (M_j)_{j=m}^{j_1}\)と置くと、「\(m < j'_0\)」または「\(m = j'_0\)かつ\(M_{0,j'_1} = M_{1,j'_1}\)かつ\(\textrm{Br}(M)\)が降順」ならば、一意な\(t_1 \in T_{\textrm{B}}\)が存在して以下を満たす:
	\begin{penumerate}
		\item \(\textrm{Trans}(M) = D_{M_{1,0}} t_1\)である。
		\item \(\textrm{Trans}(M') = D_{M_{1,m}} t_1\)である。
	\end{penumerate}
\end{lemma}

\begin{hideableproof}
	\begin{indented}
		\item 以下\nameref{FirstNodesとJointsの単調性}は断りなく用いる。
		\item \nameref{簡約性の切片への遺伝性}と\nameref{単項性の切片への遺伝性}より\(M'\)は簡約かつ単項であり、\nameref{Transが単項性を保つこと}から\(\textrm{Trans}(M)\)と\(\textrm{Trans}(M')\)は単項である。
		\item \nameref{Transの最左単項成分の左端の基本性質}から一意な\(t_1 \in T_{\textrm{PS}}\)が存在して\(\textrm{Trans}(M) = D_{M_{1,0}} t_1\)となる。
		\item \nameref{FirstNodesとTrMaxとJointsの関係}より、任意の\(J \in \mathbb{N}\)に対し\(J \leq J_1\)ならば\(\textrm{Joints}(M)_J \leq \textrm{TrMax}(M) < \textrm{FirstNodes}(M)_J\)である。
		\item 「\(m < j'_0\)」または「\(m = j'_0\)かつ\(M_{0,j'_1} = M_{1,j'_1}\)かつ\(\textrm{Br}(M)\)が降順」であるので、いずれの場合も\(m \leq j'_0\)である。
		\item \nameref{簡約性と係数の関係}から\(M\)と\(M'\)は条件(A)と(B)を満たす。\(M\)が条件(A)と(B)を満たすことから任意の\(j \in \mathbb{N}\)に対し\(j \leq \textrm{TrMax}(M)\)ならば\(M_j = (M_{1,0} + j, M_{1,0} + j)\)である。\(M\)が条件(A)を満たすことから、\(M_{0,j_0}+1 = M_{0,j_1}\)であり、また任意の\(J \in \mathbb{N}\)に対し\(J \leq J_1\)ならば\(M_{0,\textrm{FirstNodes}(M)_J} = M_{0,\textrm{Joints}(M)_J} + 1 = M_{1,0} + \textrm{Joints}(M)_J + 1\)である。特に\(M_{0,j'_1} = M_{0,j'_0}+1 = M_{1,0} + j'_0 + 1\)である。
		\item \nameref{条件(V)の下での右端の親の基本性質}より、一意な\(j_0 \in \mathbb{N}\)が存在して以下を満たす:
		\begin{penumerate}
			\item \((0,j_0) <_M^{\textrm{Next}} (0,j_1)\)である。
			\item \(j'_0 \leq j_0\)である。
			\item \(m < j_0\)または\(M_{0,j_1} = M_{1,j_1}\)である。
			\item \(m = j_0\)ならば\(j_0 < \textrm{TrMax}(M)\)である。
		\end{penumerate}
		\item
		\item \(\textrm{Trans}(M') = D_{M_{1,m}} t_1\)となることを\(j_1 - \textrm{TrMax}(M)\)に関する数学的帰納法で示す。
		\item \(j_1 - \textrm{TrMax}(M) = 1\)とする。
		\begin{indented}
			\item \(M = ((j,j))_{j=M_{1,0}}^{M_{1,0} + \textrm{TrMax}(M)} \oplus_{\mathbb{N}^2} M_{m_1}\)である。
			\item \(J_1 = 0\)かつ\(j'_1 = j_1\)かつ\(j'_0 = j_0\)である。従って\(M_{0,j_1} = M_{0,j'_1} = M_{1,0} + j'_0 + 1 \leq M_{1,0} + \textrm{TrMax}(M) + 1 = M_{0,\textrm{TrMax}(M)}+1\)である。
			\item \(m < j_0\)ならば、\(M_{0,j_1} = M_{1,0} + j'_0 + 1 = M_{1,0} + j_0 + 1 > M_{1,0} + m + 1= M_{0,m} + 1\)であるので\nameref{Predが公差(1,1)のペア数列のTransの基本性質} (1)と(2)と(3)より\(\textrm{Trans}(M')\)は\(\textrm{Trans}(M) = D_{M_{1,0}} t_1\)の左端を\(D_{M_{1,m}}\)に置き換えたものである。
			\item \(m = j_0\)かつ\(M_{0,j_1} = M_{1,j_1}\)とする。
			\begin{indented}
				\item \nameref{簡約性と係数の基本性質}より\(M_{1,j_0} \leq M_{0,j_0} = M_{0,j_1}-1 = M_{1,j_1}-1 < M_{1,j_1}\)より\((1,j_0) <_M^{\textrm{Next}} (1,j_1)\)となり、\((1,\textrm{TrMax}(M)) <_M^{\textrm{Next}} (1,j_1)\)でないので\(j_0 < \textrm{TrMax}(M)\)である。
				\item \(m = j_0 < \textrm{TrMax}(M)\)と\nameref{Predが公差(1,1)のペア数列のTransの基本性質} (2)より\(\textrm{Trans}(M')\)は\(\textrm{Trans}(M) = D_{M_{1,0}} t_1\)の左端を\(D_{M_{1,m}}\)に置き換えたものである。
			\end{indented}
			\item 以上よりいずれの場合も\(\textrm{Trans}(M') = D_{M_{1,m}} t_1\)である。
		\end{indented}
		\item
		\item \(j_1 - \textrm{TrMax}(M) > 1\)とする。
		\begin{indented}
			\item \(\textrm{Trans}\)の再帰的定義中に導入した記号を\(M\)と\(M'\)に対し定め、\(M\)と\(M'\)に対する適用であることを明示するために右肩に\(M\)と\(M'\)を乗せて表記する。
			\item \(j_1^M = j_1\)かつ\(j_1^{M'} = j_1-m\)であり、\((0,j_0) <_M^{\textrm{Next}} (0,j_1) = (0,j_1^M)\)かつ\((0,j_0-m) <_{M'}^{\textrm{Next}} (0,j_1-m) = (0,j_1^{M'})\)より\(j_0^M = j_0\)かつ\(j_0^{M'} = j_0-m\)である。特に\(j_1^M-j_0^M = j_1-j_0 = j_1^{M'}-j_0^{M'}\)である。また\(M_{j_0^M} = M_{j_0} = M'_{j_0^{M'}}\)かつ\(M_{j_1^M} = M_{j_1} = M'_{j_1^{M'}}\)である。\(M\)が条件(A)を満たすことから\(M'_{j_0^{M'}}+1 = M_{0,j_0^M}+1 = M_{0,j_1^M} = M'_{0,j_1^{M'}}\)である。
			\item \(j_1^M \geq j_1^{M'} = j_1-m > \textrm{TrMax}(M)+1-m \geq 1\)より\(\textrm{Pred}(M)\)と\(\textrm{Pred}(M')\)はいずれも零項でなく、\nameref{Transが零項性を保つこと}から\(t_1^M = \textrm{Trans}(\textrm{Pred}(M)) \neq 0\)かつ\(t_1^{M'} = \textrm{Trans}(\textrm{Pred}(M')) \neq 0\)である。従って\(M\)と\(M'\)に対し条件(I)~(VI)が意味を持つ。
			\item \(m = j_0\)かつ\(M_{0,j_1} = M_{1,j_1}\)ならば、既に示したように\(j_0 < \textrm{TrMax}(M)\)であるので、\(M_{1,j_0^M}+1 = M_{1,j_0}+1 = M_{0,j_0}+1 = M_{0,j_1} = M_{1,j_1} = M_{1,j_1^M}\)かつ\(M'_{1,j_0^{M'}}+1 = M_{1,j_0^M}+1 = M_{1,j_1^M} = M'_{1,j_1^{M'}}\)であり、更に\(j_1^M-j_0^M = j_1^{M'}-j_0^{M'}\)かつ\(j_1-j_0 > (\textrm{TrMax}(M)+1) - j_0 > 1\)であることから\(M\)と\(M'\)は条件(V)を満たす。
			\item \(m < j_0\)ならば、\(j_0^M = j_0 > m \geq 0\)かつ\(j_0^{M'} = j_0-m > 0\)であるので\nameref{許容性の切片への遺伝性}から\(j_0^M\)の\(M\)許容性と\(j_0^{M'}\)の\(M'\)許容性は同値であり、更に\(M_{j_0^M} = M'_{j_0^{M'}}\)かつ\(M_{j_1^M} = M'_{j_1^{M'}}\)であることから\(M\)と\(M'\)は条件(I)~(VI)のうち同じものを満たす。
			\item 以上より、いずれの場合も\(M\)と\(M'\)は条件(I)~(VI)のうち同じものを満たす。
			\item
			\item \(J_0 := \textrm{Lng}(\textrm{Br}(\textrm{Pred}(M)))\)と置く。
			\item \nameref{簡約性の切片への遺伝性}と\nameref{単項性の始切片への遺伝性}より\(\textrm{Pred}(M)\)は簡約かつ単項である。
			\item \(\textrm{TrMax}(M) < j_1-1 = j'_1-1\)から\(\textrm{Lng}(\textrm{Pred}(M))-1 = j_1-1 > \textrm{TrMax}(M) = \textrm{TrMax}(\textrm{Pred}(M))\)であるので\(J_0 \geq 0\)である。
			\item \(j'_1 = j_1\)ならば、\(\textrm{Br}(\textrm{Pred}(M)) = (\textrm{Br}(M)_J)_{J=0}^{J_1-1}\)であり、特に\(J_0 = J_1-1\)である。
			\item \(j'_1 < j_1\)ならば、\(\textrm{Br}(\textrm{Pred}(M)) = (\textrm{Br}(M)_J)_{J=0}^{J_1-1} \oplus_{T_{\textrm{PS}}} ((M_j)_{j=j'_1}^{j_1-1})\)であり、特に\(J_0 = J_1\)である。
			\item 従っていずれの場合も\(J_0 \leq J_1\)であり、\(\textrm{FirstNodes}(\textrm{Pred}(M)) = (\textrm{FirstNodes}(M)_J)_{J=0}^{J_0}\)かつ\(\textrm{Joints}(\textrm{Pred}(M)) = (\textrm{Joints}(M)_J)_{J=0}^{J_0}\)である。
			\item 「\(m < \textrm{Joints}(\textrm{Pred}(M'))_{J_0}\)」または「\(m = \textrm{Joints}(\textrm{Pred}(M'))_{J_0}\)かつ\(M_{0,\textrm{FirstNodes}(\textrm{Pred}(M'))_{J_0}} = M_{1,\textrm{FirstNodes}(\textrm{Pred}(M'))_{J_0}}\)かつ\(\textrm{Br}(\textrm{Pred}(M))\)が降順」であることを示す。
			\item \(m < j'_0\)ならば、\(m < j'_0 = \textrm{Joints}(M)_{J_1} \leq \textrm{Joints}(M)_{J_0} = \textrm{Joints}(\textrm{Pred}(M))_{J_0}\)である。
			\item \(m = j'_0\)とする。
			\item 仮定より\(M_{0,j'_1} = M_{1,j'_1}\)かつ\(\textrm{Br}(M)\)が降順である。
			\item \(j'_0 < \textrm{Joints}(M)_{J_0}\)ならば、\(m = j'_0 < \textrm{Joints}(M)_{J_0} = \textrm{Joints}(\textrm{Pred}(M))_{J_0}\)である。
			\item \(j'_0 = \textrm{Joints}(M)_{J_0}\)とする。
			\begin{indented}
				\item \(M\)が条件(A)を満たすことから\(M_{0,\textrm{FristNodes}(M)_{J_0}} = M_{0,\textrm{Joints}(M)_{J_0}}+1 = M_{0,j'_0}+1 = M_{0,j'_1}\)である。従って\(\textrm{Br}(M)\)の降順性から\(M_{0,\textrm{FirstNodes}(M)_{J_0}} \leq M_{0,j'_1} = M_{1,j'_1} \leq M_{1,\textrm{FirstNodes}(M)_{J_0}}\)である。一方で\nameref{簡約性と係数の基本性質}より\(M_{0,\textrm{FirstNodes}(M)_{J_0}} \geq M_{1,\textrm{FirstNodes}(M)_{J_0}}\)である。
				\item 以上より\(\textrm{Pred}(M)_{0,\textrm{FirstNodes}(\textrm{Pred}(M))_{J_0}} = M_{0,\textrm{FirstNodes}(M)_{J_0}} = M_{1,\textrm{FirstNodes}(M)_{J_0}} = \textrm{Pred}(M)_{1,\textrm{FirstNodes}(\textrm{Pred}(M))_{J_0}}\)である。
				\item \(j'_1 = j_1\)ならば、\(\textrm{Br}(\textrm{Pred}(M)) = (\textrm{Br}(M)_J)_{J=0}^{J_1-1}\)より\(((\textrm{Br}(\textrm{Pred}(M))_J)_0)_{J=0}^{J_0} = ((\textrm{Br}(M)_J)_0)_{J=0}^{J_1-1}\)であるので、\(\textrm{Br}(M)\)の降順性から\(\textrm{Br}(\textrm{Pred}(M))\)は降順である。
				\item \(j'_1 < j_1\)ならば、\(\textrm{Br}(\textrm{Pred}(M)) = (\textrm{Br}(M)_J)_{J=0}^{J_1-1} \oplus_{T_{\textrm{PS}}} ((M_j)_{j=j'_1}^{j_1-1})\)より\(((\textrm{Br}(\textrm{Pred}(M))_J)_0)_{J=0}^{J_0} = ((\textrm{Br}(M)_J)_0)_{J=0}^{J_1-1} \oplus_{T_{\textrm{PS}}} ((M_{j'_1})) = ((\textrm{Br}(M)_J)_0)_{J=0}^{J_1}\)であるので、\(\textrm{Br}(M)\)の降順性から\(\textrm{Br}(\textrm{Pred}(M))\)は降順である。
				\item 従っていずれの場合も\(\textrm{Br}(\textrm{Pred}(M))\)は降順である。
			\end{indented}
			\item 以上より、いずれの場合も「\(m < \textrm{Joints}(\textrm{Pred}(M'))_{J_0}\)」または「\(m = \textrm{Joints}(\textrm{Pred}(M'))_{J_0}\)かつ\(M_{0,\textrm{FirstNodes}(\textrm{Pred}(M'))_{J_0}} = M_{1,\textrm{FirstNodes}(\textrm{Pred}(M'))_{J_0}}\)かつ\(\textrm{Br}(\textrm{Pred}(M))\)が降順」である。
			\item
			\item 帰納法の仮定から\footnote{上までで\(\textrm{Pred}(M)\)が\(M\)と同じ条件を満たすことを確認したので、帰納法の仮定を\(\textrm{Pred}(M)\)に適用することができる。}、\(t_1^M = \textrm{Trans}(\textrm{Pred}(M))\)は単項でありかつその左端を\(D_{M_{1,m}}\)に置き換えたものが\(t_1^{M'} = \textrm{Trans}(\textrm{Pred}(M'))\)である。
			\item \(s_1^M D_{v^M} t_2^M b_1^M  = s_1^M c_1^M b_1^M = t_1^M\)かつ\(s_1^{M'} D_{v^{M'}} t_2^{M'} b_1^{M'} = s_1^{M'} c_1^{M'} b_1^{M'} = t_1^{M'}\)であるので、\(s_1^M D_{v^M}\)と\(s_1^{M'} D_{v^{M'}}\)は左端を除いて一致し、\(t_2^M b_1^M = t_2^{M'} b_1^{M'}\)である。\nameref{順序数項のカッコの個数が左右で等しいこと}より\(b_1^M = b_1^{M'}\)となるので、\(t_2^M = t_2^{M'}\)である。
			\item \(M\)と\(M'\)が条件(I)~(VI)のうち同じものを満たし、\(M_{j_0^M} = M'_{j_0^{M'}}\)かつ\(M_{j_1^M} = M'_{j_1^{M'}}\)かつ\(t_2^M = t_2^{M'}\)であるので、\(c_2^M\)と\(c_2^{M'}\)の定義から、一意な\(t' \in T_{\textrm{B}}\)が存在して\(c_2^M = D_{v^M} t'\)かつ\(c_2^{M'} = D_{v^{M'}} t'\)である。
			\item \(D_{M_{1,0}} t_1 = \textrm{Trans}(M) = s_1^M c_2^M b_1^M = s_1^M D_{v^M} t' b_1^M\)かつ\(\textrm{Trans}(M') = s_1^{M'} D_{v^{M'}} t' b_1^{M'}\)であるので\(D_{M_{1,0}} t_1\)と\(\textrm{Trans}(M')\)は左端を除いて一致する。従って\nameref{Transの最左単項成分の左端の基本性質}より\(\textrm{Trans}(M') = D_{M_{1,m}} t_1\)である。
		\end{indented}
	\end{indented}
\end{hideableproof}

\iffull{それでは本題に戻る。}\fi

\begin{hideableproof}[\nameref{条件(II)か(IV)の下での終切片とTransの関係}の証明]
	\begin{indented}
		\item 以下\nameref{FirstNodesとJointsの単調性}は断りなく用いる。
		\item \(j'_1 := \textrm{FirstNodes}(M)_{J_1}\)と置く。
		\item \nameref{FirstNodesとTrMaxとJointsの関係}より、任意の\(J \in \mathbb{N}\)に対し\(J \leq J_1\)ならば\(\textrm{Joints}(M)_J \leq \textrm{TrMax}(M) < \textrm{FirstNodes}(M)_J\)である。
		\item \nameref{簡約性と係数の関係}から\(M\)は条件(A)と(B)を満たす。\(M\)が条件(A)と(B)を満たすことから任意の\(j \in \mathbb{N}\)に対し\(j \leq \textrm{TrMax}(M)\)ならば\(M_j = (M_{1,0} + j, M_{1,0} + j)\)である。また\(M\)が条件(A)を満たすことから任意の\(J \in \mathbb{N}\)に対し\(J \leq J_1\)ならば\(M_{0,\textrm{FirstNodes}(M)_J} = M_{0,\textrm{Joints}(M)_J} + 1 = M_{1,0} + \textrm{Joints}(M)_J + 1\)である。特に\(N'_{0,0} = M_{0,j'_0} = M_{1,0} + j'_0 = M_{0,j'_1}-1\)である。
		\item \nameref{条件(V)の下での終切片とTransの関係}から、一意な\(t_1 \in T_{\textrm{PS}}\)が存在して以下を満たす:
		\begin{penumerate}
			\item \(\textrm{Trans}(N) = D_{M_{1,0}} t_1\)である。
			\item \(\textrm{Trans}(N') = D_{M_{1,j'_0}} t_1\)である。
		\end{penumerate}
		\item
		\item \(t_1\)の各単項成分が\(D_{M_{1,j'_0}+1} 0\)以上であることを示す。
		\item \(\textrm{TrMax}(M) = m_1\)とする。
		\begin{indented}
			\item 既に示したように\(t_1 = D_{M_{1,\textrm{TrMax}(M)}} 0\)である。
			\item \(\textrm{TrMax}(M) > j'_0\)より\(M_{1,\textrm{TrMax}(M)} > M_{1,j'_0}\)であるので、\(t_1 = D_{M_{1,\textrm{TrMax}(M)}} 0\)の唯一の単項成分\(D_{M_{1,\textrm{TrMax}(M)}} 0\)は\(D_{M_{1,j'_0}+1} 0\)以上である。
		\end{indented}
		\item
		\item \(\textrm{TrMax}(M) < m_1\)とする。
		\begin{penumerate}
			\item[] \(J_0 > 0\)であり、\(\textrm{Joints}(M)_{J_0-1} \geq j'_0 > 0\)と\(N\)が強許容であることから、\nameref{強単項性の下での部分表現の単項成分の基本性質}より以下が成り立つ\footnote{\(\textrm{TrMax}(N) = \textrm{TrMax}(M)\)から\(N_{\textrm{TrMax}(N)} = M_{\textrm{TrMax}(M)}\)であり、\nameref{PとIdxSumの合成の特徴付け}から\(\textrm{FirstNodes}(N)_{J_0-1} = \textrm{FirstNodes}(M)_{J_0-1}\)かつ\(\textrm{Joints}(N)_{J_0-1} = \textrm{Joints}(M)_{J_0-1}\)であるので\(N_{\textrm{FirstNodes}(N)_{J_0-1}} = M_{\textrm{FirstNodes}(M)_{J_0-1}}\)かつ\(N_{\textrm{Joints}(N)_{J_0-1}} = M_{\textrm{Joints}(M)_{J_0-1}}\)となることに注意する。}:
			\begin{penumerate}
				\setcounter{penumerateii}{1}
				\item \(M_{0,\textrm{FirstNodes}(M)_{J_0-1}} = M_{1,\textrm{FirstNodes}(M)_{J_0-1}}\)ならば、\(t_1\)の各単項成分は\(D_{M_{1,\textrm{FirstNodes}(M)_{J_0-1}}} 0\)以上である。
				\item \(0 < \textrm{Joints}(M)_{J_0-1} < \textrm{TrMax}(M)\)かつ\(M_{0,\textrm{FirstNodes}(M)_{J_0-1}} > M_{1,\textrm{FirstNodes}(M)_{J_0-1}}\)ならば、\(t_1\)の各単項成分は\(D_{M_{1,\textrm{Joints}(M)_{J_0-1}}} 0\)以上である。
				\item \(0 < \textrm{Joints}(M)_{J_0-1} = \textrm{TrMax}(M)\)ならば、\(t_1\)の各単項成分は\(D_{M_{1,\textrm{TrMax}(M)}} 0\)以上である。
			\end{penumerate}
			\item[]
			\setcounter{penumeratei}{1}
			\item \(M_{0,\textrm{FirstNodes}(M)_{J_0-1}} = M_{1,\textrm{FirstNodes}(M)_{J_0-1}}\)とする。
			\begin{indented}
				\item \(t_1\)の各単項成分は\(D_{M_{1,\textrm{FirstNodes}(M)_{J_0-1}}} 0\)以上である。
				\item \nameref{簡約性と係数の基本性質}より\(M_{1,\textrm{FirstNodes}(M)_{J_0-1}} = M_{0,\textrm{FirstNodes}(M)_{J_0-1}} \geq M_{0,j'_1} = M_{0,j'_0}+1\)であるので\(t_1\)の各単項成分は\(D_{M_{1,j'_0}+1} 0\)以上である。
			\end{indented}
			\item[]
			\item \(0 < \textrm{Joints}(M)_{J_0-1} < \textrm{TrMax}(M)\)かつ\(M_{0,\textrm{FirstNodes}(M)_{J_0-1}} > M_{1,\textrm{FirstNodes}(M)_{J_0-1}}\)とする。
			\begin{indented}
				\item \(t_1\)の各単項成分は\(D_{M_{1,\textrm{Joints}(M)_{J_0-1}}} 0\)以上である。
				\item \(\textrm{LastStep}\)の定義から、\(M_{0,\textrm{FirstNodes}(M)_{J_0-1}} > M_{1,\textrm{FirstNodes}(M)_{J_0-1}}\)より\(M_{0,\textrm{FirstNodes}(M)_{J_0-1}} > M_{0,j'_1}\)である。従って\(M_{1,\textrm{Joints}(M)_{J_0-1}} = M_{0,\textrm{Joints}(M)_{J_0-1}} = M_{0,\textrm{FirstNodes}(M)_{J_0-1}}-1 \geq M_{0,j'_1} = M_{0,j'_0}+1\)であるので\(t_1\)の各単項成分は\(D_{M_{1,j'_0}+1} 0\)以上である。
			\end{indented}
			\item[]
			\item \(0 < \textrm{Joints}(M)_{J_0-1} = \textrm{TrMax}(M)\)とする。
			\begin{indented}
				\item \(t_1\)の各単項成分は\(D_{M_{1,\textrm{TrMax}(M)}} 0\)以上である。
				\item \(j'_0 = \textrm{TrMax}(M)\)と仮定して矛盾を導く。
				\begin{indented}
					\item \(\textrm{Joints}(M)_{J_0-1} = \textrm{TrMax}(M) = j'_0 \leq \textrm{Joints}(M)_{J_0-1}\)より\(\textrm{Joints}(M)_{J_0-1} = \textrm{TrMax}(M) = j'_0\)である。従って\(M_{0,\textrm{FirstNodes}(M)_{J_0-1}} = M_{0,\textrm{Joints}(M)_{J_0-1}}+1 = M_{0,j'_0}+1 = M_{0,j'_1}\)となるので、\(\textrm{LastStep}\)の定義から\(M_{0,\textrm{FirstNodes}(M)_{J_0-1}} = M_{1,\textrm{FirstNodes}(M)_{J_0-1}}\)である。
					\item \(\textrm{Joints}(M)_0 = \textrm{TrMax}(M) = \textrm{Joints}(M)_{J_0-1} \leq \textrm{Joints}(M)_0\)より\(\textrm{Joints}(M)_0 = \textrm{TrMax}(M) = \textrm{Joints}(M)_{J_0-1}\)である。従って\(M_{0,\textrm{FirstNodes}(M)_0} = M_{0,\textrm{Joints}(M)_0}+1 = M_{0,\textrm{TrMax}(M)}+1\)である。
					\item また\(M_{0,\textrm{FirstNodes}(M)_0} = M_{0,\textrm{Joints}(M)_0}+1 = M_{0,\textrm{Joints}(M)_{J_0-1}}+1 = M_{0,\textrm{FirstNodes}(M)_{J_0-1}}\)となるので、\(M\)の降順性から\(M_{1,\textrm{FirstNodes}(M)_0} \geq M_{1,\textrm{FirstNodes}(M)_{J_0-1}} = M_{0,\textrm{FirstNodes}(M)_{J_0-1}} = M_{0,\textrm{Joints}(M)_0}+1 = M_{0,\textrm{TrMax}(M)}+1 = M_{1,\textrm{TrMax}(M)}+1\)である。
					\item \(\textrm{FirstNodes}(M)_0 = \textrm{TrMax}(M)+1\)であり、\(M_{0,\textrm{FirstNodes}(M)_0} = M_{0,\textrm{TrMax}(M)}+1\)かつ\(M_{1,\textrm{FirstNodes}(M)_0} \geq M_{1,\textrm{TrMax}(M)}+1\)から\((0,\textrm{TrMax}(M)) <_M^{\textrm{Next}} (0,\textrm{TrMax}(M)+1) \)となり、これは\(\textrm{TrMax}\)の定義に矛盾する。
				\end{indented}
				\item 以上より\(j'_0 < \textrm{TrMax}(M)\)である。
				\item 従って\(M_{1,\textrm{TrMax}(M)} = M_{0,\textrm{TrMax}(M)} \geq M_{0,j'_0}+1\)となるので、\(t_1\)の各単項成分は\(D_{M_{0,j'_0}+1} 0\)以上である。
			\end{indented}
			\item[] 以上より、いずれの場合も\(t_1\)の各単項成分は\(D_{M_{0,j'_0}+1} 0\)以上である。
		\end{penumerate}
		\item
		\item \nameref{条件(V)の下での右端の親の基本性質}より、一意な\(m_0 \in \mathbb{N}\)が存在して以下を満たす:
		\begin{penumerate}
			\item \((0,m_0) <_N^{\textrm{Next}} (0,m_1)\)である。
			\item \(j'_0 \leq m_0\)である。
			\item \(j'_0 < m_0\)または\(N_{0,m_1} = N_{1,m_1}\)である。
			\item \(j'_0 = m_0\)ならば\(m_0 < \textrm{TrMax}(N)\)である。
		\end{penumerate}
		\item \(\textrm{Trans}\)の再帰的定義中に導入した記号を\(M\)と\(M'\)に対し定め、\(M\)と\(M'\)に対する適用であることを明示するために右肩にそれぞれ\(M\)と\(M'\)を乗せて表記する。
		\item \(j_1^M = j_1 > m_1 > m_0 \geq j'_0 > 0\)より\(j_1^M > 2\)であるので\(\textrm{Pred}(M)\)は零項でない。\nameref{Transが零項性を保つこと}から\(t_1^M = \textrm{Trans}(\textrm{Pred}(M)) \neq 0\)であるので、\(M\)に対し条件(I)~(VI)が意味を持つ。
		\item \(j_1^{M'} = j_1-j'_0 > m_1-j'_0 > m_0-j'_0 \geq 0\)より\(j_1^{M'} > 1\)であるので\(\textrm{Pred}(M')\)は零項でない。\nameref{Transが零項性を保つこと}から\(t_1^{M'} = \textrm{Trans}(\textrm{Pred}(M')) \neq 0\)であるので、\(M'\)に対し条件(I)~(VI)が意味を持つ。
		\item \(j'_1 \leq j_1\)と\nameref{Pの各成分の非複項性}から\((0,j'_1) \leq_M (0,j_1)\)であり、\((0,j'_0) <_M^{\textrm{Next}} (0,j'_1)\)であるので\(j'_0 \leq j_0^M\)である。\((0,j_0^M) <_M^{\textrm{Next}} (0,j_1^M) = (0,j_1)\)すなわち\((0,j_0^M-j'_0) <_{M'}^{\textrm{Next}} (0,j_1-j'_0) = (0,j_1^{M'})\)より\(j_0^{M'} = j_0^M-j'_0\)である。
		\item
		\item ここで一旦\(j'_1 = j_1\)とする\footnote{後の場合分け中に同じ議論を繰り返すことになるため、場合分けの前に一度まとめて議論をする。}。
		\begin{indented}
			\item \(j_1^M = j_1 = j'_1\)である。\((0,j'_0) <_M^{\textrm{Next}} (0,j'_1) = (0,j_1^M)\)より\(j_0^M = j'_0\)であり、更に\(0 < j'_0 < \textrm{TrMax}(M)\)より\(\textrm{TrMax}\)の定義から\(j_0^M\)は非\(M\)許容かつ\(j_{-1}^M = 0\)となる。また\(M_{0,j'_1} > M_{1,j'_1}\)から\(M_{1,j_0^M} = M_{1,j'_0} = M_{0,j'_0} = M_{0,j'_1}-1 \geq M_{1,j'_1} = M_{1,j_1^M}\)である。
			\item \(j_0^M\)が非\(M\)許容でかつ\(M_{1,j_0^M} \geq M_{1,j_1^M}\)であるので、\(M\)は条件(IV)を満たす。また\nameref{s_1とb_1の空性と基点の関係}より\(s_1^M = ()\)かつ\(D_{v^M} t_2^M = c_1^M = t_1^M\)かつ\(b_1^M = ()\)である。
			\item \(j_1^{M'} = j_1-j'_0 = j'_1-j'_0\)かつ\(j_0^{M'} = j_0^M-j'_0 = 0\)である。従って\(j_0^{M'}\)は\(M'\)許容かつ\(j_{-1}^{M'} = j_0^{M'} = 0\)となる。また\(M_{0,j'_1} > M_{1,j'_1}\)から\(M'_{1,j_0^{M'}} = M'_{1,0} = M_{1,j'_0} = M_{0,j'_0} = M_{0,j'_1}-1 \geq M_{1,j'_1} = M_{1,j_1} = M'_{1,j'_1-j'_0} = M'_{1,j_1^{M'}}\)である。
			\item \(j_0^{M'}\)が\(M'\)許容でかつ\(M'_{1,j_0^{M'}} \geq M'_{1,j_1^{M'}}\)であるので、\(M\)は条件(III)を満たす。また\nameref{s_1とb_1の空性と基点の関係}より\(s_1^{M'} = ()\)かつ\(D_{v^{M'}} t_2^{M'} = c_1^{M'} = t_1^{M'}\)かつ\(b_1^{M'} = ()\)である。
		\end{indented}
		\item 以下では\(j'_1 = j_1\)を課さない一般の状況に戻る。
		\item
		\item \(j_1 \geq \textrm{FirstNodes}(M)_{J_0} > m_1\)より\(j_1-m_1 > 0\)である。
		\item 一意な\(t_2 \in T_{\textrm{B}}\)が存在して以下を満たすことを\(j_1-m_1\)に関する数学的帰納法で示す:
		\begin{penumerate}
			\setcounter{penumeratei}{2}
			\item \(\textrm{Trans}(M') = D_{M_{1,j'_0}}(t_1 + t_2)\)かつ\(t_2 \neq 0\)である。
			\item \(\textrm{Trans}(M) = D_{M_{1,0}}(t_1 + D_{M_{1,j'_0}}(t_1 + t_2))\)である。
		\end{penumerate}
		\item
		\item \(j_1-m_1 = 1\)とする。
		\begin{indented}
			\item \(t_2 := D_{M_{1,j'_1}} 0\)と置く。
			\item \(t_2 \neq 0\)である。
			\item \(J_1 = J_0\)かつ\(j'_1 = j_1\)かつ\(\textrm{Pred}(M') = N'\)かつ\(\textrm{Pred}(M) = N\)である。
			\item \nameref{強単項性の切片への遺伝性}より\(M'\)は強許容であり、\(\textrm{Lng}(M')-1 = m_1 \geq \textrm{TrMax}(M) > j'_0 > 0\)より\(\textrm{Lng}(M')-1 > 1\)であるので、\(M'\)に対し\nameref{部分表現の単項成分とPredの関係} (1)~(4)のいずれかが成り立つ。
			\item \(\textrm{Joints}(M')_{J_1} = \textrm{Joints}(M)_{J_1}-j'_0 = 0\)であるので\(\textrm{Joints}(M')_{J_1}\)は\(M'\)許容であり、従って\(M'\)に対し\nameref{部分表現の単項成分とPredの関係} (2)は成り立たない。
			\item \(M_{0,j'_0}+1 > M_{0,j'_0} \geq M_{1,j'_0}\)かつ\(M_{0,j'_0}+1 = M_{0,j'_1} > M_{1,j'_1}\)であるので、\(t_1\)の各単項成分が\(D_{M_{0,j'_0}+1} 0\)以上かつ\(\textrm{Trans}(\textrm{Pred}(M')) = D_{M_{1,j'_0}} t_1\)であることから、\(M'\)に対し\nameref{部分表現の単項成分とPredの関係} (3)と(4)はいずれも成り立たない。
			\item 以上より、\(M'\)に対し\nameref{部分表現の単項成分とPredの関係} (1)が成り立つ。すなわち\(\textrm{Trans}(M') = D_{M_{1,j'_0}}(t_1 + t_2)\)である。
			\item 既に示したように\(j_1^M = j'_1\)かつ\(j_0^M = j'_0\)であり、\(j_0^M\)は非\(M\)許容であり、\(j_{-1}^M = 0\)であり、\(M\)は条件(IV)を満たし、\(s_1^M = ()\)かつ\(D_{v^M} t_2^M = t_1^M\)かつ\(b_1^M = ()\)である。
			\item \(D_{v^M} t_2^M = t_1^M = \textrm{Trans}(\textrm{Pred}(M)) = \textrm{Trans}(N) = D_{M_{1,0}} t_1\)より\(v^M = M_{1,0}\)かつ\(t_2^M = t_1\)となる。\(M_{1,j_0^M} = M_{1,j'_0} = M_{0,j'_0}\)であり\(t_2^M = t_1\)の各単項成分が\(D_{M_{0,j'_0}+1}\)以上であることから、\(t_2^M\)の最右単項成分の左端は\(D_{M_{1,j_0^M}}\)でない。従って\(t_3^M = t_2^M = t_1\)かつ\(t_4^M = t_2^M = t_1\)である。
			\item 以上より\(\textrm{Trans}(M) = s_1^M c_2^M b_1^M = c_2^M = D_{v^M}(t_3^M + D_{M_{1,j_0^M}}(t_4^M + D_{M_{1,j_1^M}} 0)) = D_{M_{1,0}}(t_1 + D_{M_{1,j'_0}}(t_1 + t_2))\)である。
		\end{indented}
		\item
		\item \(j_1-m_1 > 1\)とする。
		\begin{indented}
			\item \nameref{強単項性の切片への遺伝性}より\(\textrm{Pred}(M)\)は強単項である。
			\item \(J'_1 := \textrm{Lng}(\textrm{Br}(\textrm{Pred}(M)))-1\)と置く。
			\item
			\item \(j'_1 = j_1\)とする。
			\begin{indented}
				\item \nameref{PとIdxSumの合成の特徴付け}から\(J'_1 \geq J_1-1\)であり、\(\textrm{FirstNodes}(M)_{J_0} = m_1+1 < j_1 = j'_1\)より\(J_0 < J_1\)であるので、\(J'_1 \geq J_1-1 \geq J_0 \geq 0\)かつ\(\textrm{LastStep}(\textrm{Pred}(M)) = J_0\)である。
				\item \(J'_1 \geq J_0\)より、\(\textrm{LastStep}\)の定義から\(M_{0,\textrm{FirstNodes}(M)_{J'_1}} > M_{1,\textrm{FirstNodes}(M)_{J'_1}}\)である。
			\end{indented}
			\item
			\item \(j'_1 < j_1\)とする。
			\begin{indented}
				\item \nameref{PとIdxSumの合成の特徴付け}から\(J'_1 = J_1 \geq 0\)かつ\(M_{0,\textrm{FirstNodes}(M)_{J'_1}} = M_{0,\textrm{FirstNodes}(M)_{J_1}} > M_{1,\textrm{FirstNodes}(M)_{J_1}} = M_{1,\textrm{FirstNodes}(M)_{J'_1}}\)である。
				\item \((\textrm{Pred}(M)_j)_{j=0}^{\textrm{FirstNodes}(\textrm{Pred}(M))_{J'_1}} = (M_j)_{j=0}^{\textrm{FirstNodes}(M){J_1}}\)より、\(\textrm{LastStep}(\textrm{Pred}(M)) = J_0\)である。
			\end{indented}
			\item
			\item 以上より、いずれの場合も\(J'_1 \geq J_0 \geq 0\)かつ\(\textrm{LastStep}(\textrm{Pred}(M)) = J_0\)かつ\(M_{0,\textrm{FirstNodes}(M)_{J'_1}} > M_{1,\textrm{FirstNodes}(M)_{J'_1}}\)である。
			\item 従って\(\textrm{Pred}(M)_{0,\textrm{FirstNodes}(\textrm{Pred}(M))_{J'_1}} = M_{0,\textrm{FirstNodes}(M)_{J'_1}} > M_{1,\textrm{FirstNodes}(M)_{J'_1}} = \textrm{Pred}(M)_{1,\textrm{FirstNodes}(\textrm{Pred}(M))_{J'_1}}\)であり、また\(J_0 \leq J'_1\)より\(M_{1,\textrm{FirstNodes}(\textrm{Pred}(M))_{J'_1}} = M_{1,j'_1}\)であるので\(\textrm{Joints}(\textrm{Pred}(M))_{J'_1} = M_{1,\textrm{Joints}(\textrm{Pred}(M))_{J'_1}} - M_{1,0} = M_{1,\textrm{FirstNodes}(\textrm{Pred}(M))_{J'_1}} - 1 - M_{1,0} = M_{1,j'_1} - 1 - M_{1,0} = M_{1,j'_0} - M_{1,0} = j'_0 > 0\)となる。
			\item 帰納法の仮定から\footnote{上までの議論により、\(\textrm{Pred}(M)\)も\(M\)と同じ条件を満たすことが確認されたので\(\textrm{Pred}(M)\)に対し帰納法の仮定が適用可能である。}、一意な\(t'_2 \in T_{\textrm{B}}^2\)が存在して以下を満たす\footnote{\(\textrm{TrMax}(\textrm{Pred}(M)) = \textrm{TrMax}(M)\)かつ\(\textrm{Pred}(M)_{1,0} = M_{1,0}\)かつ\(\textrm{Pred}(M)_{\textrm{Joints}(\textrm{Pred}(M))_{J'_1}} = M_{j'_0}\)であることに注意する。}:
			\begin{penumerate}
				\setcounter{penumeratei}{2}
				\item \(\textrm{Trans}(\textrm{Pred}(M')) = D_{M_{1,j'_0}}(t_1 + t'_2)\)かつ\(t'_2 \neq 0\)となる。
				\item \(t_1^M = \textrm{Trans}(\textrm{Pred}(M)) = D_{M_{1,0}}(t_1 + D_{M_{1,j'_0}}(t_1 + t'_2))\)である。
			\end{penumerate}
			\item \(t'_2 \neq 0\)と\nameref{部分表現の単項成分とPredの関係}より、一意な\(t_2 \in T_{\textrm{PS}}\)が存在して\(\textrm{Trans}(M') = D_{M_{1,j'_0}}(t_1 + t_2)\)である。
			\item \(\textrm{TrMax}(M)+1 \leq m_1 < j_1\)と\(P\)の定義から\((0,j'_1-1) <_M^{\textrm{Next}} (0,j'_1)\)でないので、\(j'_1\)は\(M\)許容である。
			\item
			\item \(j'_1 = j_1\)とする。
			\begin{indented}
				\item 既に示したように\(j_1^M = j'_1\)かつ\(j_0^M = j'_0\)であり、\(j_0^M\)は非\(M\)許容であり、\(j_{-1}^M = 0\)であり、\(M\)は条件(IV)を満たし、\(s_1^M = ()\)かつ\(D_{v^M} t_2^M = t_1^M\)かつ\(b_1^M = ()\)である。
				\item \(D_{v^M} t_2^M = t_1^M = D_{M_{1,0}}(t_1 + D_{M_{1,j'_0}}(t_1 + t'_2))\)より\(v^M = M_{1,0}\)かつ\(t_2^M = t_1 + D_{M_{1,j'_0}}(t_1 + t'_2)\)であり、\(t_2^M\)の最右単項成分\(D_{M_{1,j'_0}}(t_1 + t'_2)\)の左端は\(D_{M_{1,j'_0}} = D_{M_{1,j_0^M}}\)となる。従って\(t_3^M = t_1\)かつ\(t_4^M = t_1 + t'_2\)かつ\(c_2^M = D_{v^M}(t_3^M + D_{M_{1,j_0^M}}(t_4^M + D_{M_{1,j_1^M}} 0)) = D_{M_{1,0}}(t_1 + D_{M_{1,j'_0}}(t_1 + t'_2 + D_{M_{1,j_1}} 0))\)である。
				\item 既に示したように\(j_1^{M'} = j_1-j'_0 = j'_1-j'_0\)かつ\(j_0^{M'} = 0\)であり、\(j_0^{M'}\)は\(M'\)許容であり、\(j_{-1}^{M'} = 0\)であり、\(M\)は条件(III)を満たし、\(s_1^{M'} = ()\)かつ\(D_{v^{M'}} t_2^{M'} = t_1^{M'}\)かつ\(b_1^{M'} = ()\)である。
				\item \(D_{v^{M'}} t_2^{M'} = t_1^{M'} = D_{M_{1,j'_0}}(t_1 + t'_2)\)より\(v^{M'} = M_{1,j'_0}\)かつ\(t_2^{M'} = t_1 + t'_2\)であり、\(t_2^{M'}\)の最右単項成分\(D_{M_{1,j'_0}}(t_1 + t'_2)\)の左端は\(D_{M_{1,j'_0}} = D_{M_{1,j_0^M}}\)となる。従って\(c_2^{M'} = D_{v^{M'}}(t_2^{M'} + D_{M'_{1,j_1^{M'}}} 0) = D_{M_{1,j'_0}}(t_1 + t'_2 + D_{M_{1,j_1}} 0)\)である。
				\item \(D_{M_{1,j'_0}}(t_1 + t_2) = \textrm{Trans}(M') = s_1^{M'} c_2^{M'} b_1^{M'} = D_{M_{1,j'_0}}(t_1 + t'_2 + D_{M_{1,j_1}} 0)\)より\(t_2 = t'_2 + D_{M_{1,j_1}} 0\)である。従って\(\textrm{Trans}(M) = s_1^M c_2^M b_1^M = D_{M_{1,0}}(t_1 + D_{M_{1,j'_0}}(t_1 + t'_2 + D_{M_{1,j_1}} 0)) = D_{M_{1,0}}(t_1 + D_{M_{1,j'_0}}(t_1 + t_2))\)である。
			\end{indented}
			\item
			\item \(j'_1 < j_1\)とする。
			\begin{indented}
				\item \nameref{Pの各成分の非複項性}から\((0,j'_1) \leq_M (0,j_1)\)であるので\(j'_1 \leq j_0^M\)であり、\(j'_1\)が\(M\)許容であることから\(j'_1 \leq j_{-1}^M\)である。\(j_0^{M'} = j_0^M-j'_0\)かつ\(j_{-1}^M-j'_0 \geq j'_1-j'_0 > 0\)より、\nameref{許容化の切片への遺伝性}から\(j_{-1}^{M'} = j_{-1}^M-j'_0 > 0\)である。
				\item \nameref{s_1とb_1の空性と基点の関係}より\(s_1^{M'} \neq ()\)である。\((D_{M_{1,j'_0}}(t_1 + t'_2),c_1^{M'}) = (t_1^{M'},c_1^{M'}) \in T_{\textrm{B}}^{\textrm{Marked}}\)かつ\(s_1^{M'} \neq ()\)より\((t'_2,c_1^{M'}) \in T_{\textrm{B}}^{\textrm{Marked}}\)である。従って、ある\((s'_1,b'_1) \in (\Sigma^{< \omega})^2\)が存在して\((s'_1,c_1^{M'},b'_1)\)は\(t'_2\)のscb分解である。
				\item \(s_1^{M'} c_1^{M'} b_1^{M'} = t_1^{M'} = D_{M_{1,j'_0}}(t_1 + t'_2) = D_{M_{1,j'_0}}(t_1 + s'_1 c_1^{M'} b'_1)\)かつ\(s_1^M c_1^M b_1^M = t_1^M = D_{M_{1,0}}(t_1 + D_{M_{1,j'_0}}(t_1 + t'_2)) = D_{M_{1,0}}(t_1 + D_{M_{1,j'_0}}(t_1 + s'_1 c_1^{M'} b'_1))\)より、\nameref{scb分解の置換可能性}と\nameref{scb分解の合成則}と\nameref{加法とscb分解の関係}から\(D_{M_{1,j'_0}}(t_1 + t_2) = \textrm{Trans}(M') = s_1^{M'} c_2^{M'} b_1^{M'} = D_{M_{1,j'_0}}(t_1 + s'_1 c_2^{M'} b'_1)\)かつ\(\textrm{Trans}(M) = s_1^M c_2^M b_1^M = D_{M_{1,0}}(t_1 + D_{M_{1,j'_0}}(t_1 + s'_1 c_2^{M'} b'_1))\)である。
				\item 従って\(t_2 = s'_1 c_2^{M'} b'_1\)かつ\(\textrm{Trans}(M) = D_{M_{1,0}}(t_1 + D_{M_{1,j'_0}}(t_1 + s'_1 c_2^{M'} b'_1)) = D_{M_{1,0}}(t_1 + D_{M_{1,j'_0}}(t_1 + t_2))\)である。
			\end{indented}
			\item 以上より、いずれの場合も\(\textrm{Trans}(M) = D_{M_{1,0}}(t_1 + D_{M_{1,j'_0}}(t_1 + t_2))\)である。
		\end{indented}
	\end{indented}
\end{hideableproof}


\subsection{条件(II)の下での展開規則}

\begin{proposition}[条件(II)の下での\(\textrm{Trans}\)と基本列の交換関係]\label{条件(II)の下でのTransと基本列の交換関係}
	任意の\(M \in ST_{\textrm{PS}} \cap PT_{\textrm{PS}}\)と\(n \in \mathbb{N}_{+}\)に対し、\(\textrm{Trans}\)の再帰的定義中に導入した記号を用い、\(L := \textrm{Red}((M_j)_{j=j_{-1}}^{j_1})\)と置くと、\(j_1 > 1\)かつ\(M\)が条件(II)を満たすならば\footnotemark{}、\(P(t_2)_{J_1}\)の左端が\(D_{M_{1,j_0}}\)であるか否かに従って\(m_n := n-1\)または\(m_n := n-2\)と置くと、以下が成り立つ:
	\begin{penumerate}
		\item \(m_n = -1\)ならば\(\textrm{Trans}(M[n]) = s_1 D_{M_{1,j_{-1}}} t_2 b_1\)である。
		\item \(m_n \geq 0\)ならば\(\textrm{Trans}(M[n]) = \textrm{Trans}(M)[m_n]\)である。
		\item \(\textrm{Mark}(M[n],j_{-1}) = D_{M_{1,j_{-1}}} (t_3 + (D_{M_{1,j_0}} t_4) \times (m_n+1))\)である。
		\item \(\textrm{Trans}(M[n]) < \textrm{Trans}(M)\)である。
	\end{penumerate}
\end{proposition}
\footnotetext{\nameref{標準形の簡約性}から\(M\)は簡約であり、\(j_1 > 1\)より\(t_1 \neq 0\)であるので\(M\)に対し条件(II)が意味を持つ。}

\nameref{条件(II)の下でのTransと基本列の交換関係}を証明するための準備としていくつかの補題を示す。

\begin{lemma}[第\(0\)種型基本列の基本不等式]\label{第0種型基本列の基本不等式}
	任意の\(M in T_{\textrm{PS}}\)と\(n,r' \in \mathbb{N}_{+}\)と\(q,q' \in \mathbb{N}\)に対し、\(j_1 := \textrm{Lng}(M)-1\)と置くと、\((0,j_0) <_M^{\textrm{Next}} (0,j_1)\)を満たす一意な\(j_0 \in \mathbb{N}\)が存在し\(M_{1,j_1} = 0\)かつ\(q,q' \leq n-1\)かつ\(r' \in j_1-j_0\)ならば、\(M[n]_{0,j_0+q(j_1-j_0)} < M[n]_{0,q'(j_1-j_0)+r'}\)である。
\end{lemma}

\begin{hideableproof}
	\begin{indented}
		\item \(M_{1,j_1} = 0\)より\(M[n] = (M_j)_{j=0}^{j_0-1} \oplus \bigoplus_{\mathbb{N}^2} ((M_j)_{j=j_0}^{j_1-1})_{k=0}^{n-1}\)である。
		\item \((0,j_0) <_M^{\textrm{Next}} (0,j_1)\)より、任意の\(j \in \mathbb{N}\)に対し\(j_0 < j < j_1\)ならば\(M_{0,j} \geq M_{0,j_1} > M_{0,j_0}\)である。
		\item \(j_0 < j_0+r' < j_1\)より\(M_{0,j_0} < M_{0,j_0+r'}\)であり、\(M[n]_{j_0+q(j_1-j_0)} = M_{j_0}\)かつ\(M[n]_{j_0+q'(j_1-j_0)+r'} = M_{j_0+r'}\)より\(M[n]_{0,j_0+q(j_1-j_0)} < M[n]_{0,j_0+q'(j_1-j_0)+r'}\)となる。
	\end{indented}
\end{hideableproof}

\begin{lemma}[第\(0\)種型基本列の基本分岐規則]\label{第0種型基本列の基本分岐規則}
	任意の\(M \in RT_{\textrm{PS}}\)と\(n \in \mathbb{N}_{+}\)と\(q \in \mathbb{N}\)に対し、\(j_1 := \textrm{Lng}(M)-1\)と置き、\((0,j_0) <_M^{\textrm{Next}} (0,j_1)\)を満たす一意な\(j_0 \in \mathbb{N}\)が存在するとし、\(M_{1,j_1} = 0\)かつ\(q \leq n-1\)かつ\(j_0\)が非\(M\)許容ならば、\((0,j_0-1) <_{M[n]}^{\textrm{Next}} (0,j_0+q(j_1-j_0))\)かつ\((1,j_0-1) <_{M[n]}^{\textrm{Next}} (1,j_0+q(j_1-j_0))\)である。
\end{lemma}

\begin{hideableproof}
	\begin{indented}
		\item \(j_{-1} := \textrm{Adm}_M(j_0)\)と置く。
		\item \(j_0\)が非\(M\)許容であることから\(j_{-1} < j_0\)であり、任意の\(j'_0 \in \mathbb{N}\)に対し\(j_{-1} < j'_0 \leq j_0\)ならば\((1,j'_0-1) <_M^{\textrm{Next}} (1,j'_0)\)である。特に\((1,j_0-1) <_M^{\textrm{Next}} (1,j_0)\)となり、\nameref{簡約性と係数の関係}より\(M\)は条件(A)と(B)を満たすことから、\(M_{0,j_0-1}+1 = M_{0,j_0}\)かつ\(M_{1,j_0-1}+1 = M_{1,j_0}\)である。
		\item 更に \(M_{1,j_1} = 0\)より\(M[n] = (M_j)_{j=0}^{j_0-1} \oplus \bigoplus_{\mathbb{N}^2} ((M_j)_{j=j_0}^{j_1-1})_{k=0}^{n-1}\)であるので\(M[n]_{j_0-1} = M_{j_0-1}\)かつ\(M[n]_{j_0+q(j_1-j_0)} = M_{j_0}\)であることから、\(M[n]_{0,j_0-1} = M[n]_{0,j_0+q(j_1-j_0)}-1 < M[n]_{0,j_0+q(j_1-j_0)}\)かつ\(M[n]_{1,j_0-1} = M[n]_{1,j_0+q(j_1-j_0)}-1 < M[n]_{1,j_0+q(j_1-j_0)}\)である。
		\item 一方で任意の\(j'_0 \in \mathbb{N}\)に対し、\(j_0-1 < j'_0 < j_0+q(j_1-j_0)\)ならば\nameref{第0種型基本列の基本不等式}より\(M[n]_{j'_0} \geq M[n]_{j_0+q(j_1-j_0)}\)である。以上より、\((0,j_0-1) <_{M[n]}^{\textrm{Next}} (0,j_0+q(j_1-j_0))\)かつ\((1,j_0-1) <_{M[n]}^{\textrm{Next}} (1,j_0+q(j_1-j_0))\)である。
	\end{indented}
\end{hideableproof}

\begin{lemma}[第\(0\)種型基本列の基本基点関係]\label{第0種型基本列の基本基点関係}
	任意の\(M \in RT_{\textrm{PS}}\)と\(n \in \mathbb{N}_{+}\)に対し、\(j_1 := \textrm{Lng}(M)-1\)と置き、\((0,j_0) <_M^{\textrm{Next}} (0,j_1)\)を満たす一意な\(j_0 \in \mathbb{N}\)が存在するとし\(j_{-1} := \textrm{Adm}_M(j_0)\)と置くと、\(M_{1,j_1} = 0\)ならば以下が成り立つ:
	\begin{penumerate}
		\item \(n > 1\)ならば\((M[n],j_0+(n-1)(j_1-j_0)) \in RT_{\textrm{PS}}^{\textrm{Marked}}\)である。
		\item \(j_0\)が非\(M\)許容ならば\((M[n],j_{-1}) \in RT_{\textrm{PS}}^{\textrm{Marked}}\)である。
	\end{penumerate}
\end{lemma}

\begin{hideableproof}
	\begin{penumerate}
		\item[] \nameref{簡約性が基本列で保たれること}から\(M[n]\)は簡約である。
		\item[] \(M_{1,j_1} = 0\)より\((M[n]_j)_{j=j_0+(n-1)(j_1-j_0)}^{j_0+n(j_1-j_0)-1} = (M_j)_{j=j_0}^{j_1-1}\)である。
		\item \nameref{第0種型基本列の基本不等式}より\((0,j_0) \leq_M (0,j_1-1)\)であるので、\((0,j_0+(n-1)(j_1-j_0)) \leq_{M[n]} (0,j_0+(n-1)(j_1-j_0)+(j_1-j_0-1)) = (0,j_0+n(j_1-j_0)-1)\)である。
		\item[] \nameref{第0種型基本列の基本分岐規則}より\((1,j_0-1) <_{M[n]}^{\textrm{Next}} (1,j_0+(n-1)(j_1-j_0))\)であり、\(n > 1\)より\(j_0+(n-1)(j_1-j_0)-1 > j_0-1\)であるので\((1,j_0+(n-1)(j_1-j_0)-1) \leq_{M[n]} (1,j_0+(n-1)(j_1-j_0))\)でない。従って\(j_0+(n-1)(j_1-j_0)\)は\(M[n]\)許容であり、\((M[n],j_0+(n-1)(j_1-j_0)) \in RT_{\textrm{PS}}^{\textrm{Marked}}\)である。
		\item \(j_0\)の非\(M\)許容性より\((0,j_{-1}) \leq_M (0,j_0-1)\)であるので\((0,j_{-1}) \leq_{M[n]} (0,j_0-1)\)であり、\nameref{第0種型基本列の基本分岐規則}より\((0,j_0-1) <_{M[n]}^{\textrm{Next}} (0,j_0+(n-1)(j_1-j_0))\)であり、(1)より\((0,j_0+(n-1)(j_1-j_0)) \leq_{M[n]} (0,j_0+n(j_1-j_0)-1)\)であるので、\((0,j_{-1}) \leq_{M[n]} (0,j_0+n(j_1-j_0)-1)\)である。
		\item[] \nameref{許容性の切片への遺伝性}より\(j_{-1}\)は\(\textrm{Pred}(M)\)許容である。\(j_0\)の非\(M\)許容性より\(j_{-1} < j_0 \leq j_1-1\)であり、\((M[n]_j)_{j=0}^{j_1-1} = \textrm{Pred}(M)\)であるので、再び\nameref{許容性の切片への遺伝性}より\(j_{-1}\)は\(M[n]\)許容である。以上より、\((M[n],j_{-1}) \in RT_{\textrm{PS}}^{\textrm{Marked}}\)である。
	\end{penumerate}
\end{hideableproof}

\iffull{それでは本題に戻る。}\fi

\begin{hideableproof}[\nameref{条件(II)の下でのTransと基本列の交換関係}の証明]
	\begin{indented}
		\item \nameref{標準形の簡約性}から\(M\)は簡約であり、\nameref{簡約性と係数の関係}より\(M\)は条件(A)と(B)を満たす。
		\item \(M\)が条件(II)を満たすことから、\(M_{1,j_0} \geq M_{1,j_1}\)かつ\(j_0\)が非\(M\)許容すなわち\(j_{-1} < j_0\)である。
		\item \nameref{Markの左端の基本性質}より\(v = M_{1,j_{-1}}\)である。\nameref{scb分解の置換可能性}より\((s_1,c_2,b_1) = (s_1,D_{M_{1,j_{-1}}}(t_3 + D_{M_{1,j_0}}(t_4 + D_0 0)),b_1)\)は\(\textrm{Trans}(M)\)のscb分解であり、\nameref{scb分解と基本列の関係} (1-2)より
		\begin{eqnarray*}
		\textrm{Trans}(M)[n] = s_1 D_{M_{1,j_{-1}}}(t_3 + (D_{M_{1,j_0}} t_4) \times (n+1)) b_1
		\end{eqnarray*}
		\item である。特に\(\textrm{Trans}(M) \neq 0\)である。
		\item
		\item (1)~(4)を\(n \in \mathbb{N}_{+}\)に関する数学的帰納法で示す。
		\item \(n=1\)とする。
		\begin{indented}
			\item \(M[n] = \textrm{Pred}(M)\)より\(\textrm{Trans}(M[n]) = \textrm{Trans}(\textrm{Pred}(M)) = t_1\)かつ\(\textrm{Mark}(M[n],j_{-1}) = \textrm{Mark}(\textrm{Pred}(M),j_{-1}) = c_1\)である。
			\item \(P(t_2)_{J_1}\)の左端が\(D_{M_{1,j_0}}\)であるとする。
			\begin{indented}
				\item \(m_n = n-1 = 0\)であり、
				\begin{eqnarray*}
				t_3 & = & \Sigma_{\textrm{B}} (P(t_2)_J)_{J=0}^{J_1-1} \\
				c_1 & = & D_v t_2 = D_{M_{1,j_{-1}}} (\Sigma_{\textrm{B}} (P(t_2)_J)_{J=0}^{J_1-1} + P(t_2)_{J_1}) = D_{M_{1,j_{-1}}} (t_3 + D_{M_{1,j_0}} t_4)
				\end{eqnarray*}
				\item であるので\(\textrm{Mark}(M[n],j_{-1}) = c_1 = D_{M_{1,j_{-1}}} (t_3 + (D_{M_{1,j_0}} t_4) \times (m_n+1))\)であり、
				\begin{eqnarray*}
				\textrm{Trans}(M[n]) = t_1 = s_1 c_1 b_1 = s_1 D_{M_{1,j_{-1}}} (t_3 + (D_{M_{1,j_0}} t_4)) b_1 = \textrm{Trans}(M)[n-1]
				\end{eqnarray*}
				\item である。従って\(\textrm{Trans}(M) \neq 0\)と\cite{buc1} Lemma 3.2より\(\textrm{Trans}(M[n]) = \textrm{Trans}(M)[n-1] < \textrm{Trans}(M)\)である。
			\end{indented}
			\item
			\item \(P(t_2)_{J_1}\)の左端が\(D_{M_{1,j_0}}\)でないとする。
			\begin{eqnarray*}
			c_1 & = & D_v t_2 = D_{M_{1,j_{-1}}} t_2 \\
			c_2 & = & D_v(t_2 + D_{M_{1,j_0}} 0) = D_{M_{1,j_{-1}}}(t_2 + D_{M_{1,j_0}} 0)
			\end{eqnarray*}
			\begin{indented}
				\item であるので\(\textrm{Mark}(M[n],j_{-1}) = c_1 = D_{M_{1,j_{-1}}} t_2\)であり、
				\begin{eqnarray*}
				\textrm{Trans}(M[n]) & = & t_1 = s_1 c_1 b_1 = s_1 D_{M_{1,j_{-1}}} t_2 b_1 \\
				\textrm{Trans}(M) & = & s_1 c_2 b_1 = s_1 D_{M_{1,j_{-1}}}(t_2 + D_{M_{1,j_0}} 0) b_1
				\end{eqnarray*}
				\item である。従って\nameref{部分表現の不等式の延長性}より、\(t_2 < t_2 + D_{M_{1,j_0}} 0\)から\(\textrm{Trans}(M[n]) < \textrm{Trans}(M)\)である。
			\end{indented}
		\end{indented}
		\item
		\item \(n > 1\)とする。
		\begin{indented}
			\item \(N := (M[n]_j)_{j=0}^{j_0+(n-1)(j_1-j_0)}\)と置く。
			\item \(M\)の単項性より\((0,0) \leq_M (0,j_0-1)\)であるので\((0,0) \leq_N (0,j_0-1)\)であり、\nameref{第0種型基本列の基本分岐規則}より\((0,j_0-1) <_{M[n]}^{\textrm{Next}} (0,j_0+(n-1)(j_1-j_0))\)であり\((0,j_0-1) <_N^{\textrm{Next}} (0,j_0+(n-1)(j_1-j_0))\)となるので、\(N\)は単項である。\(M\)が条件(B)を満たすことから\(N_{0,0} = M_{0,0} = M_{1,0} = N_{1,0}\)となる。\(N\)の単項性と\(N_{0,0} = N_{1,0}\)より、\(N\)は条件(B)を満たす。
			\item
			\item \(N\)が条件(A)を満たすことを示す。
			\item \(i \in \{0,1\}\)と\(j'_0,j'_1 \in \mathbb{N}\)とし、\((i,j'_0) <_N^{\textrm{Next}} (i,j'_1)\)とする。
			\item \(j'_1 \leq j_0\)ならば、\((i,j'_0) <_M^{\textrm{Next}} (i,j'_1)\)かつ\(M\)が条件(A)を満たすことから\(N_{i,j'_0}+1 = M_{i,j'_0}+1 = M_{i,j'_1} = N_{i,j'_1}\)である。
			\item \(j'_0 < j_0 < j'_1\)とする。
			\begin{indented}
				\item \(j'_1-j_0\)を\(j_1-j_0\)で割った商と余りを\(q_1,r_1 \in \mathbb{N}\)と置く。
				\item \(q_1 \leq n-1\)かつ\(r_1 < j_1-j_0\)である。
				\item \nameref{第0種型基本列の基本分岐規則}と\nameref{第0種型基本列の基本不等式}より\((0,j_0-1) <_N^{\textrm{Next}} (0,j_0+q_1(j_1-j_0)) \leq_N (0,j_0+q_1(j_1-j_0)+r_1) = (0,j'_1)\)であるので、\((0,j_0-1) \leq_N (0,j_1)\)である。更に\(j'_0 < j_0\)かつ\((i,j'_0) <_N^{\textrm{Next}} (i,j'_1)\)より\((0,j'_0) \leq_N (0,j_0-1)\)となるので、\((0,j'_0) \leq_M (0,j_0-1)\)となる。
				\item 任意の\(j \in \mathbb{N}\)に対し、\((0,j'_0) \leq_M (0,j) \leq_M (0,j_0+r)\)ならば\((i,j) \leq_M (i,j_0+r_1)\)でないことを示す。
				\begin{indented}
					\item \(j \leq j_0-1\)ならば、\((0,j'_0) \leq_M (0,j) \leq_M (0,j_0+r)\)かつ\((0,j'_0) \leq_M (0,j_0-1)\)より\((0,j) \leq_M (0,j_0-1)\)でありすなわち\((0,j) \leq_N (0,j_0-1) \leq_N (0,j'_1)\)となるので、\((i,j'_0) <_N^{\textrm{Next}} (i,j'_1)\)と\nameref{親の基本性質}から\(M_{i,j} = N_{i,j} \geq N_{i,j'_1} = M_{i,j_0+r_1}\)となる。
					\item \(j > j_0-1\)ならば、\((0,j) \leq_M (0,j_0+r)\)より\((0,j+q_1(j_1-j_0)) \leq_N (0,j_0+q_1(j_1-j_0)+r) = (0,j'_1)\)となり、\((i,j'_0) <_N^{\textrm{Next}} (i,j'_1)\)と\nameref{親の基本性質}から\(M_{i,j} = N_{i,j+q_1(j_1-j_0)} \geq N_{i,j'_1} = M_{i,j_0+r_1}\)となる。
				\end{indented}
				\item 以上より、いずれの場合も\(M_{i,j} \geq M_{i,j_0+r}\)であり、特に\((i,j) \leq_M (i,j_0+r_1)\)でない。
				\item 従って\((i,j'_0) <_M^{\textrm{Next}} (i,j_0+r_1)\)であり、\(M\)が条件(A)を満たすことから\(N_{i,j'_0}+1 = M_{i,j'_0}+1 = M_{i,j_0+r_1} = N_{i,j'_1}\)である。
			\end{indented}
			\item
			\item \(j_0 \leq j'_0\)とする。
			\begin{indented}
				\item \(j'_1-j_0\)を\(j_1-j_0\)で割った商と余りを\(q_1,r_1 \in \mathbb{N}\)と置く。
				\item \(q_1 \leq n-1\)かつ\(r_1 < j_1-j_0\)である。
				\item \nameref{第0種型基本列の基本分岐規則}と\nameref{第0種型基本列の基本不等式}より\((0,j_0-1) <_N^{\textrm{Next}} (0,j_0+q_1(j_1-j_0)) \leq_N (0,j_0+q_1(j_1-j_0)+r_1) = (0,j'_1)\)であるので、\(j_0-1 < j_0 \leq j'_0\)かつ\((i,j'_0) <_N^{\textrm{Next}} (i,j'_1)\)より\(j_0+q_1(j_1-j_0) \leq j'_0 < j'_1\)である。
				\item 更に\((N_j)_{j=j_0+q_1(j_1-j_0)}^{j'_1} = (M_j)_{j=j_0}^{j_0+r_1}\)であるので、\((i,j'_0-q_1(j_1-j_0)) <_M^{\textrm{Next}} (i,j_0+r_1)\)である。\(M\)が条件(A)を満たすことから\(N_{i,j'_0}+1 = M_{i,j'_0-q_1(j_1-j_0)}+1 = M_{i,j_0+r_1} = N_{i,j'_1}\)となる。
			\end{indented}
			\item 以上よりいずれの場合も\(N_{i,j'_0}+1 = N_{i,j'_1}\)である。従って\(N\)は条件(A)を満たす。
			\item \(N\)が条件(A)と(B)を満たすことから、\nameref{簡約性と係数の関係}より\(N\)は簡約である。
			\item
			\item \(L' := \textrm{Red}((M[n]_j)_{j=j_0+(n-1)(j_1-j_0)}^{j_0+n(j_1-j_0)-1})\)と置く。
			\item \(\textrm{Trans}(M[n]) = s_1 D_{M_{1,j_{-1}}}(t_3 + (D_{M_{1,j_0}} t_4) \times m_n + \textrm{Trans}(L')) b_1\)となることを示す。
			\item \(\textrm{Trans}\)の再帰的定義中に導入した記号を\(N\)に対して定義し、\(N\)に対しての適用であることを明示するために右肩に\(N\)を乗せて表記する。
			\item \(j_{-1} < j_0 < j_1\)より\(j_1^N = \textrm{Lng}(N)-1 = j_0+(n-1)(j_1-j_0) > j_{-1} + (j_1-j_0) \geq 1\)となるので\(\textrm{Pred}(N)\)は零項であり、\nameref{Transが零項性を保つこと}から\(t_1^N = \textrm{Trans}(\textrm{Pred}(N)) \neq 0\)である。\(N\)が単項であることと合わせ、\(N\)に対して条件(I)~(VI)が意味を持つ。
			\item \nameref{第0種型基本列の基本分岐規則}より\((0,j_0-1) <_N^{\textrm{Next}} (0,j_0+(n-1)(j_1-j_0))\)であり、\(j_0\)が非\(M\)許容かつ\(M\)が条件(A)を満たすことから\(N_{1,j_0-1}+1 = M_{1,j_0-1}+1 = M_{1,j_0} = N_{1,j_0+(n-1)(j_1-j_0)} > 0\)であり、かつ\((j_0-1)+1 = j_0 < j_0+(n-1)(j_1-j_0)\)であることから、\(N\)は条件(V)を満たす。
			\item \(j_1^N = j_0+(n-1)(j_1-j_0)\)であり、\nameref{第0種型基本列の基本分岐規則}より\((0,j_0-1) <_N^{\textrm{Next}} (0,j_0+(n-1)(j_1-j_0))\)であるので、\(j_0^N = j_0-1\)である。\(j_0\)が非\(M\)許容であるので\(j_{-1} = \textrm{Adm}_M(j_0-1) = \textrm{Adm}_N(j_0-1) = \textrm{Adm}_N(j_0^N)\)となる。\nameref{Markの左端の基本性質}より\(v^N = N_{1,\textrm{Adm}_N(j_0^N)} = N_{1,j_{-1}} = M_{1,j_{-1}}\)である。
			\item \(\textrm{Pred}(N) = (M[n]_j)_{j=0}^{j_0+(n-1)(j_1-j_0)-1} = M[n-1]\)であるので\(c_1^N = \textrm{Mark}(\textrm{Pred}(N),\textrm{Adm}_N(j_0^N)) = \textrm{Mark}(M[n-1],j_{-1})\)であり、帰納法の仮定より
			\begin{eqnarray*}
			D_{v^N} t_2^N = c_1^N = \textrm{Mark}(M[n-1],j_{-1}) = D_{M_{1,j_{-1}}}(t_3 + (D_{M_{1,j_0}} t_4) \times m_n)
			\end{eqnarray*}
			\item かつ
			\begin{eqnarray*}
			& & s_1^N c_1^N b_1^N = t_1^N = \textrm{Trans}(\textrm{Pred}(N)) = \textrm{Trans}(M[n-1]) \\
			& = & \left\{ \begin{array}{ll} \textrm{Trans}(M)[m_{n-1}] & (m_{n-1} \geq 0) \\ s_1 D_{M_{1,j_{-1}}}(t_2) & (m_{n-1} = -1) \end{array} \right. \\
			& = & \left\{ \begin{array}{ll} s_1 D_{M_{1,j_{-1}}} (t_3 + (D_{M_{1,j_0}} t_4) \times m_n) b_1 & (m_n \geq 1) \\ s_1 D_{M_{1,j_{-1}}}(t_2) & (m_n = 0) \end{array} \right. \\
			& = & s_1 D_{M_{1,j_{-1}}} (t_3 + (D_{M_{1,j_0}} t_4) \times m_n) b_1
			\end{eqnarray*}
			\item となる。従って\(t_2^N = t_3 + (D_{M_{1,j_0}} t_4) \times m_n\)かつ\(s_1^N = s_1\)かつ\(b_1^N = b_1\)である。\(N\)が条件(V)を満たすことから
			\begin{eqnarray*}
			c_2^N = D_{v^N}(t_2^N + D_{N_{1,j_1^N}} 0) = D_{M_{1,j_{-1}}}(t_3 + (D_{M_{1,j_0}} t_4) \times m_n + D_{M_{1,j_0}} 0)
			\end{eqnarray*}
			\item となるので
			\begin{eqnarray*}
			\textrm{Trans}(N) = s_1^N c_2^N b_1^N = s_1 D_{M_{1,j_{-1}}}(t_3 + (D_{M_{1,j_0}} t_4) \times m_n + D_{M_{1,j_0}} 0) b_1
			\end{eqnarray*}
			\item である。更に\nameref{第0種型基本列の基本基点関係} (1)より\((M[n],j_0+(n-1)(j_1-j_0)) \in RT_{\textrm{PS}}^{\textrm{Marked}}\)であり、\nameref{TransのMarkと切片による表示}より
			\begin{eqnarray*}
			\textrm{Trans}(M[n]) = s_1 D_{M_{1,j_{-1}}}(t_3 + (D_{M_{1,j_0}} t_4) \times m_n + \textrm{Mark}(M[n],j_0+(n-1)(j_1-j_0))) b_1
			\end{eqnarray*}
			\item である。
			\item \(j_0\)が非\(M\)許容であることから\((1,j_0) <_M^{\textrm{Next}} (1,j_0+1)\)であり、\(M\)が条件(II)を満たすことから\(M_{1,j_0} \geq 0 = M_{1,j_1}\)であるので、\(j_1 > j_0+1\)であり、よって\(\textrm{Lng}(L')-1 = j_1-j_0-1 > 0\)である。
			\item \nameref{直系先祖による切片とRedとIncrFirstの関係}より
			\begin{eqnarray*}
			& & (M_j)_{j=j_0}^{j_1-1} = (M[n]_j)_{j=j_0+(n-1)(j_1-j_0)}^{j_0+n(j_1-j_0)-1} = \textrm{IncrFirst}^{M[n]_{0,j_0+(n-1)(j_1-j_0)}-M[n]_{1,j_0+(n-1)(j_1-j_0)}}(L') \\
			& = & \textrm{IncrFirst}^{M_{0,j_0}-M_{1,j_0}}(L')
			\end{eqnarray*}
			\item となり、\nameref{RedのIncrFirst不変性}と\(L'\)が簡約であることから\(\textrm{Red}((M_j)_{j=j_0}^{j_1-1}) = L'\)である。\nameref{MarkのTransによる表示}と\nameref{Transの(IncrFirst,Red)不変P同変性}より
			\begin{eqnarray*}
			& & \textrm{Mark}(M[n],j_0+(n-1)(j_1-j_0)) = \textrm{Trans}((M[n]_j)_{j=j_0+(n-1)(j_1-j_0)}^{j_0+n(j_1-j_0)-1}) = \textrm{Trans}((M_j)_{j=j_0}^{j_1-1}) \\
			& = & \textrm{Trans}(L')
			\end{eqnarray*}
			\item となる。従って
			\begin{eqnarray*}
			\textrm{Trans}(M[n]) = s_1 D_{M_{1,j_{-1}}}(t_3 + (D_{M_{1,j_0}} t_4) \times m_n + \textrm{Trans}(L')) b_1
			\end{eqnarray*}
			\item となる。
			\item
			\item \(\textrm{Trans}(L') = D_{M_{1,j_0}} t_4\)であることを示す。
			\item \(L := \textrm{Red}((M_j)_{j=j_{-1}}^{j_1-1})\)と置く。
			\item \((0,j_{-1}) \leq_M (0,j_0) <_M^{\textrm{Next}} (0,j_1)\)と\nameref{直系先祖の木構造} (1)から\((0,j_{-1}) \leq_M (0,j_1-1)\)である。従って\nameref{標準形の直系先祖による切片の簡約化の強単項性}より\(L\)は強単項である。
			\item \(j_{-1} < j_0\)と\nameref{許容性の切片への遺伝性}から\(j_0-j_{-1}\)は非\(L\)許容であり、\nameref{許容化の切片への遺伝性}から\(\textrm{Adm}_L(j_0-j_{-1}) = j_{-1}-j_{-1} = 0\)である。従って\(\textrm{TrMax}\)の定義より\(0 < j_0-j_{-1} < \textrm{TrMax}(L)\)である。
			\item \(\textrm{Lng}(L)-1 = j_1-j_{-1}\)であり、\((0,j_0) <_M^{\textrm{Next}} (0,j_1)\)より\((0,j_0-j_{-1}) <_L^{\textrm{Next}} (0,j_1-j_{-1})\)であり、\(L_{1,j_0-j_{-1}} = M_{1,j_0} \geq M_{1,j_1} = L_{1,j_1-j'_0}\)である。従って\(\textrm{TrMax}\)の定義より\(\textrm{TrMax}(L) < j_1-j'_0\)となるので、\(\textrm{Lng}(\textrm{Br}(L)) \geq 0\)である。
			\item \nameref{簡約性と係数の関係}より\(L\)は条件(A)と(B)を満たす。\(L\)が条件(A)と(B)を満たすことから\(L_{0,j_1-j_{-1}} = L_{0,j_1-j_{-1}}+1 = L_{1,j_1-j_{-1}}+1 \geq L_{0,j_1-j_{-1}}+1 > L_{0,j_1-j_{-1}}\)である。
			\item 従って\nameref{条件(II)か(IV)の下での終切片とTransの関係}から、ある\((t'_1,t'_2) \in T_{\textrm{B}}^2\)が存在して以下を満たす:
			\begin{penumerate}
				\item \(\textrm{Trans}(L') = D_{L_{1,j_0-j_{-1}}}(t'_1 + t'_2)\)かつ\(t'_2\)である。
				\item \(\textrm{Trans}(L) = D_{L_{1,0}}(t'_1 + D_{L_{1,j_0-j_{-1}}}(t'_1 + t'_2))\)である。
			\end{penumerate}
			\item \nameref{右端第2基点のMarkの基本性質}と\nameref{MarkのTransによる表示}と\nameref{Transの(IncrFirst,Red)不変P同変性}と\nameref{MarkのTransによる表示}より
			\begin{eqnarray*}
			& & D_{L_{1,0}}(t'_1 + D_{L_{1,j_0-j_{-1}}}(t'_1 + t'_2)) = \textrm{Trans}(L) = \textrm{Trans}((M_j)_{j=j_{-1}}^{j_1}) = \textrm{Mark}(M,j_{-1}) = c_2  \\
			& = & D_{M_{1,j_{-1}}}(t_3 + D_{M_{1,j_0}} t_4)
			\end{eqnarray*}
			\item であるので\(D_{L_{1,j_0-j_{-1}}}(t'_1 + t'_2)\)は\(t_3 + D_{M_{1,j_0}} t_4\)の最右単項成分\(D_{M_{1,j_0}} t_4\)に等しく、以上より
			\begin{eqnarray*}
			\textrm{Trans}(L') = D_{L_{1,j_0-j_{-1}}}(t'_1 + t'_2) = D_{M_{1,j_0}} t_4
			\end{eqnarray*}
			\item である。
			\item
			\item 以上より、
			\begin{eqnarray*}
			& & \textrm{Trans}(M[n]) = s_1 D_{M_{1,j_{-1}}}(t_3 + (D_{M_{1,j_0}} t_4) \times m_n + \textrm{Trans}(L')) b_1 \\
			& = & s_1 D_{M_{1,j_{-1}}}(t_3 + (D_{M_{1,j_0}} t_4) \times m_n + D_{M_{1,j_0}} t_4) b_1 \\
			& = & s_1 D_{M_{1,j_{-1}}}(t_3 + (D_{M_{1,j_0}} t_4) \times (m_n+1)) b_1 = \textrm{Trans}(M)[m_n]
			\end{eqnarray*}
			\item である。
		\end{indented}
	\end{indented}
\end{hideableproof}


\subsection{条件(III)か(IV)の下での展開規則}

\begin{proposition}[条件(III)か(IV)の下での\(\textrm{Trans}\)と基本列の交換関係]\label{条件(III)か(IV)の下でのTransと基本列の交換関係}
	任意の\(M \in ST_{\textrm{PS}} \cap PT_{\textrm{PS}}\)と\(n \in \mathbb{N}_{+}\)に対し、\(\textrm{Trans}\)の再帰的定義中に導入した記号を用い、\((1,j_{-2}) <_M^{\textrm{Next}} (1,j_1)\)を満たす一意な\(j_{-2} \in \mathbb{N}\)が存在するとし\(j_{-3} := \textrm{Adm}_M(j_{-2})\)と置くと、\(j_1 > 1\)かつ\(M\)が条件(III)か(IV)を満たすならば\footnotemark{}、以下が成り立つ:
	\begin{penumerate}
		\item \(\textrm{Trans}(M[n]) \leq \textrm{Trans}(M)[n-1]\)である。
		\item \(\textrm{Trans}(M[n]) < \textrm{Trans}(M)\)である。
		\item \(\textrm{Trans}(M)[n-1] < \textrm{Trans}(M[n+1])\)である。
	\end{penumerate}
\end{proposition}
\footnotetext{\nameref{標準形の簡約性}から\(M\)は簡約であり、\(j_1 > 1\)より\(t_1 \neq 0\)であるので\(M\)に対し条件(III)と(IV)が意味を持つ。}

\nameref{条件(III)か(IV)の下でのTransと基本列の交換関係}を証明するための準備としていくつかの補題を示す。

\begin{lemma}[右端の非許容直系先祖の基本性質]\label{右端の非許容直系先祖の基本性質}
	任意の\(M \in ST_{\textrm{PS}} \cap PT_{\textrm{PS}}\)と\(m_0,m_1 \in \mathbb{N}\)に対し、\(j_1 := \textrm{Lng}(M)-1\)と置き、\(m_{-1} := \textrm{Adm}_M(m_0)\)と置き、\(N := (M_j)_{j=m_{-1}}^{j_1}\)と置き、\((0,m_0) <_M^{\textrm{Next}} (0,m_1) \leq_M (0,j_1)\)として\(N' := (M_j)_{j=m_0}^{j_1}\)と置き、\(J_1 := \textrm{Lng}(\textrm{Br}(\textrm{Red}(N)))-1\)と置くと、\((1,m_1-1) <_M^{\textrm{Next}} (1,m_1)\)でなくかつ\(m_0\)が非\(M\)許容ならば、\(J_1 \geq 0\)かつ\(0 < m_0-m_{-1} < \textrm{TrMax}(\textrm{Red}(N))\)かつ\(m_0-m_{-1} = \textrm{Joints}(\textrm{Red}(N))_{J_1}\)かつ\(\textrm{FirstNodes}(\textrm{Red}(N))_{J_1} = m_1-m_{-1}\)である。
\end{lemma}

\begin{hideableproof}
	\begin{indented}
		\item \(m_0\)が非\(M\)許容より\(m_{-1} < m_0\)であるので、\(m_{-1}-m_0 > 0\)である。
		\item 任意の\(j \in \mathbb{N}\)に対し、\(0 < j \leq \textrm{TrMax}(\textrm{Red}(N))\)ならば、\((1,j-1) <_{\textrm{Red}(N)}^{\textrm{Next}} (1,j)\)であり\nameref{直系先祖のRed不変性}より\((1,j-1) <_{N}^{\textrm{Next}} (1,j)\)すなわち\((1,j+m_{-1}-1) <_M^{\textrm{Next}} (1,j+m_{-1})\)である。従って\((1,m_1-1) <_M^{\textrm{Next}} (1,m_1)\)でないことから\(\textrm{TrMax}(\textrm{Red}(N)) < m_1-m_{-1}\)である。
		\item \((0,m_0) <_M^{\textrm{Next}} (0,m_1) \leq_M (0,j_1)\)より\((0,m_0-m_{-1}) <_N^{\textrm{Next}} (0,m_1-m_{-1}) \leq_N (0,j_1-m_{-1})\)であり、\nameref{直系先祖のRed不変性}より\((0,m_0-m_{-1}) <_{\textrm{Red}(N)}^{\textrm{Next}} (0,m_1-m_{-1}) \leq_{\textrm{Red}(N)} (0,j_1-m_{-1})\)である。
		\item \nameref{許容化の切片への遺伝性}より\(\textrm{Adm}_N(m_0-m_{-1}) = \textrm{Adm}_M(m_0)-m_{-1} = 0\)であり、\nameref{許容化のRed不変性}から\(\textrm{Adm}_{\textrm{Red}(N)}(m_0-m_{-1}) = 0\)である。従って\(m_0-m_{-1} \leq \textrm{TrMax}(\textrm{Red}(N))\)である。
		\item \(0 < m_0-m_{-1} \leq \textrm{TrMax}(\textrm{Red}(N))\)より\(\textrm{TrMax}(\textrm{Red}(N)) > 0\)である。\(\textrm{TrMax}\)の定義より\(\textrm{TrMax}(\textrm{Red}(N))\)は\(\textrm{Red}(N)\)許容であるので、\(\textrm{Adm}_{\textrm{Red}(N)}(\textrm{TrMax}(\textrm{Red}(N))) = \textrm{TrMax}(\textrm{Red}(N)) > 0 = \textrm{Adm}_{\textrm{Red}(N)}(m_0-m_{-1})\)である。従って\(m_0-m_{-1} < \textrm{TrMax}(\textrm{Red}(N))\)となる。
		\item \(m_0-m_{-1} \leq \textrm{TrMax}(\textrm{Red}(N)) < m_1-m_{-1}\)と\((0,m_0-m_{-1}) <_{\textrm{Red}(N)}^{\textrm{Next}} (0,m_1-m_{-1}) \leq_{\textrm{Red}(N)} (0,j_1-m_{-1})\)と\nameref{Pの各成分の非複項性}より\(\textrm{FirstNodes}(\textrm{Red}(N))_{J_1} = m_1-m_{-1}\)かつ\(m_0-m_{-1} = \textrm{Joints}(\textrm{Red}(N))_{J_1}\)である。
	\end{indented}
\end{hideableproof}

\begin{lemma}[条件(III)~(V)の下での右端の置き換えと\(\textrm{Trans}\)の関係]\label{条件(III)~(V)の下での右端の置き換えとTransの関係}
	任意の\(M \in ST_{\textrm{PS}} \cap PT_{\textrm{PS}}\)に対し、\(\textrm{Trans}\)の再帰的定義中に導入した記号を用い、\((1,j_{-2}) <_M^{\textrm{Next}} (1,j_1)\)を満たす一意な\(j_{-2} \in \mathbb{N}\)が存在するとして\(N' := (M_j)_{j=j_{-2}}^{j_1}\)と置き、\(L' := (M_j)_{j=j_{-2}}^{j_1-1} \oplus_{\mathbb{N}^2}((M_{0,j_1},M_{1,j_{-2}}))\)と置くと、\(j_{-2} < j_1-1\)ならば一意な\((s,b) \in (\Sigma^{< \omega})^2\)が存在して以下を満たす:
	\begin{penumerate}
		\item \((s,D_{M_{1,j_1}} 0,b)\)は\(\textrm{Trans}(N')\)のscb分解である。
		\item \(j_{-2} = j_0\)または\(j_0\)が\(M\)許容であるならば、\((s,D_{M_{1,j_{-2}}} 0,b)\)は\(\textrm{Trans}(L')\)のscb分解である。
		\item \(j_{-2} < j_0\)かつ\(j_0\)が非\(M\)許容であるならば、\((s,D_{M_{1,j_0}}(t_2 + D_{M_{1,j_0}} 0),b)\)は\(\textrm{Trans}(L')\)のscb分解である。
	\end{penumerate}
\end{lemma}

\begin{hideableproof}
	\begin{indented}
		\item \(j_{-3} := \textrm{Adm}_M(j_{-2})\)と置く。
		\item \(N := (M_j)_{j=j_{-3}}^{j_1}\)と置く。
		\item \((1,j_{-3}) \leq_M (1,j_{-2}) <_M^{\textrm{Next}} (1,j_1)\)より\((0,j_{-3}) \leq_M (0,j_{-2}) \leq_M (0,j_1)\)であるので、\nameref{標準形の直系先祖による切片の簡約化の強単項性}より\(\textrm{Red}(N)\)と\(\textrm{Red}(N')\)は強単項であり、特に簡約かつ単項である。
		\item \((N'_{0,j})_{j=0}^{j_1-j_{-2}} = (L'_{0,j})_{j=0}^{j_1-j_{-2}}\)と\nameref{直系先祖のRed不変性}から\(L'\)は単項であり、再び\nameref{Redが単項性を保つこと}から\(\textrm{Red}(L')\)は簡約かつ単項である。
		\item \(\textrm{Trans}\)の再帰的定義中に導入した記号を\(\textrm{Red}(N)\)と\(\textrm{Red}(N')\)と\(\textrm{Red}(L')\)に対して定め、\(\textrm{Red}(N)\)と\(\textrm{Red}(N')\)と\(\textrm{Red}(L')\)に対する適用であることを明示するために右肩にそれぞれ\(N\)と\(N'\)と\(L'\)を載せて表記する。
		\item \(j_1^N = j_1-j_{-3}\)かつ\(j_1^{N'} = j_1^{L'} = j_1-j_{-2}\)であり、\nameref{直系先祖のRed不変性}から\(j_0^N = j_0-j_{-3}\)かつ\(j_0^{N'} = j_0^{L'} = j_0-j_{-2}\)かつ\((1,0) \leq_{\textrm{Red}(N)} (1,j_{-2}-j_{-3}) <_{\textrm{Red}(N)}^{\textrm{Next}} (1,j_1-j_{-3}) = (1,j_1^N)\)かつ\((1,0) <_{\textrm{Red}(N')}^{\textrm{Next}} (1,j_1-j_{-2}) = (1,j_1^{N'})\)である。従って\(\textrm{Red}(N')_{1,0} <_{\textrm{Red}(N')}^{\textrm{Next}} \textrm{Red}(N')_{1,j_1^{N'}}\)である。
		\item \nameref{簡約性と係数の関係}から\(M\)と\(\textrm{Red}(N')\)は条件(A)と(B)を満たす。\(M\)が条件(A)を満たすことと\(j_{-3} = \textrm{Adm}_M(j_{-2})\)より、\((N_j)_{j=0}^{j_{-2}-j_{-3}} = \textrm{IncrFirst}^{N_{0,0}-N_{1,0}}(((j,j))_{j=N_{1,0}}^{N_{1,j_{-2}-j_{-3}}})\)かつ\(N_{0,0}-N_{1,0} = M_{0,j_{-3}}-M_{1,j_{-3}} = M_{0,j_{-2}}-M_{1,j_{-2}} = N'_{0,0}-N'_{0,1}\)である。
		\item \nameref{直系先祖による切片とRedとIncrFirstの関係}より\(\textrm{IncrFirst}^{N'_{0,0}-N'_{1,0}}(\textrm{Red}(N)) = \textrm{IncrFirst}^{N_{0,0}-N_{1,0}}(\textrm{Red}(N)) = N\)かつ\(\textrm{IncrFirst}^{N'_{0,0}-N'_{1,0}}(\textrm{Red}(N')) = N'\)である。
		\item 従って
		\begin{eqnarray*}
		\textrm{Red}(N) & = & (N_{0,j}-N'_{0,0}+N'_{1,0},N_{1,j})_{j=0}^{j_1^{N'}} = ((j,j))_{j=N_{1,0}}^{N'_{1,0}-1} \oplus_{\mathbb{N}^2} (N'_{0,j}-N'_{0,0}+N'_{1,0},N'_{1,j})_{j=0}^{j_1^{N'}} \\
		\textrm{Red}(N') & = & (N'_{0,j}-N'_{0,0}+N'_{1,0},N'_{1,j})_{j=0}^{j_1^{N'}}
		\end{eqnarray*}
		\item となるので、\(\textrm{Red}(N) = ((j,j))_{j=N_{1,0}}^{N'_{1,0}-1} \oplus_{\mathbb{N}^2} \textrm{Red}(N')\)である。特に\(\textrm{Br}(\textrm{Red}(N)) = \textrm{Br}(\textrm{Red}(N'))\)である。
		\item \(J_1 := \textrm{Lng}(\textrm{Br}(\textrm{Red}(N')))-1\)と置く。
		\item \(\textrm{Red}(N')_{1,0} <_{\textrm{Red}(N')}^{\textrm{Next}} \textrm{Red}(N')_{1,j_1^{N'}}\)かつ\(0 < j_1-1-j_{-2} < j_1^{N'}-1\)より\(\textrm{Red}(N')_{1,j_1^{N'}-1} <_{\textrm{Red}(N')}^{\textrm{Next}} \textrm{Red}(N')_{1,j_1^{N'}}\)でない。従って\(\textrm{TrMax}(\textrm{Red}(N')) < j_1^{N'}\)であり、\(J_1 \geq 0\)である。
		\item
		\item \(R := \textrm{Pred}(\textrm{Red}(N')) \oplus_{\mathbb{N}^2} (N'_{0,j_1^{N'}}-N'_{0,0}+N'_{1,0},N'_{1,0})\)と置く。
		\item \(R\)が簡約であることを示す。
		\item \(R\)の定義から\(\leq_{\textrm{Red}(N')}\)と\(\leq_R\)は\((\{0,1\} \times \mathbb{N}) \setminus \{(1,j_1^{N'})\}\)上で一致する。特に\(\textrm{Red}(N')\)の単項性から\(R\)は単項である。
		\item \((1,0) <_{\textrm{Red}(N')}^{\textrm{Next}} (1,j_1^{N'})\)と\nameref{親の基本性質} (2)より、\(\textrm{Red}(N')_{1,0} < \textrm{Red}(N')_{1,j_1^{N'}}\)かつ任意の\(j \in \mathbb{N}\)に対し\(0 < j < j_1^{N'}\)かつ\((0,j) \leq_{\textrm{Red}(N')} (0,j_1^{N'})\)ならば\(R_{1,j} = N'_{1,j} = \textrm{Red}(N')_{1,j} \geq \textrm{Red}(N')_{1,j_1^{N'}} > \textrm{Red}(N')_{1,0} = N'_{1,0} = R_{1,j_1^{N'}}\)である。
		\item 従って、任意の\(j \in \mathbb{N}\)に対し\(0 < j < j_1^{N'}\)ならば\((1,j) \leq_R (1,j_1^{N'})\)でない。更に\(R_{1,0} = N'_{1,0} = R_{1,j_1^{N'}}\)であるので\((1,0) \leq_R (1,j_1^{N'})\)でない。すなわち\((1,j) <_R^{\textrm{Next}} (1,j_1^{N'})\)を満たす一意な\(j \in \mathbb{N}\)は存在しない。
		\item 以上より、任意の\(i \in \{0,1\}\)と\(j'_1,j'_2 \in \mathbb{N}\)に対し、\((i,j'_1) <_R^{\textrm{Next}} (i,j'_1)\)である必要十分条件は\((i,j'_1) <_{\textrm{Red}(N')}^{\textrm{Next}} (i,j'_2) \neq (1,j_1^{N'})\)である。更に\(\textrm{Red}(N')\)が条件(A)を満たすことと\(R = \textrm{Pred}(\textrm{Red}(N')) \oplus_{\mathbb{N}^2} (\textrm{Red}(N')_{0,j_1^{N'}},N'_{1,0})\)であることから、\(R\)は条件(A)を満たす。
		\item \(R\)の単項性と\(R_{0,0} = \textrm{Pred}(\textrm{Red}(N'))_{0,0} = N'_{1,0}\)から、\(R\)は条件(B)を満たす。従って\nameref{簡約性と係数の関係}から\(R\)は簡約である。
		\item また\(\textrm{IncrFirst}^{N'_{0,0}-N'_{1,0}}(R) = L'\)より、\nameref{RedのIncrFirst不変性}から\(\textrm{Red}(L') = \textrm{Red}(R) = R\)である。
		\item
		\item \(\textrm{Pred}(N') = \textrm{Pred}(L')\)より\(t_1^{N'} = t_1^{L'}\)であり、\(\textrm{Red}(N')\)の単項性と\nameref{Transが零項性を保つこと}から\(t_1^{N'} \neq 0\)である。従って\(\textrm{Red}(N')\)と\(\textrm{Red}(L')\)に対し条件(I)~(VI)が意味を持つ。
		\item \(\textrm{Pred}(N') = \textrm{Pred}(L')\)かつ\(j_0^{N'} = j_0^{L'} < j_1^{L'}\)より\(j_{-1}^{N'} = j_{-1}^{L'}\)である。従って\(c_1^{N'} = c_1^{L'}\)であり、\(D_{v^{N'}} t_2^{N'} = c_1^{N'} = c_1^{L'} = D_{v^{L'}} t_2^{L'}\)であるので\(v^{N'} = v^{L'}\)かつ\(t_2^{N'} = t_2^{L'}\)である。また\(t_1^{N'} = t_1^{L'}\)と\(c_1^{N'} = c_1^{L'}\)と\nameref{scb分解の一意性}より\(s_1^{N'} = s_1^{L'}\)かつ\(b_1^{N'} = b_1^{L'}\)である。
		\item \((1,0) <_{\textrm{Red}(N')}^{\textrm{Next}} (1,j_1-j_{-2}) = (1,j_1^{N'})\)より\(\textrm{Red}(N')_{1,j_1^{N'}} > \textrm{Red}(N')_{1,0} \geq 0\)であり、\((1,0) <_{\textrm{Red}(N')}^{\textrm{Next}} (1,j_1^{N'})\)かつ\(0 < j_1-1-j_{-2} = j_1^{N'}-1\)であるので、\(\textrm{Red}(N')\)は条件(III)か(IV)か(V)を満たす。
		\item \(\textrm{Red}(N')\)が条件(A)を満たすことと\(\textrm{Red}(N')_{1,0} <_{\textrm{Red}(N')}^{\textrm{Next}} \textrm{Red}(N')_{1,j_1^{N'}}\)より\(\textrm{Red}(N')_{1,0} = \textrm{Red}(N')_{1,j_1^{N'}}-1\)であるので、
		\begin{eqnarray*}
		& & \textrm{Red}(L')_{1,j_0^{L'}} - \textrm{Red}(L')_{1,j_1^{L'}} = R_{1,j_0^{N'}} - R_{1,j_1^{N'}} = \textrm{Red}(N')_{1,j_0^{N'}} - N'_{1,0} \\
		& = & \textrm{Red}(N')_{1,j_0^{N'}} - \textrm{Red}(N')_{1,0} = \textrm{Red}(N')_{1,j_0^{N'}} - \textrm{Red}(N')_{1,j_1^{N'}} + 1 > \textrm{Red}(N')_{1,j_0^{N'}} - \textrm{Red}(N')_{1,j_1^{N'}}
		\end{eqnarray*}
		\item である。従って\(\textrm{Red}(N')\)が条件(III)か(IV)を満たすならば、\(\textrm{Red}(N')_{1,j_0^{N'}} - \textrm{Red}(N')_{1,j_1^{N'}} \geq 0\)であるので\(\textrm{Red}(L')_{1,j_0^{L'}} - \textrm{Red}(L')_{1,j_1^{L'}} > 0\)となり、\(\textrm{Red}(L')\)は条件(V)と(VI)のいずれも満たさない。\(\textrm{Red}(N')\)が条件(V)を満たすならば、\(\textrm{Red}(N')_{1,j_0^{N'}}+1 = \textrm{Red}(N')_{1,j_1^{N'}}\)より\(\textrm{Red}(L')_{1,j_0^{L'}} - \textrm{Red}(L')_{1,j_1^{L'}} = 0\)となるので\(\textrm{Red}(L')\)は条件(V)と(VI)のいずれも満たさない。
		\item 以上より、いずれの場合も\(\textrm{Red}(L')\)は条件(V)と(VI)のいずれも満たさない。
		\item \nameref{許容性の切片への遺伝性}より以下が同値である:
		\begin{penumerate}
			\item \(j_0^{N'}\)が\(N'\)許容である。
			\item \(j_0^{L'}\)が\(L'\)許容である。
			\item \(j_{-2} = j_0\)または\(j_0\)が\(M\)許容である。
		\end{penumerate}
		\item 従って\nameref{Redが許容性を保つこと}から以下が同値である:
		\begin{penumerate}
			\item \(j_0^{N'}\)が\(\textrm{Red}(N')\)許容である。
			\item \(j_0^{L'}\)が\(\textrm{Red}(L')\)許容である。
			\item \(j_{-2} = j_0\)または\(j_0\)が\(M\)許容である。
		\end{penumerate}
		\item \(j_{-2} = j_0\)または\(j_0\)が\(M\)許容ならば\(\textrm{Red}(L')\)は条件(I)か(III)を満たし、\(j_{-2} < j_0\)かつ\(j_0\)が非\(M\)許容ならば\(\textrm{Red}(L')\)は条件(II)か(IV)を満たす。
		\item
		\item \(j_{-2} = j_0\)または\(j_0\)が\(M\)許容であるとする。
		\begin{indented}
			\item \(\textrm{Red}(N')\)は条件(III)か(V)を満たし、\(\textrm{Red}(L')\)は条件(I)か(III)を満たす。
			\item \(t_2^{N'} = t_2^{L'}\)と\nameref{Transの(IncrFirst,Red)不変P同変性}より
			\begin{eqnarray*}
			\textrm{Trans}(N') & = & \textrm{Trans}(\textrm{Red}(N')) = s_1^{N'} c_2^{N'} b_1^{N'} = s_1^{N'} D_{v^{N'}}(t_2^{N'} + D_{\textrm{Red}(N')_{1.j_1^{N'}}} 0) b_1^{N'} \\
			& = & s_1^{N'} D_{v^{N'}}(t_2^{N'} + D_{N'_{1,j_1^{N'}}} 0) b_1^{N'} = s_1^{N'} D_{v^{N'}}(t_2^{N'} + D_{M_{1,j_1}} 0) b_1^{N'} \\
			\textrm{Trans}(L') & = & \textrm{Trans}(\textrm{Red}(L')) = s_1^{L'} c_2^{L'} b_1^{L'} = s_1^{L'} D_{v^{L'}}(t_2^{L'} + D_{\textrm{Red}(L')_{1,j_1^{L'}}} 0) b_1^{L'} \\
			& = & s_1^{N'} D_{v^{N'}}(t_2^{N'} + D_{L'_{1,j_1^{N'}}} 0) b_1^{N'} = s_1^{N'} D_{v^{N'}}(t_2^{N'} + D_{N'_{1,0}} 0) b_1^{N'} \\
			& = & s_1^{N'} D_{v^{N'}}(t_2^{N'} + D_{M_{1,j_{-2}}} 0) b_1^{N'} \\
			\end{eqnarray*}
			\item であるので、\nameref{scb分解の合成則}と\nameref{加法とscb分解の関係}より一意な\((s,b) \in (\Sigma^{< \omega})^2\)が存在して\((s,D_{M_{1,j_1}} 0,b)\)と\((s,D_{M_{1,j_{-2}}} 0,b)\)はそれぞれ\(\textrm{Trans}(N')\)と\(\textrm{Trans}(L')\)のscb分解である。\qedhere
		\end{indented}
		\item
		\item \(j_{-2} < j_0\)かつ\(j_0\)が非\(M\)許容であるとする。
		\begin{indented}
			\item \(\textrm{Red}(N')\)は条件(IV)を満たし、\(\textrm{Red}(L')\)は条件(II)か(IV)を満たす。
			\item \(t_2^{N'} = t_2^{L'}\)かつ\(\textrm{Red}(N')_{1,j_0^{N'}} = R_{1,j_0^{N'}} = \textrm{Red}(L')_{1,j_0^{L'}}\)より\(t_3^{N'} = t_3^{L'}\)かつ\(t_4^{N'} = t_4^{L'}\)であるので、\nameref{Transの(IncrFirst,Red)不変P同変性}より
			\begin{eqnarray*}
			\textrm{Trans}(N') & = & \textrm{Trans}(\textrm{Red}(N')) = s_1^{N'} c_2^{N'} b_1^{N'} = s_1^{N'} D_{v^{N'}}(t_3^{N'} + D_{\textrm{Red}(N')_{1,j_0^{N'}}}(t_4^{N'} + D_{\textrm{Red}(N')_{1.j_1^{N'}}} 0)) b_1^{N'} \\
			& = & s_1^{N'} D_{v^{N'}}(t_3^{N'} + D_{R_{1,j_0^{N'}}}(t_4^{N'} + D_{N'_{1,j_1^{N'}}} 0)) b_1^{N'} = s_1^{N'} D_{v^{N'}}(t_3^{N'} + D_{R_{1,j_0^{N'}}}(t_4^{N'} + D_{M_{1,j_1}} 0)) b_1^{N'} \\
			\textrm{Trans}(L') & = & \textrm{Trans}(\textrm{Red}(L')) = s_1^{L'} c_2^{L'} b_1^{L'} = s_1^{L'} D_{v^{L'}}(t_3^{L'} + D_{\textrm{Red}(L')_{1,j_0^{L'}}}(t_4^{L'} + D_{\textrm{Red}(L')_{1,j_1^{L'}}} 0)) b_1^{L'} \\
			& = & s_1^{N'} D_{v^{N'}}(t_3^{N'} + D_{R_{1,j_0^{N'}}}(t_4^{N'} + D_{L'_{1,j_1^{N'}}} 0)) b_1^{N'} = s_1^{N'} D_{v^{N'}}(t_3^{N'} + D_{R_{1,j_0^{N'}}}(t_4^{N'} + D_{N'_{1,0}} 0)) b_1^{N'} \\
			& = & s_1^{N'} D_{v^{N'}}(t_3^{N'} + D_{R_{1,j_0^{N'}}}(t_4^{N'} + D_{M_{1,j_{-2}}} 0)) b_1^{N'} \\
			\end{eqnarray*}
			\item であるので、\nameref{scb分解の合成則}と\nameref{加法とscb分解の関係}より一意な\((s,b) \in (\Sigma^{< \omega})^2\)が存在して\((s,D_{M_{1,j_1}} 0,b)\)と\((s,D_{M_{1,j_{-2}}} 0,b)\)はそれぞれ\(\textrm{Trans}(N')\)と\(\textrm{Trans}(L')\)のscb分解である。
		\end{indented}
	\end{indented}
\end{hideableproof}

\begin{lemma}[条件(III)~(VI)の下での展開規則の基本性質]\label{条件(III)~(VI)の下での展開規則の基本性質}
	任意の\(M \in ST_{\textrm{PS}} \cap PT_{\textrm{PS}}\)に対し、\(\textrm{Trans}\)の再帰的定義中に導入した記号を用い、\((1,j_{-2}) <_M^{\textrm{Next}} (1,j_1)\)を満たす一意な\(j_{-2} \in \mathbb{N}\)が存在するとし\(N' := (M_j)_{j=j_{-2}}^{j_1}\)と置き、\(L' := (M'_j)_{j=j_{-2}}^{j_1}\)と置き、各\(n \in \mathbb{N}_{+}\)に対し\(L_n := M[n] \oplus_{\mathbb{N}^2} ((M_{0,j_{-2}}+n(M_{0,j_1}-M_{0,j_{-2}}),M_{1,j_{-2}}))\)と置くと\footnotemark{}、\(j_1 > 1\)\footnotemark{}ならば以下が成り立つ:
	\begin{penumerate}
		\item \(M\)が条件(III)か(IV)を満たすならば\(j_{-2} < j_0\)であり、\(M\)が条件(V)か(VI)を満たすならば\(j_{-2} = j_0\)である。
		\item 任意の\(n \in \mathbb{N}_{+}\)に対し、\(L_n\)は簡約かつ単項である。
		\item \(\leq_M\)と\(\leq_{L_1}\)の\((\{0,1\} \times \mathbb{N}) \setminus \{(1,j_1)\}\)への制限は一致する。
		\item 「\(M\)が(VI)を満たすかまたは\(j_0\)が\(M\)許容である」ならば\(L_1\)は条件(I)か(III)を満たし、「\(M\)が(VI)を満たさずかつ\(j_0\)が非\(M\)許容である」ならば\(L_1\)は条件(II)か(IV)を満たす\footnotemark{}。
		\item 任意の\(n \in \mathbb{N}_{+}\)に対し、\(n > 1\)ならば一意な\((s',b') \in (\Sigma^{< \omega})^2\)が存在して以下を満たす:
		\begin{indented}
			\item[(5-1)] \((s',D_{M_{1,j_{-2}}} 0,b')\)は\(\textrm{Trans}(L_{n-1})\)のscb分解である。
			\item[(5-2)] \((s',\textrm{Trans}(L'),b')\)は\(\textrm{Trans}(L_n)\)のscb分解である。
			\item[(5-3)] \((s',\textrm{Trans}(\textrm{Pred}(N')),b')\)は\(\textrm{Trans}(M[n])\)のscb分解である。
		\end{indented}
	\end{penumerate}
\end{lemma}
\addtocounter{footnote}{-2}
\footnotetext{\((1,j_{-2}) <_M^{\textrm{Next}} (1,j_1)\)より\((0,j_{-2}) \leq_M (0,j_1)\)なので\(M_{0,j_1}-M_{0,j_{-2}} > 0\)であり、\(L_n\)の各成分は自然数となる。}
\addtocounter{footnote}{1}
\footnotetext{この時\(\textrm{Pred}(M)\)が零項でないので\(M\)に対し条件(I)~(VI)が意味を持つ。}
\addtocounter{footnote}{1}
\footnotetext{この時\(\textrm{Pred}(L_1) = \textrm{Pred}(M)\)は零項でないので\(L_1\)に対し条件(I)~(VI)が意味を持つ。}

\begin{hideableproof}
	\begin{indented}
		\item (1)
		\item \(M\)が条件(III)か(IV)を満たすならば、\(M_{1,j_0} \geq M_{1,j_1}\)であるので\((1,j_0) \leq_M (1,j_1)\)でなく、従って\(j_{-2} < j_0\)である。
		\item \(M\)が条件(V)か(VI)を満たすならば、\(M_{1,j_0} < M_{1,j_1}\)であるので\((1,j_0) <_M^{\textrm{Next}} (1,j_1)\)であり、従って\(j_{-2} = j_0\)である。
		\item
		\item (2)
		\item \nameref{標準形の簡約性}より\(M\)は簡約であり、\nameref{簡約性が基本列で保たれること}から\(M[n+1]\)は簡約である。\(M_{1,j_1} > M_{1,j_{-2}} > 0\)と\nameref{非複項性と基本列の関係} (2)より\(M[n+1]\)は単項である。更に
		\begin{eqnarray*}
		L_n = M[n] \oplus_{\mathbb{N}^2} ((M_{0,j_1},M_{1,j_{-2}})) = M[n] \oplus_{\mathbb{N}^2} ((M_{0,j_1},M_{1,j_1}-1)) = (M[n+1]_j)_{j=0}^{j_{-2}+n(j_1-j_{-2})}
		\end{eqnarray*}
		\item であるので、\nameref{簡約性の切片への遺伝性}と\nameref{単項性の始切片への遺伝性}より\(L_n\)は簡約かつ単項である。
		\item
		\item (3)
		\item \(L_1 = M[1] \oplus_{\mathbb{N}^2} (M_{0,j_1},M_{0,j_{-2}}) = \textrm{Pred}(M) \oplus_{\mathbb{N}^2} (M_{0,j_1},M_{0,j_{-2}})\)であるので、\(\leq_M\)と\(\leq_{L_1}\)の\((\{0,1\} \times \mathbb{N}) \setminus \{(1,j_1)\}\)への制限は一致する。特に\((0,j_0) <_{L_1}^{\textrm{Next}} (0,j_1)\)である。
		\item \(\textrm{Trans}\)の再帰的定義中に導入した記号を\(L_1\)に対しても定め、\(L_1\)に対する適用であることを明示するために右肩に\(L_1\)を乗せて表記する。
		\item \(\textrm{Pred}(L_1) = \textrm{Pred}(M)\)より\(t_1^{L_1} = t_1 \neq 0\)であるので\(L_1\)に対して条件(I)~(VI)が意味を持つ。
		\item \((0,j_0) <_{L_1}^{\textrm{Next}} (0,j_1)\)かつ\(L_{1,j_0} = M_{1,j_0} = L_{1,j_1}\)より\(L_1\)は条件(V)と(VI)のいずれも満たさない。
		\item \(M\)が条件(VI)を満たすとする。
		\begin{indented}
			\item \(j_0+1 = j_1\)である。\(L_{1,j_0} = L_{1,j_1} = L_{1,j_0+1}\)であるので\((1,j_0) <_{L_1}^{\textrm{Next}} (1,j_0+1)\)でない。従って\(j_0\)は\(L_1\)許容である。
			\item 以上より、\(L_1\)は条件(I)か(III)を満たす。
		\end{indented}
		\item \(j_0\)が\(M\)許容であるとする。
		\begin{indented}
			\item \(j_0\)が\(M\)許容であることから、\((0,j_0) <_M^{\textrm{Next}} (0,j_0+1)\)でないかまたは\(M_{1,j_0} \geq M_{1,j_0+1}\)である。
			\item \((0,j_0) <_M^{\textrm{Next}} (0,j_0+1)\)でないならば、\((0,j_0) <_{L_1}^{\textrm{Next}} (0,j_0+1)\)でないので\(j_0\)は\(L_1\)許容である。
			\item \(M_{1,j_0} \geq M_{1,j_0+1}\)ならば、\(j_0 < j_1\)より\(L_{1,j_0} = M_{1,j_0} \geq M_{1,j_1} \geq L_{1,j_1}\)であるので\(j_0\)は\(L_1\)許容である。
			\item 従っていずれの場合も\(j_0\)は\(L_1\)許容である。
			\item 以上より、\(L_1\)は条件(I)か(III)を満たす。
		\end{indented}
		\item \(M\)が(VI)を満たさずかつ\(j_0\)が非\(M\)許容であるとする。
		\begin{indented}
			\item \(j_0\)が非\(M\)許容であることから\((1,j_0-1) <_M^{\textrm{Next}} (1,j_0) <_M^{\textrm{Next}} (1,j_0+1)\)である。
			\item \(M_{1,j_1} > M_{1,j_{-2}} \geq 0\)より\(M\)は条件(I)と(II)のいずれも満たさない。従って\(M\)は条件(IV)か(V)を満たすので\(M_{1,j_0}+1 \geq M_{1,j_1}\)となり、\(L_{1,j_0} = M_{1,j_0} \geq M_{1,j_1}-1 = L_{1,j_1}\)となる。
			\item \((1,j_0) <_M^{\textrm{Next}} (1,j_1)\)でない。従って\(j_0+1 < j_1\)となるので、\((1,j_0-1) <_{L_1}^{\textrm{Next}} (1,j_0) <_{L_1}^{\textrm{Next}} (1,j_0+1)\)である。従って\(j_0\)は非\(L_1\)許容である。
			\item 以上より\(L_1\)は条件(II)か(IV)を満たす。\qedhere
		\end{indented}
		\item
		\item (5)
		\item \(\textrm{Lng}(L_n)-1 = j_{-2}+n(j_1-j_{-2})\)である。
		\item \(j_{-2}+(n-1)(j_1-j_{-2})\)が非\(L_n\)許容であると仮定し矛盾を導く。
		\begin{indented}
			\item \(j_{-2}+(n-1)(j_1-j_{-2})\)が非\(L_n\)許容より\((1,j_{-2}+(n-1)(j_1-j_{-2})-1) <_{L_n}^{\textrm{Next}} (1,j_{-2}+(n-1)(j_1-j_{-2}))\)であるので、
			\begin{eqnarray*}
			& & M_{0,j_1-1} = (L_n)_{0,j_{-2}+(n-1)(j_1-j_{-2})-1}-(n-2)(M_{0,j_1}-M_{0,j_{-2}}) \\
			& < & (L_n)_{0,j_{-2}+(n-1)(j_1-j_{-2})}-(n-2)(M_{0,j_1}-M_{0,j_{-2}}) \\
			& = & (M_{0,j_{-2}}+(n-1)(M_{0,j_1}-M_{0,j_{-2}}))-(n-2)(M_{0,j_1}-M_{0,j_{-2}}) = M_{0,j_1}
			\end{eqnarray*}
			\item かつ\(M_{1,j_1-1} = (L_n)_{1,j_{-2}+(n-1)(j_1-j_{-2})-1} < (L_n)_{1,j_{-2}+(n-1)(j_1-j_{-2})} = M_{1,j_{-2}} < M_{1,j_1}\)である。
			\item \(M_{0,j_1-1} < M_{0,j_1}\)かつ\(M_{1,j_1-1} < M_{1,j_{-2}} < M_{1,j_1}\)より\((1,j_1-1) <_M^{\textrm{Next}} (1,j_1)\)すなわち\(j_{-2} = j_1-1\)となるが、これは\(M_{1,j_1-1} < M_{1,j_{-2}}\)に矛盾する。
		\end{indented}
		\item 以上より、\(j_{-2}+(n-1)(j_1-j_{-2})\)は\(L_n\)許容である。更に\((1,j_{-2}) <_M^{\textrm{Next}} (1,j_1)\)より\((0,j_{-2}) \leq_M (0,j_1)\)であるので、\((0,j_{-2}+(n-1)(j_1-j_{-2})) \leq_{L_n} (0,j_{-2}+n(j_1-j_{-2}))\)である。従って\((L_n,j_{-2}+(n-1)(j_1-j_{-2})) \in T_{\textrm{PS}}^{\textrm{Marked}}\)である。
		\item \nameref{MarkのTransによる表示}より
		\begin{eqnarray*}
		\textrm{Mark}(L_n,j_{-2}+(n-1)(j_1-j_{-2})) = \textrm{Trans}(((L_n)_j)_{j=j_{-2}+(n-1)(j_1-j_{-2})}^{j_{-2}+n(j_1-j_{-2})}) = \textrm{Trans}(L')
		\end{eqnarray*}
		\item であるので、\(((L_n)_j)_{j=0}^{j_{-2}+(n-1)(j_1-j_{-2})} = L_{n-1}\)と\nameref{TransのMarkと切片による表示}より一意な\((s',b') \in (\Sigma^{< \omega})^2\)が存在して以下を満たす:
		\begin{penumerate}
			\item \((s',D_{M_{1,j_{-2}}} 0,b')\)は\(\textrm{Trans}(L_{n-1})\)のscb分解である。
			\item \((s',\textrm{Trans}(L'),b')\)は\(\textrm{Trans}(L_n)\)のscb分解である。
		\end{penumerate}
		\item \(j_{-2}+(n-1)(j_1-j_{-2})\)が\(L_n\)許容であり\(j_{-2}+(n-1)(j_1-j_{-2}) < j_{-2}+n(j_1-j_{-2})\)かつ\(M[n] = \textrm{Pred}(L_n)\)であることから、\nameref{基点の切片への遺伝性}より\((M[n],j_{-2}+(n-1)(j_1-j_{-2})) \in T_{\textrm{PS}}^{\textrm{Marked}}\)である。
		\item \nameref{簡約性が基本列で保たれること}から\(M[n]\)は簡約であるので、\nameref{MarkのTransによる表示}より
		\begin{eqnarray*}
		\textrm{Mark}(M[n],j_{-2}+(n-1)(j_1-j_{-2})) = \textrm{Trans}(((M[n])_j)_{j=j_{-2}+(n-1)(j_1-j_{-2})}^{j_{-2}+n(j_1-j_{-2})}) = \textrm{Trans}(\textrm{Pred}(N'))
		\end{eqnarray*}
		\item である。従って\(((M[n])_j)_{j=0}^{j_{-2}+(n-1)(j_1-j_{-2})} = L_{n-1}\)と\nameref{TransのMarkと切片による表示}より、\((s',\textrm{Trans}(\textrm{Pred}(N')),b')\)は\(\textrm{Trans}(M[n])\)のscb分解である。\qedhere\NoEndMark
	\end{indented}
\end{hideableproof}

\begin{lemma}[条件(III)~(VI)の下での\(\textrm{Trans}\)とscb分解の関係]\label{条件(III)~(VI)の下でのTransとscb分解の関係}
	任意の\(M \in RT_{\textrm{PS}} \cap PT_{\textrm{PS}}\)に対し、\(j_1 := \textrm{Lng}(M)-1\)と置いて\(j_1 > 1\)とし、\(\textrm{Trans}\)の再帰的定義中に導入した記号を用い、\((1,j_{-2}) <_M^{\textrm{Next}} (1,j_1)\)を満たす一意な\(j_{-2} \in \mathbb{N}\)が存在するとし、\(j_{-3} := \textrm{Adm}_M(j_{-2})\)と置き、\(N := (M_j)_{j=j_{-3}}^{j_1}\)と置くと、一意な\((s',b') \in (\Sigma^{< \omega})^2\)が存在して\((s',\textrm{Trans}(N),b')\)は\(\textrm{Trans}(M)\)の第\(1\)種scb分解である。
\end{lemma}

\begin{hideableproof}
	\begin{indented}
		\item \((0,j_{-3}) \leq_M (0,j_{-2}) \leq_M (0,j_1)\)かつ\(j_{-3}\)は\(M\)許容であるので、\((M,j_{-3}) \in T_{\textrm{PS}}^{\textrm{Marked}}\)である。
		\item \(N := (M_j)_{j=j_{-3}}^{j_1}\)と置く。
		\item \((0,j_{-3}) \leq_M (0,j_{-2})\)かつ\((1,j_{-2}) <_M^{\textrm{Next}} (1,j_1)\)より\(j_{-3} < j_1\)かつ\((0,j_{-3}) \leq_M (0,j_{-2}) \leq_M (0,j_1)\)である。
		\begin{indented}
			\item \((M,j_{-3}) \in T_{\textrm{PS}}^{\textrm{Marked}}\)より、一意な\((s',b') \in (\Sigma^{< \omega})^2\)が存在して\((s',\textrm{Mark}(M,j'_{-2}),b')\)は\(\textrm{Trans}(M)\)のscb分解である。
			\item \nameref{MarkのTransによる表示}より\((s',\textrm{Trans}(N),b')\)は\(\textrm{Trans}(M)\)のscb分解である。
		\end{indented}
		\item
		\item \((0,j'_{-2}) \leq_M (0,j_1)\)と\nameref{単項性の直系先祖による切片への遺伝性}から、\(N\)は単項である。従って\nameref{Redが単項性を保つこと}より\(\textrm{Red}(N)\)は簡約かつ単項である。
		\item \((0,j'_{-2}) \leq_M (0,j_1)\)より\((0,0) \leq_N (0,j_1-j'_{-2})\)であり、\nameref{直系先祖のRed不変性}より\((0,0) \leq_{\textrm{Red}(N)} (0,j_1-j'_{-2})\)である。
		\item \((0,0) \leq_{\textrm{Red}(N)} (0,j_1-j'_{-2})\)かつ\(0\)が\(\textrm{Red}(N)\)許容であることから、一意な\(J_1 \in \mathbb{N}\)と\(n \in \mathbb{N}_M^{< \omega}\)が存在して以下を満たす:
		\begin{indented}
			\item \(\textrm{Lng}(n) = J_1+1\)である。
			\item \(n_0 = 0\)である。
			\item 任意の\(J \in \mathbb{N}\)に対し、\(J < J_1\)ならば\((0,n_J) <_{\textrm{Red}(N)}^{\textrm{NextAdm}} (0,n_{J+1})\)である。
			\item \(n_{J_1} = j_1-j'_{-2}\)である。
		\end{indented}
		\item 任意の\(J \in \mathbb{N}\)に対し、\(0 < J < J_1\)ならば、\(n_J+j'_{-2} > n_0+j'_{-2} = j'_{-2} \geq j_{-2}\)であり、\((0,n_J) \leq_{\textrm{Red}(N)} (0,n_{J_1}) = (0,j_1-j'_{-2})\)と\nameref{直系先祖のRed不変性}より\((0,n_J) \leq_N (0,j_1-j'_{-2})\)すなわち\((0,n_J+j'_{-2}) \leq_M (0,j_1)\)であるので、\((1,j_{-2}) <_M^{\textrm{Next}} (1,j_1)\)と\nameref{親の基本性質} (2)より\(M_{1,n_J+j'_{-2}} \geq M_{1,j_1}\)である。
		\item \(J'_1 := \textrm{Lng}(\textrm{RightAnces}(N))-1\)と置く。
		\item \(\textrm{Red}(N)\)が単項であるので、\(\textrm{RightAnces}\)の再帰的定義より一意な\(n' \in \mathbb{N}\)が存在して以下が成り立つ:
		\begin{indented}
			\item \(\textrm{RightAnces}(N) = (\textrm{Red}(N)_{1,n'_{J'}})_{J'=0}^{J'_1}\)である。
			\item \(n'_0 = 0\)である。
			\item 任意の\(J' \in \mathbb{N}\)に対し、\(J' \leq J'_1\)ならば\((0,0) \leq_{\textrm{Red}(N)} (0,n'_{J'}) \leq_M (0,j_1-j'_{-2})\)である。
			\item 任意の\(J' \in \mathbb{N}\)に対し、\(J' < J'_1\)ならば\(n'_{J'} < n'_{J'+1}\)である。
			\item \(n'_{J'_1} = j_1-j'_{-2}\)である。
			\item 任意の\(J' \in \mathbb{N}\)に対し、\(J' \leq J'_1\)ならば一意な\(J \in \mathbb{N}\)が存在して\(J \leq J_1\)かつ以下のいずれかを満たす:
			\begin{indented}
				\item[(a)] \(n'_{J'} = n_J\)である。
				\item[(b)] \((\textrm{Red}(N)_j)_{j=0}^{n_J-1}\)が零項でなく\((\textrm{Red}(N)_j)_{j=0}^{n_J}\)は条件(II)か(IV)を満たし\footnote{\nameref{Transが零項性を保つこと}から\(\textrm{Trans}(\textrm{Pred}((\textrm{Red}(N)_j)_{j=0}^{n_J})) \neq 0\)であり、\(\textrm{Red}(N)\)の単項性と\nameref{単項性の始切片への遺伝性}から\((\textrm{Red}(N)_j)_{j=0}^{n_J}\)は単項であり、\nameref{簡約性の切片への遺伝性}から\((\textrm{Red}(N)_j)_{j=0}^{n_J}\)は簡約であるので、\((\textrm{Red}(N)_j)_{j=0}^{n_J}\)に対して条件(I)~(VI)が意味を持つ。}かつ\((0,n'_{J'}) <_{\textrm{Red}(N)}^{\textrm{Next}} (0,n_J)\)である。
			\end{indented}
		\end{indented}
		\item
		\item 任意の\(J' \in \mathbb{N}\)に対し\(0 < J' \leq J'_1\)ならば\(\textrm{RightNodes}(\textrm{Trans}(N))_{J'} \geq \textrm{RightNodes}(\textrm{Trans}(N))_{J'_1}\)であることを示す。
		\begin{indented}
			\item \nameref{直系先祖による切片とRedとIncrFirstの関係}より\(\textrm{IncrFirst}^{N_{0,0}-N_{1,0}}(\textrm{Red}(N)) = N\)であるので、特に任意の\(j \in \mathbb{N}\)に対し\(j \leq j_1-j_{-3}\)ならば\(\textrm{Red}_{1,j} = N_{1,j} = M_{1,j+j_{-3}}\)である。
			\item 一意な\(J \in \mathbb{N}\)が存在して\(0 < J \leq J_1\)かつ\(J'\)に対して条件(a)か(b)を満たす。
			\item \(J\)が(a)を満たすならば、\(n_J\)が\(\textrm{Red}(N)\)許容でありかつ\(\textrm{Adm}_M(j_{-2}) = j_{-3} < n_J+j_{-3}\)であるので\(n_J+j_{-3} > j_{-2}\)であり、従って\(j_{-2} < n_J+j_{-3} = n'_{J'}+j_{-3}\)かつ\((0,n'_{J'}+j_{-3}) = (0,n_J+j_{-3}) \leq_M (0,j_1)\)である。
			\item \(J\)が(b)を満たすとする。
			\begin{indented}
				\item \((0,n'_{J'}) <_{\textrm{Red}(N)}^{\textrm{Next}} (0,n_J)\)と\nameref{直系先祖のRed不変性}より\(n'_{J'} < n_J\)かつ\((0,n'_{J'}) <_N^{\textrm{Next}} (0,n_J)\)すなわち\((0,n'_{J'}+j_{-3}) <_M^{\textrm{Next}} (0,n_J+j_{-3}) \leq_M (0,j_1)\)である。
				\item \((0,j_{-2}) \leq_M (0,j_1)\)より\((0,j_{-2}-j_{-3}) \leq_N (0,j_1-j_{-3})\)であり、\nameref{直系先祖のRed不変性}より\((0,j_{-2}-j_{-3}) \leq_{\textrm{Red}(N)} (0,j_1-j_{-3})\)である。
				\item \(j_{-2}-j_{-3} < n_J\)かつ\((0,j_{-2}-j_{-3}) \leq_{\textrm{Red}(N)} (0,j_1-j_{-3}) = (0,n_{J_1})\)かつ\((0,n_J) \leq_{\textrm{Red}(N)} (0,n_{J_1})\)より\((0,j_{-2}-j_{-3}) \leq_{\textrm{Red}(N)} (0,n_J)\)かつ\((0,n'_{J'}) <_{\textrm{Red}(N)}^{\textrm{Next}} (0,n_J)\)であるので、\((0,j_{-2}-j_{-3}) \leq_{\textrm{Red}(N)} (0,n'_{J'})\)である。
				\item \((0,n'_{J'}) <_{\textrm{Red}(N)}^{\textrm{Next}} (0,n_J) \leq_{\textrm{Red}(N)} (0,n_{J_1}) = (0,j_1-j_{-3})\)と\nameref{直系先祖のRed不変性}より\((0,n'_{J'}) <_N^{\textrm{Next}} (0,n_J) \leq_N (0,j_1-j_{-3})\)すなわち\((0,n'_{J'}+j_{-3}) <_M^{\textrm{Next}} (0,n_J+j_{-3}) \leq_M (0,j_1)\)である。
				\item \(j_{-2}-j_{-3} = n'_{J'}\)と仮定して矛盾を導く。
				\begin{indented}
					\item \((0,j_{-2}) = (0,n'_{J'}+j_{-3}) <_M^{\textrm{Next}} (0,n_J+j_{-3}) \leq_M (0,j_1)\)であり、\((1,j_{-2}) <_M^{\textrm{Next}} (1,j_1)\)と\nameref{親の基本性質} (2)から\(M_{1,n_J+j_{-3}} \geq M_{1,j_1} > M_{1,j_{-2}}\)である。
					\item 従って\(\textrm{Red}(N)_{1,n_J} = M_{1,n_J+j_{-3}} > M_{1,j_{-2}} = \textrm{Red}(N)_{1,j_{-2}-j_{-3}}\)となるので、\((0,j_{-2}-j_{-3}) <_{\textrm{Red}(N)}^{\textrm{Next}} (0,n_J)\)より\((\textrm{Red}(N)_j)_{j=0}^{n_J}\)は条件(V)か(VI)を満たし、\((\textrm{Red}(N)_j)_{j=0}^{n_J}\)は条件(II)か(IV)を満たすことと矛盾する。
				\end{indented}
				\item 以上より\(j_{-2}-j_{-3} < n'_{J'}\)すなわち\(j_{-2} < n'_{J'}+j_{-3}\)である。
			\end{indented}
			\item 以上より、いずれの場合も\(j_{-2} < n'_{J'}+j_{-3}\)かつ\((0,n'_{J'}+j_{-3}) \leq_M (0,j_1)\)である。
			\item \(j_{-2} < n'_{J'}+j_{-3}\)かつ\((0,n'_{J'}+j_{-3}) \leq_M (0,j_1)\)かつ\((1,j_{-2}) <_M^{\textrm{Next}} (1,j_1)\)より、\nameref{親の基本性質} (2)から\(M_{1,n'_{J'}+j'_{-2}} \geq M_{1,j_1}\)である。従って\nameref{RightNodesとRightAncesの関係}より
			\begin{eqnarray*}
			& & \textrm{RightNodes}(\textrm{Trans}(N))_{J'} = \textrm{RightAnces}(N)_{J'} = \textrm{Red}(N)_{1,n'_{J'}} = M_{1,n'_{J'}+j'_{-2}} \\
			& \geq & M_{1,j_1} = \textrm{Red}(N)_{1,n'_{J'_1}} =  \textrm{RightAnces}(N)_{J'_1} = \textrm{RightNodes}(\textrm{Trans}(N))_{J'_1}
			\end{eqnarray*}
			\item である。
		\end{indented}
		\item
		\item \(M_{1,j_{-3}} \leq M_{1,j_{-2}} < M_{1,j_1}\)であるので、\nameref{RightNodesとRightAncesの関係}より
		\begin{eqnarray*}
		& & \textrm{RightNodes}(\textrm{Trans}(N))_0 = \textrm{RightAnces}(N)_0 = \textrm{Red}(N)_{1,n'_0} = M_{1,j_{-3}} \\
		& < & M_{1,j_1} = \textrm{RightAnces}(N)_{J'_1} = \textrm{RightNodes}(\textrm{Trans}(N))_{J'_1}
		\end{eqnarray*}
		\item である。以上より、\((s',\textrm{Trans}(N),b')\)は\(\textrm{Trans}(M)\)の第\(1\)種scb分解である。
	\end{indented}
\end{hideableproof}

\begin{lemma}[条件(III)~(V)の下での切片のscb分解]\label{条件(III)~(V)の下での切片のscb分解}
	任意の\(M \in ST_{\textrm{PS}} \cap PT_{\textrm{PS}}\)と\(n \in \mathbb{N}_{+}\)に対し、\(\textrm{Trans}\)の再帰的定義中に導入した記号を用い、\((1,j_{-2}) <_M^{\textrm{Next}} (1,j_1)\)を満たす一意な\(j_{-2} \in \mathbb{N}\)が存在するとし\(N' := (M_j)_{j=j_{-2}}^{j_1}\)と置くと、\(M\)が条件(VI)を満たさずかつ\(\textrm{Adm}_M(j_{-2}) = j_{-1}\)ならば、一意な\((s'_1,b'_1) \in (\Sigma^{< \omega})^2\)が存在して以下を満たす:
	\begin{penumerate}
		\item \((D_{M_{1,j_{-1}}} s'_1,D_{M_{1,j_1}} 0,b'_1)\)は\(c_2\)のscb分解である。
		\item \((D_{M_{1,j_{-2}}} s'_1,D_{M_{1,j_1}} 0,b'_1)\)は\(\textrm{Trans}(N')\)のscb分解である。
		\item \(\textrm{Trans}(\textrm{Pred}(N')) = D_{M_{1,j_{-2}}} t_2\)である。
	\end{penumerate}
\end{lemma}

\begin{hideableproof}
	\begin{indented}
		\item 以下では\nameref{条件(III)~(VI)の下での展開規則の基本性質}を断りなく用いる。
		\item
		\item (1)
		\item \nameref{標準形の簡約性}より\(M\)は簡約である。
		\item \nameref{右端第1基点のMarkの基本性質}から\(\textrm{Mark}(M,j_1) = D_{M_{1,j_1}} 0\)であり、\nameref{右端第2基点のMarkの基本性質}から\(\textrm{Mark}(M,j_{-1}) = c_2\)である。
		\item 従って\(j_{-1} \leq j_0 < j_1\)と\nameref{scb分解の自明性の判定条件}と\nameref{Transの最左単項成分の左端の基本性質}と\nameref{Markが順序関係を保つこと}から一意な\((s'_1,b'_1) \in (\Sigma^{< \omega})^2\)が存在して\((D_{M_{1,j_{-1}}} s'_1,D_{M_{1,j_1}} 0,b'_1)\)は\(c_2\)のscb分解である。
		\item
		\item (2)
		\item \(N := (M_j)_{j=j_{-1}}^{j_1}\)と置く。
		\item \((0,j_{-1}) \leq_M (0,j_0) <_M^{\textrm{Next}} (0,j_1)\)と\nameref{標準形の直系先祖による切片の簡約化の強単項性}から\(\textrm{Red}(N)\)は強単項である。
		\item \nameref{Transの(IncrFirst,Red)不変P同変性}と\nameref{MarkのTransによる表示}から\(\textrm{Trans}(\textrm{Red}(N)) = \textrm{Trans}(N) = \textrm{Mark}(M,j_{-1}) = c_2\)である。
		\item \(j_{-1} = \textrm{Adm}_M(j_{-2}) \leq j_{-2} \leq j_0\)である。\nameref{許容化の切片への遺伝性}より\(\textrm{Adm}_N(j_{-2}-j_{-1}) = \textrm{Adm}_M(j_{-2})-j_{-1} = 0\)かつ\(\textrm{Adm}_N(j_0-j_{-1}) = \textrm{Adm}_M(j_0)-j_{-1} = 0\)であり、\nameref{許容化のRed不変性}から\(\textrm{Adm}_{\textrm{Red}(N)}(j_{-2}-j_{-1}) = 0\)かつ\(\textrm{Adm}_{\textrm{Red}(N)}(j_0-j_{-1}) = 0\)である。従って\(j_{-2}-j_{-1} \leq j_0-j_{-1} \leq \textrm{TrMax}(\textrm{Red}(N))\)である。
		\item \(M\)が条件(III)か(IV)を満たすならば、\(M_{1,j_0} \geq M_{1,j_1}\)より\((1,j_0) <_M^{\textrm{Next}} (1,j_1)\)でなく、従って\(j_{-2} < j_0 < j_1\)となるので\(j_{-2} < j_1-1\)であり、特に\((1,j_1-1) <_M^{\textrm{Next}} (1,j_1)\)でない。
		\item \(M\)が条件(V)を満たすならば、\(j_0 < j_1-1\)より\((0,j_1-1) <_M^{\textrm{Next}} (0,j_1)\)でなく、特に\((1,j_1-1) <_M^{\textrm{Next}} (1,j_1)\)でない。
		\item 従って、いずれの場合も\((1,j_1-1) <_M^{\textrm{Next}} (1,j_1)\)でない。
		\item \(J_1 := \textrm{Lng}(\textrm{Br}(\textrm{Red}(N)))\)と置く。
		\item \((1,j_1-1) \leq_M (1,j_1)\)でないので、\((1,j_1-1-j_{-1}) \leq_N (1,j_1-j_{-1})\)でない。\nameref{直系先祖のRed不変性}より\((1,j_1-1-j_{-1}) \leq_{\textrm{Red}(N)} (1,j_1-j_{-1})\)でないので、\(\textrm{TrMax}(\textrm{Red}(N)) < j_1-j_{-1}\)である。従って\(J_1 \geq 0\)である。
		\item \((0,j_0) <_M^{\textrm{Next}} (0,j_1)\)より\((0,j_0-j_{-1}) <_N^{\textrm{Next}} (0,j_1-j_{-1})\)であり、\nameref{直系先祖のRed不変性}より\((0,j_0-j_{-1}) <_{\textrm{Red}(N)}^{\textrm{Next}} (0,j_1-j_{-1})\)である。
		\item \((0,j_0-j_{-1}) <_{\textrm{Red}(N)}^{\textrm{Next}} (0,j_1-j_{-1})\)かつ\(j_0-j_{-1} \leq \textrm{TrMax}(\textrm{Red}(N)) < j_1-j_{-1}\)より、\nameref{Pの各成分の非複項性}から\(\textrm{FirstNodes}(\textrm{Red}(N))_{J_1} = j_1-j_{-1}\)かつ\(\textrm{Joints}(\textrm{Red}(N))_{J_1} = j_0-j_{-1}\)となる。
		\item \((1,j_{-2}) <_M^{\textrm{Next}} (1,j_1)\)より\((1,j_{-2}-j_{-1}) <_N^{\textrm{Next}} (1,j_1-j_{-1})\)であり、\nameref{直系先祖のRed不変性}より\((1,j_{-2}-j_{-1}) <_{\textrm{Red}(N)}^{\textrm{Next}} (1,j_1-j_{-1})\)である。
		\item \(M\)が条件(III)か(IV)を満たすならば、\(j_{-2} < j_0\)より\(j_{-2}-j_{-1} < j_0-j_{-1} = \textrm{Joints}(\textrm{Red}(N))_{J_1}\)である。
		\item \(M\)が条件(V)を満たすとする。
		\begin{indented}
			\item \(M_{1,j_0} < M_{1,j_0}+1 = M_{1,j_1}\)より\((1,j_0) <_M^{\textrm{Next}} (1,j_1)\)であるので\(j_{-2} = j_0\)である。特に\nameref{FirstNodesとTrMaxとJointsの関係}から\(j_{-2}-j_{-1} = j_0-j_{-1} = \textrm{Joints}(\textrm{Red}(N))_{J_1} \leq \textrm{TrMax}(\textrm{Red}(N))\)である。
			\item \nameref{簡約性と係数の関係}から\(\textrm{Red}(N)\)は条件(A)と(B)を満たす。\(\textrm{Red}(N)\)が条件(A)と(B)を満たすことと\(j_{-2}-j_1  \leq \textrm{TrMax}(\textrm{Red}(N))\)から\(\textrm{Red}(N)_{0,j_{-2}-j_{-1}} = \textrm{Red}(N)_{1,j_{-2}-j_{-1}}\)である。\(\textrm{Red}(N)\)が条件(A)を満たすことと\((1,j_{-2}-j_{-1}) <_{\textrm{Red}(N)}^{\textrm{Next}} (1,j_1-j_{-1})\)より\(\textrm{Red}(N)_{0,j_1-j_{-1}} = \textrm{Red}(N)_{0,j_{-2}-j_{-1}}+1 = \textrm{Red}(N)_{1,j_{-2}-j_{-1}}+1 = \textrm{Red}(N)_{1,j_1-j_{-1}}\)である。
			\item \(\textrm{Red}(N)\)が強単項であることから、\(\textrm{Br}(\textrm{Red}(N))\)は降順である。
		\end{indented}
		\item 以上より、「\(j_{-2}-j_{-1} < \textrm{Joints}(\textrm{Red}(N))_{J_1}\)」または「\(j_{-2}-j_{-1} = \textrm{Joints}(\textrm{Red}(N))_{J_1}\)かつ\(\textrm{Red}(N)_{0,j_1-j_{-1}} = \textrm{Red}(N)_{1,j_1-j_{-1}}\)かつ\(\textrm{Br}(\textrm{Red}(N))\)が降順」である。
		\item 従って\((D_{M_{1,j_{-1}}} s'_1,D_{M_{1,j_1}} 0,b'_1)\)が\(\textrm{Trans}(\textrm{Red}(N)) = c_2\)のscb分解であることと\nameref{Transの(IncrFirst,Red)不変P同変性}と\nameref{条件(V)の下での終切片とTransの関係}から\((D_{M_{1,j_{-2}}} s'_1,D_{M_{1,j_1}} 0,b'_1)\)は\(\textrm{Trans}(N')\)のscb分解である。
		\item
		\item (3)
		\item \nameref{簡約性の切片への遺伝性}と\nameref{単項性の始切片への遺伝性}より\(\textrm{Pred}(\textrm{Red}(N))\)は簡約かつ単項である。また\nameref{RedとPredの可換性}より\(\textrm{Red}(\textrm{Pred}(N)) = \textrm{Pred}(\textrm{Red}(N))\)である。
		\item \nameref{Transの(IncrFirst,Red)不変P同変性}と\nameref{MarkのTransによる表示}から
		\begin{eqnarray*}
		\textrm{Trans}(\textrm{Pred}(\textrm{Red}(N))) & = & \textrm{Trans}(\textrm{Red}(\textrm{Pred}(N))) = \textrm{Trans}(\textrm{Pred}(N)) \\
		& = & \textrm{Mark}(\textrm{Pred}(M),j_{-1}) = c_1 = D_{M_{1,j_{-1}}} t_2
		\end{eqnarray*}
		\item である。
		\item \nameref{簡約性と係数の関係}より\(M\)は条件(A)と(B)を満たすので、\(j_{-2}-j_{-1} < j_0-j_{-1} \leq \textrm{TrMax}(\textrm{Red}(N)) = \textrm{TrMax}(N)\)より\(N_{0,0}-N_{1,0} = N_{0,j_{-2}-j_{-1}}-N_{1,j_{-2}-j_{-1}} = N'_{0,0}-N'_{1,0}\)である。
		\item \nameref{直系先祖による切片とRedとIncrFirstの関係}より\(\textrm{IncrFirst}^{N_{0,0}-N_{1,0}}(\textrm{Red}(\textrm{Pred}(N))) = \textrm{Pred}(N)\)かつ\(\textrm{IncrFirst}^{N'_{0,0}-N'_{1,0}}(\textrm{Red}(\textrm{Pred}(N'))) = \textrm{Pred}(N')\)であるので、\((\textrm{Pred}(N)_j)_{j=j_{-2}-j_{-1}}^{j_1-1-j_{-1}} = \textrm{Pred}(N')\)より\((\textrm{Red}(\textrm{Pred}(N))_j)_{j=j_{-2}-j_{-1}}^{j_1-1-j_{-1}} = \textrm{Red}(\textrm{Pred}(N'))\)である。
		\item \(J_0 := \textrm{Lng}(\textrm{Br}(\textrm{Pred}(\textrm{Red}(N))))\)と置く。
		\item \(\textrm{TrMax}(\textrm{Red}(N)) < j_1-1-j_{-1}\)とする。
		\begin{indented}
			\item \(\textrm{TrMax}(\textrm{Red}(N)) < j_1-1-j_{-1}\)より\(\textrm{TrMax}(\textrm{Pred}(\textrm{Red}(N))) = \textrm{TrMax}(\textrm{Red}(N))\)であるので、特に\(j_{-2}-j_{-1} < j_0-j_{-1} \leq \textrm{TrMax}(\textrm{Pred}(\textrm{Red}(N))) < j_1-j_{-1}\)である。従って\(J_0 \geq 0\)である。\nameref{Predが1で表されること}と\nameref{Pと基本列の関係}から\(J_1-1 \leq J_0 \leq J_1\)である。
			\item \((1,j_{-2}-j_{-1}) <_{\textrm{Red}(N)}^{\textrm{Next}} (1,j_1-j_{-1})\)と\((0,j_0-j_{-1}) <_{\textrm{Red}(N)}^{\textrm{Next}} (0,j_1-j_{-1})\)と\nameref{直系先祖の木構造} (1)より\((0,j_{-2}-j_{-1}) \leq_{\textrm{Red}(N)} (0,j_0-j_{-1}) \leq_{\textrm{Red}(N)} (0,j_1-1-j_{-1})\)であるので、\((0,j_{-2}-j_{-1}) \leq_{\textrm{Pred}(\textrm{Red}(N))} (0,j_0-j_{-1}) \leq_{\textrm{Pred}(\textrm{Red}(N))} (0,j_1-1-j_{-1})\)である。
			\item \((0,j_0-j_{-1}) \leq_{\textrm{Pred}(\textrm{Red}(N))} (0,j_1-1-j_{-1})\)かつ\(j_0-j_{-1} \leq \textrm{TrMax}(\textrm{Pred}(\textrm{Red}(N))) < j_1-1-j_{-1}\)より、\nameref{Pの各成分の非複項性}から\(j_0-j_{-1} \leq \textrm{Joints}(\textrm{Pred}(\textrm{Red}(N)))_{J_0}\)である。特に\(j_{-2}-j_{-1} < j_0-j_{-1} \leq \textrm{Joints}(\textrm{Pred}(\textrm{Red}(N)))_{J_0}\)である。
			\item 従って\(\textrm{Trans}(\textrm{Red}(\textrm{Pred}(N))) = \textrm{Trans}(\textrm{Pred}(\textrm{Red}(N))) = D_{M_{1,j_{-1}}} t_2\)かつ\((\textrm{Red}(\textrm{Pred}(N))_j)_{j=j_{-2}-j_{-1}}^{j_1-1-j_{-1}} = \textrm{Red}(\textrm{Pred}(N'))\)であることと\nameref{Transの(IncrFirst,Red)不変P同変性}と\nameref{条件(V)の下での終切片とTransの関係}から\(\textrm{Trans}(\textrm{Pred}(N')) = \textrm{Trans}(\textrm{Red}(\textrm{Pred}(N'))) = D_{M_{1,j_{-2}}} t_2\)である。
		\end{indented}
		\item \(\textrm{TrMax}(\textrm{Red}(N)) = j_1-1-j_{-1}\)とする。
		\begin{indented}
			\item \(\textrm{TrMax}(\textrm{Red}(\textrm{Pred}(N))) = \textrm{TrMax}(\textrm{Pred}(\textrm{Red}(N))) = \textrm{TrMax}(\textrm{Red}(N)) = j_1-1-j_{-1}\)である。
			\item 従って\(\textrm{Trans}(\textrm{Red}(\textrm{Pred}(N))) = D_{M_{1,j_{-1}}} t_2\)かつ\((\textrm{Red}(\textrm{Pred}(N))_j)_{j=j_{-2}-j_{-1}}^{j_1-1-j_{-1}} = \textrm{Red}(\textrm{Pred}(N'))\)かつ\(j_{-2}-j_{-1} < j_0-j_{-1} \leq j_1-1-j_{-1}\)であることと\nameref{Transの(IncrFirst,Red)不変P同変性}と\nameref{公差(1,1)のペア数列のTransの基本性質}から\(\textrm{Trans}(\textrm{Pred}(N')) = \textrm{Trans}(\textrm{Red}(\textrm{Pred}(N'))) = D_{M_{1,j_{-2}}} t_2\)である。
		\end{indented}
	\end{indented}
\end{hideableproof}

\begin{lemma}[条件(III)~(V)の下での各種scb分解]\label{条件(III)~(V)の下での各種scb分解}
	任意の\(M \in ST_{\textrm{PS}} \cap PT_{\textrm{PS}}\)と\(n \in \mathbb{N}_{+}\)に対し、\(\textrm{Trans}\)の再帰的定義中に導入した記号を用い、\((1,j_{-2}) <_M^{\textrm{Next}} (1,j_1)\)を満たす一意な\(j_{-2} \in \mathbb{N}\)が存在するとし\(N' := (M_j)_{j=j_{-2}}^{j_1}\)と置き、\(L' := (M_j)_{j=j_{-2}}^{j_1-1} \oplus_{\mathbb{N}^2} ((M_{0,j_1},M_{1,j_{-2}}))\)と置き、\(L_n := M[n] \oplus_{\mathbb{N}^2} ((M_{0,j_{-2}}+n(M_{0,j_1}-M_{0,j_{-2}}),M_{1,j_{-2}}))\)と置くと\footnotemark{}、\(M\)が条件(VI)を満たさずかつ「\(j_{-2} < j_0\)または\(j_0\)が\(M\)許容」かつ\(\textrm{Adm}_M(j_{-2}) = j_{-1}\)ならば、一意な\((s'_1,b'_1) \in (\Sigma^{< \omega})^2\)が存在して以下を満たす:
	\begin{penumerate}
		\item \((D_{M_{1,j_{-1}}} s'_1,D_{M_{1,j_1}} 0,b'_1)\)は\(c_2\)のscb分解である。
		\item \((D_{M_{1,j_{-2}}} s'_1,D_{M_{1,j_1}} 0,b'_1)\)と\((D_{M_{1,j_{-2}}} s'_1,D_{M_{1,j_{-2}}} 0,b'_1)\)はそれぞれ\(\textrm{Trans}(N')\)と\(\textrm{Trans}(L')\)のscb分解である。
		\item \(\textrm{Trans}(\textrm{Pred}(N')) = D_{M_{1,j_{-2}}} t_2\)である。
		\item \(\textrm{Trans}(L_n) = s_1 D_{M_{1,j_{-1}}} (s'_1 D_{M_{1,j_{-2}}})^n 0 (b'_1)^n b_1\)である。
		\item \(\textrm{Trans}(M[n]) = s_1 D_{M_{1,j_{-1}}} (s'_1 D_{M_{1,j_{-2}}})^{n-1} t_2 (b'_1)^{n-1} b_1\)である。
	\end{penumerate}
\end{lemma}
\footnotetext{\((1,j_{-2}) <_M^{\textrm{Next}} (1,j_1)\)より\((0,j_{-2}) \leq_M (0,j_1)\)なので\(M_{0,j_1}-M_{0,j_{-2}} > 0\)であり、\(L_n\)の各成分は自然数となる。}

\begin{hideableproof}
	\begin{indented}
		\item 以下では\nameref{条件(III)~(VI)の下での展開規則の基本性質}を断りなく用いる。
		\item \nameref{条件(III)~(V)の下での切片のscb分解}より一意な\((s'_1,b'_1) \in (\Sigma^{< \omega})^2\)が存在して以下を満たす:
		\begin{penumerate}
			\item \((D_{M_{1,j_{-1}}} s'_1,D_{M_{1,j_1}} 0,b'_1)\)は\(c_2\)のscb分解である。
			\item \((D_{M_{1,j_{-2}}} s'_1,D_{M_{1,j_1}} 0,b'_1)\)は\(\textrm{Trans}(N')\)のscb分解である。
			\item \(\textrm{Trans}(\textrm{Pred}(N')) = D_{M_{1,j_{-2}}} t_2\)である。
		\end{penumerate}
		\item 以上より(1)と(3)が従う。
		\item
		\item (2)
		\item \(j_{-2} < j_0\)または\(j_0\)が\(M\)許容であるので、\(L'_{j_1-j_{-2}} = (M_{0,j_1},M_{1,j_{-2}}) = (N'_{0,j_1-j_{-2}},N'_{1,0})\)と\(j_0\)の\(M\)許容性と\nameref{条件(III)~(V)の下での右端の置き換えとTransの関係}より、\((D_{M_{1,j_{-2}}} s'_1,D_{M_{1,j_{-2}}} 0,b'_1)\)は\(\textrm{Trans}(L')\)のscb分解である。
		\item
		\item (4)
		\item \(M\)が条件(A)を満たしかつ\((1,j_{-2}) <_M^{\textrm{Next}} (1,j_1)\)であることから\(M_{1,j_{-2}} = M_{1,j_1}-1\)である。
		\item \(\textrm{Trans}(L_n) = s_1 D_{M_{1,j_{-1}}} (s'_1 D_{M_{1,j_{-2}}})^n 0 (b'_1)^n b_1\)であることを\(n\)に関する数学的帰納法で示す。
		\item \(n = 1\)とする。
		\begin{indented}
			\item \(\textrm{Trans}\)の再帰的定義中に導入した記号を\(L_1\)に対しても定め、\(L_1\)に対する適用であることを明示するために右肩に\(L_1\)を乗せて表記する。
			\item \(\textrm{Pred}(L_1) = \textrm{Pred}(M)\)より\(s_1^{L_1} = s_1\)かつ\(b_1^{L_1} = b_1\)であり、\(M\)が条件(III)か(IV)のどちらを満たすかに従って\(L_1\)が条件(I)か(III)のいずれかか条件(II)か(IV)のいずれか満たすことと\((D_{M_{1,j_{-1}}} s'_1,D_{M_{1,j_1}} 0,b'_1)\)が\(c_2\)のscb分解であることと\(\textrm{Trans}\)の定義から\((D_{M_{1,j_{-1}}} s'_1,D_{M_{1,j_{-2}}} 0,b'_1)\)は\(c_2^{L_1}\)のscb分解となる。従って
			\begin{eqnarray*}
			\textrm{Trans}(L_n) = \textrm{Trans}(L_1) = s_1^{L_1} c_2^{L_1} b_1^{L_1} = s_1 D_{M_{1,j_{-1}}} s'_1 D_{M_{1,j_{-2}}} 0 b'_1 b_1 = s_1 D_{M_{1,j_{-1}}} (s'_1 D_{M_{1,j_{-2}}})^n 0 (b'_1)^n b_1
			\end{eqnarray*}
			\item である。
		\end{indented}
		\item
		\item \(n > 1\)とする。
		\begin{indented}
			\item 帰納法の仮定から
			\begin{eqnarray*}
			\textrm{Trans}(L_{n-1}) & = & s_1 D_{M_{1,j_{-1}}} (s'_1 D_{M_{1,j_{-2}}})^{n-1} 0 (b'_1)^{n-1} b_1 \\
			& = & (s_1 D_{M_{1,j_{-1}}} (s'_1 D_{M_{1,j_{-2}}})^{n-2} s'_1) D_{M_{1,j_{-2}}} 0 ((b'_1)^{n-1} b_1)
			\end{eqnarray*}
			\item であるので、
			\begin{eqnarray*}
			\textrm{Trans}(L_n) & = & (s_1 D_{M_{1,j_{-1}}} (s'_1 D_{M_{1,j_{-2}}})^{n-2} s'_1) \textrm{Trans}(L') ((b'_1)^{n-1} b_1) \\
			& = & (s_1 D_{M_{1,j_{-1}}} (s'_1 D_{M_{1,j_{-2}}})^{n-2} s'_1) D_{M_{1,j_{-2}}} s'_1 D_{M_{1,j_{-2}}} 0 b'_1 ((b'_1)^{n-1} b_1) \\
			& = & s_1 D_{M_{1,j_{-1}}} (s'_1 D_{M_{1,j_{-2}}})^n 0 (b'_1)^n b_1
			\end{eqnarray*}
			\item である。
		\end{indented}
		\item
		\item (5)
		\item \(n=1\)とする。
		\begin{indented}
			\item \(M[n] = \textrm{Pred}(M)\)より
			\begin{eqnarray*}
			\textrm{Trans}(M[n]) & = & t_1 = s_1 c_1 b_2 = s_1 D_{M_{1,j_{-1}}} t_2 b_1 \\
			& = & s_1 D_{M_{1,j_{-1}}} (s'_1 D_{M_{1,j_{-2}}})^{n-1} t_2 (b'_1)^{n-1} b_1
			\end{eqnarray*}
			\item である。
		\end{indented}
		\item
		\item \(n > 1\)とする。
		\begin{eqnarray*}
		\textrm{Trans}(L_n) & = & s_1 D_{M_{1,j_{-1}}} (s'_1 D_{M_{1,j_{-2}}})^n 0 (b'_1)^n b_1 \\
		& = & (s_1 D_{M_{1,j_{-1}}} (s'_1 D_{M_{1,j_{-2}}})^{n-2}) s'_1 D_{M_{1,j_{-2}}} s'_1 D_{M_{1,j_{-2}}} 0 b'_1 ((b'_1)^{n-1} b_1) \\
		& = & (s_1 D_{M_{1,j_{-1}}} (s'_1 D_{M_{1,j_{-2}}})^{n-2}) s'_1 \textrm{Trans}(L') ((b'_1)^{n-1} b_1)
		\end{eqnarray*}
		\begin{indented}
			\item より
			\begin{eqnarray*}
			\textrm{Trans}(M[n]) & = & (s_1 D_{M_{1,j_{-1}}} (s'_1 D_{M_{1,j_{-2}}})^{n-2}) s'_1 \textrm{Trans}(\textrm{Pred}(N')) ((b'_1)^{n-1} b_1) \\
			& = & (s_1 D_{M_{1,j_{-1}}} (s'_1 D_{M_{1,j_{-2}}})^{n-2}) s'_1 D_{M_{1,j_{-2}}} t_2 ((b'_1)^{n-1} b_1) \\
			& = & s_1 D_{M_{1,j_{-1}}} (s'_1 D_{M_{1,j_{-2}}})^{n-1} t_2 (b'_1)^{n-1} b_1
			\end{eqnarray*}
			\item である。
		\end{indented}
	\end{indented}
\end{hideableproof}

\begin{lemma}[条件(III)か(IV)の下での各種scb分解]\label{条件(III)か(IV)の下での各種scb分解}
	任意の\(M \in ST_{\textrm{PS}} \cap PT_{\textrm{PS}}\)と\(n \in \mathbb{N}_{+}\)に対し、\(\textrm{Trans}\)の再帰的定義中に導入した記号を用い、\((1,j_{-2}) <_M^{\textrm{Next}} (1,j_1)\)を満たす一意な\(j_{-2} \in \mathbb{N}\)が存在するとし\(j_{-3} := \textrm{Adm}_M(j_{-2})\)と置き、\(N := (M_j)_{j=j_{-3}}^{j_1}\)と置き、\(N' := (M_j)_{j=j_{-2}}^{j_1}\)と置き、\(L' := (M_j)_{j=j_{-2}}^{j_1-1} \oplus_{\mathbb{N}^2} ((M_{0,j_1},M_{1,j_{-2}}))\)と置き、\(L_n := M[n] \oplus_{\mathbb{N}^2} ((M_{0,j_{-2}}+n(M_{0,j_1}-M_{0,j_{-2}}),M_{1,j_{-2}}))\)と置くと\footnotemark{}、\(j_1 > 1\)かつ\(M\)が条件(III)か(IV)を満たし\footnotemark{}かつ\(j_{-3} < j_{-1}\)ならば、一意な\((s'_0,s'_1,s'_2,b'_2,b'_1,b'_0) \in (\Sigma^{< \omega})^6\)が存在して以下を満たす:
	\begin{penumerate}
		\item \((s'_0,\textrm{Trans}(N),b'_0)\)は\(\textrm{Trans}(M)\)のscb分解である。
		\item \((D_{M_{1,j_{-3}}} s'_1,c_1,b'_1)\)と\((D_{M_{1,j_{-3}}} s'_1,c_2,b'_1)\)はそれぞれ\(\textrm{Trans}(\textrm{Pred}(N))\)と\(\textrm{Trans}(N)\)のscb分解である。
		\item \((s'_2,D_{M_{1,j_1}} 0,b'_2)\)は\(c_2\)のscb分解である。
		\item \((D_{M_{1,j_{-2}}} s'_1,c_1,b'_1)\)と\((D_{M_{1,j_{-2}}} s'_1,c_2,b'_1)\)と\((D_{M_{1,j_{-2}}} s'_1 s'_2,D_{M_{1,j_{-2}}} 0,b'_2 b'_1)\)はそれぞれ\(\textrm{Trans}(\textrm{Pred}(N'))\)と\(\textrm{Trans}(N')\)と\(\textrm{Trans}(L')\)のscb分解である。
		\item \(\textrm{Trans}(L_n) = s'_0 D_{M_{1,j_{-3}}} (s'_1 s'_2 D_{M_{1,j_{-2}}})^n 0 (b'_2 b'_1)^n b'_0\)である。
		\item \(\textrm{Trans}(M[n]) = s'_0 D_{M_{1,j_{-3}}} (s'_1 s'_2 D_{M_{1,j_{-2}}})^{n-1} s'_1 c_1 b'_1 (b'_2 b'_1)^{n-1} b'_0\)である。
	\end{penumerate}
\end{lemma}
\addtocounter{footnote}{-1}
\footnotetext{\((1,j_{-2}) <_M^{\textrm{Next}} (1,j_1)\)より\((0,j_{-2}) \leq_M (0,j_1)\)なので\(M_{0,j_1}-M_{0,j_{-2}} > 0\)であり、\(L_n\)の各成分は自然数となる。}
\addtocounter{footnote}{1}
\footnotetext{\nameref{標準形の簡約性}から\(M\)は簡約であり、\(j_1 > 1\)より\(t_1 \neq 0\)であるので\(M\)に対し条件(III)と(IV)が意味を持つ。}

\begin{hideableproof}
	\begin{indented}
		\item 以下では\nameref{条件(III)~(VI)の下での展開規則の基本性質}を断りなく用いる。
		\item
		\item (1)
		\item \nameref{MarkのTransによる表示}から\(\textrm{Trans}(N) = \textrm{Mark}(M,j_{-3})\)であるので、一意な\((s'_0,b'_0) \in (\Sigma^{< \omega})^2\)が存在して\((s'_0,\textrm{Trans}(N),b'_0)\)は\(\textrm{Trans}(M)\)のscb分解である。
		\item
		\item (2)
		\item \nameref{標準形の簡約性}より\(M\)は簡約である。
		\item \(j_{-3} < j_{-1}\)と\nameref{scb分解の自明性の判定条件}と\nameref{Transの最左単項成分の左端の基本性質}と\nameref{Markが順序関係を保つこと}から一意な\((s'_1,b'_1) \in (\Sigma^{< \omega})^2\)が存在して\((D_{M_{1,j_{-3}}} s'_1,c_2,b'_1)\)は\(\textrm{Trans}(N)\)のscb分解である。
		\item \nameref{TransのMarkとPredによる表示}より\((D_{M_{1,j_{-3}}} s'_1,c_1,b'_1)\)は\(\textrm{Trans}(\textrm{Pred}(N))\)のscb分解である。
		\item
		\item (3)
		\item \nameref{右端第1基点のMarkの基本性質}と\nameref{Markが順序関係を保つこと}から一意な\((s'_2,b'_2) \in (\Sigma^{< \omega})^2\)が存在して\((s'_2,D_{M_{1,j_1}} 0,b'_2) = (s'_2,\textrm{Mark}(M,j_1),b'_2)\)は\(c_2\)のscb分解である。
		\item
		\item (4)
		\item \((1,j_{-3}) \leq_M (1,j_{-2}) <_M^{\textrm{Next}} (1,j_1)\)と\nameref{標準形の直系先祖による切片の簡約化の強単項性}から\(\textrm{Red}(N)\)は強単項である。\nameref{簡約性と係数の関係}より\(\textrm{Red}(N)\)は条件(A)と(B)を満たす。
		\item \(J_1 := \textrm{Lng}(\textrm{Br}(\textrm{Red}(N)))\)と置く。
		\item \(M\)が条件(III)か(IV)を満たすことから\(M_{1,j_0} \geq M_{1,j_1}\)であり、従って\((1,j_0) \leq_M (1,j_1)\)でなくすなわち\((1,j_0-j_{-3}) \leq_N (1,j_1-j_{-3})\)でない。\nameref{直系先祖のRed不変性}より\((1,j_0-j_{-3}) \leq_{\textrm{Red}(N)} (1,j_1-j_{-3})\)でないので、\(\textrm{TrMax}(\textrm{Red}(N)) < j_1-j_{-3}\)である。従って\(J_1 \geq 0\)である。
		\item \((1,j_{-2}) <_M^{\textrm{Next}} (1,j_1)\)かつ\((0,j_0) <_M^{\textrm{Next}} (0,j_1)\)より\((1,j_{-2}-j_{-3}) <_N^{\textrm{Next}} (1,j_1-j_{-3})\)かつ\((0,j_0-j_{-3}) <_N^{\textrm{Next}} (0,j_1-j_{-3})\)であり、\nameref{直系先祖のRed不変性}より\((1,j_{-2}-j_{-3}) <_{\textrm{Red}(N)}^{\textrm{Next}} (1,j_1-j_{-3})\)かつ\((0,j_0-j_{-3}) <_{\textrm{Red}(N)}^{\textrm{Next}} (0,j_1-j_{-3})\)である。
		\item \nameref{許容化の切片への遺伝性}より\(\textrm{Adm}_N(j_{-2}-j_{-3}) = 0\)かつ\(\textrm{Adm}_N(j_0-j_{-3}) = j_{-1}-j_{-3}\)であり、\nameref{許容化のRed不変性}から\(\textrm{Adm}_{\textrm{Red}(N)}(j_{-2}-j_{-3}) = 0\)かつ\(\textrm{Adm}_{\textrm{Red}(N)}(j_0-j_{-3}) = j_{-1}-j_{-3}\)である。
		\item \(\textrm{Adm}_{\textrm{Red}(N)}(j_{-2}-j_{-3}) = 0\)より\(j_{-2}-j_{-3} \leq \textrm{TrMax}(\textrm{Red}(N))\)である。\(\textrm{Adm}_{\textrm{Red}(N)}(j_0-j_{-3}) = j_{-1}-j_{-3} > 0\)より\(j_0-j_{-3} \geq \textrm{TrMax}(\textrm{Red}(N))\)である。
		\item \(m_0 := \textrm{Joints}(\textrm{Red}(N))_{J_1}\)と置く。
		\item \(m_1 := \textrm{FirstNodes}(\textrm{Red}(N))_{J_1}\)と置く。
		\item \((1,j_{-2}-j_{-3}) <_{\textrm{Red}(N)}^{\textrm{Next}} (1,j_1-j_{-3})\)かつ\(j_{-2}-j_{-3} \leq \textrm{TrMax}(\textrm{Red}(N)) < j_1-j_{-3}\)より、\(j_{-2}-j_{-3} \leq m_0\)である。
		\item
		\item \(j_{-2}-j_{-3} = m_0\)ならば、\(\textrm{Red}(N)_{0,m_1} = \textrm{Red}(N)_{1,m_1}\)かつ\(\textrm{Br}(\textrm{Red}(N))\)が降順であることを示す。
		\begin{indented}
			\item \(\textrm{Red}(N)\)が強単項であるので、\(\textrm{Br}(\textrm{Red}(N))\)は降順である。
			\item \nameref{Pの各成分の非複項性}から\((0,0) \leq_{\textrm{Br}(\textrm{Red}(N))_{J_1}} (0,j_1-j_{-3}-m_1)\)であるので、\((0,m_1) \leq_{\textrm{Red}(N)} (0,j_1-j_{-3})\)である。特に、\((0,j_{-2}-j_{-3}) = (0,m_0) <_{\textrm{Red}(N)}^{\textrm{Next}} (0,m_1) \leq_{\textrm{Red}(N)} (0,j_1-j_{-3})\)である。
			\item \((1,j_{-2}-j_{-3}) <_{\textrm{Red}(N)}^{\textrm{Next}} (1,j_1-j_{-3})\)かつ\((0,j_{-2}-j_{-3}) <_{\textrm{Red}(N)}^{\textrm{Next}} (0,m_1) \leq_{\textrm{Red}(N)} (0,j_1-j_{-3})\)より、\nameref{直系先祖の木構造} (2)から\((1,j_{-2}-j_{-3}) <_{\textrm{Red}(N)}^{\textrm{Next}} (1,m_1)\)である。
			\item \(\textrm{Red}(N)\)が条件(A)と(B)を満たすことから\(j_{-2}-j_{-3} \leq \textrm{TrMax}(\textrm{Red}(N))\)であるので\(\textrm{Red}(N)_{0,j_{-2}-j_{-3}} = \textrm{Red}(N)_{1,j_{-2}-j_{-3}}\)である。再び\(\textrm{Red}(N)\)が条件(A)を満たすことと\((0,j_{-2}-j_{-3}) <_{\textrm{Red}(N)}^{\textrm{Next}} (0,m_1)\)と\((1,j_{-2}-j_{-3}) <_{\textrm{Red}(N)}^{\textrm{Next}} (1,m_1)\)から
			\begin{eqnarray*}
			\textrm{Red}(N)_{0,m_1} = \textrm{Red}(N)_{0,j_{-2}-j_{-3}}-1 = \textrm{Red}(N)_{1,j_{-2}-j_{-3}}-1 = \textrm{Red}(N)_{1,m_1}
			\end{eqnarray*}
			\item となる。
		\end{indented}
		\item 以上より「\(j_{-2}-j_{-3} < m_0\)」または「\(j_{-2}-j_{-3} = m_0\)かつ\(\textrm{Red}(N)_{0,m_1} = \textrm{Red}(N)_{1,m_1}\)かつ\(\textrm{Br}(\textrm{Red}(N))\)が降順」である。
		\item 従って\((D_{M_{1,j_{-3}}} s'_1,c_2,b'_1)\)が\(\textrm{Trans}(N)\)のscb分解であることと\nameref{Transの(IncrFirst,Red)不変P同変性}と\nameref{条件(V)の下での終切片とTransの関係}から\((D_{M_{1,j_{-2}}} s'_1,c_2,b'_1)\)は\(\textrm{Trans}(N')\)のscb分解である。
		\item
		\item \((1,j_0) \leq_M (1,j_1)\)でなくかつ\((1,j_{-2}) <_M^{\textrm{Next}} (1,j_1)\)かつ\((0,j_0) <_M^{\textrm{Next}} (0,j_1)\)より\(j_{-2} < j_0 < j_1\)である。従って\(j_1-1-j_{-3} \geq j_0-j_{-3} > j_{-2}-j_{-3} \geq 0\)であるので、\(\textrm{Red}(N)\)の強単項性と\nameref{強単項性の切片への遺伝性}より\(\textrm{Pred}(\textrm{Red}(N))\)は強単項である。
		\item \(\textrm{Adm}_{\textrm{Red}(N)}(j_{-2}-j_{-3}) = 0\)かつ\(j_{-2}-j_{-3} < j_1-1-j_{-3}\)より\(\textrm{Adm}_{\textrm{Pred}(\textrm{Red}(N))}(j_{-2}-j_{-3}) = 0\)である。従って\(j_{-2}-j_{-3} \leq \textrm{TrMax}(\textrm{Pred}(\textrm{Red}(N)))\)である。
		\item \(J_0 := \textrm{Lng}(\textrm{Br}(\textrm{Pred}(\textrm{Red}(N))))\)と置く。
		\item \(\textrm{TrMax}(\textrm{Red}(N)) < j_1-j_{-3}\)より\(\textrm{TrMax}(\textrm{Pred}(\textrm{Red}(N))) = \textrm{TrMax}(\textrm{Red}(N))\)であり、すなわち\(\textrm{TrMax}(\textrm{Pred}(\textrm{Red}(N))) \leq j_1-1-j_{-3}\)である。
		\item \(\textrm{TrMax}(\textrm{Red}(N)) < j_1-1-j_{-3}\)とする。
		\begin{indented}
			\item \(\textrm{TrMax}(\textrm{Red}(N)) < j_1-1-j_{-3}\)より\(\textrm{TrMax}(\textrm{Pred}(\textrm{Red}(N))) = \textrm{TrMax}(\textrm{Red}(N))\)であるので、特に\(j_{-2}-j_{-3} \leq \textrm{TrMax}(\textrm{Pred}(\textrm{Red}(N))) < j_1-1-j_{-3}\)である。従って\(J_0 \geq 0\)である。
			\item \((1,j_{-2}-j_{-3}) <_{\textrm{Red}(N)}^{\textrm{Next}} (1,j_1-j_{-3})\)より\((0,j_{-2}-j_{-3}) \leq_{\textrm{Red}(N)} (0,j_1-j_{-3})\)であり、\(j_{-2}-j_{-3} < j_1-1-j_{-3}\)と\nameref{直系先祖の木構造} (1)から\((0,j_{-2}-j_{-3}) \leq_{\textrm{Red}(N)} (0,j_1-1-j_{-3})\)すなわち\((0,j_{-2}-j_{-3}) \leq_{\textrm{Pred}(\textrm{Red}(N))} (0,j_1-1-j_{-3})\)である。
			\item \(\textrm{Adm}_{\textrm{Red}(N)}(j_{-2}-j_{-3}) = 0\)かつ\(\textrm{Adm}_{\textrm{Red}(N)}(j_0-j_{-3}) = j_{-1}-j_{-3}\)より\(\textrm{Adm}_{\textrm{Pred}(\textrm{Red}(N))}(j_{-2}-j_{-3}) = 0\)かつ\(\textrm{Adm}_{\textrm{Pred}(\textrm{Red}(N))}(j_0-j_{-3}) = j_{-1}-j_{-3}\)である。
			\item \(\textrm{Adm}_{\textrm{Pred}(\textrm{Red}(N))}(j_{-2}-j_{-3}) = 0\)より\(j_{-2}-j_{-3} \leq \textrm{TrMax}(\textrm{Pred}(\textrm{Red}(N)))\)である。\(\textrm{Adm}_{\textrm{Pred}(\textrm{Red}(N))} = j_{-1}-j_{-3} > 0\)より\(j_0-j_{-3} \geq \textrm{TrMax}(\textrm{Red}(N))\)である。
			\item \(m'_0 := \textrm{Joints}(\textrm{Pred}(\textrm{Red}(N)))_{J_0}\)と置く。
			\item \(m'_1 := \textrm{FirstNodes}(\textrm{Pred}(\textrm{Red}(N)))_{J_0}\)と置く。
			\item \((0,j_{-2}-j_{-3}) \leq_{\textrm{Pred}(\textrm{Red}(N))} (0,j_1-1-j_{-3})\)かつ\(j_{-2}-j_{-3} \leq \textrm{TrMax}(\textrm{Pred}(\textrm{Red}(N))) < j_1-1-j_{-3}\)より、\(j_{-2}-j_{-3} \leq m'_0\)である。
			\item
			\item \(j_{-2}-j_{-3} = m'_0\)ならば、\(\textrm{Pred}(\textrm{Red}(N))_{0,m'_1} = \textrm{Pred}(\textrm{Red}(N))_{1,m'_1}\)かつ\(\textrm{Br}(\textrm{Pred}(\textrm{Red}(N)))\)が降順であることを示す。
			\begin{indented}
				\item \(\textrm{Pred}(\textrm{Red}(N))\)が強単項であるので、\(\textrm{Br}(\textrm{Pred}(\textrm{Red}(N)))\)は降順である。
				\item \nameref{Predが1で表されること}と\nameref{Pと基本列の関係}から\(J_1-1 \leq J_0 \leq J_1\)であるので、\nameref{FirstNodesとJointsの単調性}より
				\begin{eqnarray*}
				m_0 = \textrm{Joints}(\textrm{Red}(N))_{J_1} & \leq & \textrm{Joints}(\textrm{Red}(N))_{J_0} = \textrm{Joints}(\textrm{Pred}(\textrm{Red}(N)))_{J_0} =m'_0 \\
				m_1 = \textrm{FirstNodes}(\textrm{Red}(N))_{J_1} & \geq & \textrm{FirstNodes}(\textrm{Red}(N))_{J_0} = \textrm{FirstNodes}(\textrm{Pred}(\textrm{Red}(N)))_{J_0} =m'_1
				\end{eqnarray*}
				\item である。
				\item \(j_{-2}-j_{-3} = m'_0\)かつ\(j_{-2}-j_{-3} \leq m_0 \leq m'_0\)より\(j_{-2}-j_{-3} = m_0 = m'_0\)であるので、\(\textrm{Red}(N)_{0,m_1} = \textrm{Red}(N)_{1,m_1}\)かつ\(\textrm{Br}(\textrm{Red}(N))\)が降順である。\(\textrm{Red}(N)\)が条件(A)を満たすことから\(\textrm{Red}(N)_{0,m'_1} = \textrm{Red}(N)_{0,m'_0}+1 = \textrm{Red}(N)_{0,m_0}+1 = \textrm{Red}(N)_{0,m_1}\)である。
				\item \(m'_1 \leq m_1\)と\(\textrm{Red}(N)_{0,m'_1} = \textrm{Red}(N)_{0,m_1}\)より、\(\textrm{Br}(\textrm{Red}(N))\)が降順であることから\(\textrm{Red}(N)_{1,m'_1} \geq \textrm{Red}(N)_{1,m_1} = \textrm{Red}(N)_{0,m_1} = \textrm{Red}(N)_{0,m'_1}\)である。一方で\nameref{簡約性と係数の基本性質}より\(\textrm{Red}(N)_{0,m'_1} \geq \textrm{Red}(N)_{1,m'_1}\)であるので、\(\textrm{Red}(N)_{0,m'_1} = \textrm{Red}(N)_{1,m'_1}\)すなわち\(\textrm{Pred}(\textrm{Red}(N))(N)_{0,m'_1} = \textrm{Pred}(\textrm{Red}(N))_{1,m'_1}\)である。
			\end{indented}
			\item 以上より「\(j_{-2}-j_{-3} < m'_0\)」または「\(j_{-2}-j_{-3} = m'_0\)かつ\(\textrm{Pred}(\textrm{Red}(N))_{0,m'_1} = \textrm{Pred}(\textrm{Red}(N))_{1,m'_1}\)かつ\(\textrm{Br}(\textrm{Pred}(\textrm{Red}(N)))\)が降順」である。
			\item 従って\((D_{M_{1,j_{-3}}} s'_1,c_1,b'_1)\)が\(\textrm{Trans}(\textrm{Pred}(N))\)のscb分解であることと\nameref{Transの(IncrFirst,Red)不変P同変性}と\nameref{RedとPredの可換性}と\nameref{条件(V)の下での終切片とTransの関係}から\((D_{M_{1,j_{-2}}} s'_1,c_1,b'_1)\)は\(\textrm{Trans}(\textrm{Pred}(N'))\)のscb分解である。
		\end{indented}
		\item
		\item \(\textrm{TrMax}(\textrm{Red}(N)) = j_1-1-j_{-3}\)とする。
		\begin{indented}
			\item \(\textrm{TrMax}(\textrm{Pred}(\textrm{Red}(N))) = \textrm{TrMax}(\textrm{Red}(N)) = j_1-1-j_{-3}\)である。
			\item 従って\((D_{M_{1,j_{-3}}} s'_1,c_1,b'_1)\)が\(\textrm{Trans}(\textrm{Pred}(N))\)のscb分解であることと\nameref{RedとPredの可換性}と\nameref{Transの(IncrFirst,Red)不変P同変性}と\nameref{公差(1,1)のペア数列のTransの基本性質}から\((D_{M_{1,j_{-2}}} s'_1,c_1,b'_1)\)は\(\textrm{Trans}(\textrm{Pred}(N'))\)のscb分解である。
		\end{indented}
		\item 以上より、いずれの場合も\((D_{M_{1,j_{-2}}} s'_1,c_1,b'_1)\)は\(\textrm{Trans}(\textrm{Pred}(N'))\)のscb分解である。
		\item \nameref{scb分解の合成則}より\((D_{M_{1,j_{-2}}} s'_1 s'_2,D_{M_{1,j_1}} 0,b'_2 b'_1)\)は\(\textrm{Trans}(N')\)のscb分解であり、\(L'_{j_1-j_{-2}} = (M_{0,j_1},M_{1,j_{-2}}) = (N'_{0,j_1-j_{-2}},N'_{1,0})\)と\(j_{-2} < j_0\)と\nameref{条件(III)~(V)の下での右端の置き換えとTransの関係}より、\((D_{M_{1,j_{-2}}} s'_1 s'_2,D_{M_{1,j_{-2}}} 0,b'_2 b'_1)\)は\(\textrm{Trans}(L')\)のscb分解である。
		\item
		\item (5)
		\item \(s_1 c_2 b_1 = \textrm{Trans}(M) = s'_0 \textrm{Trans}(N) b'_0 = s'_0 D_{M_{1,j_{-3}}} s'_1 c_2 b'_1 b'_0\)より、\nameref{scb分解の一意性} (1)から\(s_1 = s'_0 D_{M_{1,j_{-3}}} s'_1\)かつ\(b_1 = b'_1 b'_0\)である。
		\item \(\textrm{Trans}(L_n) = s'_0 D_{M_{1,j_{-3}}} (s'_1 s'_2 D_{M_{1,j_{-2}}})^n 0 (b'_2 b'_1)^n b'_0\)であることを\(n\)に関する数学的帰納法で示す。
		\item \(n = 1\)とする。
		\begin{indented}
			\item \(\textrm{Pred}(L_1) = \textrm{Pred}(M)\)であるので、\(\textrm{Trans}\)の定義から\(s_1^{L_1} = s_1\)かつ\(b_1^{L_1} = b_1\)かつ\((s'_2,D_{M_{1,j_{-2}}} 0,b'_2) = (s'_2,\textrm{Mark}(L_1,j_1),b'_2)\)は\(c_2^{L_1}\)のscb分解となり、
			\begin{eqnarray*}
			\textrm{Trans}(L_n) & = & \textrm{Trans}(L_1) = s_1^{L_1} c_2^{L_1} b_1^{L_1} = s_1 s'_2 D_{M_{1,j_{-2}}} 0 b'_2 b_1 = s'_0 D_{M_{1,j_{-3}}} s'_1 s'_2 D_{M_{1,j_{-2}}} 0 b'_2 b'_1 b'_0 \\
			& = & s'_0 D_{M_{1,j_{-3}}} (s'_1 s'_2 D_{M_{1,j_{-2}}})^n 0 (b'_2 b'_1)^n b'_0
			\end{eqnarray*}
			\item である。
		\end{indented}
		\item
		\item \(n > 1\)とする。
		\begin{indented}
			\item 帰納法の仮定から
			\begin{eqnarray*}
			\textrm{Trans}(L_{n-1}) & = & s'_0 D_{M_{1,j_{-3}}} (s'_1 s'_2 D_{M_{1,j_{-2}}})^{n-1} 0 (b'_2 b'_1)^{n-1} b'_0 \\
			& = & (s'_0 D_{M_{1,j_{-3}}} (s'_1 s'_2 D_{M_{1,j_{-2}}})^{n-2} s'_1 s'_2) D_{M_{1,j_{-2}}} 0 ((b'_2 b'_1)^{n-1} b'_0)
			\end{eqnarray*}
			\item であるので、
			\begin{eqnarray*}
			\textrm{Trans}(L_n) & = & (s'_0 D_{M_{1,j_{-3}}} (s'_1 s'_2 D_{M_{1,j_{-2}}})^{n-2} s'_1 s'_2) \textrm{Trans}(L') ((b'_2 b'_1)^{n-1} b'_0)  \\
			& = & s'_0 D_{M_{1,j_{-3}}} (s'_1 s'_2 D_{M_{1,j_{-2}}})^{n-2} s'_1 s'_2 D_{M_{1,j_{-2}}} s'_1 s'_2 D_{M_{1,j_{-2}}} 0 b'_2 b'_1 (b'_2 b'_1)^{n-1} b'_0 \\
			& = & s'_0 D_{M_{1,j_{-3}}} (s'_1 s'_2 D_{M_{1,j_{-2}}})^n 0 (b'_2 b'_1)^n b'_0
			\end{eqnarray*}
			\item である。
		\end{indented}
		\item
		\item (6)
		\item \(n=1\)とする。
		\begin{indented}
			\item \(M[n] = \textrm{Pred}(M)\)より
			\begin{eqnarray*}
			& & \textrm{Trans}(M[n]) = t_1 = s_1 c_1 b_2 = s'_0 D_{M_{1,j_{-3}}} s'_1 c_1 b'_1 b'_0 \\
			& = & s'_0 D_{M_{1,j_{-3}}} (s'_1 s'_2 D_{M_{1,j_{-2}}})^{n-1} s'_1 c_1 b'_1 (b'_2 b'_1)^{n-1} b'_0
			\end{eqnarray*}
			\item である。
		\end{indented}
		\item
		\item \(n > 1\)とする。
		\begin{eqnarray*}
		\textrm{Trans}(L_n) & = & s'_0 D_{M_{1,j_{-3}}} (s'_1 s'_2 D_{M_{1,j_{-2}}})^n 0 (b'_2 b'_1)^n b'_0 \\
		& = & (s'_0 D_{M_{1,j_{-3}}} (s'_1 s'_2 D_{M_{1,j_{-2}}})^{n-2} s'_1 s'_2) D_{M_{1,j_{-2}}} s'_1 s'_2 D_{M_{1,j_{-2}}} 0 b'_2 b'_1 ((b'_2 b'_1)^{n-1} b'_0) \\
		& = & (s'_0 D_{M_{1,j_{-3}}} (s'_1 s'_2 D_{M_{1,j_{-2}}})^{n-2} s'_1 s'_2) \textrm{Trans}(L') ((b'_2 b'_1)^{n-1} b'_0)
		\end{eqnarray*}
		\begin{indented}
			\item より
			\begin{eqnarray*}
			\textrm{Trans}(M[n]) & = & (s'_0 D_{M_{1,j_{-3}}} (s'_1 s'_2 D_{M_{1,j_{-2}}})^{n-2} s'_1 s'_2) \textrm{Trans}(\textrm{Pred}(N')) ((b'_2 b'_1)^{n-1} b'_0) \\
			& = & (s'_0 D_{M_{1,j_{-3}}} (s'_1 s'_2 D_{M_{1,j_{-2}}})^{n-2} s'_1 s'_2) D_{M_{1,j_{-2}}} s'_1,c_1,b'_1 ((b'_2 b'_1)^{n-1} b'_0) \\
			& = & s'_0 D_{M_{1,j_{-3}}} (s'_1 s'_2 D_{M_{1,j_{-2}}})^{n-1} s'_1 c_1 b'_1 (b'_2 b'_1)^{n-1} b'_0
			\end{eqnarray*}
			\item である。
		\end{indented}
	\end{indented}
\end{hideableproof}

\begin{lemma}[条件(III)か(IV)の下での基本列の基本性質]\label{条件(III)か(IV)の下での基本列の基本性質}
	任意の\(M \in ST_{\textrm{PS}} \cap PT_{\textrm{PS}}\)と\(n \in \mathbb{N}_{+}\)に対し、\(\textrm{Trans}\)の再帰的定義中に導入した記号を用い、\((1,j_{-2}) <_M^{\textrm{Next}} (1,j_1)\)を満たす一意な\(j_{-2} \in \mathbb{N}\)が存在するとすると、\(j_1 > 1\)かつ\(M\)が条件(III)か(IV)を満たすならば\footnotemark{}、以下が成り立つ:
	\begin{penumerate}
		\item \(M[n] = M[n+1][1]^{j_1-j_{-2}}\)である。
		\item \(\textrm{Trans}(M)[n-1] = \textrm{Trans}(M[n+1][1]^{j_1-1-j_{-2}})\)である。
		\item ある\((s',c'_1,c'_2,b') \in (\Sigma^{< \omega})^4\)が存在し、以下を満たす;
		\begin{indented}
			\item[(3-1)] \(c'_1\)と\(c'_2\)は\(c'_1 < c'_2\)を満たす単項である。
			\item[(3-2)] \((s',c'_1,b')\)は\(\textrm{Trans}(M[n])\)のscb分解である。
			\item[(3-2)] \((s',c'_2,b')\)は\(\textrm{Trans}(M)[n]\)のscb分解である。
		\end{indented}
	\end{penumerate}
\end{lemma}
\footnotetext{\nameref{標準形の簡約性}から\(M\)は簡約であり、\(j_1 > 1\)より\(t_1 \neq 0\)であるので\(M\)に対し条件(III)と(IV)が意味を持つ。}

\begin{hideableproof}
	\begin{indented}
		\item \nameref{簡約性と係数の関係}より\(M\)は条件(A)と(B)を満たす。\(M\)が条件(A)を満たすことから、\(M_{1,j_{-2}} = M_{1,j_1}-1\)である。
		\item \(M\)が条件(III)か(IV)を満たすことから\(M_{1,j_0} \geq M_{1,j_1}\)であるので\((1,j_0) <_M^{\textrm{Next}} (1,j_1)\)でなく、従って\(j_{-2} < j_0\)である。
		\item \(j_{-3} := \textrm{Adm}_M(j_{-2})\)と置く。
		\item \(N := (M_j)_{j=j_{-3}}^{j_1}\)と置く。
		\item \(N' := (M_j)_{j=j_{-2}}^{j_1}\)と置く。
		\item \(L' := (M_j)_{j=j_{-2}}^{j_1-1} \oplus_{\mathbb{N}^2} ((M_{0,j_1},M_{1,j_{-2}}))\)と置く。
		\item \(L_n := M[n] \oplus_{\mathbb{N}^2} ((M_{0,j_{-2}}+n(M_{0,j_1}-M_{0,j_{-2}}),M_{1,j_{-2}}))\)と置く\footnote{\((1,j_{-2}) <_M^{\textrm{Next}} (1,j_1)\)より\((0,j_{-2}) \leq_M (0,j_1)\)なので\(M_{0,j_1}-M_{0,j_{-2}} > 0\)であり、\(L_n\)の各成分は自然数となる。}。
		\item \nameref{Predが1で表されること}から、
		\begin{eqnarray*}
		L_n & = & \textrm{Pred}^{j_1-1-j_{-2}}(M[n+1]) = M[n+1][1]^{j_1-1-j_{-2}} \\
		M[n] & = & \textrm{Pred}(L_n) = M[n+1][1]^{j_1-j_{-2}}
		\end{eqnarray*}
		\item である。
		\item
		\item \(j_{-3} = j_{-1}\)とする。
		\begin{indented}
			\item \(j_{-2} < j_0\)と\nameref{条件(III)~(V)の下での各種scb分解}よりある\((s'_1,b'_1) \in (\Sigma^{< \omega})^2\)が存在して以下を満たす:
			\begin{penumerate}
				\item \((D_{M_{1,j_{-1}}} s'_1,D_{M_{1,j_1}} 0,b'_1)\)は\(c_2\)のscb分解である。
				\setcounter{penumeratei}{3}
				\item \(\textrm{Trans}(L_n) = s_1 D_{M_{1,j_{-1}}} (s'_1 D_{M_{1,j_{-2}}})^n 0 (b'_1)^n b_1\)である。
				\item \(\textrm{Trans}(M[n]) = s_1 D_{M_{1,j_{-1}}} (s'_1 D_{M_{1,j_{-2}}})^{n-1} t_2 (b'_1)^{n-1} b_1\)である。
			\end{penumerate}
			\item \(s'_1 D_{M_{1,j_{-2}}} t_2 b'_1 \in T_{\textrm{B}}\)かつ\(s'_1 D_{M_{1,j_{-2}}} t_2 b'_1 > t_2\)であることを示す。
			\item \(M\)が条件(III)を満たすならば、\(D_{M_{1,j_{-1}}} s'_1 D_{M_{1,j_1}} 0 b'_1 = c_2 = D_v(t_2+D_{M_{1,j_1}} 0)\)より\(s'_1 D_{M_{1,j_1}} 0 b'_1 = t_2+D_{M_{1,j_1}} 0 \in T_{\textrm{B}}\)であり、従って\(s'_1 D_{M_{1,j_1}} 0 b'_1 > t_2\)である。
			\item \(M\)が条件(IV)を満たすとする。
			\begin{indented}
				\item \(t_2\)の右端単項成分の左端が\(D_{M_{1,j_0}}\)であるならば、\(t_2 = t_3 + D_{M_{1,j_0}} t_4 < t_3+D_{M_{1,j_0}}(t_4+D_{M_{1,j_{-2}}} 0)\)である。
				\item \(t_2\)の右端単項成分の左端が\(D_{M_{1,j_0}}\)でないならば、\(t_2 = t_3 < t_3+D_{M_{1,j_0}}(t_4+D_{M_{1,j_{-2}}} 0)\)である。
				\item 従っていずれの場合も\(t_2 < t_3+D_{M_{1,j_0}}(t_4+D_{M_{1,j_{-2}}} 0)\)である。
				\item \(D_{M_{1,j_{-1}}} s'_1 D_{M_{1,j_1}} 0 b'_1 = c_2 = D_v(t_3+D_{M_{1,j_0}}(t_4+D_{M_{1,j_1}} 0))\)より\(s'_1 D_{M_{1,j_1}} 0 b'_1 = t_3+D_{M_{1,j_0}}(t_4+D_{M_{1,j_1}} 0) \in T_{\textrm{B}}\)である。従って\nameref{加法とscb分解の関係}より\(s'_1 D_{M_{1,j_{-2}}} 0 b'_1 = t_3+D_{M_{1,j_0}}(t_4+D_{M_{1,j_{-2}}} 0) \in T_{\textrm{B}}\)である。
				\item また\nameref{条件(II)か(IV)の下でt_2が0でないこと}より\(t_2 > 0\)であるので\(D_{M_{1,j_{-2}}} t_2 > D_{M_{1,j_{-2}}} 0\)である。従って\nameref{部分表現の不等式の延長性}より
				\begin{eqnarray*}
				s'_1 D_{M_{1,j_{-2}}} t_2 b'_1 > s'_1 D_{M_{1,j_{-2}}} 0 b'_1 = t_3+D_{M_{1,j_0}}(t_4+D_{M_{1,j_{-2}}} 0) > t_2
				\end{eqnarray*}
				\item である。
			\end{indented}
			\item
			\item \(n = 1\)とする。
			\begin{indented}
				\item \(s' := s_1\)と置く。
				\item \(c'_1 := c_1\)と置く。
				\item \(c'_2 := c_2\)と置く。
				\item \nameref{c_1とc_2の大小関係}より\(c'_1\)と\(c'_2\)は単項でありかつ\(c'_1 < c'_2\)である。
			\end{indented}
			\item \(n > 1\)とする。
			\begin{indented}
				\item \(s' := s_1 D_{M_{1,j_{-1}}} (s'_1 D_{M_{1,j_{-2}}})^{n-2} s'_1\)と置く。
				\item \(c'_1 := D_{M_{1,j_{-2}}} t_2\)と置く。
				\item \(c'_2 := D_{M_{1,j_{-2}}} s'_1 D_{M_{1,j_{-2}}} t_2 b'_1\)と置く。
				\item \(s'_1 D_{M_{1,j_{-2}}} t_2 b'_1 \in T_{\textrm{B}}\)であるので、定義から\(c'_1\)と\(c'_2\)は単項である。\(t_2 < s'_1 D_{M_{1,j_{-2}}} t_2 b'_1\)より、\(c'_1 = D_{M_{1,j_{-1}}} t_2 < D_{M_{1,j_{-1}}} s'_1 D_{M_{1,j_{-2}}} t_2 b'_1 = c'_2\)である。
			\end{indented}
			\item \(b' := (b'_1)^{n-1} b_1\)と置く。
			\item 定義から\((s',c'_1,b')\)と\((s',c'_2,b)\)はそれぞれ\(\textrm{Trans}(M[n])\)と\(\textrm{Trans}(L_n)\)のscb分解である。
			\item \nameref{MarkのTransによる表示}より\(\textrm{Trans}((M_j)_{j=j_{-1}}^{j_1}) = c_2 = D_{M_{1,j_{-1}}} s'_1 D_{M_{1,j_1}} 0 b'_1\)であるので、\nameref{条件(III)~(VI)の下でのTransとscb分解の関係}より\((s_1,D_{M_{1,j_{-1}}} s'_1 D_{M_{1,j_1}} 0 b'_1,b_1)\)は\(\textrm{Trans}(M)\)の第\(1\)種scb分解である。従って\nameref{scb分解と基本列の関係} (2)より
			\begin{eqnarray*}
			\textrm{Trans}(M)[n-1] & = & s_1 D_{M_{1,j_{-1}}} (s'_1 D_{M_{1,j_1}-1})^n 0 (b'_1)^n b_1 = s_1 D_{M_{1,j_{-1}}} (s'_1 D_{M_{1,j_{-2}}})^n 0 (b'_1)^n b_1 \\
			& = & \textrm{Trans}(L_n) = \textrm{Trans}(M[n+1][1]^{j_1-1-j_{-2}})
			\end{eqnarray*}
			\item である。従って\((s',c'_2,b')\)は\(\textrm{Trans}(M)[n-1]\)のscb分解である。
		\end{indented}
		\item
		\item \(j_{-3} < j_{-1}\)とする。
		\begin{penumerate}
			\item[] \nameref{条件(III)か(IV)の下での各種scb分解}より、ある\((s'_0,s'_1,s'_2,b'_2,b'_1,b'_0) \in (\Sigma^{< \omega})^6\)が存在して以下を満たす:
			\item \((s'_0,\textrm{Trans}((M_j)_{j=j_{-3}}^{j_1}),b'_0)\)は\(\textrm{Trans}(M)\)のscb分解である。
			\item \((D_{M_{1,j_{-3}}} s'_1,c_2,b'_1)\)は\(\textrm{Trans}((M_j)_{j=j_{-3}}^{j_1})\)のscb分解である。
			\item \((s'_2,D_{M_{1,j_1}} 0,b'_2)\)は\(c_2\)のscb分解である。
			\setcounter{penumeratei}{4}
			\item \(\textrm{Trans}(L_n) = s'_0 D_{M_{1,j_{-3}}} (s'_1 s'_2 D_{M_{1,j_{-2}}})^n 0 (b'_2 b'_1)^n b'_0\)である。
			\item \(\textrm{Trans}(M[n]) = s'_0 D_{M_{1,j_{-3}}} (s'_1 s'_2 D_{M_{1,j_{-2}}})^{n-1} s'_1 c_1 b'_1 (b'_2 b'_1)^{n-1} b'_0\)である。
			\item[] \(s' := s'_0 D_{M_{1,j_{-3}}} (s'_1 s'_2 D_{M_{1,j_{-2}}})^{n-1} s'_1\)と置く。
			\item[] \(c'_1 := c_1\)と置く。
			\item[] \(c'_2 := c_2\)と置く。
			\item[] \(b' := (b'_2 b'_1)^{n-1} b'_0\)と置く。
			\item[] \nameref{c_1とc_2の大小関係}より\(c'_1\)と\(c'_2\)は単項でありかつ\(c'_1 < c'_2\)である。
			\item[] 定義から\((s',c'_1,b')\)と\((s',c'_2,b')\)はそれぞれ\(\textrm{Trans}(M[n])\)と\(\textrm{Trans}(L_n)\)のscb分解である。
			\item[] \nameref{条件(III)~(VI)の下でのTransとscb分解の関係}より\((s'_0,\textrm{Trans}((M_j)_{j=j_{-3}}^{j_1}),b'_0) = (s'_0,D_{M_{1,j_{-3}}} s'_1 s'_2 D_{M_{1,j_1}} 0 b'_2 b'_1,b'_0)\)は\(\textrm{Trans}(M)\)の第\(1\)種scb分解である。従って\nameref{scb分解と基本列の関係} (2)より
			\begin{eqnarray*}
			\textrm{Trans}(M)[n-1] & = & s'_0 D_{M_{1,j_{-3}}} (s'_1 s'_2 D_{M_{1,j_1}-1})^n 0 (b'_2 b'_1)^n b'_0 = s'_0 D_{M_{1,j_{-3}}} (s'_1 s'_2 D_{M_{1,j_{-2}}})^n 0 (b'_2 b'_1)^n b'_0\\
			& = & \textrm{Trans}(L_n) = \textrm{Trans}(M[n+1][1]^{j_1-1-j_{-2}})
			\end{eqnarray*}
			\item[] である。従って\((s',c'_2,b')\)は\(\textrm{Trans}(M)[n-1]\)のscb分解である。\qedhere\NoEndMark
		\end{penumerate}
	\end{indented}
\end{hideableproof}

\iffull{それでは本題に戻る。}\fi

\begin{hideableproof}[\nameref{条件(III)か(IV)の下でのTransと基本列の交換関係}の証明]
	\begin{penumerate}
		\item \(\textrm{Pred}(M[n+1][1]^{j_1-1-j_{-2}}) = M[n+1][1]^{j_1-j_{-2}}\)より、\nameref{PredのTransに関する降下性}と\nameref{条件(III)か(IV)の下での基本列の基本性質} (1)と(2)から従う。
		\item \(M\)は単項であるので\nameref{Transが零項性を保つこと}から\(\textrm{Trans}(M) \neq 0\)である。(1)と\(\textrm{Trans}(M) \neq 0\)と\cite{buc1} Lemma 3.2より即座に従う。
		\item \nameref{Predが1で表されること}と\nameref{PredのTransに関する降下性}と\nameref{条件(III)か(IV)の下での基本列の基本性質} (2) から従う。
	\end{penumerate}
\end{hideableproof}


\subsection{条件(V)の下での展開規則}

\begin{proposition}[条件(V)の下での\(\textrm{Trans}\)と基本列の交換関係]\label{条件(V)の下でのTransと基本列の交換関係}
	任意の\(M \in ST_{\textrm{PS}} \cap PT_{\textrm{PS}}\)と\(n \in \mathbb{N}_{+}\)に対し、\(\textrm{Trans}\)の再帰的定義中に導入した記号を用い、\(j_0\)が\(M\)許容ならば\(m_n := n-1\)と置き、\(j_0\)が非\(M\)許容ならば\(m_n := n\)と置くと、\(j_1 > 1\)かつ\(M\)が条件(V)を満たすならば\footnotemark{}、以下が成り立つ:
	\begin{penumerate}
		\item \(\textrm{Trans}(M[n]) \leq \textrm{Trans}(M)[m_n]\)である。
		\item \(\textrm{Trans}(M[n]) < \textrm{Trans}(M)\)である。
		\item \(\textrm{Trans}(M)[m_n] \leq \textrm{Trans}(M[n+1])\)である。
	\end{penumerate}
\end{proposition}
\footnotetext{\nameref{標準形の簡約性}から\(M\)は簡約であり、\(j_1 > 1\)より\(t_1 \neq 0\)であるので\(M\)に対し条件(V)が意味を持つ。}

\nameref{条件(V)の下でのTransと基本列の交換関係}を証明するための準備としていくつかの補題を示す。

\begin{lemma}[条件(V)の下での\(\textrm{Joints}\)と\(\textrm{FirstNodes}\)と\(t_2\)の基本性質]\label{条件(V)の下でのJointsとFirstNodesとt_2との基本性質}
	任意の\(M \in ST_{\textrm{PS}} \cap PT_{\textrm{PS}}\)に対し、\(\textrm{Trans}\)の再帰的定義中に導入した記号を用い、\(N := (M_j)_{j=j_{-1}}^{j_1}\)と置き、\(J_1 := \textrm{Lng}(\textrm{Br}(\textrm{Red}(N)))-1\)と置くと、\((1,j_0) <_M^{\textrm{Next}} (1,j_1)\)かつ\(j_0\)が非\(M\)許容でありかつ\(j_0 < j_1-1\)ならば以下が成り立つ:
	\begin{penumerate}
		\item \(J_1 \geq 0\)かつ\(j_0-j_{-1} = \textrm{Joints}(\textrm{Red}(N))_{J_1}\)かつ\(\textrm{FirstNodes}(\textrm{Red}(N))_{J_1} = j_1-j_{-1}\)である。
		\item \(\textrm{Red}(N)_{0,j_1-j_{-1}} = \textrm{Red}(N)_{1,j_1-j_{-1}}\)である。
		\item \(t_2\)の各単項成分は\(D_{M_{1,j_1}} 0\)以上である。
	\end{penumerate}
\end{lemma}

\begin{hideableproof}
	\begin{indented}
		\item \nameref{標準形の簡約性}より\(M\)は簡約である。\(j_0 < j_1-1\)より\(j_1 > j_0+1 \geq 1\)であるので\(\textrm{Pred}(M)\)は零項でない。特に\nameref{Transが零項性を保つこと}から\(t_1 \neq 0\)である。
		\item 従って\(M\)に対し条件(I)~(VI)が意味を持つ。
		\item \nameref{簡約性と係数の関係}より\(M\)は条件(A)と(B)を満たす。\(M\)が条件(A)を満たしかつ\((0,j_0) <_M^{\textrm{Next}} (0,j_1)\)かつ\((1,j_0) <_M^{\textrm{Next}} (1,j_1)\)であることから\(M_{0,j_0} = M_{0,j_1}-1\)かつ\(M_{1,j_0} = M_{1,j_1}-1\)である。従って\(M\)は条件(V)を満たす。
		\item \(N := (M_j)_{j=j_{-1}}^{j_1}\)と置く。
		\item \nameref{標準形の直系先祖による切片の簡約化の強単項性}より\(\textrm{Red}(N)\)は強単項である。
		\item \nameref{Transの(IncrFirst,Red)不変P同変性}と\nameref{右端第2基点のMarkの基本性質}と\nameref{MarkのTransによる表示}から
		\begin{eqnarray*}
		D_v(t_2 + D_{M_{1,j_1}} 0) = c_2 = \textrm{Mark}(M,j_{-1}) = \textrm{Trans}(N) = \textrm{Trans}(\textrm{Red}(N))
		\end{eqnarray*}
		\item である。
		\item \(J_1 := \textrm{Lng}(\textrm{Br}(\textrm{Red}(N)))-1\)と置く。
		\item \((1,j_0) <_M^{\textrm{Next}} (1,j_1)\)かつ\(j_0 < j_1-1\)であるので\((1,j_1-1) <_M (1,j_1)\)でない。従って\nameref{右端の非許容直系先祖の基本性質}より\(J_1 \geq 0\)かつ\(0 < j_0-j_{-1} < \textrm{TrMax}(\textrm{Br}(\textrm{Red}(N)))\)かつ\(j_0-j_{-1} = \textrm{Joints}(\textrm{Red}(N))_{J_1}\)かつ\(\textrm{FirstNodes}(\textrm{Red}(N)) = j_1-j_{-1}\)である。
		\item 任意の\(i \in {0,1}\)に対し、\((i,j_0) <_M^{\textrm{Next}} (i,j_1)\)より\((i,j_0-j_{-1}) <_N^{\textrm{Next}} (i,j_1-j_{-1})\)であり、\nameref{直系先祖のRed不変性}から\((i,j_0-j_{-1}) <_{\textrm{Red}(N)}^{\textrm{Next}} (i,j_1-j_{-1})\)である。
		\item \nameref{簡約性と係数の関係}より\(\textrm{Red}(N)\)は条件(A)と(B)を満たす。\(\textrm{Red}(N)\)が条件(A)と(B)を満たしかつ\(j_0-j_{-1} < \textrm{TrMax}(\textrm{Br}(\textrm{Red}(N)))\)であることから\(\textrm{Red}(N)_{0,j_0-j_{-1}} = \textrm{Red}(N)_{1,j_0-j_{-1}}\)であり、\(\textrm{Red}(N)\)が条件(A)を満たしかつ\((0,j_0-j_{-1}) <_{\textrm{Red}(N)}^{\textrm{Next}} (0,j_1-j_{-1})\)かつ\((1,j_0-j_{-1}) <_{\textrm{Red}(N)}^{\textrm{Next}} (1,j_1-j_{-1})\)であることから、
		\begin{eqnarray*}
		\textrm{Red}(N)_{0,j_1-j_{-1}} = \textrm{Red}(N)_{0,j_0-j_{-1}}+1 = \textrm{Red}(N)_{1,j_0-j_{-1}}+1 = \textrm{Red}(N)_{1,j_1-j_{-1}}
		\end{eqnarray*}
		\item である。
		\item \(0 < j_0-j_{-1} < \textrm{TrMax}(\textrm{Br}(\textrm{Red}(N)))\)かつ\(\textrm{Red}(N)_{0,j_1-j_{-1}} = \textrm{Red}(N)_{1,j_1-j_{-1}}\)であるので、\(\textrm{Trans}(\textrm{Red}(N)) = D_v(t_2 + D_{M_{1,j_1}} 0)\)と\nameref{強単項性の下での部分表現の単項成分の基本性質}より\(t_2 + D_{M_{1,j_1}} 0\)の各単項成分は\(D_{\textrm{Red}(N)_{1,j_1-j_{-1}}} 0\)以上である。
		\item \nameref{直系先祖による切片とRedとIncrFirstの関係}より\(\textrm{Red}(N)_{1,j_1} = M_{1,j_1}\)であるので、\(t_2\)の各単項成分は\(D_{M_{1,j_1}} 0\)以上である。
	\end{indented}
\end{hideableproof}

\begin{lemma}[条件(V)の下での各種scb分解]\label{条件(V)の下での各種scb分解}
	任意の\(M \in ST_{\textrm{PS}} \cap PT_{\textrm{PS}}\)と\(n \in \mathbb{N}_{+}\)に対し、\(\textrm{Trans}\)の再帰的定義中に導入した記号を用い、\(N' := (M_j)_{j=j_0}^{j_1}\)と置き、\(L' := (M_j)_{j=j_0}^{j_1-1} \oplus_{\mathbb{N}^2} ((M_{0,j_1},M_{1,j_0}))\)と置き、\(L_n := M[n] \oplus_{\mathbb{N}^2} ((M_{0,j_0}+n(M_{0,j_1}-M_{0,j_0}),M_{1,j_0}))\)と置くと\footnotemark{}、\(j_1 > 1\)かつ\(M\)が条件(V)を満たし\footnotemark{}かつ\(j_0\)が非\(M\)許容であるならば、一意な\((s'_1,b'_1) \in (\Sigma^{< \omega})^2\)が存在して以下を満たす:
	\begin{penumerate}
		\item \((D_{M_{1,j_{-1}}} s'_1,D_{M_{1,j_1}} 0,b'_1)\)は\(c_2\)のscb分解である。
		\item \((D_{M_{1,j_0}} s'_1,D_{M_{1,j_1}} 0,b'_1)\)と\((D_{M_{1,j_0}} s'_1,D_{M_{1,j_0}} 0,b'_1)\)はそれぞれ\(\textrm{Trans}(N')\)と\(\textrm{Trans}(L')\)のscb分解である。
		\item \(\textrm{Trans}(\textrm{Pred}(N')) = D_{M_{1,j_0}} t_2\)である。
		\item \(\textrm{Trans}(L_n) = s_1 D_{M_{1,j_{-1}}} (s'_1 D_{M_{1,j_0}})^{n+1} 0 (b'_1)^{n+1} b_1\)である。
		\item \(\textrm{Trans}(M[n]) = s_1 D_{M_{1,j_{-1}}} (s'_1 D_{M_{1,j_0}})^n t_2 (b'_1)^n b_1\)である。
	\end{penumerate}
\end{lemma}
\addtocounter{footnote}{-1}
\footnotetext{\((0,j_0) <_M^{\textrm{Next}} (0,j_1)\)より\((0,j_0) \leq_M (0,j_1)\)なので\(M_{0,j_1}-M_{0,j_0} > 0\)であり、\(L_n\)の各成分は自然数となる。}
\addtocounter{footnote}{1}
\footnotetext{\nameref{標準形の簡約性}から\(M\)は簡約であり、\(j_1 > 1\)より\(t_1 \neq 0\)であるので\(M\)に対し条件(V)が意味を持つ。}

\begin{hideableproof}
	\begin{indented}
		\item 以下では\nameref{条件(III)~(VI)の下での展開規則の基本性質}を断りなく用いる。
		\item \(M\)が条件(V)を満たすので\(M_{1,j_0} < M_{1,j_0}+1 = M_{1,j_1}\)である。従って\((1,j_0) <_M^{\textrm{Next}} (1,j_1)\)である。
		\item \nameref{条件(III)~(V)の下での切片のscb分解}より一意な\((s'_1,b'_1) \in (\Sigma^{< \omega})^2\)が存在して以下を満たす:
		\begin{penumerate}
			\item \((D_{M_{1,j_{-1}}} s'_1,D_{M_{1,j_1}} 0,b'_1)\)は\(c_2\)のscb分解である。
			\item \((D_{M_{1,j_0}} s'_1,D_{M_{1,j_1}} 0,b'_1)\)は\(\textrm{Trans}(N')\)のscb分解である。
			\item \(\textrm{Trans}(\textrm{Pred}(N')) = D_{M_{1,j_0}} t_2\)である。
		\end{penumerate}
		\item 以上より(1)と(3)が従う。
		\item
		\item (2)
		\item \(N := (M_j)_{j=j_{-1}}^{j_1}\)と置く。
		\item \nameref{標準形の直系先祖による切片の簡約化の強単項性}より\(\textrm{Red}(N)\)は強単項であるので、\(\textrm{Br}(\textrm{Red}(N))\)は降順である。
		\item \nameref{右端第2基点のMarkの基本性質}と\nameref{MarkのTransによる表示}から\(\textrm{Trans}(N) = \textrm{Mark}(M,j_{-1}) = c_2\)であるので、\((D_{M_{1,j_{-1}}} s'_1,D_{M_{1,j_1}} 0,b'_1)\)は\(\textrm{Trans}(N)\)のscb分解である。
		\item \(J_1 := \textrm{Lng}(\textrm{Br}(\textrm{Red}(N)))-1\)と置く。
		\item \(M\)が条件(V)を満たすことから\(j_0+1 < j_1\)すなわち\(j_0 < j_1-1\)である。従って\nameref{条件(V)の下でのJointsとFirstNodesとt_2との基本性質}より以下が成り立つ:
		\begin{penumerate}
			\item \(J_1 \geq 0\)かつ\(j_0-j_{-1} = \textrm{Joints}(\textrm{Red}(N))_{J_1}\)かつ\(\textrm{FirstNodes}(\textrm{Red}(N))_{J_1} = j_1-j_{-1}\)である。
			\item \(\textrm{Red}(N)_{0,j_1-j_{-1}} = \textrm{Red}(N)_{1,j_1-j_{-1}}\)である。
		\end{penumerate}
		\item \(\textrm{Red}(N)_{0,j_1-j_{-1}} = \textrm{Red}(N)_{1,j_1-j_{-1}}\)かつ\(\textrm{Br}(\textrm{Red}(N))\)が降順であるので、\((D_{M_{1,j_{-1}}} s'_1,D_{M_{1,j_1}} 0,b'_1)\)は\(\textrm{Trans}(N)\)のscb分解であることと\nameref{条件(V)の下での終切片とTransの関係}より\((D_{M_{1,j_0}} s'_1,D_{M_{1,j_1}} 0,b'_1)\)は\(\textrm{Trans}(N')\)のscb分解である。
		\item \((1,j_0) <_M^{\textrm{Next}} (1,j_1)\)かつ\(L'_{j_1-j_0} = (M_{0,j_1},M_{1,j_0}) = (N'_{0,j_1-j_0},N'_{1,0})\)であることから、\((D_{M_{1,j_0}} s'_1,D_{M_{1,j_1}} 0,b'_1)\)が\(\textrm{Trans}(N')\)のscb分解であることと\nameref{条件(III)~(V)の下での右端の置き換えとTransの関係}より\((D_{M_{1,j_0}} s'_1,D_{M_{1,j_0}} 0,b'_1)\)は\(\textrm{Trans}(L')\)のscb分解である。
		\item
		\item (4)
		\item \nameref{標準形の簡約性}より\(M\)は簡約である。\nameref{簡約性と係数の関係}より\(M\)は条件(A)と(B)を満たす。\(M\)が条件(A)を満たしかつ\((0,j_0) <_M^{\textrm{Next}} (0,j_1)\)かつ\((1,j_0) <_M^{\textrm{Next}} (1,j_1)\)であることから\(M_{0,j_0} = M_{0,j_1}-1\)かつ\(M_{1,j_0} = M_{1,j_1}-1\)である。
		\item \(\textrm{Trans}(L_n) = s_1 D_{M_{1,j_{-1}}} (s'_1 D_{M_{1,j_0}})^{2n} 0 (b'_1)^{2n} b_1\)であることを\(n\)に関する数学的帰納法で示す。
		\item \(n = 1\)とする。
		\begin{indented}
			\item \(\textrm{Trans}\)の再帰的定義中に導入した記号を\(L_1\)に対しても定め、\(L_1\)に対する適用であることを明示するために右肩に\(L_1\)を乗せて表記する。
			\item \(L_1\)の定義から\(\leq_{L_1}\)と\(\leq_M\)の\((\{0,1\} \times \mathbb{N}) \setminus \{(1,j_1)\}\)への制限は一致する。\(j_1^{J_1} = j_1\)であり、\((0,j_0) <_M^{\textrm{Next}} (0,j_1)\)より\((0,j_0) <_{L_1}^{\textrm{Next}} (0,j_1) = (0,j_1^{L_1})\)であるので、\(j_0^{L_1} = j_0\)かつ\((L_1)_{1,j_0^{L_1}} = (L_1)_{1,j_0} = M_{1,j_0} = (L_1)_{1,j_1} = (L_1)_{1,j_1^{L_1}}\)である。
			\item \(j_{-1}^{L_1} = j_{-1}\)であり、\(j_0\)が非\(M\)許容であることから\(j_0^{L_1} = j_0 > j_{-1} = j_{-1}^{L_1}\)となるので、\(j_0^{L_1}\)は非\(L_1\)許容である。
			\item \(\textrm{Pred}(L_1) = \textrm{Pred}(M)\)より\(t_1^{L_1} = t_1 \neq 0\)かつ\(c_1^{L_1} = c_1\)かつ\(t_2^{L_1} = t_2\)かつ\(s_1^{L_1} = s_1\)かつ\(b_1^{L_1} = b_1\)である。\(t_1^{L_1} \neq 0\)より\(L_1\)に対して条件(I)~(VI)が意味を持つ。
			\item \(M\)が条件(VI)を満たさずかつ\(j_0\)が非\(M\)許容であることから、\(L_1\)は条件(II)か(IV)を満たす。
			\item \nameref{Markの左端の基本性質}より\(v^{L_1} = (L_1)_{1,j_{-1}^{L_1}} = M_{1,j_{-1}}\)である。
			\item \((1,j_0) <_M^{\textrm{Next}} (1,j_1)\)かつ\(j_0\)が非\(M\)許容かつ\(j_0 < j_1-1\)より、\nameref{条件(V)の下でのJointsとFirstNodesとt_2との基本性質}から\(t_2\)の各単項成分は\(D_{M_{1,j_1}} 0\)以上である。更に\((L_1)_{1,j_0^{L_1}} = M_{1,j_0} = M_{1,j_1}-1 < M_{1,j_1}\)より、\(t_2^{L_1} = t_2\)の右端単項成分の左端は\(D_{(L_1)_{1,j_0^{L_1}}}\)でない。
			\item 従って\(t_3^{L_1} = t_2^{L_1} = t_2\)かつ\(t_4^{L_1} = t_2^{L_1} = t_2\)であり、
			\begin{eqnarray*}
			c_2^{L_1} = D_{v^{L_1}}(t_3^{L_1} + D_{(L_1)_{1,j_0^{L_1}}}(t_4^{L_1} + D_{(L_1)_{1,j_1^{L_1}}} 0)) = D_{M_{1,j_{-1}}}(t_2 + D_{M_{1,j_0}}(t_2 + D_{M_{1,j_0}} 0))
			\end{eqnarray*}
			\item \((D_{M_{1,j_{-1}}} s'_1,D_{M_{1,j_1}} 0,b'_1)\)が\(c_2= D_v(t_2 + D_{M_{1,j_1}} 0)\)のscb分解であることから\((D_{M_{1,j_0}} s'_1,D_{M_{1,j_1}} 0,b'_1)\)は\(D_{M_{1,j_0}}(t_2 + D_{M_{1,j_1}} 0)\)のscb分解であり、\nameref{加法とscb分解の関係}より\((D_{M_{1,j_0}} s'_1,D_{M_{1,j_0}} 0,b'_1)\)は\(D_{M_{1,j_0}}(t_2 + D_{M_{1,j_0}} 0)\)のscb分解であり、\((D_{M_{1,j_{-1}}} s'_1,D_{M_{1,j_0}} s'_1 D_{M_{1,j_0}} 0 b'_1,b'_1)\)は\(c_2^{L_1} = D_{M_{1,j_{-1}}}(t_2 + D_{M_{1,j_0}}(t_2 + D_{M_{1,j_0}} 0))\)のscb分解である。
			\item 以上より
			\begin{eqnarray*}
			\textrm{Trans}(L_n) = \textrm{Trans}(L_1) = s_1^{L_1} c_2^{L_1} b_1^{L_1} = s_1 D_{M_{1,j_{-1}}} s'_1 D_{M_{1,j_0}} s'_1 D_{M_{1,j_0}} 0 b'_1 b'_1 b_1 \\
			& = & s_1 D_{M_{1,j_{-1}}} (s'_1 D_{M_{1,j_0}})^{n+1} 0 (b'_1)^{n+1} b_1
			\end{eqnarray*}
			\item である。
		\end{indented}
		\item
		\item \(n > 1\)とする。
		\begin{indented}
			\item 帰納法の仮定から
			\begin{eqnarray*}
			\textrm{Trans}(L_{n-1}) & = & s_1 D_{M_{1,j_{-1}}} (s'_1 D_{M_{1,j_0}})^n 0 (b'_1)^n b_1 \\
			& = & (s_1 D_{M_{1,j_{-1}}} (s'_1 D_{M_{1,j_0}})^{n-1} s'_1) D_{M_{1,j_0}} 0 ((b'_1)^n b_1)
			\end{eqnarray*}
			\item であるので、
			\begin{eqnarray*}
			\textrm{Trans}(L_n) & = & (s_1 D_{M_{1,j_{-1}}} (s'_1 D_{M_{1,j_0}})^{n-1} s'_1) \textrm{Trans}(L') ((b'_1)^n b_1) \\
			& = & s_1 D_{M_{1,j_{-1}}} (s'_1 D_{M_{1,j_0}})^{n-1} s'_1 D_{M_{1,j_0}} s'_1 D_{M_{1,j_0}} 0 b'_1 (b'_1)^n b_1 \\
			& = & s_1 D_{M_{1,j_{-1}}} (s'_1 D_{M_{1,j_0}})^{n+1} 0 (b'_1)^{n+1} b_1
			\end{eqnarray*}
			\item である。
		\end{indented}
		\item
		\item (5)
		\item \(n=1\)とする。
		\begin{indented}
			\item \(M[n] = \textrm{Pred}(M)\)より
			\begin{eqnarray*}
			\textrm{Trans}(M[n]) & = & t_1 = s_1 c_1 b_2 = s_1 D_{M_{1,j_{-1}}} t_2 b_1 \\
			& = & s_1 D_{M_{1,j_{-1}}} (s'_1 D_{M_{1,j_0}})^{2n-2} t_2 (b'_1)^{2n-2} b_1
			\end{eqnarray*}
			\item である。
		\end{indented}
		\item
		\item \(n > 1\)とする。
		\begin{eqnarray*}
		\textrm{Trans}(L_n) & = & s_1 D_{M_{1,j_{-1}}} (s'_1 D_{M_{1,j_0}})^{n+1} 0 (b'_1)^{n+1} b_1 \\
		& = & (s_1 D_{M_{1,j_{-1}}} (s'_1 D_{M_{1,j_0}})^{n-1} s'_1) D_{M_{1,j_0}} s'_1 D_{M_{1,j_0}} 0 b'_1 ((b'_1)^n b_1) \\
		& = & (s_1 D_{M_{1,j_{-1}}} (s'_1 D_{M_{1,j_0}})^{n-1} s'_1) \textrm{Trans}(L') ((b'_1)^n b_1)
		\end{eqnarray*}
		\begin{indented}
			\item より
			\begin{eqnarray*}
			\textrm{Trans}(M[n]) & = & (s_1 D_{M_{1,j_{-1}}} (s'_1 D_{M_{1,j_0}})^{n-1} s'_1) \textrm{Trans}(\textrm{Pred}(N')) ((b'_1)^n b_1) \\
			& = & (s_1 D_{M_{1,j_{-1}}} (s'_1 D_{M_{1,j_0}})^{n-1} s'_1) D_{M_{1,j_0}} t_2 ((b'_1)^n b_1) \\
			& = & s_1 D_{M_{1,j_{-1}}} (s'_1 D_{M_{1,j_0}})^n t_2 (b'_1)^n b_1
			\end{eqnarray*}
			\item である。
		\end{indented}
	\end{indented}
\end{hideableproof}

\begin{lemma}[条件(V)の下での基本列のscb分解]\label{条件(V)の下での基本列のscb分解}
	任意の\(M \in ST_{\textrm{PS}} \cap PT_{\textrm{PS}}\)と\(n \in \mathbb{N}_{+}\)に対し、\(\textrm{Trans}\)の再帰的定義中に導入した記号を用い、\(j_0\)が\(M\)許容ならば\(m_n := n-1\)と置き、\(j_0\)が非\(M\)許容ならば\(m_n := n\)と置くと、\(j_1 > 1\)かつ\(M\)が条件(V)を満たすならば\footnotemark{}、一意な\(u \in \mathbb{N}\)と\((s'_0,b'_0) \in (\Sigma^{< \omega})^2\)と\(t' \in T_{\textrm{B}}\)が存在して以下を満たす:
	\begin{penumerate}
		\item \((s'_0,D_u t_2,b'_0)\)は\(\textrm{Trans}(M[n])\)のscb分解である。
		\item \((s'_0,D_u(t_2 + D_{M_{1,j_0}} 0),b'_0)\)は\(\textrm{Trans}(M)[m_n]\)のscb分解である。
		\item \((s'_0,D_u(t_2 + D_{M_{1,j_0}} t'),b'_0)\)は\(\textrm{Trans}(M[n+1])\)のscb分解である。
	\end{penumerate}
\end{lemma}
\footnotetext{\nameref{標準形の簡約性}から\(M\)は簡約であり、\(j_1 > 1\)より\(t_1 \neq 0\)であるので\(M\)に対し条件(V)が意味を持つ。}

\begin{hideableproof}
	\begin{indented}
		\item \(M\)が条件(V)を満たすので\(M_{1,j_0} < M_{1,j_0}+1 = M_{1,j_1}\)である。従って\((1,j_0) <_M^{\textrm{Next}} (1,j_1)\)である。
		\item 任意の\((s'_1,b'_1) \in (\Sigma^{< \omega})^2\)に対し、\((D_{M_{1,j_{-1}}} s'_1,D_{M_{1,j_1}} 0,b'_1)\)が\(c_2\)のscb分解であるならば\(\textrm{Trans}(M)[m_n] = s_1 D_{M_{1,j_{-1}}} (s'_1 D_{M_{1,j_0}})^{m_n+1} 0 (b'_1)^{m_n+1} b_1\)かつ任意の\(t' \in T_{\textrm{B}}\)に対し\(s'_1 D_{M_{1,j_0}} t' b'_1 = t_2 + D_{M_{1,j_0}} t'\)であることを示す。
		\item \nameref{条件(III)~(VI)の下でのTransとscb分解の関係}より\((s_1,c_2,b_1) = (s_1,D_{M_{1,j_{-1}}} s'_1 D_{M_{1,j_1}} 0 b'_1,b_1)\)は\(\textrm{Trans}(M)\)の第\(1\)種scb分解であり、\nameref{scb分解と基本列の関係} (2)より
		\begin{eqnarray*}
		\textrm{Trans}(M)[m_n] = s_1 D_{M_{1,j_{-1}}} (s'_1 D_{M_{1,j_1}}-1)^{m_n+1} 0 (b'_1)^{m_n+1} b_1 = s_1 D_{M_{1,j_{-1}}} (s'_1 D_{M_{1,j_0}})^{m_n+1} 0 (b'_1)^{m_n+1} b_1
		\end{eqnarray*}
		\item である。\(D_{M_{1,j_{-1}}} s'_1 D_{M_{1,j_1}} 0 b'_1 = c_2 = D_v(t_2 + D_{M_{1,j_1}} 0)\)より\(s'_1 D_{M_{1,j_1}} 0 b'_1 = t_2 + D_{M_{1,j_1}} 0\)である。従って\nameref{加法とscb分解の関係}より\(t_2 + D_{M_{1,j_0}} t' = s'_1 D_{M_{1,j_0}} t' b'_1\)である。
		\item
		\item \(j_0\)が\(M\)許容であるとする。
		\begin{indented}
			\item \nameref{条件(III)~(V)の下での各種scb分解}より、一意な\((s'_1,b'_1) \in (\Sigma^{< \omega})^2\)が存在して以下を満たす:
			\begin{penumerate}
				\item \((D_{M_{1,j_{-1}}} s'_1,D_{M_{1,j_1}} 0,b'_1)\)は\(c_2\)のscb分解である。
				\setcounter{penumeratei}{4}
				\item \(\textrm{Trans}(M[n]) = s_1 D_{M_{1,j_{-1}}} (s'_1 D_{M_{1,j_0}})^{n-1} t_2 (b'_1)^{n-1} b_1\)である。
				\item[] (5)' \(\textrm{Trans}(M[n+1]) = s_1 D_{M_{1,j_{-1}}} (s'_1 D_{M_{1,j_0}})^n t_2 (b'_1)^n b_1\)である。
			\end{penumerate}
			\item \(n = 1\)とする。
			\begin{indented}
				\item \(u := M_{1,j_{-1}}\)と置く。
				\item \(s'_0 := s_1\)と置く。
			\end{indented}
			\item \(n > 1\)とする。
			\begin{indented}
				\item \(u := M_{1,j_0}\)と置く。
				\item \(s'_0 := s_1 D_{M_{1,j_{-1}}} (s'_1 D_{M_{1,j_0}})^{n-2} s'_1\)と置く。
			\end{indented}
			\item \(b'_0 := (b'_1)^{n-1} b_1\)と置く。
			\item \(t' := t_2\)と置く。
			\item 既に示したように、\(\textrm{Trans}(M)[m_n] = s_1 D_{M_{1,j_{-1}}} (s'_1 D_{M_{1,j_0}})^{m_n+1} 0 (b'_1)^{m_n+1} b_1\)かつ\(s'_1 D_{M_{1,j_0}} 0 b'_1 = t_2 + D_{M_{1,j_0}} 0\)かつ\(s'_1 D_{M_{1,j_0}} t' b'_1 = t_2 + D_{M_{1,j_0}} t'\)である。従って
			\begin{eqnarray*}
			\textrm{Trans}(M[n]) & = & s_1 D_{M_{1,j_{-1}}} (s'_1 D_{M_{1,j_0}})^{n-1} t_2 (b'_1)^{n-1} b_1 = s'_0 D_u t_2 b'_0 \\
			\textrm{Trans}(M)[m_n] & = & s_1 D_{M_{1,j_{-1}}} (s'_1 D_{M_{1,j_0}})^n 0 (b'_1)^n b_1 = s'_0 D_u(t_2 + D_{M_{1,j_0}} 0) b'_0 \\
			\textrm{Trans}(M[n+1]) & = & s_1 D_{M_{1,j_{-1}}} (s'_1 D_{M_{1,j_0}})^n t_2 (b'_1)^n b_1 = s'_0 D_u(t_2 + D_{M_{1,j_0}} t') b'_0
			\end{eqnarray*}
			\item である。
		\end{indented}
		\item
		\item \(j_0\)が非\(M\)許容であるとする。
		\begin{indented}
			\item \nameref{条件(V)の下での各種scb分解}より、一意な\((s'_1,b'_1) \in (\Sigma^{< \omega})^2\)が存在して以下を満たす:
			\begin{penumerate}
				\item \((D_{M_{1,j_{-1}}} s'_1,D_{M_{1,j_1}} 0,b'_1)\)は\(c_2\)のscb分解である。
				\setcounter{penumeratei}{4}
				\item \(\textrm{Trans}(M[n]) = s_1 D_{M_{1,j_{-1}}} (s'_1 D_{M_{1,j_0}})^n t_2 (b'_1)^n b_1\)である。
				\item[] (5)' \(\textrm{Trans}(M[n+1]) = s_1 D_{M_{1,j_{-1}}} (s'_1 D_{M_{1,j_0}})^{n+1} t_2 (b'_1)^{n+1} b_1\)である。
			\end{penumerate}
			\item \(u := M_{1,j_0}\)と置く。
			\item \(s'_0 := s_1 D_{M_{1,j_{-1}}} (s'_1 D_{M_{1,j_0}})^{n-1} s'_1\)と置く。
			\item \(b'_0 := (b'_1)^n b_1\)と置く。
			\item \(t'_0 := t_2 + D_{M_{1,j_0}} t_2\)と置く。
			\item 既に示したように、\(\textrm{Trans}(M)[m_n] = s_1 D_{M_{1,j_{-1}}} (s'_1 D_{M_{1,j_0}})^{m_n+1} 0 (b'_1)^{m_n+1} b_1\)かつ\(s'_1 D_{M_{1,j_0}} 0 b'_1 = t_2 + D_{M_{1,j_0}} 0\)かつ\(s'_1 D_{M_{1,j_0}} t_2 b'_1 = t_2 + D_{M_{1,j_0}} t_2 = t'\)である。従って
			\begin{eqnarray*}
			\textrm{Trans}(M[n]) & = & s_1 D_{M_{1,j_{-1}}} (s'_1 D_{M_{1,j_0}})^n t_2 (b'_1)^n b_1 = s'_0 D_u t_2 b'_0 \\
			\textrm{Trans}(M)[m_n] & = & s_1 D_{M_{1,j_{-1}}} (s'_1 D_{M_{1,j_0}})^{n+1} 0 (b'_1)^{n+1} b_1 = s'_0 D_{M_{1,j_0}}(t_2 + D_{M_{1,j_0}} 0) b'_0 \\
			\textrm{Trans}(M[n+1]) & = & s_1 D_{M_{1,j_{-1}}} (s'_1 D_{M_{1,j_0}})^{n+2} t_2 (b'_1)^{n+2} b_1 = s'_0 D_{M_{1,j_0}}(t_2 + D_{M_{1,j_0}} t'_0) b'_0
			\end{eqnarray*}
			\item である。
		\end{indented}
	\end{indented}
\end{hideableproof}

\iffull{それでは本題に戻る。}\fi

\begin{hideableproof}[\nameref{条件(V)の下でのTransと基本列の交換関係}の証明]
	\begin{penumerate}
		\item[] \(M\)は単項であるので\nameref{Transが零項性を保つこと}から\(\textrm{Trans}(M) \neq 0\)である。従って(2)は(1)と\cite{buc1} Lemma 3.2より即座に従う。以下では(1)と(3)を示す。
		\item[] \nameref{条件(V)の下での基本列のscb分解}より、一意な\(u \in \mathbb{N}\)と\((s'_0,b'_0) \in (\Sigma^{< \omega})^2\)と\(t' \in T_{\textrm{B}}\)が存在して以下を満たす:
		\item \((s'_0,D_u t_2,b'_0)\)は\(\textrm{Trans}(M[n])\)のscb分解である。
		\item \((s'_0,D_u(t_2 + D_{M_{1,j_0}} 0),b'_0)\)は\(\textrm{Trans}(M)[m_n]\)のscb分解である。
		\item \((s'_0,D_u(t_2 + D_{M_{1,j_0}} t'),b'_0)\)は\(\textrm{Trans}(M[n+1])\)のscb分解である。
		\item[] \(t_2 < t_2 + D_{M_{1,j_0}} 0 \leq t_2 + D_{M_{1,j_0}} t'\)であるので、\nameref{部分表現の不等式の延長性}から\(\textrm{Trans}(M[n]) < \textrm{Trans}(M)[m_n] \leq \textrm{Trans}(M[n+1])\)である。\qedhere
		\item[]
		\item[] \(j_0\)が非\(M\)許容であるとする。
		\begin{indented}
			\item \nameref{条件(V)の下での各種scb分解}より、一意な\((s'_1,b'_1) \in (\Sigma^{< \omega})^2\)が存在して以下を満たす:
			\begin{penumerate}
				\item \((D_{M_{1,j_{-1}}} s'_1,D_{M_{1,j_1}} 0,b'_1)\)は\(c_2\)のscb分解である。
				\setcounter{penumerateii}{4}
				\item \(\textrm{Trans}(M[n]) = s_1 D_{M_{1,j_{-1}}} (s'_1 D_{M_{1,j_0}})^n t_2 (b'_1)^n b_1\)である。
				\item[] (5)' \(\textrm{Trans}(M[n+1]) = s_1 D_{M_{1,j_{-1}}} (s'_1 D_{M_{1,j_0}})^{n+1} t_2 (b'_1)^{n+1} b_1\)である。
				\begin{eqnarray*}
				\textrm{Trans}(M[n]) = s_1 D_{M_{1,j_{-1}}} (s'_1 D_{M_{1,j_0}})^n t_2 (b'_1)^n b_1 \\
				\textrm{Trans}(M)[m_n] = s_1 D_{M_{1,j_{-1}}} (s'_1 D_{M_{1,j_0}})^{n+1} 0 (b'_1)^{n+1} b_1 = s_1 D_{M_{1,j_{-1}}} (s'_1 D_{M_{1,j_0}})^n s'_1 D_{M_{1,j_0}} 0 b'_1 (b'_1)^n b_1
				\end{eqnarray*}
			\end{penumerate}
			\item かつ\(t_2 < s'_1 D_{M_{1,j_0}} 0 b'_1\)より、\nameref{部分表現の不等式の延長性}から\(\textrm{Trans}(M[n]) < \textrm{Trans}(M)[m_n]\)である。
			\item \(s'_1 D_{M_{1,j_0}} 0 b'_1 \in T_{\textrm{B}}\)と\nameref{scb分解の置換可能性}より\(s'_1 D_{M_{1,j_0}} t_2 b'_1 \in T_{\textrm{B}}\)であり、
			\begin{eqnarray*}
			\textrm{Trans}(M)[m_n] = s_1 D_{M_{1,j_{-1}}} (s'_1 D_{M_{1,j_0}})^{n+1} 0 (b'_1)^{n+1} b_1 \\
			\textrm{Trans}(M[n+1]) = s_1 D_{M_{1,j_{-1}}} (s'_1 D_{M_{1,j_0}})^{n+2} t_2 (b'_1)^{n+2} b_1 = s_1 D_{M_{1,j_{-1}}} (s'_1 D_{M_{1,j_0}})^{n+1} s'_1 D_{M_{1,j_0}} t_2 b'_1 (b'_1)^{n+1} b_1
			\end{eqnarray*}
			\item かつ\(0 \leq s'_1 D_{M_{1,j_0}} t_2 b'_1\)より、\nameref{部分表現の不等式の延長性}から\(\textrm{Trans}(M)[m_n] \leq \textrm{Trans}(M[n+1])\)である。
		\end{indented}
	\end{penumerate}
\end{hideableproof}


\subsection{条件(VI)の下での展開規則}

\begin{proposition}[条件(VI)の下での\(\textrm{Trans}\)と基本列の交換関係]\label{条件(VI)の下でのTransと基本列の交換関係}
	任意の\(M \in ST_{\textrm{PS}} \cap PT_{\textrm{PS}}\)と\(n \in \mathbb{N}_{+}\)に対し、\(\textrm{Trans}\)の再帰的定義中に導入した記号を用い、\(j_0\)が\(M\)許容ならば\(m_n := n-2\)と置き、\(j_0\)が非\(M\)許容ならば\(m_n := n-1\)と置くと、\(j_1 > 1\)かつ\(M\)が条件(VI)を満たすならば\footnotemark{}、以下が成り立つ:
	\begin{penumerate}
		\item \(m_n = -1\)ならば、ある\(k \in \mathbb{N}\)が存在して\(1 < k \leq M_{1,j_1}+1\)かつ\(\textrm{Trans}(M[n]) = \textrm{Trans}(M)[0]^k\)である。
		\item \(m_n \geq 0\)ならば、\(\textrm{Trans}(M[n]) = \textrm{Trans}(M)[m_n]\)である。
		\item \(\textrm{Trans}(M[n]) < \textrm{Trans}(M)\)である。
	\end{penumerate}
\end{proposition}
\footnotetext{\nameref{標準形の簡約性}から\(M\)は簡約であり、\(j_1 > 1\)より\(t_1 \neq 0\)であるので\(M\)に対し条件(VI)が意味を持つ。}

\nameref{条件(VI)の下でのTransと基本列の交換関係}を証明するための準備としていくつかの補題を示す。

\begin{lemma}[公差\((1,0)\)のペア数列の\(\textrm{Trans}\)の基本性質]\label{公差(1,0)のペア数列のTransの基本性質}
	任意の\(u,m,j_1 \in \mathbb{N}\)に対し、\(M := ((m+j,u))_{j=0}^{j_1} \in T_{\textrm{PS}}\)と置くと
	\begin{eqnarray*}
	\textrm{Trans}(M) & = & \left\{ \begin{array}{ll} 0 & (j_1 = 0 \wedge u = 0) \\ D_u^{j_1+1} 0 & (j_1 > 0 \vee u > 0) \end{array} \right.
	\end{eqnarray*}
	となる。
\end{lemma}

\begin{hideableproof}
	\begin{indented}
		\item \(((u+j,u))_{j=0}^{j_1}\)は条件(A)と(B)を満たすので、\nameref{簡約性と係数の関係}より簡約である。従って\nameref{RedのIncrFirst不変性}より
		\begin{eqnarray*}
		\textrm{Red}(M) = \textrm{Red}(\textrm{IncrFirst}^u(M)) = \textrm{Red}(\textrm{IncrFirst}^m(((u+j,u))_{j=0}^{j_1})) = \textrm{Red}(((u+j,u))_{j=0}^{j_1}) = ((u+j,u))_{j=0}^{j_1}
		\end{eqnarray*}
		\item である。
		\item \(j_1\)に関する数学的帰納法で示す。
		\item \(j_1 = 0\)とする。
		\begin{indented}
			\item \(\textrm{Trans}\)の再帰的定義より
			\begin{eqnarray*}
			\textrm{Trans}(M) & = & \textrm{Trans}((m,u)) = \textrm{Trans}(\textrm{Red}((m,u))) = \textrm{Trans}((u,u)) \\
			& = & \left\{ \begin{array}{ll} 0 & (u = 0) \\ D_u 0 & (u > 1) \end{array} \right.
			\end{eqnarray*}
			\item である。
		\end{indented}
		\item \(j_1 = 1\)とする。
		\begin{indented}
			\item \nameref{2列ペア数列の基本性質}と\nameref{Transの(IncrFirst,Red)不変P同変性}より
			\begin{eqnarray*}
			\textrm{Trans}(M) = \textrm{Trans}((m,u),(m+1,u)) = \textrm{Trans}(\textrm{Red}((m,u),(m+1,u))) = \textrm{Trans}((u,u),(u+1,u)) = D_u D_u 0 = D_u^{j_1+1} 0
			\end{eqnarray*}
			\item である。
		\end{indented}
		\item \(j_1 > 1\)とする。
		\begin{indented}
			\item \(\textrm{Red}(M) = ((j,u))_{j=0}^{j_1}\)は簡約かつ単項である。
			\item \(\textrm{Trans}\)の再帰的定義中に導入した記号を\(\textrm{Red}(M)\)に対して定める\footnote{\(j_1\)の定義が重複するが、結果的に同じ値なので問題ない。}。
			\item 帰納法の仮定より\(\textrm{Trans}(\textrm{Pred}(M)) = D_u^{j_1} 0\)である。
			\item \nameref{RedとPredの可換性}と\nameref{Transの(IncrFirst,Red)不変P同変性}より
			\begin{eqnarray*}
			t_1 = \textrm{Trans}(\textrm{Pred}(\textrm{Red}(M))) = \textrm{Trans}(\textrm{Red}(\textrm{Pred}(M))) = \textrm{Trans}(\textrm{Pred}(M)) = D_u^{j_1} 0 \neq 0
			\end{eqnarray*}
			\item である。従って\(\textrm{Red}(M)\)に対して条件(I)~(VI)が意味を持つ。
			\item \(j_0 = j_1-1\)であり\(j_{-1} = j_0\)である。
			\item \(u = 0\)ならば\(M\)は条件(I)を満たす。
			\item \(u > 0\)ならば\(M\)は条件(III)を満たす。
			\item 従っていずれの場合も\(M\)は条件(I)か(III)を満たす。
			\item \nameref{簡約性の切片への遺伝性}より\(\textrm{Pred}(\textrm{Red}(M))\)は簡約であるので、\nameref{右端第1基点のMarkの基本性質}より\(c_1 = \textrm{Mark}(\textrm{Pred}(\textrm{Red}(M)),j_{-1}) = \textrm{Mark}(\textrm{Pred}(\textrm{Red}(M)),j_1-1) = D_u 0\)である。
			\item \(t_1 = D_u^{j_1} 0\)かつ\(c_1 = D_u 0\)より、\(s_1 = D_u^{j_1-1}\)かつ\(b_1 = ()\)である。
			\item \(D_v t_2 = c_1 = D_u 0\)より、\(v = u\)かつ\(t_2 = 0\)である。
			\item \(M\)は条件(I)か(III)を満たすことから\(c_2 = D_v(t_2 + D_u 0) = D_u D_u 0 = D_u^2 0\)である。従って\nameref{Transの(IncrFirst,Red)不変P同変性}より
			\begin{eqnarray*}
			\textrm{Trans}(M) & = & \textrm{Trans}(\textrm{Red}(M)) = s_1 c_2 b_1 = D_u^{j_1-1} D_u^2 0 = D_u^{j_1+1} 0
			\end{eqnarray*}
			\item である。
		\end{indented}
	\end{indented}
\end{hideableproof}

\begin{lemma}[公差\((1,1)\)のペア数列の\(\textrm{Trans}\)の展開規則]\label{公差(1,1)のペア数列の展開規則}
	任意の\(u,j_1 \in \mathbb{N}\)と\(n \in \mathbb{N}_{+}\)に対し、\(M := ((u+j,u+j))_{j=0}^{j_1} \in T_{\textrm{PS}}\)と置くと、\(j_1 > 1\)ならば\(\textrm{Trans}(M[n]) = D_u D_{u+j_1-1}^n 0\)である。
\end{lemma}

\begin{hideableproof}
	\begin{indented}
		\item \(M\)は条件(A)と(B)を満たすので、\nameref{簡約性と係数の関係}より\(M\)は簡約である。\nameref{簡約性が基本列で保たれること}より\(M[n]\)は簡約である。
		\item \((1,j_1-1) <_M^{\textrm{Next}} (1,j_1)\)より
		\begin{eqnarray*}
		M[n] & = & (M_j)_{j=0}^{j_1-1} \oplus_{\mathbb{N}^2} ((M_{0,j_1-1}+j,M_{1,j_1-1}))_{j=1}^{n-1} = (M_j)_{j=0}^{j_1-1} \oplus_{\mathbb{N}^2} ((u+j_1-1+j,u+j_1-1))_{j=1}^{n-1}
		\end{eqnarray*}
		\item であり、\(\textrm{Lng}(M[n])-1 = j_1-2+n\)である。
		\item \(n = 1\)ならば、\(j_1-1 = \textrm{Lng}(M[n])-1\)であるので\(j_1-1\)は\(M[n]\)許容である。
		\item \(n > 1\)ならば、\(j_1-1 < \textrm{Lng}(M[n])-1\)かつ\(M[n]_{1,j_1-1} = M_{1,j_1-1} = M[n]_{1,j_1}\)であるので\((1,j_1-1) <_{M[n]}^{\textrm{Next}} (1,j_1)\)でなく、従って\(j_1-1\)は\(M[n]\)許容である。
		\item 以上より、いずれの場合も\(j_1-1\)は\(M[n]\)許容である。
		\item \((M[n]_j)_{j=0}^{j_1-1} = (M_j)_{j=0}^{j_1-1} = ((u+j,u+j))_{j=0}^{j_1-1}\)であるので、\(j_1 > 1\)と\nameref{公差(1,1)のペア数列のTransの基本性質}より
		\begin{eqnarray*}
		\textrm{Trans}((M[n]_j)_{j=0}^{j_1-1}) = D_u D_{u+j_1-1} 0
		\end{eqnarray*}
		\item である。
		\item \((M[n]_j)_{j=j_1-1}^{j_1-2+n} = ((u+j_1-1+j,u+j_1-1))_{j=1}^{n-1}\)であるので、\(j_1 > 1\)と\nameref{公差(1,0)のペア数列のTransの基本性質}より\(\textrm{Trans}((M[n]_j)_{j=j_1-1}^{j_1-2+n}) = D_{u+j_1-1}^n 0\)である。従って\nameref{MarkのTransによる表示}より
		\begin{eqnarray*}
		\textrm{Mark}(M[n],j_1-1) = \textrm{Trans}((M[n]_j)_{j=j_1-1}^{j_1-2+n}) = D_{u+j_1-1}^n 0
		\end{eqnarray*}
		\item である。
		\item 以上より、\(j_1-1 > 0\)と\nameref{TransのMarkと切片による表示}から
		\begin{eqnarray*}
		\textrm{Trans}(M[n]) = D_u \textrm{Mark}(M[n],j_1-1) = D_u D_{u+j_1-1}^n 0
		\end{eqnarray*}
		\item である。
	\end{indented}
\end{hideableproof}

\begin{lemma}[順序数項の末尾単項の零化可能性]\label{順序数項の末尾単項の零化可能性}
	任意の\(t,t' \in T_{\textrm{B}}\)と\(s,b \in \Sigma^{< \omega}\)と\(u,v \in \mathbb{N}\)に対し、\((s,D_u(t' + D_v 0),b\))が\(t\)のscb分解であるならば、ある\(k \in \mathbb{N}\)が存在して\(0 < k \leq v+1\)かつ\((s,D_u t',b)\)が\(t[0]^k\)のscb分解である。
	となる。
\end{lemma}

\begin{hideableproof}
	\begin{indented}
		\item \(v\)についての数学的帰納法で示す。
		\item \(v = 0\)とする。
		\begin{indented}
			\item \(k := 1\)と置く。
			\item \(0 < k = 1 = v+1\)である。
			\item 順序数項の基本列の再帰的定義より
			\begin{eqnarray*}
			(s,(D_u(t' + D_0 0))[0],b) = (s,D_u((t' + D_0 0)[0]),b) = (s,D_u(t' + (D_0 0)[0]),b)= (s,D_u(t' + 0),b) = (s,D_u t',b)
			\end{eqnarray*}
			\item が\(t[0]^k = t[0]\)のscb分解である。
		\end{indented}
		\item \(v > 0\)とする。
		\begin{indented}
			\item \((D_v 0)[0] = 0\)かつ\((D_v 0)[D_{v-1} 0] = D_{v-1} 0\)であるので、順序数項の基本列の再帰的定義と\nameref{scb分解の置換可能性}より\((s,D_u t',b)\)または\((s,D_u(t' + D_{v-1} 0),b)\)が\(t[0]\)のscb分解である。
			\item \((s,D_u t',b)\)が\(t[0]\)のscb分解であるとする。
			\begin{indented}
				\item \(k := 1\)と置く。
				\item \(0 < k = 1 < v+1\)である。
				\item \((s,D_u t',b)\)は\(t[0]^k = t[0]\)のscb分解である。
			\end{indented}
			\item \((s,D_u(t' + D_{v-1} 0),b)\)が\(t[0]\)のscb分解であるとする。
			\begin{indented}
				\item 帰納法の仮定より、ある\(k' \in \mathbb{N}\)が存在して\(0 < k' \leq v\)かつ\(t[0][0]^{k'}\)が\((s,D_u t',b)\)のscb分解である。
				\item \(k := k'+1\)と置く。
				\item \(0 < 1 < k \leq v+1\)である。
				\item \(t[0]^k = t[0][0]^{k'}\)は\((s,D_u t',b)\)のscb分解である。
			\end{indented}
		\end{indented}
	\end{indented}
\end{hideableproof}

\iffull{それでは本題に戻る。}\fi

\begin{hideableproof}[\nameref{条件(VI)の下でのTransと基本列の交換関係}の証明]
	\begin{indented}
		\item \(M\)は単項であるので\nameref{Transが零項性を保つこと}から\(\textrm{Trans}(M) \neq 0\)である。従って(3)は(1)と(2)と\cite{buc1} Lemma 3.2より即座に従う。
		\item
		\item \(M_{1,j_0} < M_{1,j_0}+1 = M_{1,j_1}\)より\((1,j_0) <_M^{\textrm{Next}} (1,j_1)\)である。
		\item \(j_0+1 = j_1\)かつ\(M_{1,j_0}+1 = M_{1,j_1}\)より
		\begin{eqnarray*}
		M[n] & = & (M_j)_{j=0}^{j_1-1} \oplus_{\mathbb{N}^2} ((M_{0,j_0}+j,M_{1,j_0}))_{j=1}^{n-1}
		\end{eqnarray*}
		\item であり、\(\textrm{Lng}(M[n])-1 = j_1-2+n\)である。
		\item \(n = 1\)ならば、\(j_0 = \textrm{Lng}(M[n])-1\)であるので\(j_0\)は\(M[n]\)許容である。
		\item \(n > 1\)ならば、\(j_0 < \textrm{Lng}(M[n])-1\)かつ\(M[n]_{1,j_0} = M_{1,j_0} = M[n]_{1,j_0+1}\)であるので\((1,j_0) <_{M[n]}^{\textrm{Next}} (1,j_0+1)\)でなく、従って\(j_0\)は\(M[n]\)許容である。
		\item 従っていずれの場合も\(j_0\)は\(M[n]\)許容である。
		\item
		\item \((1,j_0) <_M^{\textrm{Next}} (1,j_1)\)より\((1,j_{-1}) <_M^{\textrm{Next}} (1,j_{-1}+1)\)であり、\(j_{-1}\)は\(M\)許容であるので\((1,j_{-1}-1) <_M^{\textrm{Next}} (1,j_{-1})\)でない。更に\((M[n]_j)_{j=0}^{j_{-1}} = (M_j)_{j=0}^{j_{-1}}\)であるので、\((1,j_{-1}-1) <_{M[n]}^{\textrm{Next}} (1,j_{-1})\)でない。以上より\(j_{-1}\)は\(M[n]\)許容である。
		\item \nameref{右端第2基点のMarkの基本性質}より\(\textrm{Trans}(M) = s_1 c_2 b_1 = s_1 \textrm{Mark}(M,j_{-1}) b_1\)であるので、\(j_{-1} > 0\)ならば\nameref{TransのMarkと切片による表示}より
		\begin{eqnarray*}
		\textrm{Trans}((M[n]_j)_{j=0}^{j_{-1}}) = \textrm{Trans}((M_j)_{j=0}^{j_{-1}}) = s_1 D_{M_{1,j_{-1}}} b_1
		\end{eqnarray*}
		\item である。
		\item
		\item \(N := (M_j)_{j=j_{-1}}^{j_1}\)と置く。
		\item \nameref{標準形の直系先祖による切片の簡約化の強単項性}より\(\textrm{Red}(N)\)は強単項である。\nameref{簡約性と係数の関係}より\(\textrm{Red}(N)\)は条件(A)と(B)を満たす。\nameref{直系先祖による切片とRedとIncrFirstの関係}より\(\textrm{IncrFirst}^{N_{0,0}-N_{1,0}}(\textrm{Red}(N)) = N\)すなわち\(\textrm{Red}(N) = ((M_{1,j}-M_{0,j_{-1}}+M_{1,j_{-1}},M_{1,j}))_{j=j_{-1}}^{j_1}\)である。
		\item \(M_{1,j_0} < M_{1,j_0}+1 = M_{1,j_1}\)より\((1,j_0) <_M^{\textrm{Next}} (1,j_1)\)であるので\((1,j_0) <_N^{\textrm{Next}} (1,j_1)\)であり、\nameref{直系先祖のRed不変性}より\((1,j_0) <_{\textrm{Red}(N)}^{\textrm{Next}} (1,j_1)\)である。
		\item \nameref{許容化の切片への遺伝性}より\(\textrm{Adm}_N(j_0-j_{-1}) = 0\)であり、\nameref{許容化のRed不変性}より\(\textrm{Adm}_{\textrm{Red}(N)}(j_0-j_{-1}) = 0\)である。
		\item \((1,j_0) <_{\textrm{Red}(N)}^{\textrm{Next}} (1,j_1)\)かつ\(\textrm{Adm}_{\textrm{Red}(N)}(j_0-j_{-1}) = 0\)より\(\textrm{TrMax}(\textrm{Red}(N)) = j_1-j_{-1}\)である。更に\(\textrm{Red}(N)\)が条件(A)と(B)を満たし\(\textrm{Red}(N)_{1,0} = M_{1,j_{-1}}\)かつ\(\textrm{Red}(N)_{1,j_1} = M_{1,j_1}\)であることから、\(\textrm{Red}(N) = ((j,j))_{j=M_{1,j_{-1}}}^{M_{1,j_1}}\)である。
		\item \(j_1-j_{-1} \geq j_1-j_0 > 0\)と\nameref{公差(1,1)のペア数列のTransの基本性質}より、\(\textrm{Trans}(\textrm{Red}(N)) = D_{M_{1,j_{-1}}} D_{M_{1,j_1}} 0\)である。従って\nameref{Transの(IncrFirst,Red)不変P同変性}と\nameref{MarkのTransによる表示}より
		\begin{eqnarray*}
		c_2 & = & \textrm{Mark}(M,j_{-1}) = \textrm{Trans}(N) = \textrm{Trans}(\textrm{Red}(N)) = D_{M_{1,j_{-1}}} D_{M_{1,j_1}} 0
		\end{eqnarray*}
		\item である。\nameref{条件(III)~(VI)の下でのTransとscb分解の関係}より\((s_1,c_2,b_1) = (s_1,D_{M_{1,j_{-1}}} D_{M_{1,j_1}} 0,b_1) = (s_1,\textrm{Trans}(N),b_1)\)は\(\textrm{Trans}(M)\)の第\(1\)種scb分解であり、\nameref{scb分解と基本列の関係}より任意の\(m \in \mathbb{N}\)に対し
		\begin{eqnarray*}
		\textrm{Trans}(M)[m] & = & s_1 D_{M_{1,j_{-1}}} D_{M_{1,j_1}-1}^{m+1} 0 b_1
		\end{eqnarray*}
		\item である。
		\item
		\item \nameref{RedとPredの可換性}より\(\textrm{Red}(\textrm{Pred}(N)) = \textrm{Pred}(\textrm{Red}(N)) = ((j,j))_{j=M_{1,j_{-1}}}^{M_{1,j_1}-1}\)である。
		\item \(j_1-j_{-1} = 1\)とする。
		\begin{indented}
			\item \(j_{-1} \leq j_0 < j_1 = j_{-1}+1\)より\(j_0 = j_{-1}\)である。従って\(j_0\)は\(M\)許容であり、\(m_n = n-2\)である。
			\item \(\textrm{Pred}(M)_{1,j_1-1} = M_{1,j_1-1} = M_{1,j_{-1}}\)と\nameref{右端第1基点のMarkの基本性質}より
			\begin{eqnarray*}
			c_1 = \textrm{Mark}(\textrm{Pred}(M),j_{-1}) = \textrm{Mark}(\textrm{Pred}(M),j_1-1) = D_{M_{1,j_{-1}}} 0
			\end{eqnarray*}
			\item である。
			\item \(n = 1\)とする。
			\begin{indented}
				\item \(m_n = -1\)である。
				\item \(M[n] = \textrm{Pred}(M)\)より
				\begin{eqnarray*}
				\textrm{Trans}(M[n]) & = & s_1 c_1 b_1 = s_1 D_{M_{1,j_{-1}}} 0 b_1 \\
				\textrm{Trans}(M)[0] & = & s_1 D_{M_{1,j_{-1}}} D_{M_{1,j_1}-1} 0 b_1
				\end{eqnarray*}
				\item であるので、\nameref{順序数項の末尾単項の零化可能性}よりある\(k' \in \mathbb{N}\)が存在して\(0 < k' \leq M_{1,j_1}\)かつ\(\textrm{Trans}(M[n])  = \textrm{Trans}(M)[0][0]^{k'} = \textrm{Trans}(M)[0]^{k+1}\)である。
				\item \(k := k'+1\)と置く。
				\item \(1 < k \leq M_{1,j_1}+1\)である。
				\item \(\textrm{Trans}(M[n]) = \textrm{Trans}(M)[0][0]^{k'} = \textrm{Trans}(M)[0]^k\)である。
			\end{indented}
			\item \(n > 1\)とする。
			\begin{indented}
				\item \(m_n \geq 0\)である。
				\item \((M[n]_j)_{j=j_{-1}}^{j_1-2+n} = ((M_{0,j_0}+j,M_{1,j_0}))_{j=0}^{n-1}\)であるので、\nameref{MarkのTransによる表示}と\nameref{公差(1,0)のペア数列のTransの基本性質}から
				\begin{eqnarray*}
				\textrm{Mark}(M[n],j_{-1}) = \textrm{Trans}((M[n]_j)_{j=j_{-1}}^{j_1-2+n}) = D_{M_{1,j_0}}^n 0
				\end{eqnarray*}
				\item である。
				\item \(j_{-1} = 0\)とする。
				\begin{indented}
					\item \nameref{s_1とb_1の空性と基点の関係}より\(s_1 = ()\)かつ\(b_1 = ()\)である。
					\item \(M[n] = ((M_{0,j_0}+j,M_{1,j_0}))_{j=0}^{n-1}\)であるので、従って\nameref{公差(1,0)のペア数列のTransの基本性質}
					\begin{eqnarray*}
					\textrm{Trans}(M[n]) & = & D_{M_{1,j_{-1}}}^n 0 = D_{M_{1,j_{-1}}} D_{M_{1,j_1}-1}^{n-1} 0 \\
					\textrm{Trans}(M)[m_n] & = & s_1 D_{M_{1,j_{-1}}} D_{M_{1,j_1}-1}^{n-1} 0 b_1 = D_{M_{1,j_{-1}}} D_{M_{1,j_1}-1}^{n-1} 0
					\end{eqnarray*}
					\item となる。すなわち\(\textrm{Trans}(M[n])  = \textrm{Trans}(M)[m_n]\)である。
				\end{indented}
				\item \(j_{-1} > 0\)とする。
				\begin{indented}
					\item \(\textrm{Trans}((M[n]_j)_{j=0}^{j_{-1}}) = s_1 D_{M_{1,j_{-1}}} 0 b_1\)と\(\textrm{Mark}(M[n],j_{-1}) = D_{M_{1,j_0}}^n 0\)から、\nameref{TransのMarkと切片による表示}より
					\begin{eqnarray*}
					\textrm{Trans}(M[n]) & = & s_1 D_{M_{1,j_0}}^n 0 b_1 \\
					\textrm{Trans}(M)[m_n] & = & s_1 D_{M_{1,j_{-1}}} D_{M_{1,j_1}-1}^{n-1} 0 b_1 = s_1 D_{M_{1,j_0}}^n 0 b_1
					\end{eqnarray*}
				\end{indented}
				\item となる。すなわち\(\textrm{Trans}(M[n])  = \textrm{Trans}(M)[m_n]\)である。
			\end{indented}
		\end{indented}
		\item
		\item \(j_1-j_{-1} > 1\)とする。
		\begin{indented}
			\item \(j_{-1} < j_1-1 = j_0\)である。従って\(j_0\)は非\(M\)許容であり、\(m_n = n-1 \geq 0\)である。
			\item \nameref{公差(1,1)のペア数列のTransの基本性質}より\(\textrm{Trans}(\textrm{Red}(\textrm{Pred}(N))) = D_{M_{1,j_{-1}}} D_{M_{1,j_1}-1} 0\)である。従って\nameref{Transの(IncrFirst,Red)不変P同変性}と\nameref{MarkのTransによる表示}より
			\begin{eqnarray*}
			c_1 = \textrm{Mark}(\textrm{Pred}(M),j_{-1}) = \textrm{Trans}(\textrm{Pred}(N)) = \textrm{Trans}(\textrm{Red}(\textrm{Pred}(N))) = D_{M_{1,j_{-1}}} D_{M_{1,j_1}-1} 0
			\end{eqnarray*}
		\end{indented}
		\item である。
		\item \(n = 1\)とする。
		\begin{indented}
			\item
			\begin{indented}
				\item \(M[n] = \textrm{Pred}(M)\)より
				\begin{eqnarray*}
				\textrm{Trans}(M[n]) & = & s_1 c_1 b_1 = s_1 D_{M_{1,j_{-1}}} D_{M_{1,j_1}-1} 0 b_1 \\
				\textrm{Trans}(M)[m_n] & = & s_1 D_{M_{1,j_{-1}}} D_{M_{1,j_1}-1} 0 b_1
				\end{eqnarray*}
				\item である。すなわち\(\textrm{Trans}(M[n]) = \textrm{Trans}(M)[m_n]\)である。
			\end{indented}
			\item \(n > 1\)とする。
			\begin{indented}
				\item \((M[n]_j)_{j=j_{-1}}^{j_1-2+n} = (M_j)_{j=j_{-1}}^{j_1-1} \oplus_{\mathbb{N}^2} ((M_{0,j_0}+j,M_{1,j_0}))_{j=1}^{n-1} = N[n]\)であるので、\nameref{Redと基本列の可換性}と\nameref{Transの(IncrFirst,Red)不変P同変性}と\nameref{MarkのTransによる表示}と\nameref{公差(1,1)のペア数列の展開規則}より
				\begin{eqnarray*}
				\textrm{Mark}(M[n],j_{-1}) & = & \textrm{Trans}((M[n]_j)_{j=j_{-1}}^{j_1-2+n}) = \textrm{Trans}(N[n]) = \textrm{Trans}(\textrm{Red}(N[n])) \\
				& = & \textrm{Trans}(\textrm{Red}(N)[n]) = \textrm{Trans}(((j,j))_{j=M_{1,j_{-1}}}^{M_{1,j_1}}[n]) = D_{M_{1,j_{-1}}} D_{M_{1,j_1}-1}^n 0
				\end{eqnarray*}
				\item である。
				\item \(j_{-1} = 0\)とする。
				\begin{indented}
					\item \nameref{s_1とb_1の空性と基点の関係}より\(s_1 = ()\)かつ\(b_1 = ()\)である。
					\item \(M[n] = (M[n]_j)_{j=j_{-1}}^{j_1-2+n}\)であるので、
					\begin{eqnarray*}
					\textrm{Trans}(M[n]) & = & \textrm{Trans}((M[n]_j)_{j=j_{-1}}^{j_1-2+n}) = D_{M_{1,j_{-1}}} D_{M_{1,j_1}-1}^n 0 \\
					\textrm{Trans}(M)[m_n] & = & s_1 D_{M_{1,j_{-1}}} D_{M_{1,j_1}-1}^n 0 b_1 = D_{M_{1,j_{-1}}} D_{M_{1,j_1}-1}^n 0
					\end{eqnarray*}
					\item となる。すなわち\(\textrm{Trans}(M[n])  = \textrm{Trans}(M)[m_n]\)である。
				\end{indented}
				\item \(j_{-1} > 0\)とする。
				\begin{indented}
					\item \(\textrm{Trans}((M[n]_j)_{j=0}^{j_{-1}}) = s_1 D_{M_{1,j_{-1}}} 0 b_1\)と\nameref{TransのMarkと切片による表示}より
					\begin{eqnarray*}
					\textrm{Trans}(M[n]) & = & s_1 \textrm{Mark}(M[n],j_{-1}) b_1 = s_1 D_{M_{1,j_{-1}}} D_{M_{1,j_0}}^n 0 b_1 \\
					\textrm{Trans}(M)[m_n] & = & s_1 D_{M_{1,j_{-1}}} D_{M_{1,j_1}-1}^n 0 b_1 = s_1 D_{M_{1,j_{-1}}} D_{M_{1,j_0}}^n 0 b_1
					\end{eqnarray*}
				\end{indented}
				\item となる。すなわち\(\textrm{Trans}(M[n])  = \textrm{Trans}(M)[m_n]\)である。
			\end{indented}
		\end{indented}
	\end{indented}
\end{hideableproof}


\subsection{主結果}

\begin{theorem}[標準形ペア数列システムの停止性]\label{標準形ペア数列システムの停止性}
	\(ST_{\textrm{PS}} \times \mathbb{N}_{+} \subset \textrm{Dom}(F)\)である。
\end{theorem}

\nameref{標準形ペア数列システムの停止性}を証明するための準備としていくつかの補題を示す。

\begin{lemma}[公差\((0,0)\)のペア数列の\(\textrm{Trans}\)の基本性質]\label{公差(0,0)のペア数列のTransの基本性質}
	任意の\(u,j_1 \in \mathbb{N}\)に対し、\(M := ((u,u))_{j=0}^{j_1}\)と置くと、
	\begin{eqnarray*}
	\textrm{Trans}(M) = \left\{ \begin{array}{ll} (D_0 0) \times j_1 & (u = 0) \\ (D_u 0) \times (j_1+1) & (u > 0) \end{array} \right.
	\end{eqnarray*}
	である。
\end{lemma}

\begin{hideableproof}
	\begin{indented}
		\item \(\textrm{Trans}\)の再帰的定義から、\(j_1\)に関する数学的帰納法により即座に従う。
	\end{indented}
\end{hideableproof}

\begin{lemma}[基本列の降下性]\label{基本列の降下性}
	任意の\(M \in ST_{\textrm{PS}}\)と\(n \in \mathbb{N}_{+}\)に対し、\(\textrm{Lng}(M) > 1\)ならば\(\textrm{Trans}(M[n]) < \textrm{Trans}(M)\)である。
\end{lemma}

\begin{hideableproof}
	\begin{indented}
		\item \nameref{標準形の簡約性}から\(M\)は簡約である。\nameref{簡約性と係数の関係}から\(M\)は条件(A)と(B)を満たす。
		\item \(\textrm{Trans}\)の再帰的定義中に導入した記号を用いる。
		\item \(J'_1 := \textrm{Lng}(P(M))-1\)と置く。
		\item \(M[n] = \textrm{Pred}(M)\)ならば\nameref{PredのTransに関する降下性}より従う。以下では\(M[n] \neq \textrm{Pred}(M)\)とする。
		\item \(M[n] \neq \textrm{Pred}(M)\)より、\(n > 1\)かつ\((0,j_0) <_M^{\textrm{Next}} (0,j_1)\)を満たす一意な\(j_0 \in \mathbb{N}\)が存在する。特に\(j_1 > j_0 \geq 0\)であり、\(\textrm{Lng}(P(M)_{J'_1}) \geq j_1-j_0 > 0\)である。すなわち\(P(M)_{J'_1} \neq ((0,0))\)である。
		\item \(P(M)_{J'_1} \neq ((0,0))\)と\(\textrm{Trans}\)の再帰的定義と\nameref{標準形の単項成分が標準形であること}から、\(M\)を\(P(M)_{J_1}\)に置き換えることで\(M\)が単項である場合に帰着される。以下では\(M\)が単項であるとする。
		\item \(t_1 = 0\)とする。
		\begin{indented}
			\item \nameref{Transが零項性を保つこと}より\(\textrm{Pred}(M)\)は零項である。従って\(j_1 = 1\)かつ\(M_{1,0} = 0\)である。\(M\)が単項でかつ条件(A)と(B)を満たすことから、\(M = ((0,0),(1,0))\)または\(M = ((0,0),(1,1))\)である。
			\item \(M = ((0,0),(1,0))\)とする。
			\begin{indented}
				\item \nameref{公差(1,0)のペア数列のTransの基本性質}から\(\textrm{Trans}(M) = D_0 D_0 0\)である。
				\item \(M[n] = ((0,0))_{j=0}^{n-1}\)であるので、\nameref{公差(0,0)のペア数列のTransの基本性質}から\(\textrm{Trans}(M[n]) = (D_0 0) \times (n-1)\)である。
				\item \(0 < D_0 0\)より\((D_0 0) \times (n-1) < D_0 D_0 0\)であるので、\(\textrm{Trans}(M[n]) < \textrm{Trans}(M)\)である。
			\end{indented}
			\item \(M = ((0,0),(1,1))\)とする。
			\begin{indented}
				\item \nameref{公差(1,1)のペア数列のTransの基本性質}から\(\textrm{Trans}(M) = D_0 D_1 0\)である。
				\item \(M[n] = ((j,0))_{j=0}^{n-1}\)であるので、\(n > 1\)と\nameref{公差(1,0)のペア数列のTransの基本性質}から\(\textrm{Trans}(M[n]) = D_0^n 0\)である。
				\item \(D_0^{n-1} 0 < D_1 0\)より\(D_0^n 0 < D_0 D_1 0\)であるので、\(\textrm{Trans}(M[n]) < \textrm{Trans}(M)\)である。
			\end{indented}
		\end{indented}
		\item \(t_1 \neq 0\)とする\footnote{この時\(M\)に対し条件(I)~(VI)が意味を持つ。}。
		\item \nameref{Transが零項性を保つこと}より\(\textrm{Pred}(M)\)は零項でない。
		\item \(M\)が条件(I)を満たすとする。
		\begin{indented}
			\item \(j_1 = 1\)とする。
			\begin{indented}
				\item \(\textrm{Pred}(M)\)は零項でなくかつ\(M\)が条件(A)と(B)を満たすことから、一意な\(u \in \mathbb{N}\)が存在して\(u > 0\)かつ\(M = ((u,u),(u+1,0))\)である。
				\item \nameref{2列ペア数列の基本性質} (1)より\(\textrm{Trans}(M) = D_u D_0 0\)である。
				\item \(M[n] = ((u,u))_{j=0}^{n-1}\)であるので、\nameref{公差(0,0)のペア数列のTransの基本性質}から\(\textrm{Trans}(M[n]) = (D_u 0) \times n\)である。
				\item \(0 < D_0 0\)より\(D_u 0 < D_u D_0 0\)であるので、\((D_u 0) \times n < D_u D_0 0\)すなわち\(\textrm{Trans}(M[n]) < \textrm{Trans}(M)\)である。
			\end{indented}
			\item \(j_1 > 1\)ならば、\nameref{条件(I)の下でのTransと基本列の交換関係} (2)より\(\textrm{Trans}(M[n]) < \textrm{Trans}(M)\)である。
		\end{indented}
		\item
		\item \(M\)が条件(II)を満たすとする。
		\begin{indented}
			\item \(M\)が単項であるので\((0,0) \leq_M (0,j_1)\)であるが、\(0\)は\(M\)許容であるため\((0,0) <_M^{\textrm{Next}} (0,j_1)\)でない。従って\(j_1 > 1\)であり、\nameref{条件(II)の下でのTransと基本列の交換関係} (4)より\(\textrm{Trans}(M[n]) < \textrm{Trans}(M)\)である。
		\end{indented}
		\item
		\item \(M\)が条件(III)か(IV)を満たすとする。
		\begin{indented}
			\item \(M[n] \neq \textrm{Pred}(M)\)より、\((1,j_{-2}) <_M^{\textrm{Next}} (1,j_1)\)を満たす一意な\(j_{-2} \in \mathbb{N}\)が存在する。\(M_{1,j_0} \geq M_{1,j_1}\)より\((1,j_0) <_M^{\textrm{Next}} (1,j_1)\)でないので、\(j_{-2} < j_0 < j_1\)である。従って\(j_1 > 1\)であり、\nameref{条件(III)か(IV)の下でのTransと基本列の交換関係} (2)より\(\textrm{Trans}(M[n]) < \textrm{Trans}(M)\)である。
		\end{indented}
		\item
		\item \(M\)が条件(V)を満たすとする。
		\begin{indented}
			\item \(j_1 > j_0+1 \geq 1\)であるので\nameref{条件(V)の下でのTransと基本列の交換関係} (2)より\(\textrm{Trans}(M[n]) < \textrm{Trans}(M)\)である。
		\end{indented}
		\item
		\item \(M\)が条件(VI)を満たすとする。
		\begin{indented}
			\item \(j_1 = 1\)とする。
			\begin{indented}
				\item \(\textrm{Pred}(M)\)は零項でなくかつ\(M\)が条件(A)と(B)を満たすことから、一意な\(u \in \mathbb{N}\)が存在して\(u > 0\)かつ\(M = ((u,u),(u+1,u+1))\)である。
				\item \nameref{2列ペア数列の基本性質} (1)より\(\textrm{Trans}(M) = D_u D_{u+1} 0\)である。
				\item \(M[n] = ((u+j,u))_{j=0}^{n-1}\)であるので、\nameref{公差(1,0)のペア数列のTransの基本性質}から\(\textrm{Trans}(M[n]) = D_u^n 0\)である。
				\item \(D_u^{n-1} 0 < D_{u+1} 0\)より\(D_u^n 0 < D_u D_{u+1} 0\)であるので、\(\textrm{Trans}(M[n]) < \textrm{Trans}(M)\)である。
			\end{indented}
			\item \(j_1 > 1\)ならば、\nameref{条件(VI)の下でのTransと基本列の交換関係} (3)より\(\textrm{Trans}(M[n]) < \textrm{Trans}(M)\)である。
		\end{indented}
	\end{indented}
\end{hideableproof}

以下、\cite{buc1}における\(OT\)と\(T_{\textrm{B}}\)の共通部分を\(OT_{\textrm{B}}\)と置く。

\begin{lemma}[順序数項の再帰構造]\label{順序数項の再帰構造}
	任意の\(t \in OT_{\textrm{B}}\)と\(c \in T_{\textrm{B}}\)と\(s,b \in \Sigma^{< \omega}\)に対し、\((s,c,b\))が\(t\)のscb分解であるならば、\(c\)は順序数項である。
	となる。
\end{lemma}

\begin{hideableproof}
	\begin{indented}
		\item \nameref{scb分解の自明性の判定条件}と順序数項の再帰的定義から、\(\textrm{Lng}(s)\)に関する数学的帰納法より即座に従う。
	\end{indented}
\end{hideableproof}

\begin{lemma}[順序数項の共終数の遺伝性]\label{順序数項の共終数の遺伝性}
	任意の\(t,t' \in T_{\textrm{B}}\)と\(s,b \in \Sigma^{< \omega}\)に対し、\(\textrm{dom}(t') = \mathbb{N}\)かつ\((s,t',b\))が\(t\)のscb分解であるならば、\(\textrm{dom}(t) = \mathbb{N}\)である。
	となる。
\end{lemma}

\begin{hideableproof}
	\begin{indented}
		\item \nameref{scb分解の自明性の判定条件}と\(\textrm{dom}\)の再帰的定義\cite{buc1} p. 204 ([].4) (iii)と([].5)から、\(\textrm{Lng}(s)\)に関する数学的帰納法より即座に従う。
	\end{indented}
\end{hideableproof}

\begin{lemma}[順序数項の末尾項の零化可能性]\label{順序数項の末尾項の零化可能性}
	任意の\(t \in OT_{\textrm{B}}\)と\(t' \in T_{\textrm{B}}\)と\(s,b \in \Sigma^{< \omega}\)と\(u \in \mathbb{N}\)に対し、\((s,D_u t',b\))が\(t\)のscb分解であるならば、ある\(k \in \mathbb{N}\)が存在して\((s,D_u 0,b)\)が\(t[0]^k\)のscb分解である。
	となる。
\end{lemma}

\begin{hideableproof}
	\begin{indented}
		\item \cite{buc1} Lemma 2.2より\((OT_{\textrm{B}},<)\)は整礎である。従って\((OT_{\textrm{B}},<)\)に対して数学的帰納法が適用可能である。
		\item \(t = s D_u t' b\)より\((s D_u,t',b\))は\(t\)のscb分解であるので、\nameref{順序数項の再帰構造}より\(t'\)は順序数項である。
		\item ある\(k \in \mathbb{N}\)が存在して\((s,D_u 0,b)\)が\(t[0]^k\)のscb分解となることを\(t'\)に関する数学的帰納法で示す。
		\item \(t' = 0\)とする。
		\begin{indented}
			\item \(k := 0\)と置く。
			\item \((s,D_u 0,b) = (s,D_u t',b)\)は\(t[0]^k = t\)のscb分解である。
		\end{indented}
		\item \(t' > 0\)とする。
		\begin{indented}
			\item \(t' \neq 0\)であるので、\(t'\)は単項または複項である。従ってある\(\tau \in T_{\textrm{B}}\)と\(\tau' \in PT_{\textrm{B}}\)が存在して\(t' = \tau + \tau'\)である。
			\item \(\tau'\)は単項であるので\(0 < \tau'\)であり、従って\(0 \geq \tau < \tau + \tau' = t'\)である。
			\item \(\textrm{dom}(\tau') = \{0\}\)とする。
			\begin{indented}
				\item \(\textrm{dom}(t') = \textrm{dom}(\tau') = \{0\}\)より\(\textrm{dom}(D_u t') = \mathbb{N}\)であるので、\nameref{順序数項の共終数の遺伝性}から\(\textrm{dom}(t) = \mathbb{N}\)である。
				\item 基本列の再帰的定義\cite{buc1} p. 204 ([].4) (i)より\((D_u t')[0] = (D_u \tau) \times (0+1) = D_u \tau \in PT_{\textrm{B}}\)であるので、\nameref{scb分解の置換可能性}と\([]\)の再帰的定義から\((s,D_u \tau,b)\)は\(t[0]\)のscb分解となる。
				\item \(\tau < t'\)と\((s,D_u \tau,b)\)が\(t[0]\)のscb分解であることから、帰納法の仮定よりある\(k' \in \mathbb{N}\)が存在して\((s,D_u 0,b)\)が\(t[0][0]^{k'}\)のscb分解となる。
			\end{indented}
			\item \(\textrm{dom}(\tau') \neq \{0\}\)とする。
			\begin{indented}
				\item 基本列の再帰的定義\cite{buc1} p. 204 ([].4) (ii)と(iii)と([].5)と\cite{buc2} p. 6のDefinitionの6から、ある\(n \in \textrm{dom}(t') \cup \mathbb{N}\)が存在して\((s,D_u(t'[n]),b)\)が\(t[0]\)のscb分解となる。
				\item \(t' > 0\)と\cite{buc1} Lemma 3.2 (a)より\(t'[n] < t'\)である。\(t'[n] < t'\)と\((s,D_u(t'[n]),b)\)が\(t[0]\)のscb分解であることから、帰納法の仮定よりある\(k' \in \mathbb{N}\)が存在して\((s,D_u 0,b)\)が\(t[0][0]^{k'}\)のscb分解となる。
			\end{indented}
			\item \(k := k'+1\)と置く。
			\item \(t[0]^k = t[0][0]^{k'}\)より、\((s,D_u 0,b)\)が\(t[0]^k\)のscb分解となる。
		\end{indented}
	\end{indented}
\end{hideableproof}

\begin{lemma}[\(\textrm{Pred}\)と\(\lbrack0\rbrack\)の関係]\label{Predと0の関係}
	任意の\(M \in RT_{\textrm{PS}} \cap PT_{\textrm{B}}\)に対し、\(\textrm{Trans}\)の再帰的定義中に導入した記号を用いると、\(j_1 > 1\)かつ\(M\)が条件(VI)を満たさず\footnotemark{}かつ\(\textrm{Trans}(M)\)が順序数項であるならば、ある\(k \in \mathbb{N}\)が存在して\(\textrm{Trans}(M)[0]^k = t_1\)である。
\end{lemma}
\footnotetext{\(j_1 > 1\)より\(t_1 \neq 0\)であるので\(M\)に対し条件(VI)が意味を持つ。}

\begin{hideableproof}
	\begin{indented}
		\item
		\begin{indented}
			\item
			\begin{indented}
				\item \nameref{簡約性と係数の関係}から\(M\)は条件(A)と(B)を満たす。
				\item \(j_1 > 1\)より\(t_1 \neq 0\)であるので、\((s_1,D_v t_2,b_1) = (s_1,c_1,b_1)\)が\(t_1\)のscb分解をなしかつ\(M\)に対し条件(I)~(VI)が意味を持つ。
				\item \(M\)が条件(I)か(III)か(V)を満たすとする。
				\begin{indented}
					\item \((s_1,D_v(t_2 + D_{M_{1,j_1}} 0),b_1) = (s_1,c_2,b_1)\)が\(\textrm{Trans}(M)\)のscb分解である。
					\item \((s_1,D_v t_2,b_1)\)と\((s_1,D_v(t_2 + D_{M_{1,j_1}} 0),b_1)\)がそれぞれ\(t_1\)と\(\textrm{Trans}(M)\)のscb分解であることから、\nameref{順序数項の末尾単項の零化可能性}よりある\(k \in \mathbb{N}\)が存在して\(t_1 = \textrm{Trans}(M)[0]^k\)となる。
				\end{indented}
				\item \(M\)が条件(II)か(IV)を満たすとする。
				\begin{indented}
					\item \(P(t_2)_{J_1}\)の左端が\(D_{M_{1,j_0}}\)であるとする。
					\begin{indented}
						\item \(t_2 = t_3 + D_{M_{1,j_0}} t_4\)である。従って\((s_1,D_v(t_3 + D_{M_{1,j_0}} t_4),b_1) = (s_1,c_1,b_1)\)と\((s_1,D_v(t_3 + D_{M_{1,j_0}}(t_4 +D_{M_{1,j_1}} 0)),b_1) = (s_1,c_2,b_1)\)はそれぞれ\(t_1\)と\(\textrm{Trans}(M)\)のscb分解である。
						\item 更に\nameref{scb分解の合成則}と\nameref{加法とscb分解の関係}よりある\((s',b') \in (\Sigma^{< \omega})^2\)が存在して\((s',D_{M_{1,j_0}} t_4,b')\)と\((s',D_{M_{1,j_0}}(t_4 + D_{M_{1,j_1}} 0),b')\)がそれぞれ\(t_1\)と\(\textrm{Trans}(M)\)のscb分解となる。
						\item \nameref{順序数項の末尾単項の零化可能性}より、ある\(k \in \mathbb{N}\)が存在して\((s',D_{M_{1,j_0}} t_4,b')\)は\(\textrm{Trans}(M)[0]^k\)のscb分解である。従って
						\begin{eqnarray*}
						t_1 = s_1 D_{M_{1,j_{-1}}} t_2 b_1 = \textrm{Trans}(M)[0]^k
						\end{eqnarray*}
						\item である。
					\end{indented}
					\item \(P(t_2)_{J_1}\)の左端が\(D_{M_{1,j_0}}\)でないとする。
					\begin{indented}
						\item \(t_3 = t_2\)かつ\(t_4 = t_2\)かつ\(c_2 = D_v(t_3 + D_{M_{1,j_0}}(t_4 + D_{M_{1,j_1}} 0)) = D_v(t_2 + D_{M_{1,j_0}}(t_2 + D_{M_{1,j_1}} 0))\)である。従って\((s_1,D_v(t_2 + D_{M_{1,j_0}}(t_2 + D_{M_{1,j_1}} 0)),b_1) = (s_1,c_2,b_1)\)は\(\textrm{Trans}(M)\)のscb分解である。
						\item 更に\nameref{scb分解の合成則}と\nameref{加法とscb分解の関係}よりある\((s',b') \in (\Sigma^{< \omega})^2\)が存在して\((s',D_{M_{1,j_0}}(t_2 + D_{M_{1,j_1}} 0),b')\)が\(\textrm{Trans}(M)\)のscb分解となる。
						\item \(\textrm{Trans}(M)\)が順序数項であることと\nameref{順序数項の末尾項の零化可能性}よりある\(k'_0 \in \mathbb{N}\)が存在して\((s',D_{M_{1,j_0}} 0,b')\)は\(\textrm{Trans}(M)[0]^{k'_0}\)のscb分解となる。
						\item \((s_1,D_{M_{1,j_{-1}}}(t_2 + D_{M_{1,j_0}}(t_2 + D_{M_{1,j_1}} 0)),b_1)\)と\((s',D_{M_{1,j_0}}(t_2 + D_{M_{1,j_1}} 0),b')\)がいずれも\(\textrm{Trans}(M)\)のscb分解でありかつ\((s',D_{M_{1,j_0}} 0,b')\)が\(\textrm{Trans}(M)[0]^{k'_0}\)のscb分解であることから、\nameref{加法とscb分解の関係} (3)より\((s_1,D_{M_{1,j_{-1}}}(t_2 + D_{M_{1,j_0}} 0),b_1)\)は\(\textrm{Trans}(M)[0]^{k'_0}\)のscb分解である。
						\item 従って\nameref{順序数項の末尾単項の零化可能性}より、ある\(k'_1 \in \mathbb{N}\)が存在して\((s_1,D_{M_{1,j_{-1}}} t_2,b_1)\)は\(\textrm{Trans}(M)[0]^{k'_0}[0]^{k'_1}\)のscb分解である。
						\item \(k := k'_0 + k'_1\)と置く。
						\begin{eqnarray*}
						t_1 = s_1 D_{M_{1,j_{-1}}} t_2 b_1 = \textrm{Trans}(M)[0]^{k'_0}[0]^{k'_1} = \textrm{Trans}(M)[0]^k
						\end{eqnarray*}
						\item である。
					\end{indented}
				\end{indented}
			\end{indented}
		\end{indented}
	\end{indented}
\end{hideableproof}

\begin{lemma}[順序数項の基本例]\label{順序数項の基本例}
	\begin{penumerate}
		\item 任意の\(u \in \mathbb{N}\)に対し、\(D_u 0 \in OT_{\textrm{B}}\)である。
		\item 任意の\(u,v \in \mathbb{N}\)に対し、\(D_u D_v 0 \in OT_{\textrm{B}}\)である。
		\item 任意の\(u \in \mathbb{N}\)と\(n \in \mathbb{N}_{+}\)に対し、\((D_u 0) \times (n-1) \in OT_{\textrm{B}}\)である。
		\item 任意の\(u \in \mathbb{N}\)と\(n \in \mathbb{N}\)に対し、\(D_u^n 0 \in OT_{\textrm{B}}\)である。
	\end{penumerate}
\end{lemma}

\begin{hideableproof}
	\begin{penumerate}
		\item[] 以下では\(2^{T_{\textrm{B}}}\)で\(T_{\textrm{B}}\)の部分集合全体の集合を表し、\(G\)で\cite{buc1} p. 201のフィルトレーションの\(\mathbb{N} \times T_{\textrm{B}}\)への制限
		\begin{eqnarray*}
		G \colon \mathbb{N} \times T_{\textrm{B}} & \to & 2^{T_{\textrm{B}}} \\
		(u,t) & \mapsto & G_u t
		\end{eqnarray*}
		\item[] を表す。\cite{buc1} p.201の方法で\(<\)を\(T_{\textrm{B}} \cup 2^{T_{\textrm{B}}}\)へ拡張する。
		\item[]
		\item \(0 \in OT_{\textrm{B}}\)と\(G_u 0 = \emptyset < D_u 0\)と\cite{buc1} p.201 (OT3)より即座に従う。
		\item \(D_v 0 \in OT_{\textrm{B}}\)と
		\begin{eqnarray*}
		G_u D_v 0 & = & \left\{ \begin{array}{ll} \{0\} \cup G_u 0 & (u \leq v) \\ \emptyset & (u > v) \end{array} \right. \\
		& = & \left\{ \begin{array}{ll} \{0\} \cup \emptyset & (u \leq v) \\ \emptyset & (u > v) \end{array} \right. \\
		& = & \left\{ \begin{array}{ll} \{0\} & (u \leq v) \\ \emptyset & (u > v) \end{array} \right. \\
		& < & D_v 0
		\end{eqnarray*}
		\item[] と\cite{buc1} p.201 (OT3)より即座に従う。
		\item \(0, D_u 0 \in OT_{\textrm{B}}\)と\cite{buc1} p.201 (OT2)より即座に従う。
		\item[]
		\item 任意の\(m \in \mathbb{N}\)に対し\(m < n\)ならば\(D_u^m 0 < D_u^n 0\)であることを\(m\)に関する数学的帰納法で示す。
		\item[] \(m = 0\)ならば、\(D_u^m 0 = 0 < D_u^n 0\)である。
		\item[] \(m > 0\)ならば、帰納法の仮定より\(D_u^{m-1} 0 < D_u^{n-1} 0\)であるので\(D_u^m 0 = D_u D_u^{m-1} 0 < D_u D_u^{n-1} 0 = D_u^n 0\)である。
		\item[] 特に\(\{D_u^m 0 \mid m \in \mathbb{N} \wedge m < n\} < D_u^n 0\)である。
		\item[]
		\item[] \(G_u D_u^n 0 = \{D_u^m 0 \mid m \in \mathbb{N} \wedge m < n\}\)かつ\(D_u^n 0 \in OT_{\textrm{B}}\)であることを\(n\)に関する数学的帰納法で示す。
		\item[] \(n = 0\)ならば、\(G_u D_u^n 0 = G_u 0 = \emptyset = \{D_u^m 0 \mid m \in \mathbb{N} \wedge m < n\}\)かつ\(D_u^n 0 = 0 \in OT_{\textrm{B}}\)である。
		\item[] \(n > 0\)とする。
		\begin{indented}
			\item 帰納法の仮定より\(G_u D_u^{n-1} 0 = \{D_u^m 0 \mid m \in \mathbb{N} \wedge m < n-1\}\)かつ\(D_u^{n-1} 0 \in OT_{\textrm{B}}\)である。
			\item \(G_u D_u^n 0 = \{D_u^{n-1} 0\} \cup G_u D_u^{n-1} 0 = \{D_u^m 0 \mid m \in \mathbb{N} \wedge m < n\}\)である。
			\item \(D_u^{n-1} 0 \in OT_{\textrm{B}}\)と\(G_u D_u^{n-1} 0 = \{D_u^m 0 \mid m \in \mathbb{N} \wedge m < n-1\} < D_u^{n-1} 0\)と\cite{buc1} p.201 (OT3)より\(D_u^n 0 \in OT_{\textrm{B}}\)である。
		\end{indented}
	\end{penumerate}
\end{hideableproof}

\begin{lemma}[\(\textrm{Trans}\)が標準形を保つこと]\label{Transが標準形を保つこと}
	任意の\(M \in ST_{\textrm{PS}}\)に対し、\(\textrm{Trans}(M) \in OT_{\textrm{B}}\)である。
\end{lemma}

\begin{hideableproof}
	\begin{indented}
		\item 以下では\nameref{順序数項の基本例}を断りなく用いる。
		\item
		\item \(k_0 := \min \{k \in \mathbb{N} \mid M \in S_kT_{\textrm{PS}}\}\)と置く\footnote{\(ST_{\textrm{PS}} = \bigcup_{k \in \mathbb{N}} S_kT_{\textrm{PS}}\)より\(\min\)は存在する。}。
		\item \(k_0\)に関する数学的帰納法で示す。
		\item \(k_0 = 0\)とする。
		\begin{indented}
			\item 一意な\((u,v) \in \mathbb{N}^2\)が存在して\(u \leq v\)かつ\(M = ((j,j))_{j=u}^{v}\)である。
			\item \(u = v = 0\)ならば、\(\textrm{Trans}\)の定義より\(\textrm{Trans}(M) = 0 \in OT_{\textrm{B}}\)である。
			\item \(u = v > 0\)ならば、\(\textrm{Trans}\)の定義より\(\textrm{Trans}(M) = D_u 0 \in OT_{\textrm{B}}\)である。
			\item \(u < v\)ならば、\nameref{公差(1,1)のペア数列のTransの基本性質}から\(\textrm{Trans}(M[n]) = D_u D_v 0 \in OT_{\textrm{B}}\)である。
		\end{indented}
		\item \(k_0 > 0\)とする。
		\begin{indented}
			\item \(N \in S_{k_0-1}T_{\textrm{PS}}\)と\(n \in \mathbb{N}_{+}\)を用いて\(M = N[n]\)と置く。
			\item 帰納法の仮定より\(\textrm{Trans}(N) \in OT_{\textrm{B}}\)である。
			\item \cite{buc1} Lemma 3.3より、任意の\(m \in \mathbb{N}\)に対し\(\textrm{Trans}(N)[m]\)と\(\textrm{Trans}(N)[0]^m\)はいずれも順序数項である。
			\item \(\textrm{Pred}(N) = M\)ならば、\nameref{Predと0の関係}よりある\(k \in \mathbb{N}\)が存在して\(\textrm{Trans}(M) = \textrm{Trans}(N)[0]^k\)となるので、\(\textrm{Trans}(M)\)は順序数項である。以下では\(\textrm{Pred}(N) \neq M\)とする。
			\item \(N[1] = \textrm{Pred}(N) \neq M = N[n]\)より特に\(n > 1\)である。
			\item \nameref{標準形の簡約性}から\(N\)は簡約である。\nameref{簡約性と係数の関係}から\(N\)は条件(A)と(B)を満たす。
			\item \(((0,0))[n] = ((0,0)) \in S_0T_{\textrm{B}}\)かつ\(k_0 > 0\)であるので、\(N \neq ((0,0))\)である。従って\(N\)は単項または複項である。
			\item \(N\)が単項であるとする。
			\begin{indented}
				\item \(\textrm{Trans}\)の再帰的定義中に導入した記号を\(N\)に対して定める。
				\item \(t_1 = 0\)とする。
				\item \nameref{Transが零項性を保つこと}より\(\textrm{Pred}(N)\)は零項である。従って\(j_1 = 1\)かつ\(N_{1,0} = 0\)である。\(N\)が単項でかつ条件(A)と(B)を満たすことから、\(N = ((0,0),(1,0))\)または\(N = ((0,0),(1,1))\)である。
				\begin{indented}
					\item \(N = ((0,0),(1,0))\)ならば、\(M = N[n] = ((0,0))_{j=0}^{n-1}\)であるので、\nameref{公差(0,0)のペア数列のTransの基本性質}から\(\textrm{Trans}(M) = (D_0 0) \times (n-1) \in OT_{\textrm{B}}\)である。
					\item \(N = ((0,0),(1,1))\)ならば、\(M = N[n] = ((j,0))_{j=0}^{n-1}\)であるので、\(n > 1\)と\nameref{公差(1,0)のペア数列のTransの基本性質}から\(\textrm{Trans}(M) = D_0^n 0 \in OT_{\textrm{B}}\)である。
				\end{indented}
				\item \(t_1 \neq 0\)とする\footnote{この時\(N\)に対し条件(I)~(VI)が意味を持つ。}。
				\item \nameref{Transが零項性を保つこと}より\(\textrm{Pred}(N)\)は零項でない。
				\item \(N\)が条件(I)を満たすとする。
				\begin{indented}
					\item \(j_1 = 1\)とする。
					\begin{indented}
						\item \(\textrm{Pred}(N)\)は零項でなくかつ\(N\)が条件(A)と(B)を満たすことから、一意な\(u \in \mathbb{N}\)が存在して\(u > 0\)かつ\(N = ((u,u),(u+1,0))\)である。
						\item \(M = N[n] = ((u,u))_{j=0}^{n-1}\)であるので、\nameref{公差(0,0)のペア数列のTransの基本性質}から\(\textrm{Trans}(N[n]) = (D_u 0) \times n \in OT_{\textrm{B}}\)である。
					\end{indented}
					\item \(j_1 > 1\)ならば、\nameref{条件(I)の下でのTransと基本列の交換関係} (1)より\(\textrm{Trans}(M) = \textrm{Trans}(N[n]) = \textrm{Trans}(N)[n-1] \in OT_{\textrm{B}}\)である。
				\end{indented}
				\item
				\item \(N\)が条件(II)を満たすとする。
				\begin{indented}
					\item \(P(t_2)_{J_1}\)の左端が\(D_{N_{1,j_0}}\)であるか否かに従って\(m_n := n-1\)または\(m_n := n-2\)と置く。
					\item \(N\)が単項であるので\((0,0) \leq_N (0,j_1)\)であるが、\(0\)は\(N\)許容であるため\((0,0) <_N^{\textrm{Next}} (0,j_1)\)でない。従って\(j_1 > 1\)である。
					\item \(n > 1\)であるので\(m_n \geq 0\)である。従って\nameref{条件(II)の下でのTransと基本列の交換関係} (2)より\(\textrm{Trans}(M) = \textrm{Trans}(N[n]) = \textrm{Trans}(N)[m_n] \in OT_{\textrm{B}}\)である。
				\end{indented}
				\item
				\item \(N\)が条件(III)か(IV)を満たすとする。
				\begin{indented}
					\item \(\textrm{Pred}(N) \neq M = N[n]\)より、\((1,j_{-2}) <_N^{\textrm{Next}} (1,j_1)\)を満たす一意な\(j_{-2} \in \mathbb{N}\)が存在する。\(N_{1,j_0} \geq N_{1,j_1}\)より\((1,j_0) <_N^{\textrm{Next}} (1,j_1)\)でないので、\(j_{-2} < j_0 < j_1\)である。従って\(j_1 > 1\)であり、\nameref{条件(III)か(IV)の下での基本列の基本性質}より以下が成り立つ:
					\begin{penumerate}
						\item \(M = N[n] = N[n+1][1]^{j_1-j_{-2}}\)である。
						\item \(\textrm{Trans}(N)[n-1] = \textrm{Trans}(N[n+1][1]^{j_1-1-j_{-2}})\)である。
					\end{penumerate}
					\item 特に\(\textrm{Pred}(N[n+1][1]^{j_1-1-j_{-2}}) = M\)かつ\(\textrm{Trans}(N[n+1][1]^{j_1-1-j_{-2}})\)が順序数項であるので、\nameref{Predと0の関係}よりある\(k \in \mathbb{N}\)が存在して\(\textrm{Trans}(M) = \textrm{Trans}(N[n+1][1]^{j_1-1-j_{-2}})[0]^k = \textrm{Trans}(N)[n-1][0]^k\)となる。従って\cite{buc1} Lemma 3.3より\(\textrm{Trans}(M)\)は順序数項である。
				\end{indented}
				\item
				\item \(N\)が条件(V)を満たすとする。
				\begin{indented}
					\item \(j_0\)が\(M\)許容ならば\(m_n := n-1\)と置く。
					\item \(j_0\)が非\(M\)許容ならば\(m_n := n\)と置く。
					\item \(j_1 > j_0+1 \geq 1\)と\nameref{条件(V)の下での基本列のscb分解}より、一意な\(u \in \mathbb{N}\)と\((s'_0,b'_0) \in (\Sigma^{< \omega})^2\)と\(t' \in T_{\textrm{B}}\)が存在して以下を満たす:
					\begin{penumerate}
						\item \((s'_0,D_u t_2,b'_0)\)は\(\textrm{Trans}(M[n])\)のscb分解である。
						\item \((s'_0,D_u(t_2 + D_{M_{1,j_0}} 0),b'_0)\)は\(\textrm{Trans}(M)[m_n]\)のscb分解である。
					\end{penumerate}
					\item 従って\nameref{順序数項の末尾単項の零化可能性}よりある\(k \in \mathbb{N}\)が存在して\(\textrm{Trans}(M)  = \textrm{Trans}(N[n]) = \textrm{Trans}(N)[m_n][0]^k\)であるので、\cite{buc1} Lemma 3.3より\(\textrm{Trans}(M)\)は順序数項である。
				\end{indented}
				\item
				\item \(N\)が条件(VI)を満たすとする。
				\begin{indented}
					\item \(j_1 = 1\)とする。
					\begin{indented}
						\item \(\textrm{Pred}(N)\)は零項でなくかつ\(N\)が条件(A)と(B)を満たすことから、一意な\(u \in \mathbb{N}\)が存在して\(u > 0\)かつ\(N = ((u,u),(u+1,u+1))\)である。
						\item \(N[n] = ((u+j,u))_{j=0}^{n-1}\)であるので、\nameref{公差(1,0)のペア数列のTransの基本性質}から\(\textrm{Trans}(N[n]) = D_u^n 0 \in OT_{\textrm{B}}\)である。
					\end{indented}
					\item \(j_1 > 1\)とする。
					\item \(j_0\)が\(M\)許容ならば\(m_n := n-2\)と置く。
					\item \(j_0\)が非\(M\)許容ならば\(m_n := n-1\)と置く。
					\item \(n > 1\)より\(m_n \geq 0\)であるので、\nameref{条件(VI)の下でのTransと基本列の交換関係} (2)より\(\textrm{Trans}(M) = \textrm{Trans}(N[n]) = \textrm{Trans}(N)[m_n] \in OT_{\textrm{B}}\)である。
				\end{indented}
			\end{indented}
			\item
			\item \(N\)が複項であるとする。
			\begin{indented}
				\item \(J_1 := \textrm{Lng}(P(N))-1\)と置く。
				\item \nameref{Pの各成分の非複項性}から\(J_1 > 0\)である。\(\textrm{Pred}(N) \neq M = N[n]\)より\(\textrm{Lng}(P(N)_{J_1}) > 1\)である。従って\nameref{Pと基本列の関係} (2)から\(P(M) = P(N[n]) = (P(N)_J)_{J=0}^{J_1-1} \oplus_{T_{\textrm{PS}}} P(P(N)_{J_1}[n])\)である。
				\item \(J_0 := \textrm{Lng}(P(P(N)_{J_1}[n]))-1\)と置く。
				\item 写像
				\begin{eqnarray*}
				\textrm{PTrans} \colon PT_{\textrm{PS}} & \to & T_{\textrm{B}} \\
				M' & \mapsto & \textrm{PTrans}(M')
				\end{eqnarray*}
				\item を以下のように定める:
				\begin{indented}
					\item \(M'\)が零項であるならば\(\textrm{PTrans}(M') = D_0 0\)である。
					\item \(M'\)が零項でないならば\(\textrm{PTrans}(M') = \textrm{Trans}(M')\)である。
				\end{indented}
				\item \(a_0 := (\textrm{Trans}(P(N)_0)) \oplus_{T_{\textrm{B}}} (\textrm{PTrans}(P(N)_J)_{J=1}^{J_1-1})\)と置く。
				\item \(a_1 := (\textrm{PTrans}(P(P(N)_{J_1}[n])_{J'}))_{J'=0}^{J_0}\)と置く。
				\item \(a_2 := (\textrm{Trans}(P(P(N)_{J_1}[n])_0)) \oplus_{T_{\textrm{B}}} (\textrm{PTrans}(P(P(N)_{J_1}[n])_{J'}))_{J'=1}^{J_0}\)と置く。
				\item \nameref{Pの各成分の非複項性}と\nameref{Transが零項性を保つこと}と\nameref{Transが単項性を保つこと}から、\(a_0\)と\(a_1\)と\(a_2\)の各成分は\(0\)または単項であり、かつ\(0\)は左端にしか現れない。
				\item \(\textrm{Trans}\)の再帰的定義から、任意の\(M' \in RT_{\textrm{PS}} \cap MT_{\textrm{PS}}\)に対し、\(J'_1 := \textrm{Lng}(M')-1\)と置くと
				\begin{eqnarray*}
				\textrm{Trans}(M') = \textrm{Trans}(P(M')_0) + \Sigma_{\textrm{B}} (\textrm{PTrans}(P(M')_J))_{J=1}^{J'_1-1} + \textrm{PTrans}(P(M')_{J'_1})
				\end{eqnarray*}
				\item となるので、特に
				\begin{eqnarray*}
				\textrm{Trans}(N) & = & \Sigma_{\textrm{B}} a_0 + \textrm{PTrans}(P(N)_{J_1}) \\
				\textrm{Trans}(M) & = & \Sigma_{\textrm{B}} (a_0 \oplus_{T_{\textrm{B}}} a_1) \\
				\textrm{Trans}(P(N)_{J_1}[n]) & = & \Sigma_{\textrm{B}} a_2 \\
				\end{eqnarray*}
				\item である。
				\item
				\item \(a_0 \oplus_{T_{\textrm{B}}} a_1\)の各成分は順序数項であることを示す。
				\item \(\textrm{Lng}(P(N)_{J_1}) > 1\)より\(P(N)_{J_1}\)は零項でない。更に\nameref{Pの各成分の非複項性}から\(P(N)_{J_1}\)は複項でもないので、単項である。\nameref{Transが単項性を保つこと}より\(\textrm{Trans}(P(N)_{J_1})\)は単項である。
				\item \(P(N)_{J_1}\)が零項でないことから\(\textrm{PTrans}(P(N)_{J_1}) = \textrm{Trans}(P(N)_{J_1})\)である。従って\(\textrm{Trans}(N) = \Sigma_{\textrm{B}} a_0 + \textrm{Trans}(P(N)_{J_1})\)である。\(\textrm{Trans}(N)\)は順序数項であるので、Buchholzの順序数表記系の再帰的定義から\(a_0\)の各成分と\(\textrm{Trans}(P(N)_{J_1})\)は順序数項でありかつ\(a_0 \oplus_{T_{\textrm{B}}} (\textrm{Trans}(P(N)_{J_1}))\)の成分から\(0\)を除いた\(PT_{\textrm{B}}\)値配列は降順である。
				\item \nameref{標準形の単項成分が標準形であること}から\(P(N)_{J_1} \in S_{k_0-1}T_{\textrm{PS}}\)であるので、\(P(N)_{J_1}[n] \in S_{k_0}T_{\textrm{PS}}\)である。
				\item \(\textrm{Lng}(P(N)_{J_1}) > 1\)かつ標準形であることから、\nameref{基本列の降下性}より\(\Sigma_{\textrm{B}} a_2 = \textrm{Trans}(P(N)_{J_1}[n]) < \textrm{Trans}(P(N)_{J_1})\)である。従って\(a_2\)の各成分は\(\textrm{Trans}(P(N)_{J_1})\)未満である。
				\item \(P(N)_{J_1}[n] \in S_{k_0-1}T_{\textrm{PS}}\)であるならば、帰納法の仮定から\(\textrm{Trans}(P(N)_{J_1}[n])\)は順序数項である。
				\item \(P(N)_{J_1}[n] \in S_{k_0-1}T_{\textrm{PS}}\)でないならば、\(k_0 = \min \{k \in \mathbb{N} \mid P(N)_{J_1} \in S_kT_{\textrm{PS}}\}\)であるので\(P(N)_{J_1}\)が単項であることから\(\textrm{Trans}(P(N)_{J_1}[n])\)は順序数項である\footnote{\(k_0 = \min \{k \in \mathbb{N} \mid N' \in S_kT_{\textrm{PS}}\}\)を満たす任意の\(N' \in PT_{\textrm{B}}\)に対し\(\textrm{Trans}(N'[n])\)が単項であることは既に場合分けの中で示した。}。
				\item 従っていずれの場合も\(\textrm{Trans}(P(N)_{J_1}[n])\)は順序数項である。
				\item \(\textrm{Trans}(P(N)_{J_1}[n])\)が順序数項でありかつ\(\textrm{Trans}(P(N)_{J_1}[n]) = \Sigma_{\textrm{B}} a_2\)であるので、Buchholzの順序数表記系の再帰的定義から\(a_2\)の各成分は順序数項でありかつ\(a_2\)の成分から\(0\)を除いた\(PT_{\textrm{B}}\)値配列は降順である。\(0\)が順序数項であることから、\(a_1\)の各成分も順序数項である。
				\item 以上より、\(a_0 \oplus_{T_{\textrm{B}}} a_1\)の各成分は順序数項である。
				\item
				\item \(a_0 \oplus_{T_{\textrm{B}}} a_1\)の成分から\(0\)を除いた\(PT_{\textrm{B}}\)値配列が降順であることを示す。。
				\item \(a_2\)が\(0\)を成分に持つとする。
				\begin{indented}
					\item \(a_2\)の左端のみ\(0\)であり、\(P(P(N)_{J_1}[n])_0\)は零項である。
					\item \(J_2 := \textrm{Lng}(P(P(N)_{J_1}[n]))-1\)と置く。
					\item \(J_2 = 0\)とする。
					\begin{indented}
						\item \(a_1 = (D_0 0)\)である。
						\item \(a_2\)の唯一の成分\(0\)が\(\textrm{Trans}(P(N)_{J_1})\)未満であり\(D_0 0\)が\(0\)の後続であることから、特に\(D_0 0 \leq \textrm{Trans}(P(N)_{J_1})\)である。\(a_0 \oplus_{T_{\textrm{B}}} (\textrm{Trans}(P(N)_{J_1}))\)の成分から\(0\)を除いた\(PT_{\textrm{B}}\)値配列は降順であったので、\(a_0 \oplus_{T_{\textrm{B}}} a_1\)の成分から\(0\)を除いた\(PT_{\textrm{B}}\)値配列も降順である。
					\end{indented}
					\item \(J_2 > 0\)とする。
					\begin{indented}
						\item \(P(N)_{J_1}\)の単項性と\nameref{非複項性と基本列の関係}から\(P(P(N)_{J_1}[n])\)の各成分は等しく零項となる。従って\(a_1 = (D_0 0)_{J=0}^{J_2}\)かつ\(a_2 = (0) \oplus_{T_{\textrm{B}}} (D_0 0)_{J=1}^{J_2}\)である。
						\item \(a_2\)の各成分が\(\textrm{Trans}(P(N)_{J_1})\)未満であることから、特に\(D_0 0 < \textrm{Trans}(P(N)_{J_1})\)である。すなわち\(a_1\)の各成分も\(\textrm{Trans}(P(N)_{J_1})\)未満である。\(a_0 \oplus_{T_{\textrm{B}}} (\textrm{Trans}(P(N)_{J_1}))\)の成分から\(0\)を除いた\(PT_{\textrm{B}}\)値配列は降順であったので、\(a_0 \oplus_{T_{\textrm{B}}} a_1\)の成分から\(0\)を除いた\(PT_{\textrm{B}}\)値配列も降順である。
					\end{indented}
				\end{indented}
				\item \(a_2\)が\(0\)を成分に持たないとする。
				\begin{indented}
					\item \nameref{Transが零項性を保つこと}から\(P(P(N)_{J_1}[n])\)のいずれの成分も零項でないので、\(a_1 = a_2\)である。特に\(a_1\)は降順でありかつ\(a_1\)の各成分は\(\textrm{Trans}(P(N)_{J_1})\)未満である。
					\item \(a_0 \oplus_{T_{\textrm{B}}} (\textrm{Trans}(P(N)_{J_1}))\)の成分から\(0\)を除いた\(PT_{\textrm{B}}\)値配列は降順であったので、\(a_0 \oplus_{T_{\textrm{B}}} a_1\)の成分から\(0\)を除いた\(PT_{\textrm{B}}\)値配列も降順である。
				\end{indented}
				\item
				\item 以上より、\(\textrm{Trans}(M) = \Sigma_{\textrm{B}} (a_0 \oplus_{T_{\textrm{B}}} a_1)\)は順序数項の降順の和である。従って順序数項の再帰的定義\cite{buc1} p. 201 (OT2)から\(\textrm{Trans}(M)\)は順序数項である。
			\end{indented}
		\end{indented}
	\end{indented}
\end{hideableproof}

\iffull{それでは本題に戻る。}\fi

\begin{hideableproof}[\nameref{標準形ペア数列システムの停止性}の証明]
	\begin{indented}
		\item \cite{buc1} Lemma 2.2より\((OT_{\textrm{B}},<)\)は整礎である。従って\((OT_{\textrm{B}},<)\)に対して数学的帰納法が適用可能である。
		\item 任意の\(t \in OT_{\textrm{B}}\)に対し、「任意の\(M \in ST_{\textrm{PS}}\)と\(n \in \mathbb{N}_{+}\)に対し\(\textrm{Trans}(M) = t\)ならば\((M,n) \in \textrm{Dom}(F)\)」であることを\(t\)に関する数学的帰納法で示す。
		\item \nameref{標準形の簡約性}から\(M\)は簡約である。
		\item \(t = 0\)とする。
		\begin{indented}
			\item \nameref{Transが零項性を保つこと}から\(M\)は零項である。従って\(\textrm{Lng}(M) = 1\)であり、\(F\)の定義より\((M,n) \in \textrm{Dom}(F)\)である。
		\end{indented}
		\item \(t = D_v 0\)を満たす一意な\(v \in \mathbb{N}_{+}\)が存在するとする。
		\begin{indented}
			\item \nameref{Transと非可算基数の関係}より\(M = ((v,v))\)である。
			\item \(\textrm{Lng}(M) = 1\)であるので、\(F\)の定義より\((M,n) \in \textrm{Dom}(F)\)である。
		\end{indented}
		\item \(t = D_v 0\)を満たす一意な\(v \in \mathbb{N}_{+}\)が存在せずかつ\(t > 0\)とする。
		\begin{indented}
			\item \nameref{1列ペア数列の基本性質}と\nameref{Transが零項性を保つこと}と\nameref{Transと非可算基数の関係}より\(\textrm{Lng}(M) > 1\)である。
			\item \nameref{基本列の降下性}より\(\textrm{Trans}(M[n]) < \textrm{Trans}(M) = t\)である。\(M\)が標準形であることと標準形の再帰的定義から\(M[n]\)も標準形であるので、\nameref{Transが標準形を保つこと}より\(\textrm{Trans}(M[n])\)は順序数項である。従って帰納法の仮定より、\((M[n],n) \in \textrm{Dom}(F)\)である。
			\item \nameref{F_Mと基本列の関係}より\((M,n) \in \textrm{Dom}(F)\)である。
		\end{indented}
	\end{indented}
\end{hideableproof}


\iffull

\section{謝辞}

ペア数列を解析するという課題は巨大数たんにいただきました。思えば遠い昔、2018年7月下旬のことです。まずは手始めに原始数列から解析しました。余談ですが、原始数列はふぃっしゅっしゅさんの著書「巨大数論」で一度読んだことがあります。2017年7月上旬のことでした。丸一日掛けてじっくり読み込んだことを今でも鮮明に覚えています。ちなみに巨大数論を読んだのも巨大数たんによる勧めからです。僕が初めて巨大数というものを知ったのもこの頃です。当時は\(\varepsilon_0\)も知らなかったので、新しいことばかりでした。その後、巨大数たん考案の「巨大数たんシステム」により原始数列を復習することができました。なので原始数列とは今回で3度目の出会いということになります。原始数列は\(\varepsilon_0\)とカントール標準形さえ理解していれば停止性も解析も証明はさほど難しくなく、特に先人の解析結果を参考にすることなく8月初旬には解析が終わりました。

次にペア数列と行きたいところでしたが、ここで「一般のバシク行列には厳密な定義が存在しない」という壁に阻まれます。幸いなことに2018年6月上旬にkoteitanさんによる数学的定式化を用いた分類法が公開されていたため、その手直しがある程度済んだ2018年8月下旬頃、ようやくバシク行列というものを学び始めることができました。ペア数列はとても難しく、原始数列とは天と地の差がありました。何度かそれでもBuchholzの表記系が大変強力であったため、証明の大まかな道筋を得ることができました。ちなみに簡約化の定義はゆきとさんによる数列解説の中で考案したものです。

ただ、大まかな道筋を厳密に書き切ることは極めて困難でした。数学的な難しさがあるわけではなく、ただ、どの議論も長々としてめんどくさいのです。表記系を用いた議論は文字列操作の組み合わせ的な命題を大量に証明する必要があり、1つ1つは「まあ書こうと思えば書けるでしょう」的な雰囲気を出していてもいざ書こうと思うと場合分けが非常に多く、いざ書き終わってみると「これは難しくはないにしても全く自明ではなかったな」という感想を抱くものばかりです。

そんなこんなで休憩時間をほぼ全てペア数列の停止性の証明を書き切ることに費やし、もう2018年11月中旬に差し掛かってしまいました。途中で何度か「本当に自分の解析結果はあっているのだろうか?」と不安になることもありました。原始数列の時と違ってペア数列は書き切る前に多少先人の解析結果を参考にしようと思ってGoogology wikiを調べてみたのですが、驚くことにそれらは数学的な証明に全く役に立たないものばかりでした。びっくりです。解析に使っている「\(\psi\)」という記号の定義が書いていない。最初はBuchholzの\(\psi\)関数かと思いきや、自分の解析結果と全く違うものばかり。どうもUNOCFとかBashicu's OCFといった未定義なものを使っているようです。未定義なだけならまだしも、その性質を調べれば多少参考になるかと思って調べてみると、表記系と順序数崩壊関数を混同している人が海外には大量にいるということが分かっただけでした。

そんな中、唯一参考になったのはKurohaKafkaさんの解析でした。Buchholzの\(\psi\)関数を用いており、解析結果も自分のものと大体あっていたので安心しました。ずれていたのは僕の解析が標準形でないペア数列を用いて書いていたためです。当時ペア数列が標準形か否かの判定方法を最初は知らなかったため、標準形に直す際にKurohaKafkaさんの解析結果やkoteitanさんとふぃっしゅっしゅさんからの教えをいただき、非常に助かりました。更にNaruyokoさんには大量のミスを直していただきました。ここに名前を挙げた方々を始めとするお世話になった方々に、今一度感謝の意を表明します。

\fi

\end{document}